\documentclass[a4paper, 12pt]{book}

%\usepackage[center]{titlesec}

\usepackage{amsfonts, amssymb, amsmath, amsthm, amsxtra}

\usepackage{foekfont}

\usepackage{MnSymbol}

\usepackage{pdfrender, xcolor}
%\pdfrender{StrokeColor=black,LineWidth=.4pt,TextRenderingMode=2}

\usepackage{minitoc}
\setcounter{tocdepth}{4}
\setcounter{minitocdepth}{4}
\setcounter{secnumdepth}{4}

\usepackage{graphicx}

\usepackage[english]{babel}
\usepackage[utf8]{inputenc}
%\usepackage{mathpazo}
%\usepackage{euler}
\usepackage{eucal}
\usepackage{bbm}
\usepackage{bm}
\usepackage{csquotes}
\usepackage[nottoc]{tocbibind}
\usepackage{appendix}
\usepackage{float}
\usepackage[T1]{fontenc}
\usepackage[
    left = \flqq{},% 
    right = \frqq{},% 
    leftsub = \flq{},% 
    rightsub = \frq{} %
]{dirtytalk}

\usepackage{imakeidx}
\makeindex

%\usepackage[dvipsnames]{xcolor}
\usepackage{hyperref}
    \hypersetup{
        colorlinks=true,
        linkcolor=teal,
        filecolor=pink,      
        urlcolor=teal,
        citecolor=magenta
    }
\usepackage{comment}

% You would set the PDF title, author, etc. with package options or
% \hypersetup.

\usepackage[backend=biber, style=alphabetic, sorting=nty]{biblatex}
    \addbibresource{bibliography.bib}

\raggedbottom

\usepackage{mathrsfs}
\usepackage{mathtools} 
\mathtoolsset{showonlyrefs}
%\usepackage{amsthm}
\renewcommand\qedsymbol{$\blacksquare$}
\usepackage{tikz-cd}
\tikzcdset{scale cd/.style={every label/.append style={scale=#1},
    cells={nodes={scale=#1}}}}
\usepackage{tikz}
\usepackage{setspace}
\usepackage[version=3]{mhchem}
\parskip=0.1in
\usepackage[margin=25mm]{geometry}

\usepackage{listings, lstautogobble}
\lstset{
	language=matlab,
	basicstyle=\scriptsize\ttfamily,
	commentstyle=\ttfamily\itshape\color{gray},
	stringstyle=\ttfamily,
	showstringspaces=false,
	breaklines=true,
	frameround=ffff,
	frame=single,
	rulecolor=\color{black},
	autogobble=true
}

\usepackage{todonotes,tocloft,xpatch,hyperref}

% This is based on classicthesis chapter definition
\let\oldsec=\section
\renewcommand*{\section}{\secdef{\Sec}{\SecS}}
\newcommand\SecS[1]{\oldsec*{#1}}%
\newcommand\Sec[2][]{\oldsec[\texorpdfstring{#1}{#1}]{#2}}%

\newcounter{istodo}[section]

% http://tex.stackexchange.com/a/61267/11984
\makeatletter
%\xapptocmd{\Sec}{\addtocontents{tdo}{\protect\todoline{\thesection}{#1}{}}}{}{}
\newcommand{\todoline}[1]{\@ifnextchar\Endoftdo{}{\@todoline{#1}}}
\newcommand{\@todoline}[3]{%
	\@ifnextchar\todoline{}
	{\contentsline{section}{\numberline{#1}#2}{#3}{}{}}%
}
\let\l@todo\l@subsection
\newcommand{\Endoftdo}{}

\AtEndDocument{\addtocontents{tdo}{\string\Endoftdo}}
\makeatother

\usepackage{lipsum}

%   Reduce the margin of the summary:
\def\changemargin#1#2{\list{}{\rightmargin#2\leftmargin#1}\item[]}
\let\endchangemargin=\endlist 

%   Generate the environment for the abstract:
\newcommand\summaryname{Abstract}
\newenvironment{abstract}%
    {\small\begin{center}%
    \bfseries{\summaryname} \end{center}}

\newtheorem{theorem}{Theorem}[section]
    \numberwithin{theorem}{subsection}
\newtheorem{proposition}{Proposition}[section]
    \numberwithin{proposition}{subsection}
\newtheorem{lemma}{Lemma}[section]
    \numberwithin{lemma}{subsection}
\newtheorem{claim}{Claim}[section]
    \numberwithin{claim}{subsection}
\newtheorem{question}{Question}[section]
    \numberwithin{question}{subsection}

\theoremstyle{definition}
    \newtheorem{definition}{Definition}[section]
        \numberwithin{definition}{subsection}

\theoremstyle{remark}
    \newtheorem{remark}{Remark}[section]
        \numberwithin{remark}{subsection}
    \newtheorem{example}{Example}[section]
        \numberwithin{example}{subsection}    
    \newtheorem{convention}{Convention}[section]
        \numberwithin{convention}{subsection}
    \newtheorem{corollary}{Corollary}[section]
        \numberwithin{corollary}{subsection}

\numberwithin{equation}{chapter}

\usepackage{fancyhdr} % Custom headers and footers
\pagestyle{fancyplain} % Makes all pages in the document conform to the custom headers and footers
\fancyhead[L]{}% Empty left header
\fancyhead[C]{} %SECTION TITLE
\fancyhead[R]{}% Empty right header
\fancyfoot[L]{}% Empty left footer
\fancyfoot[C]{\thepage}% PAGE NUMBERING
\fancyfoot[R]{}% Empty left footer

\setcounter{chapter}{-1}
\setcounter{section}{-1}

\renewcommand{\implies}{\Rightarrow}
\renewcommand{\cong}{\simeq}
\newcommand{\ladjoint}{\dashv}
\newcommand{\radjoint}{\vdash}
\newcommand{\<}{\langle}
\renewcommand{\>}{\rangle}
\newcommand{\ndiv}{\hspace{-2pt}\not|\hspace{5pt}}
\newcommand{\cond}{\blacktriangle}
\newcommand{\decond}{\triangle}
\newcommand{\solid}{\blacksquare}
\newcommand{\ot}{\leftarrow}
\renewcommand{\-}{\text{-}}
\renewcommand{\mapsto}{\leadsto}
\renewcommand{\leq}{\leqslant}
\renewcommand{\geq}{\geqslant}
\renewcommand{\setminus}{\smallsetminus}
\newcommand{\punc}{\overset{\circ}}
\renewcommand{\div}{\operatorname{div}}
\newcommand{\grad}{\operatorname{grad}}
\newcommand{\curl}{\operatorname{curl}}
\makeatletter
\DeclareRobustCommand{\cev}[1]{%
  {\mathpalette\do@cev{#1}}%
}
\newcommand{\do@cev}[2]{%
  \vbox{\offinterlineskip
    \sbox\z@{$\m@th#1 x$}%
    \ialign{##\cr
      \hidewidth\reflectbox{$\m@th#1\vec{}\mkern4mu$}\hidewidth\cr
      \noalign{\kern-\ht\z@}
      $\m@th#1#2$\cr
    }%
  }%
}
\makeatother

\newcommand{\N}{\mathbb{N}}
\newcommand{\Z}{\mathbb{Z}}
\newcommand{\Q}{\mathbb{Q}}
\newcommand{\R}{\mathbb{R}}
\newcommand{\bbC}{\mathbb{C}}
\NewDocumentCommand{\x}{e{_^}}{%
  \mathbin{\mathop{\times}\displaylimits
    \IfValueT{#1}{_{#1}}
    \IfValueT{#2}{^{#2}}
  }%
}
\NewDocumentCommand{\pushout}{e{_^}}{%
  \mathbin{\mathop{\sqcup}\displaylimits
    \IfValueT{#1}{_{#1}}
    \IfValueT{#2}{^{#2}}
  }%
}
\newcommand{\supp}{\operatorname{supp}}
\newcommand{\im}{\operatorname{im}}
\newcommand{\coim}{\operatorname{coim}}
\newcommand{\coker}{\operatorname{coker}}
\newcommand{\id}{\mathrm{id}}
\newcommand{\chara}{\operatorname{char}}
\newcommand{\trdeg}{\operatorname{trdeg}}
\newcommand{\rank}{\operatorname{rank}}
\newcommand{\trace}{\operatorname{tr}}
\newcommand{\length}{\operatorname{length}}
\newcommand{\height}{\operatorname{ht}}
\renewcommand{\span}{\operatorname{span}}
\newcommand{\e}{\epsilon}
\newcommand{\p}{\mathfrak{p}}
\newcommand{\q}{\mathfrak{q}}
\newcommand{\m}{\mathfrak{m}}
\newcommand{\n}{\mathfrak{n}}
\newcommand{\calF}{\mathcal{F}}
\newcommand{\calG}{\mathcal{G}}
\newcommand{\calO}{\mathcal{O}}
\newcommand{\F}{\mathbb{F}}
\DeclareMathOperator{\lcm}{lcm}
\newcommand{\gr}{\operatorname{gr}}
\newcommand{\vol}{\mathrm{vol}}
\newcommand{\ord}{\operatorname{ord}}
\newcommand{\projdim}{\operatorname{proj.dim}}
\newcommand{\injdim}{\operatorname{inj.dim}}
\newcommand{\flatdim}{\operatorname{flat.dim}}
\newcommand{\globdim}{\operatorname{glob.dim}}
\renewcommand{\Re}{\operatorname{Re}}
\renewcommand{\Im}{\operatorname{Im}}
\newcommand{\sgn}{\operatorname{sgn}}
\newcommand{\coad}{\operatorname{coad}}
\newcommand{\ch}{\operatorname{ch}} %characters of representations

\newcommand{\Ad}{\mathrm{Ad}}
\newcommand{\GL}{\mathrm{GL}}
\newcommand{\SL}{\mathrm{SL}}
\newcommand{\PGL}{\mathrm{PGL}}
\newcommand{\PSL}{\mathrm{PSL}}
\newcommand{\Sp}{\mathrm{Sp}}
\newcommand{\GSp}{\mathrm{GSp}}
\newcommand{\GSpin}{\mathrm{GSpin}}
\newcommand{\rmO}{\mathrm{O}}
\newcommand{\SO}{\mathrm{SO}}
\newcommand{\SU}{\mathrm{SU}}
\newcommand{\rmU}{\mathrm{U}}
\newcommand{\rmY}{\mathrm{Y}}
\newcommand{\rmu}{\mathrm{u}}
\newcommand{\rmV}{\mathrm{V}}
\newcommand{\gl}{\mathfrak{gl}}
\renewcommand{\sl}{\mathfrak{sl}}
\newcommand{\diag}{\mathfrak{diag}}
\newcommand{\pgl}{\mathfrak{pgl}}
\newcommand{\psl}{\mathfrak{psl}}
\newcommand{\fraksp}{\mathfrak{sp}}
\newcommand{\gsp}{\mathfrak{gsp}}
\newcommand{\gspin}{\mathfrak{gspin}}
\newcommand{\frako}{\mathfrak{o}}
\newcommand{\so}{\mathfrak{so}}
\newcommand{\su}{\mathfrak{su}}
%\newcommand{\fraku}{\mathfrak{u}}
\newcommand{\Spec}{\operatorname{Spec}}
\newcommand{\Spf}{\operatorname{Spf}}
\newcommand{\Spm}{\operatorname{Spm}}
\newcommand{\Spv}{\operatorname{Spv}}
\newcommand{\Spa}{\operatorname{Spa}}
\newcommand{\Spd}{\operatorname{Spd}}
\newcommand{\Proj}{\operatorname{Proj}}
\newcommand{\Gr}{\mathrm{Gr}}
\newcommand{\Hecke}{\mathrm{Hecke}}
\newcommand{\Sht}{\mathrm{Sht}}
\newcommand{\Quot}{\mathrm{Quot}}
\newcommand{\Hilb}{\mathrm{Hilb}}
\newcommand{\Pic}{\mathrm{Pic}}
\newcommand{\Div}{\mathrm{Div}}
\newcommand{\Jac}{\mathrm{Jac}}
\newcommand{\Alb}{\mathrm{Alb}} %albanese variety
\newcommand{\Bun}{\mathrm{Bun}}
\newcommand{\loopspace}{\mathbf{\Omega}}
\newcommand{\suspension}{\mathbf{\Sigma}}
\newcommand{\tangent}{\mathrm{T}} %tangent space
\newcommand{\Eig}{\mathrm{Eig}}
\newcommand{\Cox}{\mathrm{Cox}} %coxeter functors
\newcommand{\rmK}{\mathrm{K}} %Killing form
\newcommand{\km}{\mathfrak{km}} %kac-moody algebras
\newcommand{\Dyn}{\mathrm{Dyn}} %associated Dynkin quivers
\newcommand{\Car}{\mathrm{Car}} %cartan matrices of finite quivers
\newcommand{\uce}{\mathfrak{uce}} %universal central extension of lie algebras

\newcommand{\Ring}{\mathrm{Ring}}
\newcommand{\Cring}{\mathrm{CRing}}
\newcommand{\Alg}{\mathrm{Alg}}
\newcommand{\Leib}{\mathrm{Leib}} %leibniz algebras
\newcommand{\Fld}{\mathrm{Fld}}
\newcommand{\Sets}{\mathrm{Sets}}
\newcommand{\Equiv}{\mathrm{Equiv}} %equivalence relations
\newcommand{\Cat}{\mathrm{Cat}}
\newcommand{\Grp}{\mathrm{Grp}}
\newcommand{\Ab}{\mathrm{Ab}}
\newcommand{\Sch}{\mathrm{Sch}}
\newcommand{\Coh}{\mathrm{Coh}}
\newcommand{\QCoh}{\mathrm{QCoh}}
\newcommand{\Perf}{\mathrm{Perf}} %perfect complexes
\newcommand{\Sing}{\mathrm{Sing}} %singularity categories
\newcommand{\Desc}{\mathrm{Desc}}
\newcommand{\Sh}{\mathrm{Sh}}
\newcommand{\Psh}{\mathrm{PSh}}
\newcommand{\Fib}{\mathrm{Fib}}
\renewcommand{\mod}{\-\mathrm{mod}}
\newcommand{\comod}{\-\mathrm{comod}}
\newcommand{\bimod}{\-\mathrm{bimod}}
\newcommand{\Vect}{\mathrm{Vect}}
\newcommand{\Rep}{\mathrm{Rep}}
\newcommand{\Grpd}{\mathrm{Grpd}}
\newcommand{\Arr}{\mathrm{Arr}}
\newcommand{\Esp}{\mathrm{Esp}}
\newcommand{\Ob}{\mathrm{Ob}}
\newcommand{\Mor}{\mathrm{Mor}}
\newcommand{\Mfd}{\mathrm{Mfd}}
\newcommand{\Riem}{\mathrm{Riem}}
\newcommand{\RS}{\mathrm{RS}}
\newcommand{\LRS}{\mathrm{LRS}}
\newcommand{\TRS}{\mathrm{TRS}}
\newcommand{\TLRS}{\mathrm{TLRS}}
\newcommand{\LVRS}{\mathrm{LVRS}}
\newcommand{\LBRS}{\mathrm{LBRS}}
\newcommand{\Spc}{\mathrm{Spc}}
\newcommand{\Top}{\mathrm{Top}}
\newcommand{\Topos}{\mathrm{Topos}}
\newcommand{\Nil}{\mathfrak{nil}}
\newcommand{\J}{\mathfrak{J}}
\newcommand{\Stk}{\mathrm{Stk}}
\newcommand{\Pre}{\mathrm{Pre}}
\newcommand{\simp}{\mathbf{\Delta}}
\newcommand{\Res}{\mathrm{Res}}
\newcommand{\Ind}{\mathrm{Ind}}
\newcommand{\Pro}{\mathrm{Pro}}
\newcommand{\Mon}{\mathrm{Mon}}
\newcommand{\Comm}{\mathrm{Comm}}
\newcommand{\Fin}{\mathrm{Fin}}
\newcommand{\Assoc}{\mathrm{Assoc}}
\newcommand{\Semi}{\mathrm{Semi}}
\newcommand{\Co}{\mathrm{Co}}
\newcommand{\Loc}{\mathrm{Loc}}
\newcommand{\Ringed}{\mathrm{Ringed}}
\newcommand{\Haus}{\mathrm{Haus}} %hausdorff spaces
\newcommand{\Comp}{\mathrm{Comp}} %compact hausdorff spaces
\newcommand{\Stone}{\mathrm{Stone}} %stone spaces
\newcommand{\Extr}{\mathrm{Extr}} %extremely disconnected spaces
\newcommand{\Ouv}{\mathrm{Ouv}}
\newcommand{\Str}{\mathrm{Str}}
\newcommand{\Func}{\mathrm{Func}}
\newcommand{\Crys}{\mathrm{Crys}}
\newcommand{\LocSys}{\mathrm{LocSys}}
\newcommand{\Sieves}{\mathrm{Sieves}}
\newcommand{\pt}{\mathrm{pt}}
\newcommand{\Graphs}{\mathrm{Graphs}}
\newcommand{\Lie}{\mathrm{Lie}}
\newcommand{\Env}{\mathrm{Env}}
\newcommand{\Ho}{\mathrm{Ho}}
\newcommand{\rmD}{\mathrm{D}}
\newcommand{\Cov}{\mathrm{Cov}}
\newcommand{\Frames}{\mathrm{Frames}}
\newcommand{\Locales}{\mathrm{Locales}}
\newcommand{\Span}{\mathrm{Span}}
\newcommand{\Corr}{\mathrm{Corr}}
\newcommand{\Monad}{\mathrm{Monad}}
\newcommand{\Var}{\mathrm{Var}}
\newcommand{\sfN}{\mathrm{N}} %nerve
\newcommand{\Diam}{\mathrm{Diam}} %diamonds
\newcommand{\co}{\mathrm{co}}
\newcommand{\ev}{\mathrm{ev}}
\newcommand{\bi}{\mathrm{bi}}
\newcommand{\Nat}{\mathrm{Nat}}
\newcommand{\Hopf}{\mathrm{Hopf}}
\newcommand{\Dmod}{\mathrm{D}\mod}
\newcommand{\Perv}{\mathrm{Perv}}
\newcommand{\Sph}{\mathrm{Sph}}
\newcommand{\Moduli}{\mathrm{Moduli}}
\newcommand{\Pseudo}{\mathrm{Pseudo}}
\newcommand{\Lax}{\mathrm{Lax}}
\newcommand{\Strict}{\mathrm{Strict}}
\newcommand{\Opd}{\mathrm{Opd}} %operads
\newcommand{\Shv}{\mathrm{Shv}}
\newcommand{\Char}{\mathrm{Char}} %CharShv = character sheaves
\newcommand{\Huber}{\mathrm{Huber}}
\newcommand{\Tate}{\mathrm{Tate}}
\newcommand{\Affd}{\mathrm{Affd}} %affinoid algebras
\newcommand{\Adic}{\mathrm{Adic}} %adic spaces
\newcommand{\Rig}{\mathrm{Rig}}
\newcommand{\An}{\mathrm{An}}
\newcommand{\Perfd}{\mathrm{Perfd}} %perfectoid spaces
\newcommand{\Sub}{\mathrm{Sub}} %subobjects
\newcommand{\Ideals}{\mathrm{Ideals}}
\newcommand{\Isoc}{\mathrm{Isoc}} %isocrystals
\newcommand{\Ban}{\mathrm{Ban}} %Banach spaces
\newcommand{\Fre}{\mathrm{Fr\acute{e}}} %Frechet spaces
\newcommand{\Ch}{\mathrm{Ch}} %chain complexes
\newcommand{\Pure}{\mathrm{Pure}}
\newcommand{\Mixed}{\mathrm{Mixed}}
\newcommand{\Hodge}{\mathrm{Hodge}} %Hodge structures
\newcommand{\Mot}{\mathrm{Mot}} %motives
\newcommand{\KL}{\mathrm{KL}} %category of Kazhdan-Lusztig modules
\newcommand{\Pres}{\mathrm{Pres}} %presentable categories
\newcommand{\Noohi}{\mathrm{Noohi}} %category of Noohi groups
\newcommand{\Inf}{\mathrm{Inf}}
\newcommand{\LPar}{\mathrm{LPar}} %Langlands parameters
\newcommand{\ORig}{\mathrm{ORig}} %overconvergent sites
\newcommand{\Quiv}{\mathrm{Quiv}} %quivers
\newcommand{\Def}{\mathrm{Def}} %deformation functors
\newcommand{\Root}{\mathrm{Root}}
\newcommand{\gRep}{\mathrm{gRep}}
\newcommand{\Higgs}{\mathrm{Higgs}}
\newcommand{\BGG}{\mathrm{BGG}}
\newcommand{\Poiss}{\mathrm{Poiss}}

\newcommand{\Aut}{\mathrm{Aut}}
\newcommand{\Inn}{\mathrm{Inn}}
\newcommand{\Out}{\mathrm{Out}}
\newcommand{\der}{\mathfrak{der}} %derivations on Lie algebras
\newcommand{\frakend}{\mathfrak{end}}
\newcommand{\aut}{\mathfrak{aut}}
\newcommand{\inn}{\mathfrak{inn}} %inner derivations
\newcommand{\out}{\mathfrak{out}} %outer derivations
\newcommand{\Stab}{\mathrm{Stab}}
\newcommand{\Cent}{\mathrm{Cent}}
\newcommand{\Norm}{\mathrm{Norm}}
\newcommand{\cent}{\mathfrak{cent}}
\newcommand{\norm}{\mathfrak{norm}}
\newcommand{\Rad}{\operatorname{Rad}}
\newcommand{\Transporter}{\mathrm{Transp}} %transporter between two subsets of a group
\newcommand{\Conj}{\mathrm{Conj}}
\newcommand{\Diag}{\mathrm{Diag}}
\newcommand{\Gal}{\mathrm{Gal}}
\newcommand{\bfG}{\mathbf{G}} %absolute Galois group
\newcommand{\Frac}{\mathrm{Frac}}
\newcommand{\Ann}{\mathrm{Ann}}
\newcommand{\Val}{\mathrm{Val}}
\newcommand{\Chow}{\mathrm{Chow}}
\newcommand{\Sym}{\mathrm{Sym}}
\newcommand{\End}{\mathrm{End}}
\newcommand{\Mat}{\mathrm{Mat}}
\newcommand{\Diff}{\mathrm{Diff}}
\newcommand{\Autom}{\mathrm{Autom}}
\newcommand{\Artin}{\mathrm{Artin}} %artin maps
\newcommand{\sk}{\mathrm{sk}} %skeleton of a category
\newcommand{\eqv}{\mathrm{eqv}} %functor that maps groups $G$ to $G$-sets
\newcommand{\Inert}{\mathrm{Inert}}
\newcommand{\Fil}{\mathrm{Fil}}
\newcommand{\Prim}{\mathfrak{Prim}}
\newcommand{\Nerve}{\mathrm{N}}
\newcommand{\Hol}{\mathrm{Hol}} %holomorphic functions %holonomy groups
\newcommand{\Bi}{\mathrm{Bi}} %Bi for biholomorphic functions
\newcommand{\chev}{\operatorname{chev}}
\newcommand{\bfLie}{\mathbf{Lie}} %non-reduced lie algebra associated to generalised cartan matrices
\newcommand{\frakLie}{\mathfrak{Lie}} %reduced lie algebra associated to generalised cartan matrices
\newcommand{\frakChev}{\mathfrak{Chev}} 
\newcommand{\Rees}{\operatorname{Rees}}
\newcommand{\Dr}{\mathrm{Dr}} %Drinfeld's quantum double 
\newcommand{\frakDr}{\mathfrak{Dr}} %classical double of lie bialgebras
\newcommand{\Tot}{\operatorname{Tot}} %total complexes

\renewcommand{\projlim}{\varprojlim}
\newcommand{\indlim}{\varinjlim}
\newcommand{\colim}{\operatorname{colim}}
\renewcommand{\lim}{\operatorname{lim}}
\newcommand{\toto}{\rightrightarrows}
%\newcommand{\tensor}{\otimes}
\NewDocumentCommand{\tensor}{e{_^}}{%
  \mathbin{\mathop{\otimes}\displaylimits
    \IfValueT{#1}{_{#1}}
    \IfValueT{#2}{^{#2}}
  }%
}
\NewDocumentCommand{\singtensor}{e{_^}}{%
  \mathbin{\mathop{\odot}\displaylimits
    \IfValueT{#1}{_{#1}}
    \IfValueT{#2}{^{#2}}
  }%
}
\NewDocumentCommand{\hattensor}{e{_^}}{%
  \mathbin{\mathop{\hat{\otimes}}\displaylimits
    \IfValueT{#1}{_{#1}}
    \IfValueT{#2}{^{#2}}
  }%
}
\NewDocumentCommand{\brevetensor}{e{_^}}{%
  \mathbin{\mathop{\breve{\otimes}}\displaylimits
    \IfValueT{#1}{_{#1}}
    \IfValueT{#2}{^{#2}}
  }%
}
\NewDocumentCommand{\semidirect}{e{_^}}{%
  \mathbin{\mathop{\rtimes}\displaylimits
    \IfValueT{#1}{_{#1}}
    \IfValueT{#2}{^{#2}}
  }%
}
\newcommand{\eq}{\operatorname{eq}}
\newcommand{\coeq}{\operatorname{coeq}}
\newcommand{\Hom}{\mathrm{Hom}}
\newcommand{\Maps}{\mathrm{Maps}}
\newcommand{\Tor}{\mathrm{Tor}}
\newcommand{\Ext}{\mathrm{Ext}}
\newcommand{\Isom}{\mathrm{Isom}}
\newcommand{\stalk}{\mathrm{stalk}}
\newcommand{\RKE}{\operatorname{RKE}}
\newcommand{\LKE}{\operatorname{LKE}}
\newcommand{\oblv}{\mathrm{oblv}}
\newcommand{\const}{\mathrm{const}}
\newcommand{\free}{\mathrm{free}}
\newcommand{\adrep}{\mathrm{ad}} %adjoint representation
\newcommand{\NL}{\mathbb{NL}} %naive cotangent complex
\newcommand{\pr}{\operatorname{pr}}
\newcommand{\Der}{\mathrm{Der}}
\newcommand{\Frob}{\mathrm{Fr}} %Frobenius
\newcommand{\frob}{\mathrm{f}} %trace of Frobenius
\newcommand{\bfpt}{\mathbf{pt}}
\newcommand{\bfloc}{\mathbf{loc}}
\DeclareMathAlphabet{\mymathbb}{U}{BOONDOX-ds}{m}{n}
\newcommand{\0}{\mymathbb{0}}
\newcommand{\1}{\mathbbm{1}}
\newcommand{\2}{\mathbbm{2}}
\newcommand{\Jet}{\mathrm{Jet}}
\newcommand{\Split}{\mathrm{Split}}
\newcommand{\Sq}{\mathrm{Sq}}
\newcommand{\Zero}{\mathrm{Z}}
\newcommand{\SqZ}{\Sq\Zero}
\newcommand{\lie}{\mathfrak{lie}}
\newcommand{\y}{\mathrm{y}} %yoneda
\newcommand{\Sm}{\mathrm{Sm}}
\newcommand{\AJ}{\phi} %abel-jacobi map
\newcommand{\act}{\mathrm{act}}
\newcommand{\ram}{\mathrm{ram}} %ramification index
\newcommand{\inv}{\mathrm{inv}}
\newcommand{\Spr}{\mathrm{Spr}} %the Springer map/sheaf
\newcommand{\Refl}{\mathrm{Refl}} %reflection functor]
\newcommand{\HH}{\mathrm{HH}} %Hochschild (co)homology
\newcommand{\HC}{\mathrm{HC}} %cyclic (co)homology
\newcommand{\Poinc}{\mathrm{Poinc}}
\newcommand{\Simpson}{\mathrm{Simpson}}
\newcommand{\Section}{\operatorname{Sect}}

\newcommand{\bbU}{\mathbb{U}}
\newcommand{\V}{\mathbb{V}}
\newcommand{\W}{\mathbb{W}}
\newcommand{\calU}{\mathcal{U}}
\newcommand{\calW}{\mathcal{W}}
\newcommand{\rmI}{\mathrm{I}} %augmentation ideal
\newcommand{\bfV}{\mathbf{V}}
\newcommand{\C}{\mathcal{C}}
\newcommand{\D}{\mathcal{D}}
\newcommand{\scrD}{\mathscr{D}}
\newcommand{\T}{\mathscr{T}} %Tate modules
\newcommand{\calM}{\mathcal{M}}
\newcommand{\calN}{\mathcal{N}}
\newcommand{\calP}{\mathcal{P}}
\newcommand{\calQ}{\mathcal{Q}}
\newcommand{\A}{\mathbb{A}}
\renewcommand{\P}{\mathbb{P}}
\newcommand{\calL}{\mathcal{L}}
\newcommand{\scrL}{\mathscr{L}}
\newcommand{\E}{\mathcal{E}}
\renewcommand{\H}{\mathbf{H}}
\newcommand{\scrS}{\mathscr{S}}
\newcommand{\calX}{\mathcal{X}}
\newcommand{\calY}{\mathcal{Y}}
\newcommand{\calZ}{\mathcal{Z}}
\newcommand{\calS}{\mathcal{S}}
\newcommand{\calR}{\mathcal{R}}
\newcommand{\scrX}{\mathscr{X}}
\newcommand{\scrY}{\mathscr{Y}}
\newcommand{\scrZ}{\mathscr{Z}}
\newcommand{\calA}{\mathcal{A}}
\newcommand{\calB}{\mathcal{B}}
\renewcommand{\S}{\mathcal{S}}
\newcommand{\B}{\mathbb{B}}
\newcommand{\bbD}{\mathbb{D}}
\newcommand{\G}{\mathbb{G}}
\newcommand{\horn}{\mathbf{\Lambda}}
\renewcommand{\L}{\mathbb{L}}
\renewcommand{\a}{\mathfrak{a}}
\renewcommand{\b}{\mathfrak{b}}
\renewcommand{\c}{\mathfrak{c}}
\renewcommand{\d}{\mathfrak{d}}
\renewcommand{\t}{\mathfrak{t}}
\renewcommand{\r}{\mathfrak{r}}
\newcommand{\fraku}{\mathfrak{u}}
\newcommand{\frakv}{\mathfrak{v}}
\newcommand{\frake}{\mathfrak{e}}
\newcommand{\bbX}{\mathbb{X}}
\newcommand{\frakw}{\mathfrak{w}}
\newcommand{\frakG}{\mathfrak{G}}
\newcommand{\frakH}{\mathfrak{H}}
\newcommand{\frakE}{\mathfrak{E}}
\newcommand{\frakF}{\mathfrak{F}}
\newcommand{\g}{\mathfrak{g}}
\newcommand{\frakl}{\mathfrak{l}}
\newcommand{\h}{\mathfrak{h}}
\renewcommand{\k}{\mathfrak{k}}
\newcommand{\z}{\mathfrak{z}}
\newcommand{\fraki}{\mathfrak{i}}
\newcommand{\frakj}{\mathfrak{j}}
\newcommand{\del}{\partial}
\newcommand{\bbE}{\mathbb{E}}
\newcommand{\scrO}{\mathscr{O}}
\newcommand{\bbO}{\mathbb{O}}
\newcommand{\scrA}{\mathscr{A}}
\newcommand{\scrB}{\mathscr{B}}
\newcommand{\scrE}{\mathscr{E}}
\newcommand{\scrF}{\mathscr{F}}
\newcommand{\scrG}{\mathscr{G}}
\newcommand{\scrM}{\mathscr{M}}
\newcommand{\scrN}{\mathscr{N}}
\newcommand{\scrP}{\mathscr{P}}
\newcommand{\frakS}{\mathfrak{S}}
\newcommand{\frakT}{\mathfrak{T}}
\newcommand{\calI}{\mathcal{I}}
\newcommand{\calJ}{\mathcal{J}}
\newcommand{\scrI}{\mathscr{I}}
\newcommand{\scrJ}{\mathscr{J}}
\newcommand{\scrK}{\mathscr{K}}
\newcommand{\calK}{\mathcal{K}}
\newcommand{\scrV}{\mathscr{V}}
\newcommand{\scrW}{\mathscr{W}}
\newcommand{\bbS}{\mathbb{S}}
\newcommand{\scrH}{\mathscr{H}}
\newcommand{\bfA}{\mathbf{A}}
\newcommand{\bfB}{\mathbf{B}}
\newcommand{\bfC}{\mathbf{C}}
\renewcommand{\O}{\mathbb{O}}
\newcommand{\calV}{\mathcal{V}}
\newcommand{\scrR}{\mathscr{R}} %radical
\newcommand{\sfR}{\mathsf{R}} %quantum R-matrices
\newcommand{\sfr}{\mathsf{r}} %classical R-matrices
\newcommand{\rmZ}{\mathrm{Z}} %centre of algebra
\newcommand{\rmC}{\mathrm{C}} %centralisers in algebras
\newcommand{\bfGamma}{\mathbf{\Gamma}}
\newcommand{\scrU}{\mathscr{U}}
\newcommand{\rmW}{\mathrm{W}} %Weil group
\newcommand{\frakM}{\mathfrak{M}}
\newcommand{\frakN}{\mathfrak{N}}
\newcommand{\frakB}{\mathfrak{B}}
\newcommand{\frakX}{\mathfrak{X}}
\newcommand{\frakY}{\mathfrak{Y}}
\newcommand{\frakZ}{\mathfrak{Z}}
\newcommand{\frakU}{\mathfrak{U}}
\newcommand{\frakR}{\mathfrak{R}}
\newcommand{\frakP}{\mathfrak{P}}
\newcommand{\frakQ}{\mathfrak{Q}}
\newcommand{\sfX}{\mathsf{X}}
\newcommand{\sfY}{\mathsf{Y}}
\newcommand{\sfZ}{\mathsf{Z}}
\newcommand{\sfS}{\mathsf{S}}
\newcommand{\sfT}{\mathsf{T}}
\newcommand{\sfOmega}{\mathsf{\Omega}} %drinfeld p-adic upper-half plane
\newcommand{\rmA}{\mathrm{A}}
\newcommand{\rmB}{\mathrm{B}}
\newcommand{\calT}{\mathcal{T}}
\newcommand{\sfA}{\mathsf{A}}
\newcommand{\sfB}{\mathsf{B}}
\newcommand{\sfC}{\mathsf{C}}
\newcommand{\sfD}{\mathsf{D}}
\newcommand{\sfE}{\mathsf{E}}
\newcommand{\sfF}{\mathsf{F}}
\newcommand{\sfG}{\mathsf{G}}
\newcommand{\frakL}{\mathfrak{L}}
\newcommand{\K}{\mathrm{K}}
\newcommand{\rmT}{\mathrm{T}}
\newcommand{\bfv}{\mathbf{v}}
\newcommand{\bfg}{\mathbf{g}}
\newcommand{\frakV}{\mathfrak{V}}
\newcommand{\bfn}{\mathbf{n}}
\renewcommand{\o}{\mathfrak{o}}
\newcommand{\standard}{\Delta}
\newcommand{\costandard}{\nabla}
\newcommand{\simple}{\bar{\Delta}}
\newcommand{\cosimple}{\bar{\nabla}}

\newcommand{\aff}{\mathrm{aff}}
\newcommand{\ft}{\mathrm{ft}} %finite type
\newcommand{\fp}{\mathrm{fp}} %finite presentation
\newcommand{\fr}{\mathrm{fr}} %free
\newcommand{\tft}{\mathrm{tft}} %topologically finite type
\newcommand{\tfp}{\mathrm{tfp}} %topologically finite presentation
\newcommand{\tfr}{\mathrm{tfr}} %topologically free
\newcommand{\aft}{\mathrm{aft}}
\newcommand{\lft}{\mathrm{lft}}
\newcommand{\laft}{\mathrm{laft}}
\newcommand{\cpt}{\mathrm{cpt}}
\newcommand{\cproj}{\mathrm{cproj}}
\newcommand{\qc}{\mathrm{qc}}
\newcommand{\qs}{\mathrm{qs}}
\newcommand{\lcmpt}{\mathrm{lcmpt}}
\newcommand{\red}{\mathrm{red}}
\newcommand{\fin}{\mathrm{fin}}
\newcommand{\fd}{\mathrm{fd}} %finite-dimensional
\newcommand{\gen}{\mathrm{gen}}
\newcommand{\petit}{\mathrm{petit}}
\newcommand{\gros}{\mathrm{gros}}
\newcommand{\loc}{\mathrm{loc}}
\newcommand{\glob}{\mathrm{glob}}
%\newcommand{\ringed}{\mathrm{ringed}}
%\newcommand{\qcoh}{\mathrm{qcoh}}
\newcommand{\cl}{\mathrm{cl}}
\newcommand{\et}{\mathrm{\acute{e}t}}
\newcommand{\fet}{\mathrm{f\acute{e}t}}
\newcommand{\profet}{\mathrm{prof\acute{e}t}}
\newcommand{\proet}{\mathrm{pro\acute{e}t}}
\newcommand{\Zar}{\mathrm{Zar}}
\newcommand{\fppf}{\mathrm{fppf}}
\newcommand{\fpqc}{\mathrm{fpqc}}
\newcommand{\orig}{\mathrm{orig}} %overconvergent topology
\newcommand{\smooth}{\mathrm{sm}}
\newcommand{\sh}{\mathrm{sh}}
\newcommand{\op}{\mathrm{op}}
\newcommand{\cop}{\mathrm{cop}}
\newcommand{\open}{\mathrm{open}}
\newcommand{\closed}{\mathrm{closed}}
\newcommand{\geom}{\mathrm{geom}}
\newcommand{\alg}{\mathrm{alg}}
\newcommand{\sober}{\mathrm{sober}}
\newcommand{\dR}{\mathrm{dR}}
\newcommand{\rad}{\mathfrak{rad}}
\newcommand{\discrete}{\mathrm{discrete}}
%\newcommand{\add}{\mathrm{add}}
%\newcommand{\lin}{\mathrm{lin}}
\newcommand{\Krull}{\mathrm{Krull}}
\newcommand{\qis}{\mathrm{qis}} %quasi-isomorphism
\newcommand{\ho}{\mathrm{ho}} %homotopy equivalence
\newcommand{\sep}{\mathrm{sep}}
\newcommand{\unr}{\mathrm{unr}}
\newcommand{\tame}{\mathrm{tame}}
\newcommand{\wild}{\mathrm{wild}}
\newcommand{\nil}{\mathrm{nil}}
\newcommand{\defm}{\mathrm{defm}}
\newcommand{\Art}{\mathrm{Art}}
\newcommand{\Noeth}{\mathrm{Noeth}}
\newcommand{\affd}{\mathrm{affd}}
%\newcommand{\adic}{\mathrm{adic}}
\newcommand{\pre}{\mathrm{pre}}
\newcommand{\coperf}{\mathrm{coperf}}
\newcommand{\perf}{\mathrm{perf}}
\newcommand{\perfd}{\mathrm{perfd}}
\newcommand{\rat}{\mathrm{rat}}
\newcommand{\cont}{\mathrm{cont}}
\newcommand{\dg}{\mathrm{dg}}
\newcommand{\almost}{\mathrm{a}}
%\newcommand{\stab}{\mathrm{stab}}
\newcommand{\heart}{\heartsuit}
\newcommand{\proj}{\mathrm{proj}}
\newcommand{\qproj}{\mathrm{qproj}}
\newcommand{\pd}{\mathrm{pd}}
\newcommand{\crys}{\mathrm{crys}}
\newcommand{\prisma}{\mathrm{prisma}}
\newcommand{\FF}{\mathrm{FF}}
\newcommand{\sph}{\mathrm{sph}}
\newcommand{\lax}{\mathrm{lax}}
\newcommand{\weak}{\mathrm{weak}}
\newcommand{\strict}{\mathrm{strict}}
\newcommand{\mon}{\mathrm{mon}}
\newcommand{\sym}{\mathrm{sym}}
\newcommand{\lisse}{\mathrm{lisse}}
\newcommand{\an}{\mathrm{an}}
\newcommand{\ad}{\mathrm{ad}}
\newcommand{\sch}{\mathrm{sch}}
\newcommand{\rig}{\mathrm{rig}}
\newcommand{\pol}{\mathrm{pol}}
\newcommand{\plat}{\mathrm{flat}}
\newcommand{\proper}{\mathrm{proper}}
\newcommand{\compl}{\mathrm{compl}}
\newcommand{\non}{\mathrm{non}}
\newcommand{\access}{\mathrm{access}}
\newcommand{\comp}{\mathrm{comp}}
\newcommand{\tstructure}{\mathrm{t}} %t-structures
\newcommand{\pure}{\mathrm{pure}} %pure motives
\newcommand{\mixed}{\mathrm{mixed}} %mixed motives
\newcommand{\num}{\mathrm{num}} %numerical motives
\newcommand{\ess}{\mathrm{ess}}
\newcommand{\topological}{\mathrm{top}}
\newcommand{\convex}{\mathrm{cvx}}
\newcommand{\locconvex}{\mathrm{lcvx}}
\newcommand{\ab}{\mathrm{ab}} %abelian extensions
\newcommand{\inj}{\mathrm{inj}}
\newcommand{\surj}{\mathrm{surj}} %coverage on sets generated by surjections
\newcommand{\eff}{\mathrm{eff}} %effective Cartier divisors
\newcommand{\Weil}{\mathrm{Weil}} %weil divisors
\newcommand{\lex}{\mathrm{lex}}
\newcommand{\rex}{\mathrm{rex}}
\newcommand{\AR}{\mathrm{A\-R}}
\newcommand{\cons}{\mathrm{c}} %constructible sheaves
\newcommand{\tor}{\mathrm{tor}} %tor dimension
\newcommand{\connected}{\mathrm{connected}}
\newcommand{\cg}{\mathrm{cg}} %compactly generated
\newcommand{\nilp}{\mathrm{nilp}}
\newcommand{\isg}{\mathrm{isg}} %isogenous
\newcommand{\qisg}{\mathrm{qisg}} %quasi-isogenous
\newcommand{\irr}{\mathrm{irr}} %irreducible represenations
\newcommand{\indecomp}{\mathrm{indecomp}}
\newcommand{\preproj}{\mathrm{preproj}}
\newcommand{\preinj}{\mathrm{preinj}}
\newcommand{\reg}{\mathrm{reg}}
\newcommand{\semisimple}{\mathrm{ss}}
\newcommand{\integrable}{\mathrm{int}}
\newcommand{\s}{\mathfrak{s}}
\newcommand{\elliptic}{\mathrm{ell}}
\newcommand{\stab}{\mathrm{stab}}

%prism custom command
\usepackage{relsize}
\usepackage[bbgreekl]{mathbbol}
\usepackage{amsfonts}
\DeclareSymbolFontAlphabet{\mathbb}{AMSb} %to ensure that the meaning of \mathbb does not change
\DeclareSymbolFontAlphabet{\mathbbl}{bbold}
\newcommand{\prism}{{\mathlarger{\mathbbl{\Delta}}}}
\newcommand{\toroidal}{\t}
\newcommand{\extendedtoroidal}{\hat{\t}}
\newcommand{\simpleroots}{\mathbb{I}}
\renewcommand{\positive}{+} 
\renewcommand{\negative}{-}

\begin{document}

    \title{Classical limits of untwisted affine Yangians}
    
    \author{Dat Minh Ha}
    \maketitle
    
    {
      \hypersetup{} 
      \dominitoc
      \tableofcontents %sort sections alphabetically
    }

    \newpage

    \listoftodos

    \newpage

    \chapter{Introduction}

    \newpage

    \part{Yangian extended affine Lie algebras}
        \chapter{Kassel's characterisation of UCEs of current algebras}
            \begin{abstract}
                The purpose of this chapter is to recall some relevant features of the theory of universal central extensions (UCEs) of so-called \say{perfect Lie algebras}, particularly Kassel's realisation of the centre of UCEs of the kind $\g \tensor_k A$ for finite-dimensional simple Lie algebras $\g$ over an algebraically closed field $k$ of characteristic $0$, and a commutative $k$-algebra $A$ (see theorem \ref{theorem: kassel_realisation}). We require this theory in order to be able to explicitly compute bases for the centres of $\uce(\g[v^{\pm 1}, t^{\pm 1}])$ and $\uce(\g[v^{\pm 1}, t])$ (see example \ref{example: toroidal_lie_algebras_centres}); the former is necessary for the construction of \say{Yangian extended toroidal Lie algebras} in chapter \ref{chapter: yangian_EALAs} and the latter will turn out to be the classical limit of the Yangian of $\hat{\g}$ (see chapter \ref{chapter: classical_limits_of_affine_yangians}). Along the way, we will also recall some features of the structures of finite-dimensional simple Lie algebras, as well as some generalities about extensions of Lie algebras.
            \end{abstract}
    
            \minitoc
        
           \section{Some generalities on Lie algebras}
    \subsection{Structure of finite-dimensional simple Lie algebras}
        As a precursor to our main discussion, let us recall some features of the theory of finite-dimensional simple Lie algebras, particularly about their structure.

        \begin{definition}[Simple Lie algebras]
            A Lie algebra over an arbitrary commutative ring $k$ is said to be \textbf{simple} if and only if it admits no non-zero Lie ideals. 
        \end{definition}

        Over a field $k$ that is algebraically closed and of characteristic $0$, much is known about the structure of a simple Lie algebra $\g$ that is finite-dimensional when regarded as a $k$-vector space. The bulk of the content presented above is discussed in further details in any standard textbook on Lie algebras (cf. e.g. \cite{humphreys_lie_algebras} or the first half of \cite{carter_affine_lie_algebras}). Let us give a very brief recap of this theory.

        Firsly, one of the most important features of finite-dimensional simple Lie algebras (henceforth implicitly understood to be defined over a characteristic-$0$ and algebraically closed field $k$) is that each such Lie algebra, say $\g$, posses an invariant and non-degenerate $k$-bilinear form:
            $$(-, -)_{\g}$$
        which is unique up to $k^{\x}$-multiples. The canonical choice is the so-called Killing form, given by $\kappa(x, y) := \trace(\ad(x) \circ \ad(y))$ for all $x, y \in \g$, but in various other context, other natural choices such as the trace form $\trace(xy)$ are also very useful. What is important to us is that the Killing form is essentially unique: if $\kappa'$ is any invariant and non-degenerate symmetric $k$-bilinear form on $\g$ then there will exist a \textit{unique} $c \in k^{\x}$ such that $\kappa' = c \kappa$.

        Now, such a bilinear form $(-, -)_{\g}$ allows us to construct a natural grading of any simple Lie algebra $\g$ by its \say{root lattice} (to be defined shortly). One begins this process by choosing a \textbf{Cartan subalgebra} $\h$ for $\g$, which is a maximal abelian Lie subalgebra\footnote{... and it is well-known that all Cartan subalgebras of $\g$ are conjugate to one another, so this choice does not matter.} of $\g$. The restriction of $(-, -)_{\g}$ to $\h$ remains non-degenerate, so we can canonically identify $\h \xrightarrow[]{\cong} \h^*$ via said bilinear form. By picking a basis for our choice of a Catan subalgebra $\h$ and hence a dual basis:
            $$\simpleroots := \{\alpha_i\}_{i \in \simpleroots}$$
        for $\h^*$, which we shall regard as a choice of a set of \textbf{simple roots}, and writing our bilinear form $(-, -)_{\g}$ in terms of that matrix, we shall get the \textbf{Cartan matrix} of $\g$:
            $$C := (c_{ij})_{i, j \in \simpleroots}$$
        The aforementioned \textbf{root lattice} is then given by:
            $$Q := \Z \simpleroots$$
        Given an element:
            $$\mu := \sum_{i \in \simpleroots} m_i \alpha_i \in Q$$
        we define its \textbf{height} to be the sum of the coefficients:
            $$\height \mu := \sum_{i \in \simpleroots} m_i$$
        It can also be shown that $2\id - C$ is the adjacency matrix of an undirected graph without loops, called the \textbf{Dynkin diagram} of $\g$, and the \textbf{roots} of $\g$ are the roots of this Dynkin diagram.
            
        Let $V$ be a $\g$-module. Then, one can abstractly define the vector subspace of $V$ consisting of elements of \textbf{weight} $\mu \in \h^*$ to be:
            $$V_{\mu} := \{v \in V \mid \forall h \in \h: h \cdot v = \mu(h) v\}$$
        If we have a direct sum decomposition of $\h$-module:
            $$V \cong \bigoplus_{\mu \in \h^*} V_{\mu}$$
        then we will say that $V$ is a \textbf{weight module} for $\g$. Interestingly, elements of $\g_{\alpha}$ (with $\g$ acting on itself by the adjoint action) act by raising/lowering the weights of elements of $\g$-modules $V$ in the sense that:
            $$\g_{\alpha} \cdot V_{\mu} \subseteq V_{\mu + \alpha}$$
        for all weights $\alpha, \mu \in \h^*$. 
        
        As it turns out, the $Q$-grading of $\g$ that was mentioned earlier is actually a weight space decomposition for $\g$, regarded as a module over itself via the adjoint representation.
        \begin{theorem}[Root space decomposition for finite-dimensional simple Lie algebras] \label{theorem: root_space_decomposition_for_finite_dimensional_simple_lie_algebras}
            Let $\g$ be a module over itself via the adjoint representation.
            \begin{enumerate}
                \item $\g$ is a weight module over itself. As a particular consequence, $\g$ is graded by the abelian group $Q$: for every $x \in \g_{\alpha}$, one has that $\deg x = \height \alpha$.
                \item The weight space $\g_0$ is nothing but the Cartan subalgebra $\h$.
                \item For each non-zero weight $\alpha$ of this $\g$-module, $\dim_k \g_{\alpha} = 1$.
            \end{enumerate}
        \end{theorem}
        Typically, the non-zero weights $\alpha$ of the adjoint representations of $\g$ such that $\g_{\alpha} \not \cong 0$ are called \textbf{roots}. The set of roots is usually denoted by $\Phi$. These are the same as the roots that can be constructed from the Cartan matrix of $\g$.

        Elements of $Q^+ := \Z_{\geq 0} \simpleroots$ are typically regarded as being \textbf{positive} (and in particular, the simple roots are positive by convention) and conversely, elements of $Q^- := \Z_{\leq 0} \simpleroots$ are typically said to be \textbf{negative}. One can subsequently construct the sets of positive/negative roots as:
            $$\Phi^{\pm} := \Phi \cap Q^{\pm}$$
        and note that $\Phi = \Phi^+ \cup \Phi^-$.

        From theorem \ref{theorem: root_space_decomposition_for_finite_dimensional_simple_lie_algebras}, we see that for any given positive root $\alpha \in \Phi^+$ and corresponding choice of root vectors\footnote{Choices of which are unique up to non-zero scalar multiples, since subspaces of non-zero weights are equally $1$-dimensional.} $x_{\pm\alpha} \in \g_{\pm\alpha}$, one has that:
            $$[x_{\alpha}, x_{-\alpha}] = \frac12 (x_{\alpha}, x_{-\alpha})_{\g} \check{\alpha}$$
        where $\check{\alpha} \in \h$ is such that $(\alpha, \check{\alpha})_{\g} = 2$. For simplicity, the root vectors $x_{\pm \alpha}$ are typically chosen so that $(x_{\alpha}, x_{-\alpha})_{\g} = 2$.
        
        The next result is a fundamental theorem in the study of finite-dimensional simple Lie algebras over algebraically closed fields of characteristic $0$. It essentially asserts that to give such a Lie algebra via a presentation by generators and relations is the same as to specify its Cartan matrix. The result is not only practically useful, but also is the mean by which one approaches Kac-Moody algebras, where the Cartan matrix is no longer required to be positive-definite (cf. \cite[Chapters 1-5]{kac_infinite_dimensional_lie_algebras}). 
        \begin{theorem}[Serre's Theorem]
            $\g$ is isomorphic to the Lie algebra generated by the set:
                $$\{h_i, x_i^{\pm}\}_{1 \leq i \leq l}$$
            whose elements are subjected to the following relations, given for all $1 \leq i, j \leq l$:
                $$[h_i, h_j] = 0$$
                $$[h_i, x_j^{\pm}] = \pm c_{ij} x_j^{\pm}, [x_i^+, x_j^-] = \delta_{ij} h_i$$
                $$\ad(x_i^{\pm})^{1 - c_{ij}}(x_j^{\pm}) = 0$$
            This is usually referred to as the \textbf{Chevalley-Serre} presentation for $\g$; final set of relations is usually known as the \textbf{Serre relations}.
        \end{theorem}

        We end this subsection with a brief analysis of the easiest possible example of a finite-dimensional simple Lie algebra. 
        \begin{example}[$\sl_2$]
            Recall that $\sl_2(k)$ is the kernel of the trace map:
                $$\trace: \gl_2(k) \to k$$
            i.e. it is the Lie algebra of trace-zero $2 \x 2$-matrices whose Lie bracket is the usual commutator of matrices. It is of dimension $\dim_k \gl_2(k) - \dim_k k = 4 - 1 = 3$, and happens to be also generated by a set of cardinality $3$ (though this is a coincidence, due entirely to how \say{degenerate} of an example $\sl_2(k)$ is):
                $$\{h, x^+, x^-\}$$
            and the elements of this set are subjected to the relations:
                $$[h, x^{\pm}] = \pm 2 x^{\pm}, [x^+, x^-] = h$$
            One proves both of these statements by showing firstly that a basis for $\sl_2(k)$ is:
                $$\left\{ \begin{pmatrix} 1 & 0 \\ 0 & -1 \end{pmatrix}, \begin{pmatrix} 0 & 1 \\ 0 & 0 \end{pmatrix}, \begin{pmatrix} 0 & 0 \\ 1 & 0 \end{pmatrix} \right\}$$
            and then seeing that any Cartan subalgebra of $\sl_2(k)$ must therefore be isomorphic to $k \begin{pmatrix} 1 & 0 \\ 0 & -1 \end{pmatrix}$; the rest then follows. In particular, the Cartan matrix is just:
                $$\begin{pmatrix} 2 \end{pmatrix}$$
            and the Dynkin diagram consists of only a single vertex and no edges:
                $$\bullet$$
        \end{example}

    \subsection{Perfect Lie algebras and their central extensions}
        \begin{definition}[Extensions of Lie algebras]
            Let $k$ be a commutative ring and $\a$ be a Lie algebra over $k$. An \textbf{extension} of $\a$ by another Lie algebra $\z$ over $k$ is an extension of $k$-modules:
                $$0 \to \z \to \frake \xrightarrow[]{\pi} \a \to 0$$
            such that $\frake$ is also a Lie algebra over $k$. A morphism of extensions of a Lie algebra $\a$ by another Lie algebra $\z$ is a morphism of short exact sequences of Lie algebras:
                $$
                    \begin{tikzcd}
                	&& {\frake'} \\
                	0 & \z & \frake & \a & 0
                	\arrow["\varphi", from=1-3, to=2-3]
                	\arrow["{\pi'}", two heads, from=1-3, to=2-4]
                	\arrow["\pi", two heads, from=2-3, to=2-4]
                	\arrow[from=2-2, to=2-3]
                	\arrow[from=2-2, to=1-3]
                	\arrow[from=2-1, to=2-2]
                	\arrow[from=2-4, to=2-5]
                    \end{tikzcd}
                $$
            An isomorphism of extension occurs if $\varphi: \frake' \to \frake$ is a Lie algebra isomorphism. An extension:
                $$0 \to \z \to \tilde{\a} \to \a \to 0$$
            of $\a$ by $\z$ is said to be \textbf{universal} if it is initial amongst all such Lie algebra extensions, in the sense that for every other extension:
                $$0 \to \z \to \frake \to \a \to 0$$
            there must exist a unique Lie algebra homomorphism $\tilde{\a} \to \frake$ fitting into the following morphism of extensions:
                $$
                    \begin{tikzcd}
                	&& {\tilde{\a}} \\
                	0 & \z & \frake & \a & 0
                	\arrow[dashed, from=1-3, to=2-3]
                	\arrow[two heads, from=1-3, to=2-4]
                	\arrow[two heads, from=2-3, to=2-4]
                	\arrow[from=2-2, to=2-3]
                	\arrow[from=2-2, to=1-3]
                	\arrow[from=2-1, to=2-2]
                	\arrow[from=2-4, to=2-5]
                    \end{tikzcd}
                $$
        \end{definition}
        \begin{remark}
            Universal extensions are unique up to unique isomorphisms. 
        \end{remark}
        \begin{remark}[Lie brackets on extensions]
            Suppose that:
                $$0 \to \z \to \frake \to \a \to 0$$
            is an extension of Lie algebras. Then, up to a choice of so-called \say{$2$-cocycle} $\sigma$, which can be regarded as a $k$-linear map:
                $$\sigma: \bigwedge^2 \a \to \z$$:
            satisfying a certain condition, the Lie bracket on $\frake$ will be given by:
                $$\forall X, Y \in \a: \forall K, K' \in \z: [ (X, K), (Y, K') ]_{\frake} := [X, Y]_{\a} + ( K \cdot Y - K' \cdot Y + \sigma(X, Y) )$$
            The aforementioned condition is that the bracket operation $[-, -]_{\frake}$ satisfies the Jacobi identity (note that it is already bilinear and skew-symmetric by construction).
        \end{remark}
        \begin{example}[Semi-direct products of Lie algebras]
            Let $\d$ be a Lie algebra acting on another Lie algebra $\z$, i.e. let $\t$ be a $\d$-module that happens also to be a Lie algebra over $k$. The canonical extension of $\d$ by $\z$ corresponding to the $2$-cocycle:
                $$\sigma = 0$$
            is known as the \textbf{semi-direct product} of $\d$ by $\z$, and commonly denoted by:
                $$\z \rtimes \d$$
            For the sake of completeness, let us note that the Lie bracket here is given by:
                $$\forall D, D' \in \d: \forall K, K' \in \z: [ (D, K), (D', K') ]_{\z \rtimes \d} := [D, D']_{\d} + ( D \cdot K' - D' \cdot K )$$
        \end{example}
        \begin{definition}[Central extensions of Lie algebras]
            A Lie algebra extension:
                $$0 \to \z \to \frake \to \a \to 0$$
            is \textbf{central} if the elements of $\z$ are central in $\frake$. A \textbf{universal central extension} (UCE) of a Lie algebra $\a$ by another Lie algebra $\z$, typically denoted by $\uce(\a)$, is an extension that is initial amongst all such central extensions.
        \end{definition}

        A particular class of Lie algebras that happen to admit UCEs are the so-called \say{perfect} ones.
        \begin{definition}[Perfect Lie algebras]
            A Lie algebra over a commutative ring is said to be \textbf{perfect} if it is equal to its derived subalgebra. 
        \end{definition}
        \begin{example}
            Since simple Lie algebras lack non-zero ideals by definition, they are perfect. 
        \end{example}
        \begin{example}
            Let $A$ be any commutative algebra over some base commutative ring $k$, and let $\g$ be a simple Lie algebra over $k$. Endow the $k$-module $\g_A := \g \tensor_k A$ with the Lie bracket:
                $$\forall x, y \in \g: \forall f, g \in A: [x f, y g]_{\g_A} := [x, y]_{\g} fg$$
            Then, $\g_A$ will be perfect when regraded as a Lie algebra over $k$, precisely because $\g$ is simple.
        \end{example}
        \begin{example}
            Counter-examples include nilpotent and abelian Lie algebras. For the former, their derived subalgebras are always proper Lie subalgebras, while for the latter, their derived subalgebras are zero. 
        \end{example}
        \begin{proposition}[Perfect Lie algebras admit UCEs] \label{prop: perfect_lie_algebras_admit_UCEs}
            \cite[Lemma 1.10]{garland_arithmetics_of_loop_groups} If $k$ is a field and $\a$ is a perfect Lie algebra over $k$, then it will admit a UCE.
        \end{proposition}
    
           \section{The Kassel realisation of UCEs of current Lie algebras}
    \subsection{A recollection of K\"ahler differentials}
        In \cite{kassel_universal_central_extensions_of_lie_algebras}, Kassel showed that centres of UCEs of current algebras (i.e. Lie algebras of the form $\g \tensor_k A$ for some commutative $k$-algebra $A$) are identified with spaces of algebraic differential $1$-forms modulo exact forms, so before proceeding, let us take the time to recall some relevant details about algebraic differential forms. 

        There are many approaches to algebraic differential forms. Ultimately, however, we will be relying on the description of modules of differential forms by generators and relations\footnote{One reason is that this makes it clear how, should a commutative $R$-algebra $S$ be graded by some abelian group $Z$, then that grading will induce a $Z$-grading on $\Omega^1_{S/R}$ as well (see remark \ref{remark: Z_gradings_on_toroidal_lie_algebras}).}, so let us define them that way.
        \begin{definition}[Modules of K\"ahler differentials] \label{def: kahler_differentials}
            Let $R$ be a base commutative ring and let $S$ be a commutative $R$-algebra. The $S$-module of K\"ahler differentials $\Omega^1_{S/R}$ relative to the ring map $R \to S$ is then the quotient of the $S$-module $S \tensor_R S$ by the $S$-submodule generated by the relations:
                $$ss' \tensor 1 - s' \tensor s - s \tensor s'$$
            given for all $s, s' \in S$
        \end{definition}
        \begin{remark}[Diffentials and derivations] \label{remark: differentials_and_derivations}
            The definition suggests to us that K\"ahler differentials might have something to do with derivations, and indeed they do. In fact, this relationship between algebraic $1$-forms and derivations comes from a universal property that the $S$-module $\Omega^1_{S/R}$ enjoys. Namely, for any $R$-module $M$, there exists a natural isomorphism of $S$-modules\footnote{The LHS is the $S$-module of $R$-linear derivations from $S$ into $M$.}:
                $$\Der_R(S, M) \cong \Hom_S(\Omega^1_{S/R}, M)$$
            (cf. \cite[\href{https://stacks.math.columbia.edu/tag/00RO}{Tag 00RO}]{stacks}). From this universal property, one infers that $\Omega^1_{S/R}$ is isomorphic to the $S$-module generated by the set:
                $$\{ds\}_{s \in S}$$
            whose elements are constrained by the relations:
                $$d(ss') = s' ds + s ds'$$
            given for all $s, s' \in S$. The isomorphism in question is given by:
                $$ds \mapsto s \tensor 1$$
            for all $s \in S$.
            
            A particular instance of this phenomenon is that $R$-linear derivations from $S$ to itself are dual to differential $1$-forms relative to $R \to S$, in the sense that there is an $S$-module isomorphism:
                $$\Der_R(S) := \Der_R(S, S) \cong \Hom_A(\Omega^1_{S/R}, A)$$
        \end{remark}
        \begin{remark}
            If $\Omega^1_{S/R}$ is finite free of rank $n$ over $A$, e.g.:
                $$\Omega^1_{S/R} \cong \bigoplus_{1 \leq i \leq n} A dv_i$$
            then we can identify:
                $$\Der_R(S) \cong \bigoplus_{1 \leq i \leq n} A \del_{v_i}$$
            where $\del_{v_i} \in \Der_R(S)$ are the preimages under the isomorphism $\Der_R(S) \xrightarrow[]{\cong} \Hom_S(\Omega^1_{S/R}, S)$ of the $S$-linear duals of the generators $dv_i \in \Omega^1_{S/R}$. 
        \end{remark}
        
        The following well-known lemmas are very useful. Proofs can be be found in any standard reference on general commutative algebra (e.g. \cite[\href{https://stacks.math.columbia.edu/tag/00AO}{Tag 00AO}]{stacks}).
        \begin{lemma}[$1$-forms over polynomial algebras] \label{lemma: 1_forms_over_polynomial_algebras}
            \cite[\href{https://stacks.math.columbia.edu/tag/00RX}{Tag 00RX}]{stacks} Let $R$ be a commutative ring and fix some $n \in \Z_{\geq 0}$. In this case, $\Omega^1_{R[v_1, ..., v_n]/R}$ will be free and of finite rank $n$ as an $R[v_1, ..., v_n]$-module; in particular, it admits the set $\{dv_1, ..., dv_n\}$ as a $R[v_1, ..., v_n]$-linear basis.
        \end{lemma}
        \begin{lemma}[Localisation of $1$-forms] \label{lemma: localisation_1_forms}
            \cite[\href{https://stacks.math.columbia.edu/tag/031G}{Tag 031G}]{stacks} Let $k$ be a field\footnote{... so that the only prime ideal of $k$ would be $(0)$.} and fix some $n \in \Z_{\geq 0}$, and consider the canonical ring homomorphism $k \to k[v_1, ..., v_n]$. Then, for any $1 \leq i \leq n$, there will be a $k[v_1, ..., v_n][v_i^{-1}]$-module isomorphism:
                $$\Omega^1_{k[v_1, ..., v_n][v_i^{-1}]/k} \cong \Omega^1_{k[v_1, ..., v_n]/k}[v_i^{-1}]$$
        \end{lemma}

        Let us end this subsection with the following examples, which will be useful for what comes later on.
        \begin{example}
            Let $k$ be a field.
        
            Per lemma \ref{lemma: 1_forms_over_polynomial_algebras}, we know that:
                $$\Omega^1_{k[v, t]/k} \cong k[v, t] dv \oplus k[v, t] dv$$
            Using lemma \ref{lemma: localisation_1_forms}, we then see that:
                $$\Omega^1_{k[v^{\pm 1}, t^{\pm 1}]/k} \cong k[v^{\pm 1}, t^{\pm 1}] dv \oplus k[v^{\pm 1}, t^{\pm 1}] dt$$
                $$\Omega^1_{k[v^{\pm 1}, t]/k} \cong k[v^{\pm 1}, t] dv \oplus k[v^{\pm 1}, t] dt$$
        \end{example}

    \subsection{UCEs of current Lie algebras}
        \begin{convention} \label{conv: a_fixed_finite_dimensional_simple_lie_algebra}
            From now on, we fix a finite-dimensional simple Lie algebra $\g$ over an algebraically closed field $k$ of characteristic $0$, equipped with a symmetric and non-degenerate invariant $k$-bilinear form $(-, -)_{\g}$. We fix also a Cartan subalgebra $\h$ of $\g$, along with all the accompanying data (e.g. root system, Cartan matrix, etc.) as in subsection \ref{subsection: finite_dimensional_simple_lie_algebras}. 
        \end{convention}

        For the sake of establishing the terminology, let us make the following definition:
        \begin{definition}[Current algebras] \label{def: current_algebras}
            Let $A$ be a commutative algebra over $k$. The vector space:
                $$\g \tensor_k A$$
            with the following Lie bracket:
                $$[x f, y g]_{\g \tensor_k A} := [x, y]_{\g} \tensor fg$$
            (given for all $x, y \in \g$ and all $f, g \in A$) shall then be referred to as a \textbf{current algebra}. 
                
            Also, we will be abbreviating:
                $$xf := x \tensor f$$
            for $x \in \g$ and $f \in A$.
        \end{definition}
        \begin{remark}[(Multi)loop algebras]
            When $A \cong k[v_1^{\pm 1}, ..., v_n^{\pm 1}]$, it is common to refer to $\g \tensor_k A$ as a \textbf{multiloop algebra}. When $n = 1$, we will only be saying \textbf{loop algebra}.
        \end{remark}

        \begin{convention}
            Let $R \to S$ be a homomorphism of commutative rings. Then, let us write:
                $$\bar{\Omega}^1_{S/R} := \Omega^1_{S/R}/dS$$
            Note that this is only an $R$-module, not an $S$-module. Let us also write $\bar{d}: S \to \bar{\Omega}^1_{S/R}$ for the canonical composition:
                $$
                    \begin{tikzcd}
                	S & {\Omega^1_{S/R}} \\
                	& {\bar{\Omega}^1_{S/R}}
                	\arrow["d", from=1-1, to=1-2]
                	\arrow[two heads, from=1-2, to=2-2]
                	\arrow["{\bar{d}}"', from=1-1, to=2-2]
                    \end{tikzcd}
                $$
        \end{convention}

        In order to characterise centres of UCE Kassel constructed in the proof of \cite[Theorem 3.3(iii)]{kassel_universal_central_extensions_of_lie_algebras} a $k$-linear map:
            $$\e: \bigwedge^2 (\g \tensor_k A) \to \bar{\Omega}^1_{A/k}$$
        by the formula\footnote{One can also take $\e(x f, y g) := -(x, y)_{\g} f \bar{d}g$, since $-f \bar{d}g \equiv g \bar{d}f \pmod{d(A)}$. This results in an isomorphic Lie algebra. We will switch back and forth between these choices depending on necessity.}:
            $$\e(x f, y g) := (x, y)_{\g} g \bar{d}f$$
        for all $x, y \in \g$ and for all $f, g \in A$. This can be shown - relying on the $\g$-invariance of the bilinear form $(-, -)_{\g}$ - to be a $2$-cocycle of $\g \tensor_k A$ with coefficients in $\bar{\Omega}^1_{A/k}$\footnote{... i.e. a representative of an isomorphism class $[\e] \in H^2_{\Lie}(\g \tensor_k A, \bar{\Omega}^1_{A/k})$.} and hence gives a central extension $\fraku$ of $\g \tensor_k A$ by $\bar{\Omega}_{A/k}^1$ (cf. proposition \ref{prop: lie_brackets_on_extensions}), whose underlying $k$-vector space is:
            $$(\g \tensor_k A) \oplus \bar{\Omega}_{A/k}^1$$
        and whose Lie bracket is:
            $$[-, -]_{\fraku} := [-, -]_{\g \tensor_k A} + \e$$
        \begin{lemma} \label{lemma: lie_brackets_on_UCEs_of_current_algebras}
            $[-, -]_{\fraku}$ as constructed above is a well-defined Lie bracket.
        \end{lemma}
            \begin{proof}
                By construction, it is already bilinear and skew-symmetric, so the only thing to show is that it satisfies the Jacobi identity. To this end, pick $x, y, z \in \g$ and $f, g, h \in A$ and then consider the following\footnote{We chose $\e(xf, yg) := (x, y)_{\g} g \bar{d}f$ because it makes the verification that $\e$ satisfies the Jacobi identity easier.} computations in $\bar{\Omega}^1_{A/k}$:
                    $$
                        \begin{aligned}
                            & [xf, [yg, zh]_{\fraku}]_{\fraku} + [yg, [zh, xf]_{\fraku}]_{\fraku} + [zh, [xf, yg]_{\fraku}]_{\fraku}
                            \\
                            & = [xf, [y, z]_{\g} gh + \e(yg, zh)]_{\fraku} + [yg, [z, x]_{\g} hf + \e(zh, xf)]_{\fraku} + [zh, [x, y]_{\g} fg + \e(xf, yg)]_{\fraku}
                            \\
                            & = [xf, [y, z]_{\g} gh]_{\fraku} + [yg, [z, x]_{\g} hf]_{\fraku} + [zh, [x, y]_{\g} fg]_{\fraku}
                            \\
                            & = 
                            \begin{aligned}
                                & \left( [x, [y, z]_{\g}]_{\g} fgh + \e(xf, [y, z]_{\g} gh) \right)
                                \\
                                + & \left( [y, [z, x]_{\g}]_{\g} ghf + \e(yg, [z, x]_{\g} hf) \right)
                                \\
                                + & \left( [z, [x, y]_{\g}]_{\g} hfg + \e(zh, [x, y]_{\g} fg) \right)
                            \end{aligned}
                            \\
                            & = \e(xf, [y, z]_{\g} gh) + \e(yg, [z, x]_{\g} hf) + \e(zh, [x, y]_{\g} fg)
                            \\
                            & = (x, [y, z]_{\g})_{\g} gh \bar{d}f + (y, [z, x]_{\g})_{\g} hf \bar{d}g + (z, [x, y]_{\g})_{\g} fg \bar{d}h
                            \\
                            & = (x, [y, z]_{\g})_{\g} ( gh \bar{d}f + hf \bar{d}g + fg\bar{d}h )
                            \\
                            & = 0
                        \end{aligned}
                    $$
                The fourth equality comes from the fact that $\g$ is a Lie algebra, and hence any triple of elements $x, y, z \in \g$ therein satisfies the Jacobi identity:
                    $$[x, [y, z]_{\g}]_{\g} + [y, [z, x]_{\g}]_{\g} + [z, [x, y]_{\g}]_{\g} = 0$$
                The last equality came from the fact that:
                    $$gh df + hf dg + fg dh = d(fgh)$$
                in $\Omega^1_{A/k}$, per definition \ref{def: kahler_differentials} (see also remark \ref{remark: differentials_and_derivations}), which hence implies that:
                    $$gh \bar{d}f + hf \bar{d}g + fg \bar{d}h \equiv 0 \pmod{d(A)}$$
                in $\bar{\Omega}^1_{A/k}$.
            \end{proof}
        Kassel then showed that the Lie algebra $\fraku$ as above is a UCE of $\g \tensor_k A$. 
        \begin{theorem}[The Kassel realisation] \label{theorem: kassel_realisation}
            \cite[Corollary 3.5]{kassel_universal_central_extensions_of_lie_algebras} For the (perfect) Lie $k$-algebra $\g \tensor_k A$, we have that:
                $$\uce(\g \tensor_k A) \cong (\g \tensor_k A) \oplus \bar{\Omega}^1_{A/k}$$
            with Lie bracket as in lemma \ref{lemma: lie_brackets_on_UCEs_of_current_algebras}.
        \end{theorem}

           \section{Some useful examples of UCEs}
    This section is for analysing some of the instances of UCEs of current algebras that are particularly useful to our purposes. Namely, we are interested in the UCEs of the Lie algebras:
        $$\g, \g[v^{\pm 1}], \g[v^{\pm 1}, t^{\pm 1}]$$
    (corresponding to $A$ being isomorphic to $k, \bbC[v^{\pm 1}]$, and $\bbC[v^{\pm 1}, t^{\pm 1}]$ respectively). 

    \subsection{Finite-dimensional simple Lie algebras}
        Consider, firstly, the case:
            $$A := \bbC$$
        It is trivial to see that:
            $$\dim_{\bbC} \bar{\Omega}^1_{\bbC/\bbC} \cong 0$$
        from which one sees that:
            $$\uce(\g) \cong \g$$
        i.e. $\g$ is its own universal central extension, and hence every central extension of $\g$ is trivial. Of course, there are other more conventional ways to see that $\g$ admits no non-trivial central extensions, but we thought we would include this example as a particularly degenerate case of Kassel's realisation of UCEs of current algebras.

    \subsection{Affine Lie algebras}
        Now, consider:
            $$A := \bbC[v^{\pm 1}]$$
    
        \begin{example}[Affine Lie algebras] \label{example: affine_lie_algebras_centres}
            Let us compute the UCE of $\g[v^{\pm 1}]$. From this, we can construct the so-called \say{untwisted affine Kac-Moody algebra} attached to $\g$ (cf. \cite[Chapter 7]{kac_infinite_dimensional_lie_algebras}). 

            To this end, let us firstly compute the underlying vector space of the centre of $\uce(\g[v^{\pm 1}])$. Abstractly, we know that it is isomorphic to $\bar{\Omega}^1_{\bbC[v^{\pm 1}]/\bbC}$, and it is also known that:
                $$\Omega^1_{\bbC[v^{\pm 1}]/\bbC} \cong \bbC[v^{\pm 1}] dv \cong \bigoplus_{m \in \Z} \bbC  v^m dv$$
            so the only non-trivial computation to make is that of $d( \bbC[v^{\pm 1}] )$. For this, let us consider how $d$ acts on the basis elements $v^m \in \bbC[v]$:
                $$d(v^m) = m v^{m - 1} dv$$
            We see that $d(v^m) = 0$ if and only if $m = 0$, and since the set $\{v^m\}_{m \in \Z}$ is a $\bbC$-linear basis for $\bbC[v^{\pm 1}]$, the set:
                $$\{m v^{m - 1} dv\}_{m \in \Z \setminus \{0\}}$$
            therefore spans $d(\bbC[v^{\pm 1}])$. It is also easy to see this subset of $d(\bbC[v^{\pm 1}])$ is $\bbC$-linearly independent and hence is a basis for $d(\bbC[v^{\pm 1}])$. This then tells us that:
                $$\bar{\Omega}^1_{\bbC[v^{\pm 1}]/\bbC} \cong \bbC  v^{-1} \bar{d}v$$
            The underlying vector space of $\uce(\g[v^{\pm 1}])$ is thus isomorphic to:
                $$\g[v^{\pm 1}] \oplus \bbC  v^{-1} \bar{d}v$$

            We know that the Lie bracket on $\uce(\g[v^{\pm 1}])$ is given by:
                $$[x f, y g]_{\g[v^{\pm 1}]} = [x, y]_{\g} fg + (x, y)_{\g} g \bar{d}f$$
            for all $x, y \in \g$ and all $f, g \in \bbC[v^{\pm 1}]$. Since:
                $$g \bar{d}f \in \bbC  v^{-1} \bar{d}v$$
            necessarily, the bracket can be given simplier as:
                $$[x f, y g]_{\g[v^{\pm 1}]} = [x, y]_{\g} fg + (x, y)_{\g} c(f, g) v^{-1} \bar{d}v$$
            for a uniquely determined scalar $c(f, g) \in k$, which can be computed explicitly. To do this, if suffices to perform the computation for basis elements of $\bbC[v^{\pm 1}]$, i.e. we can pick $f := v^m, g := v^n$ for some $m, n \in \Z$ and then consider the following:
                $$g \bar{d}f = v^n \bar{d}(v^m) = m v^{n + m} v^{-1} \bar{d}v$$
            This expression vanishes if and only if $m + n = 0$, so we have that:
                $$c(v^m, v^n) = m \delta_{n + m, 0}$$
            For general $f, g \in \bbC[v^{\pm 1}]$, we can write this more succinctly as:
                $$c(f, g) = \Res_{v = 0}( gdf )$$
            The Lie bracket on $\uce(\g[v^{\pm 1}])$ then takes the form:
                $$[x f, y g]_{\g[v^{\pm 1}]} = [x, y]_{\g} fg + (x, y)_{\g} \Res_{v = 0}(g df) v^{-1} \bar{d}v$$
                
            Note also, that there is a non-degenerate and invariant\footnote{Because $(-, -)_{\g}$ is invariant.} symmetric bilinear form on $\g[v^{\pm 1}]$ given by:
                $$(xf, yg)_{\g[v^{\pm 1}]} := (x, y)_{\g} \Res_{v = 0}(g df)$$
            for all $x, y \in \g$ and all $f, g \in \bbC[v^{\pm 1}]$. By invariance, the extension of this bilinear form to $\uce(\g[v^{\pm 1}])$ is necessarily invariant and degenerate (see lemma \ref{lemma: extending_bilinear_forms_to_central_extensions}).

            Let us also note that unlike $\uce(\g[v^{\pm 1}])$, the UCE of the perfect Lie algebra $\g[v]$ is trivial, since:
                $$\forall f, g \in \bbC[v]: \Res_{v = 0}(g df) = 0$$
        \end{example}

    \subsection{Toroidal Lie algebras}
        \begin{example}[Toroidal Lie algebras] \label{example: toroidal_lie_algebras_centres}
            Next, let us compute the UCE of $\g[v^{\pm 1}, t^{\pm 1}]$. For this, $\bbC[v^{\pm 1}, t^{\pm 1}]$ shall be endowed with the $\Z^2$-grading given by:
                $$\deg v^r t^s := (r, s)$$
            for all $(r, s) \in \Z^2$ so that remarks \ref{remark: gradings_on_1_forms} and \ref{remark: induced_gradings_on_UCEs} can be applied. 
            
            Firstly, let us compute its underlying vector space. Thanks to remark \ref{remark: induced_gradings_on_UCEs}, we know that the only non-trivial computation to make is that of (a basis of) the $\bbC$-vector space:
                $$\bar{\Omega}^1_{\bbC[v^{\pm 1}, t^{\pm 1}]/\bbC} := \Omega^1_{\bbC[v^{\pm 1}, t^{\pm 1}]/\bbC}/d\bbC[v^{\pm 1}, t^{\pm 1}]$$
            From remark \ref{remark: gradings_on_1_forms}, it can be inferred that:
                $$( \Omega^1_{\bbC[v^{\pm 1}, t^{\pm 1}]/\bbC} )_{(r, s)} \cong \bbC v^{r - 1} t^s dv \oplus \bbC v^r t^{s - 1} dt$$
                $$( d\bbC[v^{\pm 1}, t^{\pm 1}] )_{(r, s)} = d( \bbC[v^{\pm 1}, t^{\pm 1}]_{(r, s)} ) = d( \bbC v^r t^s ) = \bbC d(v^r t^s)$$
            which hold for all $(r, s) \in \Z^2$. We thus get:
                $$\bar{\Omega}^1_{\bbC[v^{\pm 1}, t^{\pm 1}]/\bbC} \cong \bigoplus_{(r, s) \in \Z^2} \frac{ \bbC v^{r - 1} t^s dv \oplus \bbC v^r t^{s - 1} dt }{ \bbC d(v^r t^s) }$$
            Observe that when $(r, s) = (0, 0)$, we have that:
                $$\bbC v^{0 - 1} t^0 dv \oplus \bbC v^0 t^{0 - 1} dt = \bbC v^{-1} dv \oplus \bbC t^{-1} dt$$
                $$\bbC d(v^0 t^0) = \bbC d(1) = \bbC 0 = 0$$
            and so:
                $$\bar{\Omega}^1_{\bbC[v^{\pm 1}, t^{\pm 1}]/\bbC} \cong \bigoplus_{(r, s) \in \Z^2 \setminus \{(0, 0)\}} ( \bbC v^{r - 1} t^s dv \oplus \bbC v^r t^{s - 1} dt ) \oplus ( \bbC v^{-1} dv \oplus \bbC t^{-1} dt )$$
            Finally, since:
                $$v^{r - 1} t^s dv, v^r t^{s - 1} dt, v^{-1} dv, t^{-1} dt \not \in d\bbC[v^{\pm 1}, t^{\pm 1}]$$
            meaning that $v^{r - 1} t^s dv \equiv v^{r - 1} t^s \bar{d}v \pmod{d\bbC[v^{\pm 1}, t^{\pm 1}]}$ etc., we can write:
                $$\bar{\Omega}^1_{\bbC[v^{\pm 1}, t^{\pm 1}]/\bbC} \cong \bigoplus_{(r, s) \in \Z^2 \setminus \{(0, 0)\}} ( \bbC v^{r - 1} t^s \bar{d}v \oplus \bbC v^r t^{s - 1} \bar{d}t ) \oplus ( \bbC v^{-1} \bar{d}v \oplus \bbC t^{-1} \bar{d}t )$$
            
            That said, for the indexing to be slightly more suited to our purposes\footnote{Ultimately, this is a matter of personal taste. The elements $K_{r, s}, c_v, c_t$ can be regarded as idiosyncrasies that the author personally find to be visually clear (note, for instance, that our indexing is slightly different from that on \cite[p. 35]{wendlandt_formal_shift_operators_on_yangian_doubles}), should the reader feel puzzled by this re-indexing. For instance, we find $K_{0, 0} = 0$ to be good notation.}, we shall be writing $\bar{\Omega}^1_{\bbC[v^{\pm 1}, t^{\pm 1}]/\bbC}$ as a direct sum as follows:
                $$\bar{\Omega}^1_{\bbC[v^{\pm 1}, t^{\pm 1}]/\bbC} \cong ( \bigoplus_{(r, s) \in \Z^2} \bbC K_{r, s}) \oplus \bbC c_v \oplus \bbC c_t$$
            wherein:
                $$
                    K_{r, s} :=
                    \begin{cases}
                        \text{$\frac1s v^{r - 1} t^s \bar{d}v$ if $(r, s) \in \Z \x (\Z \setminus \{0\})$}
                        \\
                        \text{$-\frac1r v^r t^{-1} \bar{d}t$ if $(r, s) \in (\Z \setminus \{0\}) \x \{0\}$}
                        \\
                        \text{$0$ if $(r, s) = (0, 0)$}
                    \end{cases}
                $$
                $$c_v := v^{-1} \bar{d}v, c_t := t^{-1} \bar{d}t$$
            In particular, the coefficients $\frac1r, \frac1s$ have been included as visual aids for keeping track of whether or not the indices $r$ or $s$ may be allowed to be $0$ in $K_{r, s}$. In fact, any element of the form:
                $$v^n t^q \bar{d}(v^m t^p) \in \bar{\Omega}^1_{\bbC[v^{\pm 1}, t^{\pm 1}]/\bbC}$$
            can be written in terms of the basis vectors $K_{r, s}, c_v, c_t$ in the following manner:
                $$v^n t^q \bar{d}(v^m t^p) = (mq - np) K_{m + n, p + q} + \delta_{(m, p) + (n, q), (0, 0)} ( m c_v + p c_t )$$
            (cf. \cite[p. 35]{wendlandt_formal_shift_operators_on_yangian_doubles}).

            Finally, let us note that the Lie bracket on $\uce(\g[v^{\pm 1}, t^{\pm 1}])$ is given by:
                $$
                    \begin{aligned}
                        & [x v^m t^p, y v^n t^q]_{\uce(\g[v^{\pm 1}, t^{\pm 1}])}
                        \\
                        = & [x, y]_{\g} v^{m + n} t^{p + q} + (x, y)_{\g} v^n t^q \bar{d}(v^m t^p)
                        \\
                        = & [x, y]_{\g} v^{m + n} t^{p + q} + (x, y)_{\g} \left( (mq - np) K_{m + n, p + q} + \delta_{(m, p) + (n, q), (0, 0)} ( m c_v + p c_t ) \right)
                    \end{aligned}
                $$
            for all $x, y \in \g$ and all $(m, p), (n, q) \in \Z^2$. As a side note, let us note that interestingly, unlike how $\g[v]$ admits only the trivial UCE, $\g[v^{\pm 1}, t]$ admits a non-trivial UCE, on which the Lie bracket is given by:
                $$[x v^m t^p, y v^n t^q]_{\uce(\g[v^{\pm 1}, t])} = [x, y]_{\g} v^{m + n} t^{p + q} + (x, y)_{\g} \left( (mq - np) K_{m + n, p + q} + \delta_{m + n, 0} m c_v \right)$$
            for all $x, y \in \g$ and all $(m, p), (n, q) \in \Z \x \Z_{\geq 0}$. This also demonstrates that the vector subspace $\uce(\g[v^{\pm 1}, t])$ is closed under the Lie bracket on the larger vector space $\uce(\g[v^{\pm 1}, t^{\pm 1}])$, and therefore is a Lie subalgebra thereof.
        \end{example}
    
        \newpage
    
        \chapter{Structure of Yangian extended toroidal Lie algebras} \label{chapter: yangian_EALAs}
            \begin{abstract}
                In this chapter, we attempt to construct \say{Yangian extended toroidal Lie algebras}, which are to be extensions $\extendedtoroidal$ of a certain Lie algebra of derivations by the toroidal Lie algebra $\toroidal := \uce(\g[v^{\pm 1}, t^{\pm 1}])$. This is to rectify the problem whereby any invariant symmetric bilinear form on $\toroidal$ is necessarily degenerate, which can result in difficulties both when dealing with the representation theory of $\toroidal$ itself and when we study \say{toroidal Lie bialgebras} later on in chapter \ref{chapter: classical_limits_of_affine_yangians}.
            \end{abstract}
    
            \minitoc
        
            \section{The initial setup}
    Again, $\g$ is a finite-dimensional simple Lie algebra over an algebraically closed field of characteristic $0$. It is accompanied by all the data recalled in subsection \ref{subsection: finite_dimensional_simple_lie_algebras}. 

    \todo[inline]{All toroidal Lie algebra conventions have been moved here.}
    For our own convenience, we will also be adopting the following abbreviations:
        $$A := k[v^{\pm 1}, t^{\pm 1}], A^{\positive} := k[v^{\pm 1}, t]$$
        $$\bar{\Omega}_{[2]} := \bar{\Omega}^1_{A/k}, \bar{\Omega}_{[2]}^{\positive} := \bar{\Omega}^1_{A^{\positive}/k}$$
    and also that:
        $$\g_{[2]} := \g \tensor_k A, \g_{[2]}^{\positive} := \g \tensor_k A^{\positive}$$
    with both being understood as current algebras (in the sense of definition \ref{def: current_algebras}).

    We will then be interested in the Lie algebras:
        $$\toroidal := \uce(\g_{[2]})$$
        $$\toroidal^{\positive} := \uce(\g_{[2]}^{\positive})$$
    which, respectively, shall be referred to as the \textbf{toroidal Lie algebra} and \textbf{positive toroidal Lie algebra} associated to $\g$. In example \ref{example: toroidal_lie_algebras_centres} and remark \ref{remark: Z_gradings_on_toroidal_lie_algebras}, the structures of the underlying ($\Z$-grraded) vector spaces of these Lie algebras have already been described, and we refer the reader there for the details (in particular, the construction of a canonical basis for the underlying vector spaces of their centres). So that our notations would be suggestive, we shall be writing:
        $$\z_{[2]} := \z(\toroidal)$$
        $$\z_{[2]}^{\positive} := \z(\toroidal^{\positive})$$
    from now on; often, we might refer to these as the \textbf{(positive) toroidal centre}.
    
    Now, instead of being equipped with the usual residue bilinear form of degree $(0, 0)$, the Lie algebra $\g_{[2]}$ will be equipped with the residue bilinear form of degree $(0, -1)$:
        $$(x v^m t^p, y v^n t^q)_{\g_{[2]}} := (x, y)_{\g} \delta_{(m, p) + (n, q), (0, -1)}$$
    given for all $x, y \in \g$ and all $(m, p), (n, q) \in \Z^2$; sometimes, we will refer to this as the \textbf{Yangian form} or the \textbf{Yangian pairing}. It is easy to see that the bilinear form:
        $$(-, -)_{\g_{[2]}}$$
    is symmetric, non-degenerate, and invariant. Because $\toroidal$ has a non-trivial centre, any invariant (symmetric) bilinear form thereon (so in particular, any extension of $(-, -)_{\g_{[2]}}$ to $\toroidal$) is necessarily degenerate. The purpose of constructing \say{Yangian extended toroidal Lie algebras} is to remedy such degeneracy.
    
            \section{Construction Yangian extended toroidal Lie algebras}
    \subsection{Extending \texorpdfstring{$\toroidal$}{} by derivations to fix degeneracy}
        The point to constructing the so-called \say{Yangian extended affine Lie algebras} is to fix the issue, whereby any \textit{invariant}\footnote{Let $\a$ be a Lie algebra with non-zero centre and $(-, -)$ be such an invariant symmetric bilinear form thereon. Then, for all $x, y \in \a$ and all $K \in \z(\a)$, we will have that $(K, [x, y]) = ([K, x], y) = (0, y) = 0$, thus proving that $(-, -)$ is degenerate.} symmetric bilinear form on:
            $$\toroidal:= \uce(\g_{[2]})$$
        is necessarily degenerate. We do this by formally introducing a \say{complementary} vector space $\d_{[2]}$ whose elements shall pair non-degenerately with those of $\z_{[2]}$. We will see (cf. lemma \ref{lemma: derivation_action_on_multiloop_algebras}) that these complementary elements are in fact certain $k$-linear derivations on $A$, which then allows us to show that $\extendedtoroidal$ is a $\der_k(A)$-module (decomposing into a direct sum of the submodules $\g_{[2]}$ and $\z_{[2]}$; see lemmas \ref{lemma: derivation_action_on_multiloop_algebras} and \ref{lemma: derivation_action_on_toroidal_centres}, respectively).
        
        \begin{convention} \label{conv: orthogonal_complement_of_toroidal_centres}
            From now on, $\d_{[2]}$ shall be the $k$-vector space:
                $$\d_{[2]} \cong ( \bigoplus_{(r, s) \in \Z^2} k D_{r, s} ) \oplus k D_v \oplus k D_t$$
            such that we can endow:
                $$\extendedtoroidal := \toroidal\oplus \d_{[2]}$$
            with a $k$-bilinear form $(-, -)_{\extendedtoroidal}$ such that:
            \begin{itemize}
                \item the elements $D_{r, s}, D_v, D_t$ are graded-dual with respect to $(-, -)_{\extendedtoroidal}$ to the elements $K_{r, s}, c_v, c_t$, respectively;
                \item $(\g_{[2]}, \z_{[2]} \oplus \d_{[2]})_{\extendedtoroidal} := 0$;
                \item $(\z_{[2]}, \z_{[2]})_{\extendedtoroidal} = (\d_{[2]}, \d_{[2]})_{\extendedtoroidal} := 0$;
                \item $(-, -)_{\extendedtoroidal}|_{\Sym^2_{k}(\g_{[2]})} := (-, -)_{\g_{[2]}}$
            \end{itemize}
        \end{convention}

        We now seek to construct a Lie bracket:
            $$[-, -]_{\extendedtoroidal}$$
        on the vector space $\extendedtoroidal := \toroidal \oplus \d_{[2]}$ such that:
        \begin{itemize}
            \item $(-, -)_{\extendedtoroidal}$ is invariant with respect to $[-, -]_{\extendedtoroidal}$, and
            \item $\toroidal := \uce(\g_{[2]})$ with its Lie bracket as in theorem \ref{theorem: kassel_realisation} embeds naturally as a Lie subalgebra into $\extendedtoroidal$ with the bracket $[-, -]_{\extendedtoroidal}$. 
        \end{itemize}
        We will do this by first identifying elements of $\d_{[2]}$ as certain derivations on $A$ (prompted by the fact that $\d_{[2]}$ is dual to $\z_{[2]}$, whose elements are differential $1$-forms of a certain kind), which will be done by exploiting the invariance of the bilinear form $(-, -)_{\extendedtoroidal}$ (see lemma \ref{lemma: derivation_action_on_multiloop_algebras}). Afterwards, we shall see that $\d_{[2]}$ is actually a Lie subalgebra of the Lie algebra $\der_k(A)$ of derivations on $A$ (with its usual commutator bracket), and we will end the subsection with the conclusion that $\toroidal$ is thus naturally a $\der_k(A)$-module (see proposition \ref{prop: toroidal_lie_algebras_as_modules_over_vector_field_lie_algebras}). In fact, it is a $\d_{[2]}$-module, and $\extendedtoroidal$ is a Lie algebra extension of $\d_{[2]}$ by $\toroidal$, but this is a subtler point and thus we will dedicate the next subsection to discussing it.  

        Let us firstly see how elements of $\d_{[2]}$ might act on those of $\g_{[2]}$, with respect to some Lie bracket $[-, -]_{\extendedtoroidal}$. \todo{Previously a remark.}
        \begin{lemma}[$\d_{[2]}$ acts on $\g_{[2]}$ by derivations] \label{lemma: derivation_action_on_multiloop_algebras}
            The elements $D \in \d_{[2]}$ act as derivations on $A$, in the sense that:
                $$[D, xf]_{\extendedtoroidal} = x \xi_D(f) + K_{D, xf}$$
            where $\xi_D \in \der_k(A)$ is a derivation on $A$ that uniquely depends on the choice of $D \in \d_{[2]}$ and $K_{D, xf} \in \z_{[2]}$ is yet to be determined explicitly (see corollary \ref{coro: derivation_action_on_multiloop_algebras}, where it is shown that in fact, these elements of $\z_{[2]}$ are necessarily $0$). In particular, the basis elements $D_{r, s}$ (for any $(r, s) \in \Z^2$) and $D_v, D_t$ of $\d_{[2]}$ act as follows on the generating elements $x v^m t^p \in \g_{[2]}$ (for some $x \in \g$ and $(m, p) \in \Z^2$), respectively:
                $$[D_{r, s}, x v^m t^p]_{\extendedtoroidal} = ( rp - ms ) x v^{m - r} t^{p - s - 1} + K_{(m, p), (r, s)}(x)$$
                $$[D_v, x v^m t^p]_{\extendedtoroidal} = -m x v^m t^{p - 1} + K_{m, p}(x)$$
                $$[D_t, x v^m t^p]_{\extendedtoroidal} = -p x v^m t^{p - 1} + K_{m, p}(x)$$
            where $K_{(m, p), (r, s)}(x), K_{m, p}(x), K_{m, p}(x) \in \z_{[2]}$ are the undetermined summands.
        \end{lemma}
            \begin{proof}
                Let us first see how the basis elements of $\d_{[2]}$ act on those of $\g_{[2]}$ before worrying about whether or not elements of the former are indeed derivations.
            
                We begin by fixing $x, y \in \g$, $(m, p), (n, q) \in \Z^2$, along with some $D \in \d_{[2]}$. Then, consider the following:
                    $$
                        \begin{aligned}
                            ( D, [x v^m t^p, y v^n t^q]_{\toroidal} )_{\extendedtoroidal} & = ( D, [x, y]_{\g} v^{m + n} t^{p + q} + (x, y)_{\g} v^n t^q \bar{d}( v^m t^p ) )_{\extendedtoroidal}
                            \\
                            & = (x, y)_{\g} ( D, v^n t^q \bar{d}( v^m t^p ) )_{\extendedtoroidal}
                            \\
                            & = (x, y)_{\g} ( D, \delta_{(m, p) + (n, q), (0, 0)} ( n c_v + q c_t ) + (np - mq) K_{m + n, p + q} )_{\extendedtoroidal}
                        \end{aligned}
                    $$
                Now, without any loss of generality, let us suppose that $D \in \d_{[2]}$ is some basis element, i.e.:
                    $$D \in \{ D_{r, s}, D_v, D_t \}$$
                and consider these cases separately, for the sake of clarity:
                \begin{enumerate}
                    \item \textbf{(Case 1: $D := D_{r, s}$):} Fix some $(r, s) \in \Z^2$ and consider the following: 
                        $$
                            \begin{aligned}
                                ( D_{r, s}, [x v^m t^p, y v^n t^q]_{\toroidal} )_{\extendedtoroidal} & = (x, y)_{\g} ( D_{r, s}, \delta_{(m, p) + (n, q), (0, 0)} ( n c_v + q c_t ) + (np - mq) K_{m + n, p + q} )_{\extendedtoroidal}
                                \\
                                & = (x, y)_{\g} (np - mq) \delta_{(r, s), (m + n, p + q)}
                            \end{aligned}
                        $$
                    The assumption that $(-, -)_{\extendedtoroidal}$ is invariant with respect to $[-, -]_{\extendedtoroidal}$ then implies that:
                        $$( [D_{r, s}, x v^m t^p]_{\extendedtoroidal}, y v^n t^q )_{\extendedtoroidal} = (x, y)_{\g} (np - mq) \delta_{(r, s), (m + n, p + q)}$$
                    Now, suppose that:
                        $$[D_{r, s}, x v^m t^p]_{\extendedtoroidal} := \sum_{(a, b) \in \Z^2} \lambda_{a, b}(x) v^a t^b + K_{(m, p), (r, s)}(x) + \xi_{(m, p), (r, s)}(x)$$
                    for some $\lambda_{a, b}(x) \in \g$, $K_{(m, p), (r, s)}(x) \in \z_{[2]}$, and $\xi_{(m, p), (r, s)}(x) \in \d_{[2]}$, depending on our choices of $x \in \g$ and $(m, p) \in \Z^2$. Next, consider the following:
                        $$
                            \begin{aligned}
                                ( [D_v, x v^m t^p]_{\extendedtoroidal}, y v^n t^q )_{\extendedtoroidal} & = \left( \sum_{(a, b) \in \Z^2} \lambda_{a, b}(x) v^a t^b + K_{(m, p), (r, s)}(x) + \xi_{(m, p), (r, s)}(x), y v^n t^q \right)_{\extendedtoroidal}
                                \\
                                & = \sum_{(a, b) \in \Z^2} \left( \lambda_{a, b}(x) v^a t^b, y v^n t^q \right)_{\g_{[2]}}
                                \\
                                & = -\sum_{(a, b) \in \Z^2} (\lambda_{a, b}(x), y)_{\g} \delta_{ (a, b) + (n, q), (0, -1) }
                                \\
                                & = -(\lambda_{-n, -q - 1}(x), y)_{\g}
                            \end{aligned}
                        $$
                    which tells us that:
                        $$(x, y)_{\g} (np - mq) \delta_{(r, s), (m + n, p + q)} = -(\lambda_{-n, -q - 1}(x), y)_{\g}$$
                    The non-degeneracy of the inner product $(-, -)_{\g}$ as well as the arbitrariness of the choices of $y \in \g$ and $(n, q) \in \Z^2$ then together imply that:
                        $$\lambda_{-n, -q - 1}(x) = -(np - mq) \delta_{(r, s), (m + n, p + q)} = (mq - np) \delta_{(r, s), (m + n, p + q)}$$
                    for any fixed choices of $x \in \g$ and $(m, p) \in \Z^2$. From this, we infer that:
                        $$
                            \begin{aligned}
                                [D_{r, s}, x v^m t^p]_{\extendedtoroidal} & = \sum_{(n, q) \in \Z^2} -(np - mq) \delta_{(r, s), (m + n, p + q)} v^{-n} t^{-q - 1} + K_{(m, p), (r, s)}(x) + \xi_{(m, p), (r, s)}(x)
                                \\
                                & = ( m(s - p) - (r - m)p ) x v^{m - r} t^{p - s - 1} + K_{(m, p), (r, s)}(x) + \xi_{(m, p), (r, s)}(x)
                                \\
                                & = ( ms - rp ) x v^{m - r} t^{p - s - 1} + K_{(m, p), (r, s)}(x) + \xi_{(m, p), (r, s)}(x)
                            \end{aligned}
                        $$
                        
                    We now claim that:
                        $$\xi_{(m, p), (r, s)}(x) = 0$$
                    To this end, consider firstly the following, wherein $Z \in \z_{[2]}$ is an arbitrary choice:
                        $$
                            \begin{aligned}
                                ( [D_{r, s}, x v^m t^p]_{\extendedtoroidal}, Z )_{\extendedtoroidal} & = ( D_{r, s}, [x v^m t^p, Z]_{\toroidal} )_{\extendedtoroidal}
                                \\
                                & = (D, 0)_{\extendedtoroidal}
                                \\
                                & = 0
                            \end{aligned}
                        $$
                    Simultaneously, consider the following:
                        $$
                            \begin{aligned}
                                ( [D_{r, s}, x v^m t^p]_{\extendedtoroidal}, Z )_{\extendedtoroidal} & = \left( \sum_{(a, b) \in \Z^2} \lambda_{a, b}(x) v^a t^b + K_{(m, p), (r, s)}(x) + \xi_{(m, p), (r, s)}(x), Z \right)_{\extendedtoroidal}
                                \\
                                & = ( \xi_{(m, p), (r, s)}(x), Z )_{\extendedtoroidal}
                            \end{aligned}
                        $$
                    The previous observation along with this one imply that:
                        $$( \xi_{(m, p), (r, s)}(x), Z )_{\extendedtoroidal} = 0$$
                    for \textit{any} $Z \in \z_{[2]}$, but since $\d_{[2]}$ is graded-dual to $\z_{[2]}$ by construction, the above implies via the non-degeneracy of the inner product $(-, -)_{\extendedtoroidal}$ that:
                        $$\xi_{(m, p), (r, s)}(x) = 0$$
                    necessarily. 
    
                    We can now conclude that:
                        $$[D_{r, s}, x v^m t^p]_{\extendedtoroidal} = ( rp - ms ) x v^{m - r} t^{p - s - 1} + K_{(m, p), (r, s)}(x)$$
                    \item \textbf{(Case 2: $D := D_v$):} In this case, it is easy to see that:
                        $$
                            \begin{aligned}
                                ( D_v, [x v^m t^p, y v^n t^q]_{\toroidal} )_{\extendedtoroidal} & = (x, y)_{\g} ( D_v, \delta_{(m, p) + (n, q), (0, 0)} ( n c_v + q c_t ) + (np - mq) K_{m + n, p + q} )_{\extendedtoroidal}
                                \\
                                & = (x, y)_{\g} \delta_{(m, p) + (n, q), (0, 0)} n
                            \end{aligned}
                        $$
                    Using invariance, we then see that:
                        $$( [D_v, x v^m t^p]_{\extendedtoroidal}, y v^n t^q )_{\extendedtoroidal} = (x, y)_{\g} \delta_{(m, p) + (n, q), (0, 0)} n$$
                    Now, suppose that:
                        $$[D_v, x v^m t^p]_{\extendedtoroidal} := \sum_{(a, b) \in \Z^2} \lambda_{a, b}(x) v^a t^b + K_{m, p}(x) + \xi_{m, p}(x)$$
                    for some $\lambda_{a, b}(x) \in \g$, $K_{m, p}(x) \in \z_{[2]}$, and $\xi_{m, p}(x) \in \d_{[2]}$, depending on our choices of $x \in \g$ and $(m, p) \in \Z^2$. Then, consider the following:
                        $$
                            \begin{aligned}
                                ( [D_v, x v^m t^p]_{\extendedtoroidal}, y v^n t^q )_{\extendedtoroidal} & = \left( \sum_{(a, b) \in \Z^2} \lambda_{a, b}(x) v^a t^b + K_{m, p}(x) + \xi_{m, p}(x), y v^n t^q \right)_{\extendedtoroidal}
                                \\
                                & = \sum_{(a, b) \in \Z^2} \left( \lambda_{a, b}(x) v^a t^b, y v^n t^q \right)_{\g_{[2]}}
                                \\
                                & = -\sum_{(a, b) \in \Z^2} (\lambda_{a, b}(x), y)_{\g} \delta_{ (a, b) + (n, q), (0, -1) }
                                \\
                                & = -(\lambda_{-n, -q - 1}(x), y)_{\g}
                            \end{aligned}
                        $$
                    From this, we are able to conclude that:
                        $$(x, y)_{\g} \delta_{(m, p) + (n, q), (0, 0)} n = -(\lambda_{-n, -q - 1}(x), y)_{\g}$$
                    As this holds for all $y \in \g$ and all $(n, q) \in \Z^2$, we can infer from the above and from the non-degeneracy of the inner product $(-, -)_{\g}$ that:
                        $$\lambda_{-n, -q - 1}(x) = \delta_{(m, p) + (n, q), (0, 0)} n x$$
                    for any $x \in \g$ and any $(m, p) \in \Z^2$ (both fixed!), and hence:
                        $$
                            \begin{aligned}
                                [D_v, x v^m t^p]_{\extendedtoroidal} & = \sum_{(n, q) \in \Z^2} \delta_{(m, p) + (n, q), (0, 0)} n x v^{-n} t^{-q - 1} + K_{m, p}(x) + \xi_{m, p}(x)
                                \\
                                & = -m x v^m t^{p - 1} + K_{m, p}(x) + \xi_{m, p}(x)
                            \end{aligned}
                        $$
    
                    Now, by arguing as in \textbf{Case 1}, we will see that:
                        $$\xi_{m, p}(x) = 0$$
                    and afterwards we will be able to conclude that:
                        $$[D_v, x v^m t^p]_{\extendedtoroidal} = -m x v^m t^{p - 1} + K_{m, p}(x)$$
                    \item \textbf{(Case 3: $D := D_t$)} Arguing as when $D = D_v$, we will obtain:
                        $$[D_t, x v^m t^p]_{\extendedtoroidal} = -p x v^m t^{p - 1} + K_{m, p}(x)$$
                    for some $K_{m, p}(x) \in \z_{[2]}$.
                \end{enumerate}

                Let us now verify that elements of $\d_{[2]}$ are indeed derivations. We can identify the derivations $D_{r, s}, D_v, D_t$ explicitly in terms of $\del_v := \frac{\del}{\del v}, \del_t := \frac{\del}{\del t}$. For this, let us firstly equip $\der_{k}(A)$ - the $k$-vector space of all $k$-linear derivations on $A$ - with the following basis:
                    $$\{ v^m t^p \del_v, v^n t^q \del_t \}_{(m, p), (n, q) \in \Z^2}$$
                \begin{enumerate}
                    \item To compute $D_{r, s}$ in terms of $\del_v, \del_t$, suppose firstly that:
                        $$D_{r, s} := f(v, t) \del_v + g(v, t) \del_t$$
                    with $f(v, t), g(v, t) \in A$. Next, fix some $(m, p) \in \Z^2$ and then consider the following:
                        $$
                            \begin{aligned}
                                D_{r, s}( v^m t^p ) & = f(v, t) \del_v( v^m t^p ) + g(v, t) \del_t( v^m t^p )
                                \\
                                & = f(v, t) m v^{m - 1} t^p + g(v, t) p v^m t^{p - 1}
                            \end{aligned}
                        $$
                    At the same time, we also have that:
                        $$D_{r, s}(v^m t^p) := ( ms - rp ) v^{m - r} t^{p - s - 1}$$
                    and hence:
                        $$f(v, t) m v^{m - 1} t^p + g(v, t) p v^m t^{p - 1} = ( ms - rp ) v^{m - r} t^{p - s - 1}$$
                    From this, one infers that:
                        $$f(v, t) = s v^{-r + 1} t^{-s - 1}, g(v, t) = -r v^{-r} t^{-s}$$
                    and therefore:
                        $$D_{r, s} = s v^{-r + 1} t^{-s - 1} \del_v - r v^{-r} t^{-s} \del_t$$
                    \item One easily checks that:
                        $$D_v = -v t^{-1} \del_v$$
                    \item Likewise:
                        $$D_t = -\del_t$$
                \end{enumerate}
                Consequently, we see that elements of $\d_{[2]}$ are derivations on $A$.
            \end{proof}
        \begin{remark}
            Now that we know that the basis elements $D_{r, s}, D_v, D_t \in \d_{[2]}$ are actually certain derivations on $A$, we can also check that the commutators of the elements $D_{r, s}, D_v, D_t$ are still elements of $\d_{[2]}$. This ensures us that we can \textit{choose} to endow $\d_{[2]}$ with the structure of a Lie subalgebra of $\der_{k}(A)$, i.e. the Lie algebra structure such that:
                $$[D, D']_{\extendedtoroidal} \in \d_{[2]}$$
            for any $D, D' \in \d_{[2]}$. In general, however, we are only guaranteed that:
                $$[D, D']_{\extendedtoroidal} = \z_{[2]} \oplus \d_{[2]}$$
            We note also that it is not even clear \textit{a priori} that the $\d_{[2]}$-summand of the commutators of the form $[D, D']_{\extendedtoroidal}$ has to be the usual commutator inherited from $\der_{k}(A)$; this turns out to be true, but follows from some non-trivial computations (cf. proposition \ref{prop: lie_bracket_on_orthogonal_complement_of_toroidal_centre}). 
        \end{remark}

        Now that we know how $\d_{[2]}$ acts on $\g_{[2]}$, it remains to see how it acts on $\z_{[2]}$ to completely determine its action on $\toroidal$. \todo{Previously a remark.}
        \begin{lemma}[$\d_{[2]}$ acts on $\z_{[2]}$ by Lie derivatives] \label{lemma: derivation_action_on_toroidal_centres}
            Elements of $\d_{[2]}$ act on those of $\z_{[2]}$ as Lie derivatives. This is to say that, the elements $D \in \d_{[2]}$ act on the generating elements $f \bar{d}g \in \z_{[2]}$ (for some $f, g \in A$) in the following manner:
                $$[D, f \bar{d}g]_{\extendedtoroidal} = \xi_D(f) \bar{d}g + f \bar{d}(\xi_D(g))$$
            where $\xi_D \in \der_k(A)$ is a derivation on $A$ determined uniquely by $A$ (well-defined because $\d_{[2]}$ is a vector subspace of $\der_k(A)$ per lemma \ref{lemma: derivation_action_on_multiloop_algebras}). In particular, this means that:
                $$[\d_{[2]}, \z_{[2]}]_{\extendedtoroidal} \subseteq \z_{[2]}$$
        \end{lemma}
            \begin{proof}
                Without any loss of generality, let us consider the following for any $h, h' \in \h$ so that\footnote{We can make this assumption because ultimately, elements of $\z_{[2]}$ do not depend on those of $\g$.}:
                    $$(h, h')_{\g} = 1$$
                any $f(v, t), g(v, t) \in A$, and any $D \in \d_{[2]}$:
                    $$[ D, [h f(v, t), h' g(v, t)]_{\toroidal} ]_{\extendedtoroidal} = [ D, f(v, t) \bar{d}( g(v, t) ) ]_{\extendedtoroidal}$$
                At the same time, we have via the Jacobi identity that:
                    $$
                        \begin{aligned}
                            [ D, [h f(v, t), h' g(v, t)]_{\toroidal} ]_{\extendedtoroidal} & = [ h f(v, t), [D, h' g(v, t)]_{\extendedtoroidal} ]_{\toroidal} + [ [D, h f(v, t)]_{\extendedtoroidal}, h' g(v, t) ]_{\toroidal}
                            \\
                            & = [ h f(v, t), h' D( g(v, t) ) ]_{\toroidal} + [ h D( f(v, t) ), h' g(v, t) ]_{\toroidal}
                            \\
                            & = f(v, t) \bar{d}( D( g(v, t) ) ) + D( f(v, t) ) \bar{d}(g(v, t))
                        \end{aligned}
                    $$
                One thus sees that:
                    $$[ D, f(v, t) \bar{d}( g(v, t) ) ]_{\extendedtoroidal} = f(v, t) \bar{d}( D( g(v, t) ) ) + D( f(v, t) ) \bar{d}(g(v, t))$$
                and since the element $f(v, t) \bar{d}( g(v, t) )$ is central (via the map $\e$ mentioned earlier), this gives another description of:
                    $$[ \d_{[2]}, \z_{[2]} ]_{\extendedtoroidal}$$
                With this in mind, we return quickly to lemma \ref{lemma: derivation_action_on_multiloop_algebras}; there, we previously demonstrated that:
                    $$[ \d_{[2]}, \g_{[2]} ]_{\extendedtoroidal} \subseteq \g_{[2]} \oplus \z_{[2]}$$
                but we claim now that the following stronger fact holds:
                    $$[ \d_{[2]}, \g_{[2]} ]_{\extendedtoroidal} \subseteq \g_{[2]}$$
                To see why this is the case, suppose firstly that for any $D \in \d_{[2]}$, any $X := x f(v, t) \in \g_{[2]}$ (for some $f(v, t) \in A$), there is $K(X) \in \z_{[2]}$ depending on $X$ (and indeed, such a $K(X)$ exists by lemma \ref{lemma: derivation_action_on_multiloop_algebras}) such that:
                    $$[ D, X ]_{\extendedtoroidal} = x D( f(v, t) ) + K(X)$$
                Next, pick an arbitrary element $\xi \in \d_{[2]}$ and then consider the following:
                    $$( [ D, X ]_{\extendedtoroidal}, \xi )_{\extendedtoroidal} = (D(X) + K(X), \xi)_{\extendedtoroidal} = (K(X), \xi)_{\extendedtoroidal}$$
                wherein the last equality holds as a consequence of the fact that:
                    $$( \g_{[2]}, \d_{[2]} )_{\extendedtoroidal} = 0$$
                per the construction of the bilinear form $(-, -)_{\extendedtoroidal}$ as in convention \ref{conv: orthogonal_complement_of_toroidal_centres}.
            \end{proof}
        This lemma has two important corollaries. The first one is easy.
        \begin{corollary}
            With respect to the bracket $[-, -]_{\extendedtoroidal}$, the vector subspace $\toroidal$ is actually a Lie ideal of $\extendedtoroidal$.
        \end{corollary}
        In light of lemma \ref{lemma: derivation_action_on_toroidal_centres}, we see also that the $\z_{[2]}$-summands of the elements of $[\d_{[2]}, \g_{[2]}]_{\extendedtoroidal}$ (cf. lemma \ref{lemma: derivation_action_on_multiloop_algebras}) necessarily vanish., yielding us the following second corollary to the lemma. \todo{Previously a remark.}
        \begin{corollary}[\texorpdfstring{$\z_{[2]}$}{}-summands of elements of \texorpdfstring{$[\d_{[2]}, \g_{[2]}]_{\extendedtoroidal}$}{}] \label{coro: derivation_action_on_multiloop_algebras}
            The action of $\d_{[2]}$ on $\g_{[2]}$ as in lemma \ref{lemma: derivation_action_on_multiloop_algebras} satisfies:
                $$[\d_{[2]}, \g_{[2]}]_{\extendedtoroidal} \subseteq \g_{[2]}$$
        \end{corollary}
            \begin{proof}
                Firstly, let us note that by using invariance, we yield:
                    $$( [ D, X ]_{\extendedtoroidal}, \xi )_{\extendedtoroidal} = (X, [\xi, D]_{\extendedtoroidal})_{\extendedtoroidal} = 0$$
                wherein the last equality is due to the fact that:
                    $$[\xi, D]_{\extendedtoroidal} \in \z_{[2]} \oplus \d_{[2]}$$
                (cf. lemma \ref{lemma: derivation_action_on_multiloop_algebras}) and the fact that:
                    $$( \g_{[2]}, \z_{[2]} \oplus \d_{[2]} )_{\extendedtoroidal} = 0$$
                per the construction of the bilnear form $(-, -)_{\extendedtoroidal}$ as in convention \ref{conv: orthogonal_complement_of_toroidal_centres}. We thus have that:
                    $$(K(X), \xi)_{\extendedtoroidal} = (X, [\xi, D]_{\extendedtoroidal})_{\extendedtoroidal} = 0$$
                for every $\xi \in \d_{[2]}$. The non-degeneracy of $(-, -)_{\extendedtoroidal}$ then implies through this fact that:
                    $$K(X) = 0$$
                As such, we have that:
                    $$[ \d_{[2]}, \g_{[2]} ]_{\extendedtoroidal} \subseteq \g_{[2]}$$
                as claimed. 
            \end{proof}

        We have now reached the following conclusion, as eluded to at the beginning of the subsection:
        \begin{proposition}[$\toroidal$ as a $\der_{k}(A)$-module] \label{prop: toroidal_lie_algebras_as_modules_over_vector_field_lie_algebras}
            $\toroidal$ is a $\der_{k}(A)$-module, decomposing into a direct sum of the submodules $\g_{[2]}$ and $\z_{[2]}$. 
        \end{proposition}
        We note that even in light of this proposition and the fact that $\extendedtoroidal$ admits $\toroidal$ as a Lie ideal, we still do not yet know enough to conclude that $\extendedtoroidal$ necessarily arises as a Lie algebra extension with kernel $\toroidal$. It remains to investigate how the brackets of the form:
            $$[D, D']_{\extendedtoroidal}$$
        (for some $D, D' \in \d_{[2]}$) are given.

    \subsection{Non-uniqueness of Yangian extended toroidal Lie brackets}
        Let us now change our perspective on $\d_{[2]}$ slightly. We begin by demonstrating (cf. proposition \ref{prop: lie_bracket_on_orthogonal_complement_of_toroidal_centre} and the corollary that follows) that for any $D, D' \in \d_{[2]}$, there exists some $K(D, D') \in \z_{[2]}$ such that:
            $$[D, D']_{\extendedtoroidal} = [D, D'] + K(D, D')$$
        with $[D, D'] := DD' - D'D$ denoting the usual commutator; this allows us to regard $\d_{[2]}$ as a Lie subalgebra of $\der_k(A)$. Because we now know from the previous subsection that $\toroidal$ is a $\der_k(A)$-module, the above implies that $\toroidal$ is also a $\d_{[2]}$-module (cf. proposition \ref{prop: toroidal_lie_algebras_as_modules_over_div_0_vector_field_lie_algebras}), and we are thus able to regard $\extendedtoroidal$ as a Lie algebra extension of $\d_{[2]}$ with kernel $\toroidal$. This then begs the question about the (non-)uniqueness of the Lie algebra structure on $\extendedtoroidal$: namely, how many isomorphism classes of Lie algebra structures on $\extendedtoroidal$ are there ? Of course, there is always the semi-direct product, corresponding to the zero $2$-cocycle of $\d_{[2]}$ with coefficients in $\z_{[2]}$ (cf. example \ref{example: lie_algebra_semi_direct_products}). Additionally, there are two isomorphism classes of Lie algebra structures different from the semi-direct product, corresponding to two non-zero isomorphism classes of $2$-cocycles of $\d_{[2]}$ with coefficients in $\z_{[2]}$ (see remark \ref{remark: non_uniqueness_of_yangian_extended_lie_algebras} and theorem \ref{theorem: non_uniqueness_of_yangian_extended_lie_algebras}); these have been known since \cite{billig_energy_momentum_tensor}. 
        
        \begin{proposition}[How does $\d_{[2]}$ act on itself] \label{prop: lie_bracket_on_orthogonal_complement_of_toroidal_centre}
            Let $\d_{[2]}$ be given as in convention \ref{conv: orthogonal_complement_of_toroidal_centres}. Then:
                $$[ \d_{[2]}, \d_{[2]} ]_{\extendedtoroidal} \subset \z_{[2]} \oplus \d_{[2]}$$
            i.e. the $\g_{[2]}$-summand of any commutator of the kind $[D, D']_{\extendedtoroidal}$ (for any two $D, D' \in \d_{[2]}$) actually vanishes. Furthermore, neither the $\z_{[2]}$- nor the $\d_{[2]}$-summand of those commutators $[D, D']_{\extendedtoroidal}$ necessarily vanish in general. 
        \end{proposition}
            \begin{proof}
                For convenience, we will be abbreviating $\h_{[2]} := \h[v^{\pm}, t^{\pm 1}]$ and $\n^{\pm}_{[2]} := [v^{\pm}, t^{\pm 1}]$, with $\n^{\pm} := \bigoplus_{\alpha \in \Phi^{\pm}} \g_{\alpha}$ being the direct sums of the positive/negative roots spaces of $\g$, as usual.
            
                Pick arbitrary elements $D, D' \in \d_{[2]}$ and set:
                    $$[D, D']_{\extendedtoroidal} := X(D, D') + K(D, D') + \xi(D, D')$$
                for some $X(D, D') \in \g_{[2]}, K(D, D') \in \z_{[2]}$, and $\xi(D, D') \in \d_{[2]}$ depending on $D, D'$. Pick also a test element $y g(v, t) \in \g_{[2]}$, for some arbitrary $y \in \g$ and $g(v, t) \in A$ and set:
                    $$[D, y g(v, t)]_{\extendedtoroidal} := y D( g(v, t) ) + K_{D, Y}$$
                    $$[D', y g(v, t)]_{\extendedtoroidal} := y D'( g(v, t) ) + K_{D', Y}$$
                for some $K_{D, Y} \in \z_{[2]}$ depending on $Y$ (cf. lemma \ref{lemma: derivation_action_on_multiloop_algebras}).
                
                Via the Jacobi identity, we get that:
                    $$
                        \begin{aligned}
                            & [ [D, D']_{\extendedtoroidal}, y g(v, t) ]_{\extendedtoroidal}
                            \\
                            = & [ D, [ D', y g(v, t) ]_{\extendedtoroidal} ]_{\extendedtoroidal} + [ D', [ y g(v, t), D ]_{\extendedtoroidal} ]_{\extendedtoroidal}
                            \\
                            = & [ D, y D'( g(v, t) ) + K_{D', Y} ]_{\extendedtoroidal} - [ D', y D( g(v, t) ) + K_{D, Y} ]_{\extendedtoroidal}
                            \\
                            = & \left( y D( D'(g(v, t)) ) + K_{DD', Y} + [ D, K_{D', Y} ]_{\extendedtoroidal} \right) - \left( y D'( D(g(v, t)) ) + K_{D'D, Y} + [ D', K_{D, Y} ]_{\extendedtoroidal} \right)
                            \\
                            = & y (DD' - D'D)( g(v, t) ) + ( K_{DD', Y} - K_{D'D, Y} ) + ( [ D, K_{D', Y} ]_{\extendedtoroidal} - [ D', K_{D, Y} ]_{\extendedtoroidal} )
                        \end{aligned}
                    $$
                for some $K_{DD', Y}, K_{D'D, Y} \in \z_{[2]}$ such that:
                    $$[ D, y D'( g(v, t) ) ]_{\extendedtoroidal} := y D( D'( g(v, t) ) ) + K_{DD', Y}$$
                    $$[ D', y D( g(v, t) ) ]_{\extendedtoroidal} := y D( D'( g(v, t) ) ) + K_{D'D, Y}$$
                At the same time, we have that:
                    $$
                        \begin{aligned}
                            & [ [D, D']_{\extendedtoroidal}, y g(v, t) ]_{\extendedtoroidal}
                            \\
                            = & [ X(D, D') + K(D, D') + \xi(D, D') , y g(v, t) ]_{\extendedtoroidal}
                            \\
                            = & [ X(D, D') + \xi(D, D') , y g(v, t) ]_{\extendedtoroidal}
                            \\
                            = & [ X(D, D') , y g(v, t) ]_{\extendedtoroidal} + \left( y \xi(D, D')(g(v, t)) + K_{\xi(D, D'), Y} \right)
                        \end{aligned}
                    $$
                wherein the second equality holds thanks to the fact that $[\z_{[2]}, \g_{[2]}]_{\extendedtoroidal} = 0$, and $K_{\xi(D, D'), Y} \in \z_{[2]}$ is some element (cf. lemma \ref{lemma: derivation_action_on_multiloop_algebras}). Combining the two observations together then yields:
                    $$
                        \begin{aligned}
                            & [ X(D, D') , y g(v, t) ]_{\extendedtoroidal} + \left( y \xi(D, D')(g(v, t)) + K_{\xi(D, D'), Y} \right)
                            \\
                            = & y (DD' - D'D)( g(v, t) ) + ( K_{DD', Y} - K_{D'D, Y} ) + ( [ D, K_{D', Y} ]_{\extendedtoroidal} - [ D', K_{D, Y} ]_{\extendedtoroidal} )
                        \end{aligned}
                    $$
                There exists $K_{X(D, D'), Y} \in \z_{[2]}$ such that:
                    $$[ X(D, D') , y g(v, t) ]_{\extendedtoroidal} = [ X(D, D') , Y ]_{\extendedtoroidal} = [X(D, D'), Y]_{\g_{[2]}} + K_{X(D, D'), Y}$$
                using which we can write:
                    $$
                        \begin{aligned}
                            & [X(D, D'), Y]_{\g_{[2]}} - y \left( ( DD' - D'D) - \xi(D, D') \right)( g(v, t) )
                            \\
                            = & \left( [ D, K_{D', Y} ]_{\extendedtoroidal} - [ D', K_{D, Y} ]_{\extendedtoroidal} \right) - \left( K_{X(D, D'), Y} + K_{\xi(D, D'), Y} \right)
                        \end{aligned}
                    $$
                    
                We note right away that the LHS lies entirely in $\g_{[2]}$, whereas the RHS is an element of $\z_{[2]}$ due to the fact that $[\d_{[2]}, \z_{[2]}]_{\extendedtoroidal} \subseteq \z_{[2]}$ (cf. lemma \ref{lemma: derivation_action_on_toroidal_centres}), which tells us that $[ D, K_{D', Y} ]_{\extendedtoroidal}, [ D', K_{D, Y} ]_{\extendedtoroidal} \in \z_{[2]}$ in particular. Because $\g_{[2]}$ is centreless (as $\g$ is simple and the Lie bracket on $\g_{[2]}$ is given by extension of scalars), this observation subsequently implies that the LHS must vanish, i.e.:
                    $$[X(D, D'), Y]_{\g_{[2]}} - y \left( ( DD' - D'D) - \xi(D, D') \right)( g(v, t) ) = 0$$
                Because we have by construction that:
                    $$DD' - D'D - \xi(D, D') \in \d_{[2]}$$
                we now make the following claim: \textit{if we fix some arbitrary $E \in \g_{[2]}$ and some $P \in \d_{[2]}$ then:}
                    $$\forall H := h \varphi \in \g_{[2]}: [E, H]_{\g_{[2]}} = h P( \varphi ) \implies E = 0$$

                Using the root space decomposition for $\g$, we see that if $h \in \h$ then we then will have that $[E, H]_{\g_{[2]}} \in \n^{\pm}_{[2]}$, but at the same time, that $h P(\varphi) \in \h_{[2]}$. The only way for this to be true is that $[E, H]_{\g_{[2]}} = 0$, which is the case if and only if $E = 0$. If $h \in \n^{\pm}$, then $[E, H]_{\g_{[2]}} \in \n^{\pm}_{[2]} \oplus \h_{[2]}$ and the $\h_{[2]}$-summand will be non-zero in general; at the same time, $h P(\varphi) \in \n^{\pm}_{[2]}$ in this case, and again, the only way for these to facts to be true simultaneously is that $E = 0$ necessarily. 

                Apply the claim to the fact that:
                    $$[X(D, D'), Y]_{\g_{[2]}} = y \left( ( DD' - D'D) - \xi(D, D') \right)( g(v, t) )$$
                - and again, note that $( DD' - D'D) - \xi(D, D') \in \d_{[2]}$ - then yields:
                    $$X(D, D') = 0$$
                precisely as desired. 
            \end{proof}
        \begin{corollary}
            For any $D, D' \in \d_{[2]}$, the $\d_{[2]}$-summand of $[D, D']_{\extendedtoroidal}$ is nothing but the commutator $DD' - D'D$.
        \end{corollary} 
        In light of the above, it is also valuable to know how the $\d_{[2]}$-summand of the commutators:
            $$[D, D']_{\extendedtoroidal}$$
        are given explicitly. Later on, these computations will be used to prove that the centre of $\extendedtoroidal$ is $2$-dimensional, namely given by $k c_v \oplus k c_v$.
        \begin{lemma}[Explicitly commutators between basis elements of $\d_{[2]}$] \label{lemma: explicit_commutators_between_basis_elements_of_toroidal_central_orthogonal_complement}
            The usual commutator (i.e. $[D, D'] := DD' - D'D$) between the basis elements $D_{r, s}, D_v, D_t \in \d_{[2]}$ are given as follows:
                $$[D_v, D_t] = 0$$
                $$[D_v, D_{r, s}] = r D_{r, s + 1}$$
                $$[D_t, D_{r, s}] = D_{r, s + 1}$$
                $$[D_{a, b}, D_{r, s}] = (br - sa) D_{a + r, b + s + 1}$$
        \end{lemma}
            \begin{proof}
                \begin{enumerate}
                    \item Since we know that:
                        $$D_v = -vt^{-1} \del_v, D_t = -\del_t$$
                    (cf. lemma \ref{lemma: derivation_action_on_multiloop_algebras}), it is therefore trivial that:
                        $$[D_v, D_t] = 0$$
                    \item From lemma \ref{lemma: derivation_action_on_multiloop_algebras}, we know that:
                        $$D_v(v^m t^p) = -m v^m t^{p - 1}$$
                        $$D_{r, s}(v^m t^p) = ( ms - rp ) v^{m - r} t^{p - s - 1}$$
                    From this, we infer that:
                        $$
                            \begin{aligned}
                                [D_v, D_{r, s}](v^m t^p) & = D_v( D_{r, s}(v^m t^p) ) - D_{r, s}( D_v(v^m t^p) )
                                \\
                                & = (ms - rp) D_v( v^{m - r} t^{p - s - 1} ) + m D_{r, s}( v^m t^{p - 1} )
                                \\
                                & = -(m - r)(ms - rp) v^{m - r} t^{p - s - 2} + (ms - r(p - 1)) m v^{m - r} t^{p - s - 2}
                                \\
                                & = r(m(s + 1) - rp) v^{m - r} t^{p - (s + 1) - 1}
                                \\
                                & = r D_{r, s + 1}(v^m t^p)
                            \end{aligned}
                        $$
                    and hence:
                        $$[D_v, D_{r, s}] = r D_{r, s + 1}$$
                    \item Likewise, we can show that:
                        $$[D_t, D_{r, s}] = D_{r, s + 1}$$
                    \item Lastly, we can also show, using a completely analogous argument, that:
                        $$[D_{a, b}, D_{r, s}] = (br - sa) D_{a + r, b + s + 1}$$
                \end{enumerate}
            \end{proof}
        By exploiting invariance again, we are able to obtain the following corollary to the lemma above, concerning commutators between elements of $\z_{[2]}$ and those of $\d_{[2]}$ (which are already known to be elements of $\z_{[2]}$; cf. lemma \ref{lemma: derivation_action_on_toroidal_centres}). Due to its usefulness, we nevertheless grant it lemma-hood.
        \begin{lemma}[Explicit commutators between basis elements of $\d_{[2]}$ and $\z_{[2]}$] \label{lemma: explicit_commutators_between_central_basis_elements_and_derivations}
            In the Lie algebra $\extendedtoroidal$, one has the following non-trivial relations between elements of $\z_{[2]}$ and those of $\d_{[2]}$:
                $$
                    \forall (a, b) \in \Z^2: [D, K_{a, b}]_{\extendedtoroidal} =
                    \begin{cases}
                        \text{$((b - 1)r - sa) D_{a - r, b - s - 1}$ if $D = D_{r, s}$}
                        \\
                        \text{$-r K_{a, b - 1}$ if $D = D_v$}
                        \\
                        \text{$- D_{a, b - 1}$ if $D = D_t$}
                    \end{cases}
                $$
                $$[\d_{[2]}, c_v]_{\extendedtoroidal} = [\d_{[2]}, c_t]_{\extendedtoroidal} = 0 = 0$$
        \end{lemma}
            \begin{proof}
                For a moment, consider two arbitrary derivations $D', D \in \d_{[2]}$ along with an arbitrary element $K \in \z_{[2]}$. Suppose also that:
                    $$[D', D]_{\extendedtoroidal} = [D', D] + K(D', D)$$
                for some $K(D', D) \in \z_{[2]}$ (cf. proposition \ref{prop: lie_bracket_on_orthogonal_complement_of_toroidal_centre}). Using the invariance property of the bilinear form $(-, -)_{\extendedtoroidal}$ as well as the fact that:
                    $$(\z_{[2]}, \z_{[2]})_{\extendedtoroidal} = 0$$
                per the construction of $(-, -)_{\extendedtoroidal}$ (cf. convention \ref{conv: orthogonal_complement_of_toroidal_centres}), one gets:
                    $$([D', D], K)_{\extendedtoroidal} = ([D', D] + K(D', D), K)_{\extendedtoroidal} = ([D', D]_{\extendedtoroidal}, K)_{\extendedtoroidal} = (D', [D, K]_{\extendedtoroidal})_{\extendedtoroidal}$$
                Since we know how the ordinary commutators:
                    $$[D', D]$$
                are given and particularly, how if $D', D \in \d_{[2]}$ are basis elements then their commutator $[D', D]$ will be in the span of a single basis element of $\d_{[2]}$ (cf. lemma \ref{lemma: explicit_commutators_between_basis_elements_of_toroidal_central_orthogonal_complement}), as well as how $\d_{[2]}$ is the orthogonal complement of $\z_{[2]}$ with respect to $(-, -)_{\extendedtoroidal}$ by construction, it remains now to simply specialise to the case wherein $K$ is some basis element of $\z_{[2]}$. 
                \begin{enumerate}
                    \item If:
                        $$K = K_{a, b}$$
                    for some fixed $(a, b) \in \Z^2$ then:
                        $$([D', D], K_{a, b})_{\extendedtoroidal} = 1 \iff [D', D] = D_{a, b}$$
                    Hence, we have that:
                        $$1 = (D_{a, b}, K_{a, b})_{\extendedtoroidal} = ([D', D], K_{a, b})_{\extendedtoroidal} = (D', [D, K_{a, b}]_{\extendedtoroidal})_{\extendedtoroidal}$$
                    Using the commutators computed in lemma \ref{lemma: explicit_commutators_between_basis_elements_of_toroidal_central_orthogonal_complement}, we then see that:
                        $$
                            [D, K_{a, b}]_{\extendedtoroidal} =
                            \begin{cases}
                                \text{$((b - 1)r - sa) D_{a - r, b - s - 1}$ if $D = D_{r, s}$}
                                \\
                                \text{$-r K_{a, b - 1}$ if $D = D_v$}
                                \\
                                \text{$- D_{a, b - 1}$ if $D = D_t$}
                            \end{cases}
                        $$
                    \item If:
                        $$K = c_v$$
                    then we have that:
                        $$([D', D], c_v)_{\extendedtoroidal} = 1 \iff [D', D] = D_v$$
                    which in turn implies that:
                        $$1 = (D_v, K_v)_{\extendedtoroidal} = ([D', D], c_v)_{\extendedtoroidal} = (D', [D, c_v]_{\extendedtoroidal})_{\extendedtoroidal}$$
                    Once again, by using the commutators computed in lemma \ref{lemma: explicit_commutators_between_basis_elements_of_toroidal_central_orthogonal_complement}, we then see that:
                        $$\forall D \in \{D_{r, s}\}_{(r, s) \in \Z^2} \cup \{D_v, D_t\}: [D, c_v]_{\extendedtoroidal} = 0$$
                    and since the set $\{D_{r, s}\}_{(r, s) \in \Z^2} \cup \{D_v, D_t\}$ is a basis for $\d_{[2]}$, one thus has that:
                        $$[\d_{[2]}, c_v]_{\extendedtoroidal} = 0$$
                    as none of said commutators are elements of $k D_v$.
                    \item Likewise, one can show that:
                        $$[\d_{[2]}, c_t]_{\extendedtoroidal} = 0$$
                \end{enumerate}
            \end{proof}
        We will also see that in fact, the centre of $\extendedtoroidal$ is actually just spanend by $c_v$ and $c_t$. See proposition \ref{prop: centres_of_extended_toroidal_lie_algebras} and the discussion preceding it for more details.
        
        \begin{proposition}[$\toroidal$ as a $\d_{[2]}$-module] \label{prop: toroidal_lie_algebras_as_modules_over_div_0_vector_field_lie_algebras}
            If we regard $\d_{[2]}$ as a Lie subalgebra of $\der_{k}(A)$ with the usual commutator bracket, then $\toroidal$ will be a $\d_{[2]}$-module, not just a $\der_k(A)$-module.
        \end{proposition}
        
        In summary, we have yielded the following result:
        \begin{theorem}[Yangian extended toroidal Lie algebras] \label{theorem: extended_toroidal_lie_algebras}
            Let $\d_{[2]}$ be equipped with the commutator bracket inherited from $\der_k(A)$.
        
            When endowed with the Lie bracket $[-, -]_{\extendedtoroidal}$\footnote{... which ultimately is specified by the Lie structure on $\toroidal$ and the non-degenerate and invariant symmetric bilinear form $(-, -)_{\extendedtoroidal}$ as in convention \ref{conv: orthogonal_complement_of_toroidal_centres}.} (as determined in lemmas \ref{lemma: derivation_action_on_multiloop_algebras} and \ref{lemma: derivation_action_on_toroidal_centres}, and proposition \ref{prop: lie_bracket_on_orthogonal_complement_of_toroidal_centre}), the vector space $\extendedtoroidal := \toroidal \oplus \d_{[2]}$ becomes a Lie algebra extension:
                $$0 \to \toroidal \to \extendedtoroidal \to \d_{[2]} \to 0$$
            of $\d_{[2]}$ by $\toroidal$. 
        \end{theorem}
        \begin{remark}
            Note that we do not yet know whether or not $\extendedtoroidal$ is unique (up to isomorphisms) as a Lie algebra extension of $\d_{[2]}$ by $\toroidal$. Due to how brackets on Lie algebra extensions are given (cf. proposition \ref{prop: lie_brackets_on_extensions}), the uniqueness or non-uniqueness (up to isomorphisms) of $\extendedtoroidal$ as a Lie algebra extension of $\d_{[2]}$ by $\toroidal$ lies in how the $\z_{[2]}$-summands of brackets of the form:
                $$[D, D']_{\extendedtoroidal} \in \z_{[2]} \oplus \d_{[2]}$$
            (for all $D, D' \in \d_{[2]}$) are given. Of course, when these summands are equally $0$, one gets $\extendedtoroidal$ as the semi-direct product of $\d_{[2]}$ by $\toroidal$:
                $$\extendedtoroidal \cong \toroidal \rtimes \d_{[2]}$$
            Otherwise, one shall need to find the number of isomorphism classes of non-zero $2$-cocycles of $\d_{[2]}$ with coefficients in $\z_{[2]}$ (again, see proposition \ref{prop: lie_brackets_on_extensions} for why), or in other words, compute the dimension of the vector space $H^2_{\Lie}(\d_{[2]}, \z_{[2]})$. Through remark \ref{remark: non_uniqueness_of_yangian_extended_lie_algebras} and theorem \ref{theorem: non_uniqueness_of_yangian_extended_lie_algebras} below, we will see that there are two non-zero isomorphism classes.
        \end{remark}
        \begin{definition}[Yangian extended toroidal Lie algebras] \label{def: extended_toroidal_lie_algebras}
            We refer to $\extendedtoroidal$ as in theorem \ref{theorem: extended_toroidal_lie_algebras} as a \textbf{Yangian extended toroidal Lie algebra}. 
        \end{definition}
        \begin{remark}[Regarding terminologies]
            Our notion of Yangian extended toroidal Lie algebras does not quite coincide with the very similar notion of an \say{toroidal extended affine Lie algebra} that appeared, for instance, in \cite{billig_representations_of_toroidal_extended_affine_lie_algebras}. Ultimately, this is because the bilinear form that we have endowed $\g_{[2]}$ with is of degree $-1$ (as opposed to $0$) in the second variable. For the latter, $\d_{[2]}$ would be the Lie algebra of divergence-free algebraic vector fields on the (smooth) affine $k$-scheme $\Spec A \cong \G_m^2$.
        \end{remark}
        
        \begin{remark}[Some explicit elements of $H^2_{\Lie}(\d_{[2]}, \z_{[2]})$] \label{remark: non_uniqueness_of_yangian_extended_lie_algebras}
            From proposition \ref{prop: lie_bracket_on_orthogonal_complement_of_toroidal_centre}, we know that:
                $$[\d_{[2]}, \d_{[2]}]_{\extendedtoroidal} \subset \z_{[2]} \oplus \d_{[2]}$$
            we can obtain some $2$-cocyles $\bar{\sigma} \in H^2_{\Lie}(\d_{[2]}, \z_{[2]})$ by restricting elements $\sigma \in H^2_{\Lie}(\der_{k}(A), \z_{[2]})$, some of which are known.

            It pays to abstract the situation out to the $n$-variable case for a moment, mostly for us to make the point that the dimension of the vector space $H^2_{\Lie}(\der_{k}(k[v_1^{\pm 1}, ..., v_n^{\pm 1}]), \bar{\Omega}^1_{k[v_1^{\pm 1}, ..., v_n^{\pm 1}]/k})$ depends not on the number of variables. In \cite[pp. 5, below Equation 1.3]{billig_energy_momentum_tensor}, it was noted that\todo{Find a proper citation for this. \cite{billig_energy_momentum_tensor} does not provide one.}:
                $$H^2_{\Lie}(\der_{k}(k[v_1^{\pm 1}, ..., v_n^{\pm 1}]), \bar{\Omega}^1_{k[v_1^{\pm 1}, ..., v_n^{\pm 1}]/k}) \cong k \sigma_1 \oplus k \sigma_2$$
            with the $2$-cocyles $\sigma_1, \sigma_2$ twisting the Lie brackets:
                $$[v_1^{m_1} ... v_n^{m_n} v_p \del_{v_p}, v_1^{r_1} ... v_n^{r_n} v_q \del_{v_q}] \in \der_{k}(k[v_1^{\pm 1}, ..., v_n^{\pm 1}])$$
            being given by:
                $$\sigma_1(v_1^{m_1} ... v_n^{m_n} v_p \del_{v_p}, v_1^{r_1} ... v_n^{r_n} v_q \del_{v_q}) = r_p m_q \sum_{1 \leq i \leq n} r_i v_1^{m_1 + r_1} ... v_n^{m_n + r_n} v_i^{-1} \bar{d}(v_i)$$
                $$\sigma_2(v_1^{m_1} ... v_n^{m_n} v_p \del_{v_p}, v_1^{r_1} ... v_n^{r_n} v_q \del_{v_q}) = m_p r_q \sum_{1 \leq i \leq n} r_i v_1^{m_1 + r_1} ... v_n^{m_n + r_n} v_i^{-1} \bar{d}(v_i)$$
            for every $1 \leq p, q \leq n$ and every $(m_1, ..., m_n), (r_1, ..., r_n) \in \Z^n$. 

            Now, back to the $2$-variable case. Here, we know how the basis elements $D_{r, s}, D_v, D_t$ of $\d_{[2]}$ are given in terms of $A$-multiples of the partial derivatives $\del_v, \del_t$ (cf. lemma \ref{lemma: derivation_action_on_multiloop_algebras}), so we can exploit the bilinearity of $2$-cocycles in order to see how $\sigma_1, \sigma_2$ act on elements of $\d_{[2]}$. Knowing that the aforementioned basis elements are given by:
                $$D_{r, s} = s v^{-r + 1} t^{-s - 1} \del_v - r v^{-r} t^{-s} \del_t$$
                $$D_v = -v t^{-1} \del_v$$
                $$D_t = -\del_t$$
            we see thus that:
                $$\sigma_a(D_t, -) = 0$$
                $$\sigma_a(D_{r, s}, -), \sigma_a(D_v, -) \not = 0$$
            as elements of $\Hom_{k}(\d_{[2]}, \z_{[2]})$, where $a \in \{1, 2\}$; one proves this by looking at the powers of $v, t$ in the multiples of $\del_v, \del_t$ in the expressions for $D_{r, s}, D_v, D_t$. Subsequently, we can conclude that neither of the $2$-cocycles $\sigma_1, \sigma_2$ vanish entirely on the Lie subalgebra $\d_{[2]}$ of $\der_{k}(A)$, and hence:
                $$H^2_{\Lie}(\d_{[2]}, \z_{[2]}) \cong k \sigma_1 \oplus k \sigma_2$$
            We delegate the relevant detailed computations to the proof of theorem \ref{theorem: non_uniqueness_of_yangian_extended_lie_algebras} down below.  
        \end{remark}
        \begin{theorem}[Non-uniqueness of Yangian extended Lie algebras] \label{theorem: non_uniqueness_of_yangian_extended_lie_algebras}
            Let the vector space:
                $$\d_{[2]} := (\bigoplus_{(r, s) \in \Z^2} k D_{r, s}) \oplus k D_v \oplus k D_t$$
            as in convention \ref{conv: orthogonal_complement_of_toroidal_centres} (see lemma \ref{lemma: derivation_action_on_multiloop_algebras} also, for how the elements $D_{r, s}, D_v, D_t$ are given in terms of $\del_v, \del_t$) be viewed as a Lie subalgebra of $\der_{k}(A)$ (possible thanks to proposition \ref{prop: lie_bracket_on_orthogonal_complement_of_toroidal_centre}), which we endow with the usual commutator bracket. Then, there are two isomorphism classes of Lie algebra extensions:
                $$0 \to \toroidal \to \extendedtoroidal \to \d_{[2]} \to 0$$
            or, in cohomological terms, one has that:
                $$\dim_{k} H^2_{\Lie}(\d_{[2]}, \z_{[2]}) = \dim_{k} H^2_{\Lie}(\d_{[2]}, \toroidal) = 2$$
        \end{theorem}
            \begin{proof}
                As indicated in remark \ref{remark: non_uniqueness_of_yangian_extended_lie_algebras}, it only remains to prove that:
                    $$\sigma_a(D_t, -) = 0$$
                    $$\sigma_a(D_v, -), \sigma_a(D_{r, s}, -) \not = 0$$
                (where $a \in \{1, 2\}$) via explicit computations. For our own convenience, let us temporarily relabel the variables as:
                    $$v := v_1, t := v_2$$
                \begin{itemize}
                    \item Firstly, note that:
                        $$\sigma_1(\del_{v_2}, v_1^{r_1} v_2^{r_2} \del_{v_q}) = \sigma_1(v_1^0 v_2^0 \del_{v_2}, -) = r_1 \cdot 0 \cdot \sum_{1 \leq i \leq 2} (...) = 0$$
                        $$\sigma_2(\del_{v_2}, v_1^{r_1} v_2^{r_2} \del_{v_q}) = \sigma_2(v_1^0 v_2^0 \del_{v_2}, -) = 0 \cdot r_2 \cdot \sum_{1 \leq i \leq 2} (...) = 0$$
                    and hence we indeed have that:
                        $$\sigma_a(D_t, -) = \sigma_a(-\del_{v_2}, -) = -\sigma_a(\del_{v_2}, -) = 0$$
                    \item Secondly, to show that:
                        $$\sigma_a(D_{r, s}, -) \not = 0$$
                    it suffices to only show that:
                        $$\sigma_a(D_{r, s}, D_v) \not = 0$$
                    To this end, recall that:
                        $$D_{r, s} = s v_1^{-r + 1} v_2^{-s - 1} \del_{v_1} - r v_1^{-r} v_2^{-s} \del_{v_2}$$
                        $$D_v = -v_1 v_2^{-1} \del_{v_1}$$
                    which tells us that:
                        $$
                            \begin{aligned}
                                \sigma_1(D_{r, s}, D_v) & = ( s \cdot 1 \cdot (-r + 1) - r \cdot (-1) \cdot (-s) ) \sum_{1 \leq i \leq 2} r_i v_1^{(-r + 1) + 1} v_2^{(-s - 1) - 1} v_i^{-1} \bar{d}(v_i)
                                \\
                                & = (-2sr + 1) \sum_{1 \leq i \leq 2} r_i v_1^{-r} v_2^{-s - 2} v_i^{-1} \bar{d}(v_i)
                            \end{aligned}
                        $$
                    \item Lastly, to show that:
                        $$\sigma_a(D_v, -) \not = 0$$
                    simply note that:
                        $$\sigma_a(D_{r, s}, D_v) = -\sigma_a(D_v, D_{r, s})$$
                \end{itemize}
            \end{proof}

    \subsection{The centre of \texorpdfstring{$\extendedtoroidal$}{}}
        Let us conclude this section with the following question, which is natural now that we have a solid handle on how the Lie bracket on $\extendedtoroidal$ is given:
        \begin{question}
            What is the centre $\hat{\z}_{[2]} := \z( \extendedtoroidal )$ ? This ought to be smaller than $\z_{[2]}$ somehow, since elements of $\z_{[2]}$ need not be central in $\extendedtoroidal$. 
        \end{question}
        \begin{remark}[Computing the centre without computing all the brackets ...]
            Since $\g_{[2]}$ is centreless, we have that:
                $$\hat{\z}_{[2]} = \z( \z_{[2]} \oplus \d_{[2]} )$$
            As $\z_{[2]}$ is an abelian Lie algebra, this implies that in order to compute $\hat{\z}_{[2]}$, it suffices to explicitly compute the commutators of the form:
                $$[D, K]_{\extendedtoroidal}, [D, D']_{\extendedtoroidal}$$
            for $D, D' \in \d_{[2]}$ and $K \in \z_{[2]}$, to see which ones vanish. However, this is rather tedious and not very insightful.
            
            An alternative method is as follows: exploiting the fact that the symmetric bilinear form $(-, -)_{\extendedtoroidal}$ is both invariant and non-degenerate (by construction; cf. convention \ref{conv: orthogonal_complement_of_toroidal_centres}), we can characterise the centre $\hat{\z}_{[2]}$ as the Lie ideal of $\extendedtoroidal$ containing elements $Z$ such that:
                $$0 = ([Z, X]_{\extendedtoroidal}, Y)_{\extendedtoroidal} = (Z, [X, Y]_{\extendedtoroidal})_{\extendedtoroidal}$$
            for any $X, Y \in \extendedtoroidal$, with the first equality holding thanks to the fact that $Z$ is supposed to commute with every other element of $\extendedtoroidal$ by assumption of being central. We are thus left with the task of finding elements:
                $$Z \in \extendedtoroidal$$
            such that:
                $$(Z, [\extendedtoroidal, \extendedtoroidal]_{\extendedtoroidal})_{\extendedtoroidal} = 0$$
            Since brackets of the form:
                $$[X, Y]_{\extendedtoroidal}, [D, D']_{\extendedtoroidal}$$
            (for some $X, Y \in \g_{[2]}$ and some $D, D' \in \d_{[2]}$) are generally non-zero, their elements can not be central in $\extendedtoroidal$. As such, we have narrowed the scope of our search down to:
                $$\hat{\z}_{[2]} \subset \z_{[2]}$$

            Another way to see that:
                $$\hat{\z}_{[2]} \subset \z_{[2]}$$
            is to use the fact that $\extendedtoroidal$ is a Lie algebra extension of $\d_{[2]}$ by $\toroidal$ (cf. theorem \ref{theorem: extended_toroidal_lie_algebras}). This tells us that the centre of $\extendedtoroidal$ ought to lie inside that of $\toroidal$, i.e.:
                $$\hat{\z}_{[2]} \subset \z(\toroidal) = \z_{[2]}$$
            as per proposition \ref{prop: lie_brackets_on_extensions}.
        \end{remark}
        \begin{proposition}[Centres of extended toroidal Lie algebras] \label{prop: centres_of_extended_toroidal_lie_algebras}
            The centre $\hat{\z}_{[2]}$ is a two-dimensional (abelian) Lie subalgebra of $\z_{[2]}$, spanned by $c_v$ and $c_t$. 
        \end{proposition}
            \begin{proof}
                Since we know that:
                    $$\hat{\z}_{[2]} \subset \z_{[2]}$$
                and that the only possibly non-zero bracket with elements of $\z_{[2]}$ are elements of $[\d_{[2]}, \z_{[2]}]_{\extendedtoroidal}$, and since we also know from lemma \ref{lemma: explicit_commutators_between_central_basis_elements_and_derivations} that:
                    $$[\d_{[2]}, K]_{\extendedtoroidal} = 0 \iff K \in k c_v \oplus k c_v$$
                we can conclude immediately that:
                    $$\hat{\z}_{[2]} = k c_v \oplus k c_t$$
            \end{proof}
        \begin{remark}
            It is rather interesting that:
                $$\hat{\z}_{[2]} \cong k c_v \oplus k c_t$$
            as this is in good analogy with the affine Kac-Moody case, where the centre of $\hat{\g}$ is $1$-dimensional, namely spanned by $c_v$ (cf. example \ref{example: affine_lie_algebras_centres}).
        \end{remark}

            \section{Classification of Yangian extended toroidal Lie algebras}
    \subsection{Yangian toroidal cocycles}
        Let:
            $$0 \to \p \to \fraky \xrightarrow[]{\pi} \d \to 0$$
        be a Lie algebra extension such that $\p$ is an $\d$-module, i.e. let $\fraky$ be a twisted semi-direct product $\p \rtimes^{\sigma} \d$ (where $\sigma: \bigwedge^2 \d \to \p$ is some $2$-cocycle). In proposition \ref{prop: twisted_semi_direct_product_criterion}, we have seen that any such cocycle arises as the difference:
            $$\sigma(D, D') := [\gamma(D), \gamma(D')]_{\fraky} - \gamma( [D, D']_{\d} )$$
        for arbitrary elements $D, D' \in \d$ and for some choice of linear section:
            $$\gamma: \d \to \fraky$$
        We therefore see that to give such a linear section $\gamma: \d \to \fraky$ is the same as to give a $2$-cocycle $\sigma: \bigwedge^2 \d \to \p$, i.e. the cocycle $\sigma$ measures how far the extension $(\fraky, \pi)$ is away from splitting, i.e. from being the semi-direct product $\p \rtimes \d$ (which is more-or-less the definition of $2$-cocycles of Lie algebras; cf. definition \ref{def: twisted_semi_direct_products}).

        Keeping the above in mind, let us return to the setting of definition \ref{def: yangian_extended_toroidal_lie_algebras}. An natural question to ask, given this definition, is as follows:
        \begin{question}
            Amongst the twisted semi-direct products $\fraky(\sigma) := \toroidal \rtimes^{\sigma} \d_{[2]}$, which ones are Yangian extended toroidal Lie algebras ? 
        \end{question}
        Since $\toroidal$ automatically embeds into any twited semi-direct product $\fraky(\sigma)$ as a Lie subalgebra, and since the underlying vector space of $\fraky(\sigma)$ is $\toroidal \oplus \d_{[2]}$ by definition, in order to answer this question, it shall suffice to give a criterion on the cocycle:
            $$\sigma: \bigwedge^2 \d_{[2]} \to \toroidal$$
        so that there would exist an \textit{invariant} and \textit{non-degenerate} symmetric bilinear form $(-, -)_{\sigma}$ on $\fraky(\sigma)$. Equivalently, one can give a criterion on the corresponding linear section:
            $$\gamma: \d_{[2]} \to \fraky(\sigma)$$
        which, as mentioned above, is such that:
            $$\sigma(D, D') = [\gamma(D), \gamma(D')]_{\fraky(\sigma)} - \gamma( [D, D']_{\d_{[2]}} )$$

        For convenience, let us fix the following terminologies.
        \begin{definition}[General extended toroidal Lie algebras] \label{def: general_extended_toroidal_lie_algebras}
            Any twisted semi-direct product:
                $$\fraky(\sigma) := \toroidal \rtimes^{\sigma} \d_{[2]}$$
            shall be called an \textbf{extended toroidal Lie algebra}.
        \end{definition}
        \begin{definition}[Yangian toroidal cocycles] \label{def: yangian_toroidal_cocycles}
            Any $2$-cocyle $\sigma: \bigwedge^2 \d_{[2]} \to \toroidal$ shall be referred to as a \textbf{toroidal $2$-cocycle}. If the codomain of a toroidal $2$-cocycle is contained in $\z_{[2]} = \z(\toroidal)$, then we shall refer to said $2$-cocycle as being \textbf{central}.
            
            Any toroidal $2$-cocycle $\sigma$ such that $\fraky(\sigma)$ is a Yangian extended toroidal Lie algebra (in the sense of definition \ref{def: yangian_extended_toroidal_lie_algebras}) shall be called a \textbf{Yangian toroidal $2$-cocycle}.
        \end{definition}
        \begin{remark}
            Since we now know that Yangian extended toroidal Lie algebras are necessarily isomorphic to some twisted semi-direct product $\fraky(\sigma)$ (cf. theorem \ref{theorem: yangian_extended_toroidal_lie_algebras_preliminary_version} and corollary \ref{coro: yangian_extended_toroidal_lie_algebras_are_twisted_semi_direct_products}), definitions \ref{def: general_extended_toroidal_lie_algebras} and \ref{def: yangian_toroidal_cocycles} as above make sense.
        \end{remark}

    \subsection{Classifying Yangian extended toroidal Lie algebras}
        Fix a toroidal $2$-cocyle:
            $$\sigma: \bigwedge^2 \d_{[2]} \to \toroidal$$
        along with a \textit{non-degenerate} symmetric bilinear form:
            $$(-, -)_{\sigma}: \Sym^2_k( \fraky(\sigma) ) \to k$$
        such that:
        \begin{itemize}
            \item the restriction of $(-, -)_{\sigma}$ down to the vector subspace $\g_{[2]} \oplus \z_{[2]}$ coincides with $(-, -)_{\toroidal}$, and
            \item $(\z_{[2]}, \d_{[2]})_{\sigma} \not = 0$ and $(\g_{[2]}, \d_{[2]})_{\sigma} = 0$ and $(\d_{[2]}, \d_{[2]})_{\sigma} = 0$.
        \end{itemize}
        Our task now, as indicated above, is to find a criterion on $\sigma$ so that $(-, -)_{\sigma}$ would be invariant with respect to the Lie bracket on $\fraky(\sigma)$.
        
        Firstly, observe that, per theorem \ref{theorem: yangian_extended_toroidal_lie_algebras} and corollary \ref{coro: yangian_extended_toroidal_lie_algebras_are_twisted_semi_direct_products}, we can infer particularly that:
            $$(-, -)_{\sigma}$$
        for any $\sigma$, is invariant with respect to $[-, -]_0$, in the sense that:
            $$([X, Y]_0, Z)_{\sigma} = (X, [Y, Z]_0)_{\sigma}$$
        for any $X + D, Y + D', Z + D'' \in \toroidal \oplus \d_{[2]}$. Next, recall that by definition \ref{def: twisted_semi_direct_products} (see also: example \ref{example: lie_algebra_semi_direct_products}), the Lie bracket on the twisted semi-direct product $\fraky(\sigma)$, which henceforth will be denoted by $[-, -]_{\sigma}$, is given by:
            $$[-, -]_{\sigma} = [-, -]_0 + \sigma \circ \pi$$
        where $[-, -]_0$ denotes the Lie bracket on the semi-direct product $\fraky(0) \cong \toroidal \rtimes \d_{[2]}$ and $\pi: \fraky(\sigma) \to \d_{[2]}$ is the canonical projection. Then, consider the following:
            $$
                \begin{aligned}
                    ([X + D', Y + D'']_{\sigma}, Z + D'')_{\sigma} & = ([X + D, Y + D']_0 + \sigma(D, D'), Z)_{\sigma}
                    \\
                    & = ([X + D, Y + D']_0, Z + D'')_{\sigma} + (\sigma(D, D'), Z + D'')_{\sigma}
                    \\
                    & = (X + D, [Y + D', Z + D'']_0)_{\sigma} + (\sigma(D, D'), Z + D'')_{\sigma}
                \end{aligned}
            $$
        We see then, that it shall suffices to find conditions on $\sigma$ so that:
            $$(\sigma(D, D'), Z + D'')_{\sigma} = (X + D, \sigma(D', D''))_{\sigma}$$
        For any $\zeta \in \g_{[2]} \oplus \z_{[2]} \oplus \d_{[2]}$, let us write:
            $$\zeta := \zeta_{\g_{[2]}} + \zeta_{\z_{[2]}} + \zeta_{\d_{[2]}}$$
        for its decomposition into its $\g_{[2]}$, $\z_{[2]}$, and $\d_{[2]}$-summands. We see then, that the equation $(\sigma(D, D'), Z + D'')_{\sigma} = (X + D, \sigma(D', D''))_{\sigma}$ is equivalently to:
            $$(\sigma(D, D')_{\g_{[2]}}, Z_{\g_{[2]}})_{\sigma} + (\sigma(D, D')_{\z_{[2]}}, D'')_{\sigma} = (X_{\g_{[2]}}, \sigma(D', D'')_{\g_{[2]}})_{\sigma} + (D, \sigma(D', D'')_{\z_{[2]}})_{\sigma}$$
        As $X, Y, Z \in \toroidal$ were chosen arbitrarily, the problem can thus be rephrased as follows.
        \begin{question}
            What are the necessarily conditions on a given toroidal $2$-cocycle:
                $$\sigma: \bigwedge^2 \d_{[2]} \to \toroidal$$
            so that:
                $$(\sigma(D, D')_{\g_{[2]}}, yg)_{\sigma} + (\sigma(D, D')_{\z_{[2]}}, D'')_{\sigma} = (xf, \sigma(D', D'')_{\g_{[2]}})_{\sigma} + (D, \sigma(D', D'')_{\z_{[2]}})_{\sigma}$$
            for all $D, D', D'' \in \d_{[2]}$, and for all $x, y \in \g$ and all $f, g \in A$.
        \end{question}
        \begin{remark}
            Note that if $\sigma$ is a central toroidal $2$-cocycle, then it shall suffice to find a condition that it must satisfy so that:
                $$(\sigma(D, D')_{\z_{[2]}}, D'')_{\sigma} = (D, \sigma(D', D'')_{\z_{[2]}})_{\sigma}$$
            in order to make sure that the bilinear form $(-, -)_{\sigma}$ would be invariant with respect to $[-, -]_{\sigma}$.
        \end{remark}

        \begin{lemma}[A Yangian-ness criterion for central toroidal cocycles] \label{lemma: yangian_criterion_for_central_toroidal_cocycles}
            
        \end{lemma}
            \begin{proof}
                
            \end{proof}
        \begin{theorem}[A Yangian-ness criterion for toroidal cocycles] \label{theorem: yangian_criterion_for_toroidal_cocycles}
            
        \end{theorem}
            \begin{proof}
                
            \end{proof}
    
            \section{A root grading for Yangian extended toroidal Lie algebra} \label{section: root_grading_for_yangian_EALAs}
    As is now standard practice in infinite-dimensional Lie theory, infinite-dimensional Lie algebra induced from finite-dimensional simple Lie algebras ought to carry a grading by some kind of induced \say{higher root lattice} (e.g. affine Kac-Moody algebras are graded by the affinisations of the root lattices of the underlying finite-dimensional simple Lie algebras; cf. lemma \ref{lemma: root_grading_for_affine_lie_algebras}). There are many reasons as to why one might seek to endow Lie algebras with such gradings, but one rather important reason is that without a root grading of some sort - which in turn would give rise to some kind of triangular decomposition - one would have no hope of setting up a theory of highest-weight modules which, from practical experiences with cases such as $\g$ and $\hat{\g}$, we know to be an extremely powerful method for attacking the problem of classifying say, simple modules over Lie algebras. Therfore, it is natural to ask the question of whether or not our Yangian extended toroidal Lie algebra $\extendedtoroidal$ can also be endowed with such an induced grading, primarily because $\extendedtoroidal$ carries a non-degenerate invariant symmetric bilinear form.

    \subsection{Positive/negative extended toroidal Lie algebras}
        For the purposes of establishing a triangular decomposition for $\extendedtoroidal$, we will be firstly needing an auxiliary Lie subalgebra of $\toroidal$ that is orthogonally complementary to $\toroidal^{\positive}$ inside $\toroidal$, which is to give rise to a Lie subalgebra $\extendedtoroidal^{\negative}$ of $\extendedtoroidal$ that is orthogonally complementary to $\extendedtoroidal^{\positive}$.

        Set:
            $$A^{\negative} := t^{-1} k[v^{\pm 1}, t^{-1}]$$
            $$\g_{[2]}^{\negative} := \g \tensor_k A^{\negative}$$
        The $k$-vector space $\g_{[2]}^{\negative}$ will be endowed with the bracket given by:
            $$[x f, y g]_{\g_{[2]}^{\negative}} := [x, y]_{\g} fg$$
        for all $x, y \in \g$ and all $f, g \in A^{\neg}$. It is not hard to see that there is a direct sum decomposition of Lie algebras:
            $$\g_{[2]} \cong \g_{[2]}^{\positive} \oplus \g_{[2]}^{\negative}$$
        Furthermore, the Lie algebra $\g_{[2]}^{\negative}$ (much like $\g_{[2]}$ and $\g_{[2]}^{\positive}$) is also perfect and hence admits a UCE (cf. proposition \ref{prop: perfect_lie_algebras_admit_UCEs}), which shall be denoted by $\toroidal^{\negative}$. According to theorem \ref{theorem: kassel_realisation}, it can be identified as:
            $$\toroidal^{\negative} \cong \g_{[2]}^{\negative} \oplus \z_{[2]}^{\negative}$$
        with $\z_{[2]}^{\negative}$ being the orthogonal complement of $\z_{[2]}^{\positive}$ inside $\z_{[2]}$, i.e.:
            $$\z_{[2]} \cong \z_{[2]}^{\positive} \oplus \z_{[2]}^{\negative}$$
        This can be inferred from the computations of bases for $\z_{[2]}$ and $\z_{[2]}^{\positive}$ carried out in example \ref{example: toroidal_lie_algebras_centres}.
        
        Let us now make the following constructions:
        \begin{itemize}
            \item
                $$\d_{[2]}^{\positive} := ( \bigoplus_{(r, s) \in \Z \x \Z_{\leq 0} } k D_{r, s} ) \oplus k D_t$$
                $$\d_{[2]}^{\negative} := ( \bigoplus_{(r, s) \in \Z \x \Z_{> 0} } k D_{r, s} ) \oplus k D_v$$
            shall respectively be the Lie subalgebras of $\d_{[2]}$ which are graded-dual to $\z_{[2]}^{\positive/\negative}$ with respect to $(-, -)_{\extendedtoroidal}$;
            \item
                $$\extendedtoroidal^{\positive/\negative} := \toroidal^{\positive/\negative} \oplus \d_{[2]}^{\positive/\negative}$$
        \end{itemize}
        These constructions give rise to two orthogonally complementary Lie subalgebras of $\extendedtoroidal$ that we will make use of to establish a triangular decomposition for $\extendedtoroidal$, provided that we make the assumption that:
            $$\extendedtoroidal \cong \toroidal \rtimes \d_{[2]}$$
        from now on.
        
        \begin{lemma}[Positive/negative extended toroidal Lie algebras] \label{lemma: positive/negative_extended_toroidal_lie_algebras}
            The vector spaces $\extendedtoroidal^{\positive/\negative}$ are Lie subalgebras of $\extendedtoroidal$ with respect to the Lie bracket $[-, -]_{\extendedtoroidal}$.
        \end{lemma}
            \begin{proof}
                Given how $\d_{[2]}$ acts on $\g_{[2]}$ and on $\z_{[2]}$ (cf. lemmas \ref{lemma: derivation_action_on_multiloop_algebras} and \ref{lemma: derivation_action_on_toroidal_centres} respectively), which implies in particular that:
                    $$[\d_{[2]}, \toroidal^{\positive/\negative}] \subseteq \toroidal^{\positive/\negative}$$
                it shall suffice to only demonstrate that the vector spaces $\d_{[2]}^{\positive/\negative}$ are Lie subalgebras of $\d_{[2]}$ in order to show that $\extendedtoroidal^{\positive/\negative}$ are Lie subalgebras of $\extendedtoroidal$. To do this, it suffices to verify that:
                    $$(r, s) \in \Z \x \Z_{\leq 0} \implies [D_{r, s}, D_t] \in \d_{[2]}^{\positive}$$
                    $$(r, s) \in \Z \x \Z_{> 0} \implies [D_{r, s}, D_v] \in \d_{[2]}^{\negative}$$
                for which we rely on the assumption that:
                    $$\extendedtoroidal \cong \toroidal \rtimes \d_{[2]}$$
                This can be inferred from lemma \ref{lemma: explicit_commutators_between_basis_elements_of_toroidal_central_orthogonal_complement}.
            \end{proof}

    \subsection{Root grading and triangular decomposition for Yangian extended toroidal Lie algebras}
        For what follows, let us remind the reader that it is important that we make the assumption that:
            $$\extendedtoroidal \cong \toroidal \rtimes \d_{[2]}$$
        \begin{proposition}[Induced $Q \x \Z$-grading on $\extendedtoroidal$] \label{prop: root_grading_on_extended_toroidal_lie_algebras}
            Define the following grading on $\toroidal$\footnote{Note that we can not simply define a grading on $\g_{[2]}$ alone, since $[\g_{[2]}, \g_{[2]}]_{\toroidal} \not \subset \g_{[2]}$.}, naturally induced by the natural $Q \x \Z$-grading on $\g$.
            
            Firstly, let us declare that:
                $$\deg x v^m t^p := (\alpha, m)$$
            for all $\alpha \in \Phi$, all $x \in \g_{\alpha}$, and all $(m, p) \in \Z^2$. This defines a $Q \x \Z$-grading on $\g_{[2]}$. 
            
            If we are to extend the $Q \x \Z$-grading on $\toroidal$ as above to $\extendedtoroidal$ then the Lie bracket $[-, -]_{\extendedtoroidal}$ ought to be $Q \x \Z$-graded in a compatible manner. Given the adjoint actions of the derivations $D_{r, s}, D_v, D_t$ on the monomials $x v^m t^p \in \g_{[2]}$ (in particular, how said actions affect the $Q \x \Z$-degrees of said monomials; cf. lemma \ref{lemma: derivation_action_on_multiloop_algebras}), let us declare that:
                $$\forall (r, s) \in \Z^2: \deg D_{r, s} := (0, -r)$$
                $$\deg D_v = \deg D_t := (0, 0)$$
            We would also like the bilinear form $(-, -)_{\extendedtoroidal}$ to be of total degree $(0, 0)$, which forces:
                $$\forall (r, s) \in \Z^2: \deg K_{r, s} := (0, r)$$
                $$\deg c_v = \deg c_t := (0, 0)$$
        \end{proposition}
            \begin{proof}
                Let us check whether the constructed $Q \x \Z$-grading on $\extendedtoroidal$ is well-defined.
    
                Firstly, let us check that the grading is well-define on $\toroidal := \g_{[2]} \oplus \z_{[2]}$. To this end, pick $x, y \in \g$ and that $x \in \g_{\alpha}, y \in \g_{\beta}$ for some $\alpha, \beta \in \Phi \cup \{0\}$; also, choose some arbitrary $(m, p), (n, q) \in \Z^2$. Next, consider:
                    $$
                        \begin{aligned}
                            [x v^m t^p, y v^n t^q]_{\toroidal} & = [x, y]_{\g} v^{m + n} t^{p + q} + (x, y)_{\g} v^n t^p \bar{d}(v^m t^p)
                            \\
                            & = [x, y]_{\g} v^{m + n} t^{p + q} + (x, y)_{\g} \delta_{(m, p) + (n, q), (0, 0)} ( n c_v + q c_t ) + (np - mq) K_{m + n, p + q}
                        \end{aligned}
                    $$
                (cf. example \ref{example: toroidal_lie_algebras_centres}). Now, note that if either:
                    $$\alpha + \beta = 0, \alpha \not = 0$$
                or:
                    $$\alpha = \beta = 0$$
                (i.e. $x, y \in \h$) then:
                    $$[x, y] \in \h$$
                and hence:
                    $$\deg [x, y]_{\g} v^{m + n} t^{p + q} = \deg K_{m + n, p + q} = (0, m + n)$$
                On the other hand, if:
                    $$\alpha + \beta \not = 0$$
                then:
                    $$[x, y] \in \n^- \oplus \n^+$$
                which means in particular that at leeast either $x$ or $y$ is nilpotent under the vector representation of $\g$, and hence:
                    $$(x, y)_{\g} = 0$$
                as $(-, -)_{\g}$ is some non-zero multiple of the trace form, and traces of nilpotent matrices are equally $0$. Hence, in this case, we have that:
                    $$\deg [x v^m t^p, y v^n t^q]_{\toroidal} = \deg [x, y]_{\g} v^{m + n} t^{p + q} = (\alpha + \beta, m + n)$$
                Both cases together show that the constructed $Q \x \Z$-grading on $\toroidal$ is well-defined. 
                
                Secondly, note that from how commutators of elements of $\d_{[2]} := \bigoplus_{(r, s) \in \Z^2} k D_{r, s} \oplus k D_v \oplus k D_t$ are given (cf. lemma \ref{lemma: explicit_commutators_between_basis_elements_of_toroidal_central_orthogonal_complement}), one sees that:
                    $$\deg [D_v, D_t] = (0, 0) = \deg D_v + \deg D_t$$
                    $$\deg [D_v, D_{r, s}] = (0, -r) = \deg D_v + \deg D_{r, s}$$
                    $$\deg [D_t, D_{r, s}] = (0, -r) = \deg D_t + \deg D_{r, s}$$
                    $$\deg [D_{a, b}, D_{r, s}] = \deg D_{a + r, b + s + 1} = (0, -(a + r)) = \deg D_{a, b} + \deg D_{r, s}$$
                Thus, the constructed grading is well-defined on $\d_{[2]}$. Recall also from proposition \ref{prop: lie_bracket_on_orthogonal_complement_of_toroidal_centre} that:
                    $$[\d_{[2]}, \d_{[2]}]_{\extendedtoroidal} \subseteq \z_{[2]} \oplus \d_{[2]}$$
                with the $\d_{[2]}$-summand being the usual commutator of derivations $[-, -]$ inherited from $\der_{k}(A)$, while the $\z_{[2]}$-summand is undetermined, but can be viewed as twist of $[-, -]$ by a cocycle $\sigma \in H^2_{\Lie}(\d_{[2]}, \z_{[2]})$ (cf. theorem \ref{theorem: non_uniqueness_of_yangian_extended_lie_algebras}). For this reason, we can and must choose the restriction of $[-, -]_{\extendedtoroidal}$ down to $\d_{[2]}$ to be the usual commutator $[-, -]$ for the construction of our $Q \x \Z$-grading. 
            \end{proof}

        The following is a corollary to proposition \ref{prop: root_grading_on_extended_toroidal_lie_algebras}. One can see it to be true simply by looking at the degrees of elements of $\extendedtoroidal$. 
        \begin{theorem}[Root grading for extended toroidal Lie algebras] \label{theorem: root_grading_for_extended_toroidal_lie_algebras}
            The weight spaces of the adjoint action of $\hat{\g}$ on $\extendedtoroidal$ can be given explicitly in terms of the basis elements of the latter in the following manner:
                $$\forall (\alpha, m) \in \Phi \x \Z: \extendedtoroidal_{\alpha + m\delta} \cong \hat{\g}_{\alpha + m\delta}[t^{\pm 1}]$$
                $$
                    \forall r \in \Z \setminus \{0\}: \extendedtoroidal_{r\delta} \cong \hat{\g}_{r\delta}[t^{\pm 1}] \oplus \bigoplus_{s \in \Z} ( k K_{r, s} \oplus k D_{-r, s} )
                $$
                $$\extendedtoroidal_0 \cong \h \oplus (k c_v \oplus k c_t) \oplus (k D_v \oplus k D_t)$$
            Furthermore, $\extendedtoroidal$ is a weight module of $\hat{\g}$, i.e.:
                $$\extendedtoroidal \cong \bigoplus_{\beta \in \hat{\Phi} \cup \{0\}} \extendedtoroidal_{\beta}$$
        \end{theorem}
        \begin{corollary}
            Recall the triple of Lie algebras:
                $$\extendedtoroidal, \extendedtoroidal^{\positive}, \extendedtoroidal^{\negative}$$
            from lemma \ref{lemma: positive/negative_extended_toroidal_lie_algebras}. Via theorem \ref{theorem: root_grading_for_extended_toroidal_lie_algebras}, we see that the Lie subalgebras $\extendedtoroidal^{\positive/\negative}$ admit the following weight space decompositions, when regarded as $\hat{\g}$-modules of $\extendedtoroidal$:
                $$
                    \forall (\alpha, m) \in \Phi \x \Z:
                    \begin{cases}
                        \extendedtoroidal^{\positive}_{\alpha + m\delta} \cong \hat{\g}_{\alpha + m\delta} \tensor_{k} k[t]
                        \\
                        \extendedtoroidal^{\negative}_{\alpha + m\delta} \cong \hat{\g}_{\alpha + m\delta} \tensor_{k} t^{-1}k[t^{-1}]
                    \end{cases}
                $$
                $$
                    \forall r \in \Z \setminus \{0\}:
                    \begin{cases}
                        \text{$\extendedtoroidal^{\positive} \cong \hat{\g}_{r\delta}[t^{\pm 1}] \oplus \bigoplus_{s \in \Z_{\leq 0}} (k K_{r, s} \oplus k D_{-r, s})$ if $r > 0$}
                        \\
                        \text{$\extendedtoroidal^{\negative} \cong \hat{\g}_{r\delta}[t^{\pm 1}] \oplus \bigoplus_{s \in \Z_{> 0}} (k K_{r, s} \oplus k D_{-r, s})$ if $r < 0$}
                    \end{cases}
                $$
                $$\extendedtoroidal^{\positive}_0 \cong \h \oplus k c_v \oplus k D_t, \extendedtoroidal^{\negative}_0 \cong \h \oplus k c_t \oplus k D_v$$
            Of course, one has also from these constructions that:
                $$\extendedtoroidal^{\positive/\negative} \cong \bigoplus_{\beta \in \hat{\Phi} \cup \{0\}} \extendedtoroidal^{\positive/\negative}_{\beta}$$
        \end{corollary}
        
        \begin{remark} \label{remark: toroidal_root_systems}
            The $\hat{Q}$-grading of $\extendedtoroidal$ as in theorem \ref{theorem: root_grading_for_extended_toroidal_lie_algebras} induces pairing of weight spaces in the following manner.
            \begin{enumerate}
                \item  Firstly, note that for each real root:
                    $$\alpha + m\delta \in \Re(\hat{\Phi})$$
                the corresponding root spaces:
                    $$\extendedtoroidal_{\alpha + m\delta}$$
                are free and of rank $1$ over $k[t^{\pm 1}]$, in good analogy with how real roots of an affine Kac-Moody algebras are equally of multiplicity $1$.

                If we fix:
                    $$(\alpha, m, p), (\beta, n, q) \in \Phi \x \Z^2$$
                along with root vectors:
                    $$x_{\alpha} \in \g_{\alpha}, x_{\beta} \in \g_{\beta}$$
                then:
                    $$( x_{\alpha} v^m t^p, x_{\beta} v^n t^q )_{\extendedtoroidal} = \delta_{(\alpha, m, p) + (\beta, n, q), (0, 0, -1)}$$
                This suggest to us that for each positive real root:
                    $$\alpha + m\delta \in \hat{\Phi}^+ \cong \Phi^+ \x \Z_{\geq 0}$$
                one has the following non-trivial pairing of subspaces:
                    $$\left( \extendedtoroidal^{\negative}_{\mp (\alpha + m\delta)}, \extendedtoroidal^{\positive}_{\pm (\alpha + m\delta)} \right)_{\extendedtoroidal} \not = 0$$
                \item Observe that weight-$0$ subspace:
                    $$\extendedtoroidal_0$$
                is finite-dimensional, namely of dimension $\dim_{k} \h + 2 + 2$, with each summand of $2$ corresponding to one of the direct summands $k c_v \oplus k D_v$ and $k c_t \oplus k D_t$ of $\extendedtoroidal_0$, similar to how:
                    $$\dim_{k} \hat{\h} = \dim_{k} \h + 2$$
                in the affine Kac-Moody case, where the summand of $2$ corressponds to the direct summand of the $1$-dimensional centre and the subspace spanned by the canonical degree derivation. From this, we infer that $\extendedtoroidal$ ought to admit two distinct weights:
                    $$\delta_v := (c_v, -)_{\extendedtoroidal}, \delta_t := (c_t, -)_{\extendedtoroidal}$$
                i.e. they are dual to the central elements $c_v, c_t$ under the vector space isomorphism $\extendedtoroidal_0 \xrightarrow[]{\cong} \extendedtoroidal_0^*$ given by $H \mapsto (H, -)_{\extendedtoroidal}$. Note that because:
                    $$(\d_{[2]}, \d_{[2]})_{\extendedtoroidal} = 0$$
                per the construction of the bilinear form $(-, -)_{\extendedtoroidal}$ (cf. convention \ref{conv: orthogonal_complement_of_toroidal_centres}), we have that:
                    $$(\delta_v, \delta_v)_{\extendedtoroidal} = (\delta_t, \delta_t)_{\extendedtoroidal} = 0$$

                Note also that, once again because $(\d_{[2]}, \d_{[2]})_{\extendedtoroidal} = 0$, we also have that:
                    $$(\delta_v, \delta_t)_{\extendedtoroidal} = 0$$
            \end{enumerate}
        \end{remark}
        
        The following result is nothing but a formal consequence of the discussion above.
        \begin{proposition}[Triangular decomposition for extended toroidal Lie algebras] \label{prop: triangular_decomposition_of_extended_toroidal_lie_algebras}
            With respect to the choices of positive/negative roots as in remark \ref{remark: toroidal_root_systems}, let us set:
                $$\extendedtoroidal_{\up/\low} := \bigoplus_{\beta \in \hat{Q}^{\pm} \x \Z} \extendedtoroidal_{\beta}$$
            The $\hat{Q}$-grading of the Lie algebra $\extendedtoroidal$ (respectively, of $\extendedtoroidal^{\positive/\negative}$) induces a triangular decomposition thereof as follows:
                $$\extendedtoroidal \cong \extendedtoroidal_{\low} \oplus \extendedtoroidal_0 \oplus \extendedtoroidal_{\up}$$
                $$\extendedtoroidal^{\positive/\negative} \cong \extendedtoroidal^{\positive/\negative}_{\low} \oplus \extendedtoroidal^{\positive/\negative}_0 \oplus \extendedtoroidal^{\positive/\negative}_{\up}$$
        \end{proposition}
    
        \newpage

    \part{Classical limits of affine Yangians} 
        \chapter{Toroidal Lie bialgebras as classical limits} \label{chapter: classical_limits_of_affine_yangians}
            \begin{abstract}
                
            \end{abstract}
    
            \minitoc

            \section{Some conventions}
    \subsection{Symmetrisable Kac-Moody algebras} \label{subection: a_fixed_symmetrisable_kac_moody_algebra}
        It will be convenient to phrase certain definitions and results in terms of general symmetrisable Kac-Moody algebras, so let us fix such a Lie algebra $\fraku$. \textit{We do not require that the Cartan matrix is indecomposable}. All related constructions will be carried out over the previously fixed field $k$, which we recall to be algebraically closed and of characteristic $0$. 

        To avoid confusion with the Cartan subalgebra $\h$ of $\g$ (cf. subsection \ref{subsection: finite_dimensional_simple_lie_algebras}), let us write:
            $$\fraku_0$$
        to mean a choice of Cartan subalgebra of $\fraku$. On $\fraku$, there shall be a non-degenerate and invariant symmetric bilinear form:
            $$(-, -)_{\fraku}$$
        whose restriction to $\fraku_0$ is non-degenerate. This allows us to construct the Cartan matrix of $\fraku$:
            $$C_{\fraku} := (c_{ij})_{i, j \in \simpleroots_{\fraku}}$$
        using the same procedure used to construct the affine Cartan matrix $\hat{C}$ in subsection \ref{subsection: a_fixed_untwisted_affine_kac_moody_algebra}. Said procedure yields us also a symmetrisation:
            $$C_{\fraku} := D_{\fraku} A_{\fraku}$$
        wherein $D_{\fraku}$ is diagonal and invertible, and $A_{\fraku}$ is symmetric (cf. \cite[Chapter 2]{kac_infinite_dimensional_lie_algebras}). From the Cartan matrix $C$, one can construct an adjacency matrix of an undirected graph with weighted edges, called the Dynkin diagram of $\fraku$ (cf. \cite[Section 4.7]{kac_infinite_dimensional_lie_algebras}), whose roots form the set:
            $$\Phi_{\fraku}$$
        of roots of $\fraku$; if between any two vertices of the Dynkin diagram of $\fraku$, there is exactly one edge, then we will say that the Dynkin diagram (and likewise, $\fraku$) is \textbf{simply laced}. Let us also denote the set of simple roots of $\fraku$ by:
            $$\{\alpha_i\}_{i \in \simpleroots_{\fraku}}$$
        We will be normalising the Chevalley-Serre generators (cf. \cite[Theorem 1.4]{kac_infinite_dimensional_lie_algebras}), i.e. elements of the set:
            $$\{x_i^{\pm}, h_i\}_{i \in \simpleroots}$$
        so that:
            $$(x_i^+, x_i^-)_{\fraku} = 1$$
        for all $i \in \simpleroots_{\fraku}$. Also, let us regard $\fraku$ as a Lie algebra graded by its root lattice:
            $$Q_{\fraku} := \Z\simpleroots_{\fraku}$$
        with the grading in question being given by:
            $$\deg x = \alpha$$
        for all $\alpha \in \Phi_{\fraku} \cup \{0\}$ and all $x \in \fraku_{\alpha}$.
        
        To avoid confusion with the \say{upper/lower triangular} nilpotent Lie subalgebras $\n^{\pm} \subset \g$ (cf. subsection \ref{subsection: finite_dimensional_simple_lie_algebras}), let us instead write:
            $$\fraku_{\up/\low} := \bigoplus_{\alpha \in \Phi_{\fraku}^{\pm}} \fraku_{\alpha}$$
        Then, let us write:
            $$\b_{\up/\low} := \fraku_0 \oplus \fraku_{\up/\low}$$
        to denote the upper/lower Borel subalgebras of $\fraku$, i.e. the direct sums of the Cartan subalgebra $\fraku_0$ with the direct sums of the positive/negative root spaces. Recall that $\fraku$ admits a triangular decomposition:
            $$\fraku \cong \fraku_{\low} \oplus \fraku_0 \oplus \fraku_{\up}$$
        (cf. \cite[Theorem 1.2]{kac_infinite_dimensional_lie_algebras}). Affine Kac-Moody algebras are symmetrisable \textit{a priori} (cf. \cite[Chapter 4]{kac_infinite_dimensional_lie_algebras} and subsection \ref{subsection: a_fixed_untwisted_affine_kac_moody_algebra}), and when:
            $$\fraku \cong \hat{\g}$$
        we will defer to the more conventional notations:
            $$\hat{\h} := \fraku_0$$
            $$\hat{\n}^{\pm} := \fraku_{\up/\low}$$
        (cf. subsection \ref{subsection: a_fixed_untwisted_affine_kac_moody_algebra}).
            
        Out of technical necessity, we must right away exclude the cases where $\fraku$ is either of type $\sfA_1^{(1)}$ or of type $\sfA_2^{(2)}$ (in the notations of \cite[Chapter 4]{kac_infinite_dimensional_lie_algebras}). There seem to be evidences towards the fact that in these cases, the presentations for the associated (formal) Yangian as given in definitions \ref{def: formal_yangians_associated_to_symmetrisable_kac_moody_algebras} and \ref{def: yangians_associated_to_symmetrisable_kac_moody_algebras} must include some higher-order relations. In the former case, it is also known that the formal Yangian is not a graded flat deformation of $\rmU(\toroidal^{\positive})$, which is problematic for us.

    \subsection{Some abbreviations}
        We will also be making use of some shorthands.
        \begin{convention}
            All non-Lie algebras will automatically be assumed to be associative and unital. 
    
            Also, if $A$ is an algebra and $X_1, ..., X_n \in A$ are arbitrary elements therein, then we will be using the following shorthands:
                $$\{ X_1, ..., X_n \} := \sum_{\sigma \in S_n} X_{\sigma(1)} \cdot ... \cdot X_{\sigma(n)}$$
            and for any $X \in A$, we will be writing:
                $$\ad(X) := [X, -]$$
        \end{convention}
        \begin{convention}
            Let $V$ be a vector space.
        
            As a shorthand, we will be writing:
                $$\bar{\Delta}(X) := X \tensor 1 + 1 \tensor X$$
            and this will be understood to be an element of $\rmT(V)^{\tensor 2}$. If we have:
                $$X := \sum_{i, j} X_i \tensor X_j \in \rmT(V)^{\tensor 2}$$
            then we will be writing:
                $$X_{12} := \sum_{i, j} X_i \tensor X_j \tensor 1 \in \rmT(V)^{\tensor 3}$$
                $$X_{23} := \sum_{i, j} 1 \tensor X_i \tensor X_j \in \rmT(V)^{\tensor 3}$$
                $$X_{13} := \sum_{i, j} X_i \tensor 1 \tensor X_j \in \rmT(V)^{\tensor 3}$$
            and likewise for the other permutations. If $V$ is a Lie algebra then instead of the tensor algebra, we will typically think of these elements as living in tensor powers of the universal enveloping algebra.
        \end{convention}
    
            \section{Yangians associated to Kac-Moody algebras}
    \begin{convention}[A fixed symmetrisable Kac-Moody algebra]
        Over the course of this section, suppose that $\fraku$ is a symmetrisable Kac-Moody algebra whose Cartan matrix is is indecomposable. All related constructions will be carried out over the previously fixed field $k$, which we recall to be algebraically closed and of characteristic $0$. 

        To avoid confusion with the Cartan subalgebra $\h$ of $\g$ (cf. convention \ref{conv: a_fixed_finite_dimensional_simple_lie_algebra}), let us write:
            $$\fraku_0$$
        to mean a choice of Cartan subalgebra of $\fraku$. 
        
        On $\fraku$, there shall be a non-degenerate and invariant symmetric bilinear form:
            $$(-, -)_{\fraku}$$
        Let us also denote the set of simple roots of $\fraku$ by:
            $$\simpleroots_{\fraku} := \{\alpha_i\}_{i \in \simpleroots_{\fraku}}$$
        We will be normalising the Chevalley-Serre generators, i.e. elements of the set:
            $$\{x_i^{\pm}, h_i\}_{i \in \simpleroots}$$
        so that:
            $$(x_i^+, x_i^-)_{\fraku} = 1$$
            $$[x_i^+, x_i^-] = h_i$$
        for all $i \in \simpleroots_{\fraku}$.

        Let us also denote the Cartan matrix of $\fraku$ by:
            $$C := (c_{ij})_{i, j \in \simpleroots_{\fraku}}$$
    \end{convention}
    Out of technical necessity, we must right away exclude the cases where $\fraku$ is either of type $\sfA_1^{(1)}$ or of type $\sfA_2^{(2)}$ (in the notations of \cite[Chapter 4]{kac_infinite_dimensional_lie_algebras}). There seem to be evidences towards the fact that in these cases, the presentations for the associated (formal) Yangian as given in definitions \ref{def: formal_yangians_associated_to_symmetrisable_kac_moody_algebras} and \ref{def: yangians_associated_to_symmetrisable_kac_moody_algebras} must include some higher-order relations. In the former case, it is also known that the formal Yangian is not a graded flat deformation of $\rmU(\toroidal^{\positive})$, which is problematic for us.

    \begin{convention}
        All non-Lie algebras will automatically be assumed to be associative and unital. 

        Also, if $A$ is an algebra and $X_1, ..., X_n \in A$ are arbitrary elements therein, then we will be using the following shorthands:
            $$\{ X_1, ..., X_n \} := \sum_{\sigma \in S_n} X_{\sigma(1)} \cdot ... \cdot X_{\sigma(n)}$$
        and for any $X \in A$, we will be writing:
            $$\ad(X) := [X, -]$$
    \end{convention}

    \subsection{Presentations for Yangians}
        \begin{definition}[Formal Yangians associated to symmetrisable Kac-Moody algebras] \label{def: formal_yangians_associated_to_symmetrisable_kac_moody_algebras}
            The \textbf{formal Yangian associated to the derived subalgebra of $\fraku$} is a certain $k$-algebra:
                $$\rmY_{\hbar}(\fraku')$$
            generated by the set:
                $$\{X_{i, r}^{\pm}, H_{i, r}\}_{(i, r) \in \simpleroots_{\fraku} \x \Z_{\geq 0}}$$
            whose elements are subjected to the following relations, given for all $(i, r), (j, s) \in \simpleroots_{\fraku} \x \Z_{\geq 0}$:
                $$[H_{i, r}, H_{j, s}] = 0$$
                $$[H_{i, 0}, X_{j, s}^{\pm}] = \pm (\alpha_i, \check{\alpha}_j)_{\fraku} X_{j, s}^{\pm}$$
                $$[X_{i, r}^+, X_{j, s}^-] = \delta_{i, j} H_{i, r + s}$$
                $$[H_{i, r + 1}, X^{\pm}_{j, s}] - [H_{i, r}, X^{\pm}_{j, s + 1}] = \frac12 \hbar (\alpha_i, \check{\alpha}_j)_{\fraku} \{H_{i, r}, X^{\pm}_{j, s}\}$$
                $$[X^{\pm}_{i, r + 1}, X^{\pm}_{j, s}] - [X^{\pm}_{i, r}, X^{\pm}_{j, s + 1}] = \frac12 \hbar (\alpha_i, \check{\alpha}_j)_{\fraku} \{X^{\pm}_{i, r}, X^{\pm}_{j, s}\}$$
                $$\sum_{ \sigma \in S_{1 - c_{ij}} } \ad(X_{ i, r_{\sigma(1 - c_{ij})} }) \cdot ... \cdot \ad(X_{ i, r_{\sigma(2)} }) \cdot \ad(X_{ i, r_{\sigma(1)} }) \cdot X_{j, s}^{\pm} = 0$$
                
            To obtain the \textbf{formal Yangian associated to the Kac-Moody algebra $\fraku$} itself:
                $$\rmY_{\hbar}(\fraku)$$
            we must enlarge the generating set to:
                $$\{X_{i, r}^{\pm}, H_{i, r}\}_{(i, r) \in \simpleroots_{\fraku} \x \Z_{\geq 0}} \cup \fraku_0$$
            and then impose the following additional relations, given for all $h \in \fraku_0$ and all $(j, s) \in \simpleroots_{\fraku} \x \Z_{\geq 0}$:
                $$[h, H_{j, s}] = 0$$
                $$[h, X_{j, s}^{\pm}] = \pm \alpha_j(h) X_{j, s}^{\pm}$$
        \end{definition}
        \begin{remark}[Formal Yangians as \texorpdfstring{$k[\hbar]$}{}-algebras and the $\Z_{\geq 0}$-grading] \label{remark: positive_Z_grading_on_formal_yangians}
            Equivalently, the formal Yangians $\rmY_{\hbar}(\fraku')$ and $\rmY_{\hbar}(\fraku)$ from definition \ref{def: formal_yangians_associated_to_symmetrisable_kac_moody_algebras} can be regarded as $k[\hbar]$-algebras generated by the sets:
                $$\{X_{i, r}^{\pm}, H_{i, r}\}_{(i, r) \in \simpleroots_{\fraku} \x \{0, 1\}}$$
            and:
                $$\{X_{i, r}^{\pm}, H_{i, r}\}_{(i, r) \in \simpleroots_{\fraku} \x \{0, 1\}} \cup \fraku_0$$
            respectively, whose elements are subjected to the same relations.

            In any event, these algebras carry natural $\Z_{\geq 0}$-gradings given by:
                $$\deg \hbar = 1$$
                $$\forall (i, r) \in \simpleroots_{\fraku} \x \Z_{\geq 0}: \deg X^{\pm}_{i, r} = \deg H_{i, r} = r$$
            Later on, this grading will be used for establishing formal Yangians as Rees algebras for certain cases of the Kac-Moody algebra $\fraku$.
        \end{remark}
        \begin{remark}[The degree-$0$ graded component]
            When regarded as a $k$-algebra, both $\rmY_{\hbar}(\fraku')$ and $\rmY_{\hbar}(\fraku)$ admit the universal enveloping algebra $\rmU(\fraku)$ as a $k$-subalgebra. In particular, $\rmU(\fraku)$ is isomorphic to the degree-$0$ graded component of $\rmY_{\hbar}(\fraku')$ and $\rmY_{\hbar}(\fraku)$; this can be shown by constructing an algebra homomorphisms:
                $$\rmU(\fraku) \to \rmY_{\hbar}(\fraku'), \rmU(\fraku) \to \rmY_{\hbar}(\fraku)$$
            given on generators by:
                $$x_i^{\pm} \mapsto X_{i, 0}^{\pm}, h_i \mapsto H_{i, 0}$$
            and then showing that these maps are injective. 

            From now on, this fact will be used without explicit mention.
        \end{remark}

        Under certain mild technical restrictions, it is possible to give a presentation for the formal Yangians $\rmY_{\hbar}(\fraku)$ in terms only of generators of degrees $0$ and $1$, i.e. elements of the set:
            $$\{X_{i, r}^{\pm}, H_{i, r}\}_{(i, r) \in \simpleroots_{\fraku} \x \{0, 1\}} \cup \fraku_0$$
        This can be viewed as a generalisation of the work \cite{levendorskii_finite_type_yangians_presentation} of Levendorskii, applicable to the case where $\fraku$ is of finite-type (e.g. $\fraku \cong \g$), and is very useful for performing computations with the generators of $\rmY_{\hbar}(\fraku)$.
        
        \begin{convention}[An auxiliary generator for (formal) Yangians]
            Henceforth, let us write:
                $$T_{i, 1}(\hbar) := H_{i, 1} - \frac12 \hbar H_{i, 0}^2$$
                $$T_{i, 1} := T_{i, 1}(1) = H_{i, 1} - \frac12 H_{i, 0}^2$$
        \end{convention}
        
        The hypotheses of the following lemma are satisfied at least when $\g$ is either of finite type or of affine type, save for the types $\sfA_1^{(1)}$ and $\sfA_1^{(2)}$.
        \begin{lemma}[A Levendorskii-type presentation for Yangians of Kac-Moody algebras] \label{lemma: levendorskii_presentation}
            \cite[Theorem 2.13]{guay_nakajima_wendlandt_affine_yangian_coproduct} Choose a total ordering on the set of simple roots $\simpleroots_{\fraku}$ and suppose that the Cartan matrix $C := (c_{ij})_{i, j \in \simpleroots_{\fraku}}$ of the Kac-Moody algebra $\fraku$ is such that, for any $i < j \in \simpleroots_{\fraku}$ (with respect to some choice of total ordering on $\simpleroots_{\fraku}$) the following $2 \x 2$ submatrix is invertible:
                $$
                    \begin{pmatrix}
                        c_{ii} & c_{ij}
                        \\
                        c_{ji} & c_{ji}
                    \end{pmatrix}
                $$
            
            The formal Yangian $\rmY_{\hbar}(\fraku)$ of $\fraku$ will then be isomorphic to the associative $\bbC$-algebra generated by the set:
                $$\{ H_{i, r}, X_{i, r}^{\pm} \}_{(i, r) \in \simpleroots_{\fraku} \x \Z_{\geq 0}}$$
            whose elements are subjected to the following relations\footnote{... and it is understood that the elements $H_{i, 0} = h_i, X_{i, 0}^{\pm} = e_i^{\pm}$ satisfy the Chevalley-Serre relations defining $\fraku$; cf. \cite[Chapter 1]{kac_infinite_dimensional_lie_algebras}.}:
                $$H_{i, 0} = h_i, X_{i, 0}^{\pm} = x_i^{\pm}$$
                $$[ H_{i, r}, H_{j, s} ] = 0$$
                $$[ H_{i, 0}, X_{j, s}^{\pm} ] = \pm (\alpha_i, \check{\alpha}_j)_{\fraku} X_{j, s}^{\pm}$$
                $$[ X_{i, r}^+, X_{j, s}^- ] = \pm \delta_{ij} H_{i, r + s}$$
                $$\left[ T_{i, 1}(\hbar), X_{j, 0}^{\pm} \right] = \pm \hbar (\alpha_i, \check{\alpha}_j)_{\fraku} X_{j, 1}^{\pm}$$
                $$[ X_{i, 1}^{\pm}, X_{j, 0}^{\pm} ] - [ X_{i, 0}^{\pm}, X_{j, 1}^{\pm} ] = \pm \frac12 \hbar (\alpha_i, \check{\alpha}_j)_{\fraku} \{X_{i, 0}^{\pm}, X_{j, 0}^{\pm}\}$$
        \end{lemma}

        Let us now consider what might occur when we specialise the variable $\hbar$ to a specific value $\hbar_0 \in k$.

        Firstly, let us consider the case $\hbar_0 \not = 0$, which is much simpler than the case $\hbar_0 = 0$. 
        \begin{definition}[Yangians associated to symmetrisable Kac-Moody algebras] \label{def: yangians_associated_to_symmetrisable_kac_moody_algebras}
            By specialising $\hbar$ to some $\hbar_0 \in k^{\x}$, one obtains the \textbf{Yangian} associated to $\fraku'$ and $\fraku$, respectively:
                $$\rmY(\fraku') := \rmY_{\hbar}(\fraku')/(\hbar - \hbar_0)$$
                $$\rmY(\fraku) := \rmY_{\hbar}(\fraku)/(\hbar - \hbar_0)$$
        \end{definition}
        \begin{definition}[Rees algebras] \label{def: rees_algebras}
            Let:
                $$A := \bigoplus_{r \in \Z_{\geq 0}} A_r$$
            be a $\Z_{\geq 0}$-filtered $k$-algebra. The \textbf{Rees algebra} associated to $A$ is then the $k[\hbar]$-algebra given by:
                $$\Rees_{\hbar} A := \bigoplus_{r \in \Z_{\geq 0}} A_r \hbar^r$$
        \end{definition}
        \begin{lemma}[Basic properties of Rees algebras] \label{lemma: basic_properties_of_rees_algebras}
            \cite[Exercise I.9.5]{kassel_quantum_groups} Let:
                $$A := \bigoplus_{r \in \Z_{\geq 0}} A_r$$
            be a $\Z_{\geq 0}$-filtered $k$-algebra. Also, fix an arbitrary element $\hbar_0 \in k$.
            \begin{enumerate}
                \item There are $k$-algebra isomorphisms:
                    $$
                        (\Rees_{\hbar} A)/(\hbar - \hbar_0) \cong
                        \begin{cases}
                            \text{$A$ if $\hbar_0 \not = 0$}
                            \\
                            \text{$\gr A$ if $\hbar_0 = 0$}
                        \end{cases}
                    $$
                \item Suppose that the associated graded algebra $\gr A$ is generated by a set of homogeneous elements:
                    $$a_0, a_1, a_2, ...$$
                respectively of degree $r_0 := 0, r_1, r_2, ... \in \Z_{\geq 0}$, then $\Rees_{\hbar} A$ will be generated by the elements:
                    $$a_0, a_1 \hbar^{r_1}, a_2 \hbar^{r_2}, ...$$
            \end{enumerate}
        \end{lemma}
        \begin{convention}
            In light of lemma \ref{lemma: basic_properties_of_rees_algebras}, it is typically to just make the choice:
                $$\hbar_0 := 1$$
            and hence work with:
                $$\rmY(\fraku) := \rmY_1(\fraku)$$
        \end{convention}
        \begin{remark}[Yangians are $\Z_{\geq 0}$-filtered] \label{remark: positive_Z_filtrations_on_yangians}
            Now, the Yangian $\rmY(\fraku)$ is no longer graded but just $\Z_{\geq 0}$-filtered. The (ascending) filtration in question is given by:
                $$\deg X_{i, r}^{\pm} = \deg H_{i, r} = r$$
                
            However, whether this filtration gives - via the Rees algebra construction - a $\Z_{\geq 0}$-grading agreeing with the one on formal Yangians is a rather subtle issue, with no complete solution in general. Another difficult problem is that of explicitly computing the associated graded algebra:
                $$\gr \rmY(\fraku)$$
            When $\fraku$ is of finite type, it is known that:
                $$\gr \rmY(\fraku) \cong \rmU(\fraku[t])$$
            (in fact, Drinfeld seemed to have constructed the Yangian of a finite-dimensional simple Lie algebra specifically so that this would be true) but when $\fraku$ is a general indecomposable symmetrisable Kac-Moody algebra, what the algebra $\gr \rmY(\fraku)$ is explicitly is still not known.
            
            That said, partial answers to both of these problems have been obtained in \cite[Section 6]{guay_nakajima_wendlandt_affine_yangian_vertex_representations_and_PBW}, where the authors focused on the case where $\fraku$ is of a simply laced affine type. We defer the discussion of the details of these problems to subsection \ref{subsection: yangians_as_deformations}, where the special cases relevant to our goals are examined. 
        \end{remark}

        \begin{remark}
            When $\fraku$ is of finite type, it is also known that the formal Yangian $\rmY_{\hbar}(\fraku)$ (hence also the Yangian $\rmY(\fraku)$) carries a Hopf bialgebra structure\footnote{Recall also that Hopf bialgebra structures, if they exist, are unique up to isomorphisms.}. For organisational purposes, we would like the contents of this section to remain entirely algebraic, as opposed to algebraic and coalgebraic. As such, we will not mention this fact again until subsection \ref{subsection: manin_triples_and_quantisations_of_lie_bialgebras},where we will discuss this Hopf structures as an instance of the much more general fact that Lie bialgebras admit so-called \say{quantisations}. 
        \end{remark}

    \subsection{Finite-type and affine Yangians as deformations} \label{subsection: yangians_as_deformations}
        Let us now discuss how, when $\fraku$ is either of finite type or affine type\footnote{... and satisfying some other mild technical restrictions.}, it can be shown that $\rmY_{\hbar}(\fraku)$ is the flat deformation of the universal enveloping algebra of a certain polynomial Lie algebra. In the former case, we will see that the Lie algebra in question is:
            $$\g[t]$$
        whereas in the latter case, it is:
            $$\toroidal^{\positive} := \uce(\g[v^{\pm 1}, t])$$
        Both ought to be viewed as UCEs; it just so happens that the Lie algebra $\g[t]$ admits itself as a UCE (cf. example \ref{example: affine_lie_algebras_centres}).
    
        \begin{definition}[Graded and PBW deformations] \label{def: graded_and_PBW_deformations}
            Fix an $\Z_{\geq 0}$-graded associative algebra:
                $$U_0 := \bigoplus_{r \geq 0} U_r$$
            over a field $k$. An \textbf{$\Z_{\geq 0}$-graded deformation} of such an algebra $U_0$ is then an $\Z_{\geq 0}$-graded associative $k[\hbar]$-algebra $U_{\hbar}$, free as a $k[\hbar]$-module, and such that:
                $$U_{\hbar}/\hbar U_{\hbar} \cong U_0$$
            Now, fix some $\hbar_0 \in k^{\x}$. The algebra:
                $$U_{\hbar_0} := U_{\hbar}/(\hbar - \hbar_0)U_{\hbar}$$
            is then called the \textbf{PBW deformation} of $U_0$ at $\hbar_0$.  
        \end{definition}
    
            \section{Affine Yangians as quantisations}
    \begin{convention}[Graded duals]
        If $Z$ is an abelian group and:
            $$V := \bigoplus_{d \in Z} V_d$$
        is a $Z$-graded vector space, then we will be writing:
            $$V^{\star} := \bigoplus_{d \in Z} V_d^*$$
        for its $Z$-graded dual.
    \end{convention}

    \subsection{Manin triples and quantisations of Lie bialgebras} \label{subsection: manin_triples_and_quantisations_of_lie_bialgebras}
        \begin{remark}
            Technically speaking, we do not need to work over an algebraically closed field, and especially not $\bbC$ specifically, for the abstract machineries presented in this subsection to be true. This hypothesis is only needed for some of the examples. 
        \end{remark}
    
        \begin{definition}[Lie bialgebras] \label{def: lie_bialgebras}
            Let $\a$ be a Lie algebra over $\bbC$ equipped with a $\bbC$-linear map:
                $$\delta: \a \to \a \tensor_{\bbC} \a$$
            $\a$ will then be a \textbf{Lie bialgebra} (over $\bbC$) with \textbf{Lie cobracket} $\delta$ if the following two conditions are satisfied:
            \begin{itemize}
                \item Firstly, we require that the map:
                    $$\delta^*: (\a \tensor_{\bbC} \a)^* \to \a^*$$
                induces a Lie bracket $\a^* \tensor_{\bbC} \a^* \to \a^*$ on the $\bbC$-vector space $\a^*$. Equivalently, this is saying $\delta: \a \to \a \tensor_{\bbC} \a$ must firstly be \textbf{Lie cobracket}, i.e. it is to be skew-symmetric (i.e. its codomain is actually $\a \wedge \a$) and to satisfy the \textbf{co-Jacobi identity}:
                    $$( (1 \: 2 \: 3) + (2 \: 3 \: 1) + (3 \: 1 \: 2) ) \circ (\delta \circ \id_{\a}) \circ \delta = 0$$
                \item Secondly, we insist that the Lie cobracket $\delta$ is a Lie $1$-cocycle\footnote{In cohomological terms, one can write $\delta \in H^1_{\Lie}(\a, \a \tensor_{\bbC} \a)$.} of $\a$ with coefficients in $\a \tensor_{\bbC} \a$, which is to say that the following identity is to hold in $\rmU(\a) \tensor_{\bbC} \rmU(\a)$:
                    $$\delta( [x, y] ) = [\bar{\Delta}(x), \delta(y)] + [\delta(x), \bar{\Delta}(y)]$$
                for all $x, y \in \a$.
            \end{itemize}
            If $(\a, [-, -], \delta)$ and $(\a', [-, -]', \delta')$ are Lie bialgebras then a homomorphism between them will be a homomorphism between the underlying Lie algebras $\phi: (\a, [-, -]) \to (\a', [-, -]')$ such that:
                $$\phi^{\tensor 2} \circ \delta = \delta' \circ \phi$$
        \end{definition}
        \begin{definition}[Graded Lie bialgebras] \label{def: graded_lie_bialgberas}
            Let $(\a, [-, -], \delta)$ be a Lie bialgebra over $\bbC$, whose underlying vector space is graded by some abelian group $Z$, i.e.:
                $$\a := \bigoplus_{d \in Z} \a_d$$
            in such a way that the graded components $\a_d$ are all finite-dimensional as vector spaces over $\bbC$. This Lie bialgebra is then said to be \textbf{$Z$-graded} if and only if the following conditions are satisfied:
            \begin{itemize}
                \item $[-, -]$ is a graded Lie bracket, i.e.:
                    $$[\a_m, \a_n] \subseteq \a_{m + n}$$
                for all $m, n \in Z$.
                \item $\delta$ is a graded Lie cobracket, in the sense that for any $d \in Z$, one has that:
                    $$\delta(\a_d) \subseteq \bigoplus_{m + n = d} \a_m \tensor_{\bbC} \a_n$$
            \end{itemize}
        \end{definition}
        \begin{remark}[Duals of Lie bialgebras and Drinfeld's double construction] \label{remark: drinfeld_doubles}
            Let:
                $$(\a, [-, -]_{\a}, \delta_{\a})$$
            be a Lie bialgebra over $\bbC$.
        
            Suppose firstly that $\a$ is a finite-dimensional. Then clearly, the full linear dual $\a^*$ will also be a Lie bialgebra, whose Lie cobracket:
                $$\delta_{\a^*}: \a^* \to \a^* \tensor_{\bbC} \a^*$$
            is induced by the dual $[-, -]_{\a}^*: \a^* \to (\a \tensor_{\bbC} \a)^*$ of the Lie bracket on $\a$. Furthermore, we note that there is no natural non-discrete topology on the linear dual of a finite-dimensional vector space.

            If $\a$ is an infinite-dimensional Lie bialgebra, on the other hand, then the full linear dual $\a^*$ will not generally be a Lie bialgebra but rather a \textbf{topological Lie bialgebra}, in the sense that the codomain of its Lie cobracket $\delta_{\a^*}$ may not $\a^* \tensor_{\bbC} \a^*$, but rather in some appropriate topological completion $\a^* \hattensor_k \a^*$.

            In either event, the $\bbC$-vector space:
                $$\frakDr(\a) := \a \oplus \a^*$$
            can be made into a Lie algebra over $\bbC$, with Lie bracket given by\footnote{Note that $\frakDr(\a)^{\tensor 2} \cong \a^{\tensor 2} \oplus (\a^*)^{\tensor 2} \oplus (\a^* \tensor_{\bbC} \a \oplus \a \tensor_{\bbC} \a^*)$}:
                $$[-, -]_{\frakDr(\a)} := [-, -]_{\a} \oplus [-, -]_{\a^*} \oplus ( [-, -]_{\a} \circ (S_{\a^*} \tensor \id_{\a}) \oplus [-, -]_{\a^*}^{\op} \circ (\id_{\a^*} \tensor S_{\a}) )$$
            where:
                $$[-, -]_{\a^*}$$
            is the Lie bracket on $\a^*$ induced by $\delta_{\a}^*$ (well-defined since $\a$ is a Lie bialgebra by hypothesis) with opposite $[-, -]_{\a^*}^{\op} := -[-, -]_{\a^*}$, and:
                $$S_{\a}, S_{\a^*}$$
            denote the antipodes on the universal enveloping algebras of $\a, \a^*$ respectively. A natural non-degenerate and symmetric $\bbC$-bilinear form $(-, -)_{\frakDr(\a)}$ on $\frakDr(\a)$ which is invariant with respect to $[-, -]_{\frakDr(\a)}$ can then be constructed by declaring that:
                $$(x + \varphi, y + \psi)_{\frakDr(\a)} := \psi(x) + \varphi(y)$$
            for all $x, y \in \a$ and all $\varphi, \psi \in \a^*$. 

            A topological Lie cobracket on $\frakDr(\a)$ can also be given as:
                $$\delta_{\frakDr(\a)} := \delta_{\a} \oplus \delta_{\a^*}^{\cop}$$
            with $\delta_{\a^*}^{\cop} := -\delta_{\a^*}$ being the opposite Lie cobracket on $\a^*$, and it can be verified that this is a $1$-cocycle of $\frakDr(\a)$ with values in $\frakDr(\a) \tensor_{\bbC} \frakDr(\a)$, thus making $\frakDr(\a)$ a Lie bialgebra. This Lie bialgebra is typically called the \textbf{Drinfeld double}\footnote{Sometimes also called the \textbf{classical double}.} of $\a$ (or equivalently, of $\a^*$).
        \end{remark}
        
        \begin{remark}[Graded Drinfeld doubles] \label{remark: graded_drinfeld_doubles}
            A graded analogue of Drinfeld doubles is also available for graded Lie bialgebras:
                $$(\a, [-, -]_{\a}, \delta_{\a})$$
            with finite-dimensional graded components. When $\a$ is finite-dimensional, there is no difference from the construction given in remark \ref{remark: drinfeld_doubles}. When $\a$ is infinite-dimensional, instead of considering full linear duals as in remark \ref{remark: drinfeld_doubles}, one considers graded duals. The rest can then be carried out in the same manner, yielding a graded Lie bialgebra structure on:
                $$\frakDr(\a) := \a \oplus \a^{\star}$$
        \end{remark}
        Even though we use the same notation for Drinfeld doubles of ungraded and graded Lie bialgebras, how the underlying vector space is given should be clear from context. We will specify if needed.
        
        The construction of (graded) Drinfeld doubles leads to the following notion of \say{Manin triples}, which in some ways is easier to work with than Lie bialgebras. We do not lose information by passing to these Manin triples, however, since each of them gives rise to a Lie bialgebra and \textit{vice versa}.
        \begin{definition}[Manin triples] \label{def: manin_triples}
            A \textbf{(graded) Manin triple} is the data of a triple of (graded) Lie algebras:
                $$(\a, \a^+, \a^-)$$
            as well as a non-degenerate and invariant symmetric bilinear form $(-, -)$ satisfying the following conditions:
            \begin{itemize}
                \item Either the Lie algebras $\a, \a^+, \a^-$ are finite-dimensional, or the graded components of the Lie algebras $\a, \a^+, \a^-$ are finite-dimensional.
                \item $\a \cong \a^+ \oplus \a^-$.
                \item The bilinear form $(-, -)$ pairs the Lie subalgebras $\a^{\pm}$ isotropically, i.e. $(\a^{\pm}, \a^{\pm}) = 0$. 
                \item Via $(-, -)$, one gets an identification $\a^- \cong (\a^+)^*$; respectively, to obtain a graded Manin triple, we require that $\a^- \cong (\a^+)^{\star}$ via $(-, -)$.
            \end{itemize}
        \end{definition}
        \begin{remark}[Coboundary Lie algebras and classical R-matrices] \label{remark: classical_R_matrices}
            Let $\a$ be a Lie algebra.
        
            Let us note here that there is an alternative way to construct a Lie bialgebra structure on the vector space:
                $$\frakDr(\a) := \a \oplus \a^*$$
            as follows, which turns out to be more useful in practice (cf. e.g. example \ref{example: finite_type_yangian_manin_triple} and corollary \ref{coro: extended_toroidal_lie_bialgebras}). Namely, one makes use of the canonical elements:
                $$\sfr_{\a} \in \a \tensor_{\bbC} \a^*, \sfr_{\a^*} \in \a^* \tensor_{\bbC} \a$$
            which are to be the preimage of $\id_{\a} \in \End_k(\a), \id_{\a^*} \in \End_k(\a^*)$ under the canonical maps\footnote{Note that these maps are both injective, precisely because $(-, -)_{\frakDr(\a)}$ is non-degenerate by construction\footnote{Note that we are not making use of invariance here, and hence we do not need to assume that $\a$ is a Lie bialgebra from the start (this assumption is needed for the construction of the Lie bracket on $\frakDr(\a)$).}, and hence $\sfr_{\a}$ is well-defined.} $\a \tensor_{\bbC} \a^* \to \End_k(\a), \a^* \tensor_{\bbC} \a \to \End_k(\a^*)$ given by $x \tensor \varphi \mapsto (-, \varphi)_{\frakDr(\a)} x$ and by $x \tensor \varphi \mapsto (x, -)_{\frakDr(\a)} \varphi$, respectively, for all $x \in \a$ and all $\varphi \in \a^*$. A theorem of Drinfeld asserts that:
            \begin{enumerate}
                \item if:
                    $$\delta_{\a} := [\bar{\Delta}, \sfr_{\a}], \delta_{\a^*} := [\bar{\Delta}, \sfr_{\a^*}]$$
                then $\delta_{\a}$ and $\delta_{\a^*}$ will be Lie cobrackets on $\a$ and on $\a^*$ respectively, and
                \item if furthermore, the so-called \textbf{classical Yang-Baxter equation}:
                    $$[\sfr_{12}, \sfr_{13}] + [\sfr_{12}, \sfr_{23}] + [\sfr_{13}, \sfr_{13}]$$
                (where $\sfr := \sfr_{\a} - \sfr_{\a^*}$) is $\frakDr(\a)$-invariant, then:
                    $$\delta_{\frakDr(\a)} = \delta_{\a} \oplus \delta_{\a^*}^{\cop} = [\bar{\Delta}, \sfr]$$
                will be a Lie bialgebra structure on $\frakDr(\a)$.
            \end{enumerate}
            Consequently, $\delta_{\a}$ and $\delta_{\a^*}$ will be Lie bialgebra structures on $\a$ and $\a^*$, per the discussions in remark \ref{remark: drinfeld_doubles}.
        \end{remark}
        \begin{definition}[Classical R-matrices] \label{def: classical_R_matrices}
            Let:
                $$(\a, \a^+, \a^-)$$
            be a Manin triple, where $\a$ is equipped with a non-degenerate and invariant symmetric bilinear form $(-, -)$. Then, the canonical element $\sfr_{\a} \in \a \tensor_{\bbC} \a^*$ with respect to $(-, -)$ as in remark \ref{remark: classical_R_matrices} shall be referred to as the \textbf{classical R-matrix} of $\a$.
        \end{definition}
        \begin{lemma}[Lie bialgebras from Manin triples] \label{lemma: lie_bialgebras_from_manin_triples}
            \begin{enumerate}
                \item \cite[Proposition 1.3.4 and Lemma 1.3.5]{chari_pressley_quantum_groups} There is a bijective function from the set of finite-dimensional Lie bialgebras to the set of finite-dimensional Manin triples.
                \item Let $Z$ be an abelian group. 

                There is a bijective function from the set of $Z$-graded Lie bialgebras with finite-dimensional graded components to the set of $Z$-graded Manin triples $(\p, \p^+, \p^-)$ where each of $\p, \p^+, \p^-$ has finite-dimensional graded components.
            \end{enumerate}
            Both functions map Lie bialgebras $(\a, [-, -], \delta)$ to the Manin triples $(\frakDr(\a), \a, \a^*)$. Their inverses map Manin triples $(\a, \a^+, \a^-)$ to the Lie bialgebra $(\a^+, \delta^+)$, with $\delta^+ := [\bar{\Delta}, \sfr^+]$, where $\sfr^+$ being the classical R-matrix of $\a^+$.
        \end{lemma}
        \begin{definition}[Lie sub-bialgebras and Lie coideals] \label{def: lie_sub_bialgebras_and_lie_coideals}
            Let $(\a, [-, -], \delta)$ be a Lie bialgebra.
            \begin{itemize}
                \item A \textbf{Lie sub-bialgebra} of $\a$ is then determined by an injective Lie bialgebra homomorphism $\b \to \a$. 
                \item A \textbf{Lie coideal} therein is then a vector subspace $\b \subseteq \a$ such that:
                    $$\delta(\b) \subseteq \a \tensor_{\bbC} \b \oplus \b \tensor_{\bbC} \a$$
                or equivalently, if $\b$ is a Lie sub-bialgebra such that $\a/\b$ inherits a Lie bialgebra structure from the one on $\a$.
            \end{itemize}
        \end{definition}
        \begin{example}[The Kac-Moody Manin triple] \label{example: kac_moody_manin_triple}
            Via the \textit{a priori} non-degenerate Kac-Moody form $(-, -)_{\fraku}$, one has that:
                $$\fraku_{-\alpha} \xrightarrow[]{\cong} \fraku_{\alpha}^*$$
            for any (positive) root $\alpha$ (cf. \cite[Theorem 2.2]{kac_infinite_dimensional_lie_algebras}), and hence:
                $$\fraku_{\low} \xrightarrow[]{\cong} \fraku_{\up}^{\star}$$
            with the isomorphisms in question being given by $x \mapsto (x, -)_{\fraku}$ for all $x \in \fraku_{\alpha}$ and all roots $\alpha \in \Phi_{\fraku}$. \cite[Theorem 2.2]{kac_infinite_dimensional_lie_algebras} also asserts that:
                $$\fraku_0 \xrightarrow[]{\cong} \fraku_0^*$$
            via $h \mapsto (h, -)_{\fraku}$, given for all $h \in \fraku_0$. As such, we have that:
                $$\b_{\low} \cong \b_{\up}^{\star}$$
            Additionally, the same result from \cite{kac_infinite_dimensional_lie_algebras} tells us that:
                $$(\fraku_0, \fraku_{\up/\low})_{\fraku} = (\fraku_{\low}, \fraku_{\low})_{\fraku} = (\fraku_{\up}, \fraku_{\up})_{\fraku} = 0$$
            and that the restriction of $(-, -)_{\fraku}$ to the Cartan subalgebra $\fraku_0$ remains non-degenerate. One can therefore define a bilinear form on $\b_{\up} \oplus \b_{\low}$ by:
                $$\<x + y, x' + y'\> := (y, x')_{\fraku} + (y', x)_{\fraku}$$
            for all $x, x' \in \b_{\up}$ and all $y, y' \in \b_{\low}$. This is for the purpose of pairing the upper/lower Borel subalgebras isotropically:
                $$\<\b_{\low}, \b_{\low}\> = \<\b_{\up}, \b_{\up}\> = 0$$
                $$\<\b_{\low}, \b_{\up}\> \not = 0$$
            This new bilinear form can also be seen to be non-degenerate, and one has that:
                $$\b_{\low} \xrightarrow[]{\cong} \b_{\up}$$
            via $x \mapsto \<x, -\>$. By endowing the vector space $\b_{\up} \oplus \b_{\low}$ with the Lie bracket:
                $$[x + y, x' + y']_{\b_{\up} \oplus \b_{\low}} := [x + y, x' + y']_{\fraku}$$
            given for all $x, x' \in \b_{\up}$ and all $y, y' \in \b_{\low}$, one can also show that $\<-, -\>$ is $\b_{\up} \oplus \b_{\low}$-invariant.

            With the above, we have constructed a Manin triple:
                $$(\b_{\up} \oplus \b_{\low}, \b_{\up}, \b_{\low})$$
            Per remark \ref{remark: classical_R_matrices} and lemma \ref{lemma: lie_bialgebras_from_manin_triples}, we know that there is thus a Lie bialgebra structure on $\b_{\up} \oplus \b_{\low}$ given by:
                $$\tilde{\delta} := [\bar{\Delta}, \sfr_{\fraku}]$$
            where $\sfr_{\fraku}$ is the Casimir tensor of $\sfr_{\fraku}$ (cf. \cite[Section 2.5]{kac_infinite_dimensional_lie_algebras}). It is not hard to see that $\fraku_0$ is a Lie coideal of $\b_{\up} \oplus \b_{\low}$, and hence we get an induced Lie bialgebra structure $\delta$, given by the same formula, on $\fraku \cong (\b_{\up} \oplus \b_{\low})/\fraku_0$.
            
            It can also be shown that, on the Chevalley-Serre generators of $\fraku$, the Lie cobracket $\delta$ is given by:
                $$\delta(h_i) = 0, \delta(x_i^{\pm}) = h \tensor x_i^{\pm}$$

            Typically, this Lie bialgebra structure on $\fraku$ is called the \textbf{standard} one.
        \end{example}
        \begin{example}[The Yangian Manin triple] \label{example: finite_type_yangian_manin_triple}
            There is a $\Z$-graded Manin triple:
                $$( \g[t^{\pm 1}], \g[t], t^{-1}\g[t^{-1}] )$$
            wherein $\g[t^{\pm 1}]$ is equipped with the following \textit{a priori} invariant inner product, given for all $x, y \in \g$ and all $p, q \in \Z$:
                $$(x t^p, y t^q)_{\g[t^{\pm 1}]} := (x, y)_{\g} \delta_{p + q, -1}$$
            Corresponding to this Manin triple is a topological Lie bialgebra structure on $\g[t]$, wherein the Lie cobracket:
                $$\delta: \g[t] \to \g[t] \hattensor_k \g[t] \subset \g^{\tensor 2}[\![t_1, t_2^{-1}]\!]$$
            is given by:
                $$\delta(X)(t_1, t_2) = [ X(t_1) \tensor 1 + 1 \tensor X(t_2), \sfr_{\g} \frac{1}{t_2 - t_1} ]$$
            for all $X(t) \in \g[t]$, with $\frac{1}{t_2 - t_1}$ being a shorthand for the geometric series $\sum_{p \in \Z_{\geq 0}} t_1^p t_2^{-p - 1}$.

            In this situation, we have that:
                $$\g[t^{\pm 1}] \cong \frakDr(\g[t])$$
            or equivalently, it is the Drinfeld double of $t^{-1}\g[t]$, with the Lie bialgebra structure graded-dual to that on $\g[t]$. 
        \end{example}

        \begin{definition}[Quantisations] \label{def: quantisations}
            Let $\a$ be a Lie bialgebra over $\bbC$, say with Lie cobracket $\delta$. A \textbf{(graded) quantisation} of $\a$ is a (graded) Hopf $\bbC[\![\hbar]\!]$-bialgebra $Y_{\hbar}$, say with coproduct $\Delta_{\hbar}$, such that:
            \begin{itemize}
                \item $Y_{\hbar}$ is a (graded) flat deformation\footnote{In the sense of definition \ref{def: graded_and_PBW_deformations}.} of $\rmU(\a)$, and
                \item For any $x \in \a$, one has that:
                    $$\delta(x) \equiv \frac{1}{\hbar}(\Delta_{\hbar} - \Delta_{\hbar}^{\cop})(x) \pmod{\hbar}$$
            \end{itemize}
            We say also, that $(\a, \delta)$ is the \textbf{classical limit} of $(Y_{\hbar}, \Delta_{\hbar})$. 
        \end{definition}
        \begin{remark}
            The question of existence and uniqueness of quantisations of Lie bialgebras is a difficult one. We refer the reader to \cite{etingof_kazhdan_quantisation_1} and its sequels for details.
        \end{remark}

        Let us now consider some examples of (non-cocommutative) Hopf algebras that quantise the Lie bialgebras constructed in examples \ref{example: kac_moody_manin_triple} and \ref{example: finite_type_yangian_manin_triple}.
        \begin{example}[Quantised enveloping algebras] \label{example: finite_type_QUEs}
            Consider the finite-dimensional simple Lie algebra $\g$ with its standard Lie bialgebra structure, as constructed in example \ref{example: kac_moody_manin_triple}. In this situation, a quantisation:
                $$\rmU_{\hbar}(\g)$$
            of $\g$ is known. We are not interested in this quantum group here, so for more details, we refer the reader to \cite[Section XVII.2]{kassel_quantum_groups}. 
        \end{example}
        \begin{example}[Finite-type Yangians] \label{example: finite_type_yangians}
            The classical limit of the formal Yangian:
                $$\rmY(\g)$$
            equipped with the Hopf structure:
                $$(\Delta_{\hbar}, S_{\hbar}, )$$
            as in theorem \ref{theorem: finite_type_yangian_hopf_structure} admits the Lie bialgebra $\g[t]$ (cf. example \ref{example: finite_type_yangian_manin_triple}) as its classical limit. Therefore, by definition, the Hopf bialgebra $\rmY_{\hbar}(\g)$ is a quantisation of the current Lie bialgebra $\g[t]$. This can be proven by setting $\hbar = 0$, which degenerates $\rmY_{\hbar}(\g)$ to $\rmU(\g[t])$ with its usual Hopf structure. Also, by specialising to $\hbar = 1$, one obtains a $\Z_{\geq 0}$-filtered Hopf $\bbC$-bialgebra structure $(\Delta_1, S_1, \e_1)$ on $\rmY(\g)$, according to lemma \ref{lemma: basic_properties_of_rees_algebras}.
        \end{example}
        By cohomological arguments, one can also show that both of these quantisations are unique, but we will not go into those details.

        \todo[inline]{Write about how affine Yangians are not exactly quantisations.}

    \subsection{Toroidal Lie bialgebras} \label{subsection: toroidal_lie_bialgebras}
        The goal of this subsection is to construct a topological Lie bialgebra structure on the toroidal Lie algebra:
            $$\toroidal^{\positive} := \uce(\g_{[2]}^{\positive})$$
        so that we can realise the formal Yangian $\rmY_{\hbar}(\hat{\g})$ not merely as a $\Z_{\geq 0}$-graded flat deformation of $\rmU(\toroidal^{\positive})$, but also as a $\Z_{\geq 0}$-graded quantisation of $\toroidal^{\positive}$. We will proceed towards this goal by firstly constructing a topological Lie bialgebra structure on the Yangian extended toroidal Lie algebra:
            $$\extendedtoroidal^{\positive} := \toroidal^{\positive} \rtimes \d_{[2]}^{\positive}$$
        (cf. lemma \ref{lemma: positive/negative_yangian_extended_toroidal_lie_algebras}) and then secondly, proving that the Lie ideal $\toroidal \subset \extendedtoroidal$ is actually also a Lie coideal (cf. definition \ref{def: lie_sub_bialgebras_and_lie_coideals}), and hence a Lie sub-bialgebra of $\extendedtoroidal^{\positive}$ as well. This shall be done by checking that the codomain of the restriction:
            $$\hat{\delta}^{\positive}|_{\toroidal^{\positive}}$$
        lies inside an appropriate topological completion $\toroidal^{\positive} \hattensor_{\bbC} \toroidal^{\positive}$ of $\toroidal^{\positive} \tensor_{\bbC} \toroidal^{\positive}$ and to this end, we will need to know how $\toroidal^{\positive}$ is given by generators and relations. For this, we will firstly establish a Chevalley-Serre presentation for $\toroidal^{\positive}$. One will then see that it is actually computationally infeasible to verify that:
            $$\hat{\delta}^{\positive}(X) \in \toroidal^{\positive} \tensor_{\bbC} \toroidal^{\positive}$$
        for a Chevalley-Serre generator $X \in \toroidal^{\positive}$ of degree $> 1$. As such, we will have to exploit the existence of the Levendorskii presentation for $\rmY_{\hbar}(\hat{\g})$ - in terms of low-degree generators of degrees $\leq 1$ - in order to give a similar presentation for $\toroidal^{\positive}$ in terms of generators or degrees $\leq 1$.

        First of all, let us note that, almost as a direct consequence of lemma \ref{lemma: positive/negative_yangian_extended_toroidal_lie_algebras}, one has a topological Manin triple that gives rise to a topological Lie bialgebra structure on $\extendedtoroidal^{\positive}$.
        \begin{theorem} \label{theorem: extended_toroidal_manin_triples}
            There is a complete topological Manin triple:
                $$(\extendedtoroidal, \extendedtoroidal^{\positive}, \extendedtoroidal^{\negative})$$
            wherein $\extendedtoroidal$ is equipped with the non-degenerate invariant inner product $(-, -)_{\extendedtoroidal}$ (cf. convention \ref{conv: orthogonal_complement_of_toroidal_centres}).
        \end{theorem}
            \begin{proof}
                We know from lemma \ref{lemma: positive/negative_yangian_extended_toroidal_lie_algebras} that $\extendedtoroidal^{\positive/\negative}$ are Lie subalgebras of $\extendedtoroidal$, so it now remains to show that $(-, -)_{\extendedtoroidal}$ pairs the subalgebras $\extendedtoroidal^{\positive/\negative}$ isotropically, but this is true entirely due to how this invariant bilinear form was constructed in convention \ref{conv: orthogonal_complement_of_toroidal_centres}.
            \end{proof}
        \begin{corollary}[Lie cobracket on $\extendedtoroidal^{\positive}$] \label{coro: extended_toroidal_lie_bialgebras}
            On the extended toroidal Lie algebra $\extendedtoroidal^{\positive}$, there is a continuous Lie cobracket\footnote{Note the completion!}, making $\extendedtoroidal^{\positive}$ a complete topological Lie bialgebra:
                $$\hat{\delta}^{\positive}: \extendedtoroidal^{\positive} \to \extendedtoroidal^{\positive} \hattensor_k \extendedtoroidal^{\positive}$$
            given for any $X \in \extendedtoroidal^{\positive}$ by the following formula (cf. \cite{etingof_kazhdan_quantisation_1}):
                $$\hat{\delta}^{\positive}(X) = [ X \tensor 1 + 1 \tensor X, \sfr_{\extendedtoroidal^{\positive}} ]$$
            wherein:
                $$\sfr_{\extendedtoroidal^{\positive}} := \sfr_{\g} + \sfr_{\z_{[2]}^{\positive}} + \sfr_{\d_{[2]}^{\positive}} \in \extendedtoroidal^{\positive} \hattensor_k \extendedtoroidal^{\negative}$$
            with:
                $$\sfr_{\g_{[2]}^{\positive}} \in \g_{[2]}^{\positive} \hattensor_k \g_{[2]}^{\negative} := \sfr_{\g} v_2 \1(v_1, v_2) \1(t_1, t_2)$$
            and\footnote{Note how we are simply summing over tensor products of dual basis elements.} $\sfr_{\z_{[2]}^{\positive}} \in \z_{[2]}^{\positive} \hattensor_k \d_{[2]}^{\positive}$ and $\sfr_{\d_{[2]}^{\positive}} \in \d_{[2]}^{\positive} \hattensor_k \z_{[2]}^{\positive}$ being given by the following formulae:
                $$\sfr_{\z_{[2]}^{\positive}} := \sum_{(r, s) \in \Z \x \Z_{> 0}} K_{r, s} \tensor D_{r, s} + c_v \tensor D_v$$
                $$\sfr_{\d_{[2]}^{\positive}} := \sum_{(r, s) \in \Z \x \Z_{\leq 0}} D_{r, s} \tensor K_{r, s} + D_t \tensor c_t$$
        \end{corollary}

        \todo[inline]{Chevalley-Serre and Levendorskii presentations for $\toroidal^{\positive}$}
        For what follows, let us remind the reader that the standing assumption throughout the entirety of this section is that:
            $$\g \not \cong \sl_2(\bbC)$$
        \begin{lemma}[Chevalley-Serre presentation for positive toroidal Lie algebras] \label{lemma: chevalley_serre_presentations_for_positive_toroidal_lie_algebras}
            The Lie algebra $\toroidal^{\positive}$ is isomorphic to the Lie algebra $\s$ that is generated by the set:
                $$\{X_{i, r}^{\pm}, H_{i, r}\}_{(i, r) \in \hat{\simpleroots} \x \Z_{\geq 0}}$$
            whose elements are subjected to the following relations, given for all $i, j \in \hat{\simpleroots}$ and all $r, s \in \Z_{\geq 0}$:
                $$$$
            and when $i \not = j$, there are also the following \say{Serre relations}:
                $$$$
            The isomorphism in question is given as follows, for all $(i, r) \in \hat{\simpleroots} \x \Z_{\geq 0}$
                $$X_{i, r}^{\pm} \mapsto x_i^{\pm} t^r$$
                $$
                    H_{i, r} \mapsto
                    \begin{cases}
                        \text{$h_i t^r$ if $i \in \simpleroots$}
                        \\
                        \text{$\theta^{\vee} t^r + t^r c_v$ if $i = \theta$}
                    \end{cases}
                $$
        \end{lemma}
        \begin{proposition}[Levendorskii presentation for $\toroidal^{\positive}$] \label{prop: levendorskii_presentation_for_positive_toroidal_lie_algebras}
            The Lie algebra $\toroidal^{\positive}$ is isomorphic to the Lie algebra generated by the set:
                $$\{ X_{i, r}^{\pm}, H_{i, r} \}_{(i, r) \in \hat{\simpleroots} \x \{0, 1\}}$$
            whose elements are subjected to the following relations, given for all $(i, r), (j, s) \in \hat{\simpleroots} \x \Z_{\geq 0}$:
                $$H_{i, 0} = h_i, X_{i, 0}^{\pm} = x_i^{\pm}$$
                $$[ H_{i, r}, H_{j, s} ] = 0$$
                $$[ H_{i, 0}, X_{j, s}^{\pm} ] = \pm d_{ij} X_{j, s}^{\pm}$$
                $$[ X_{i, r}^+, X_{j, s}^- ] = \delta_{ij} H_{i, r + s}$$
                $$[ X_{i, 1}^{\pm}, X_{j, 0}^{\pm} ] - [ X_{i, 0}^{\pm}, X_{j, 1}^{\pm} ] = 0$$
            and when $i \not = j$, there are also the following \say{Serre relations}:
                $$$$
            The isomorphism in question is as in lemma \ref{lemma: chevalley_serre_presentations_for_positive_toroidal_lie_algebras}.
        \end{proposition}
            \begin{proof}
                
            \end{proof}
    
        \begin{convention}[Formal Dirac distributions] \label{conv: formal_dirac_distributions}
            We will be using the following shorthands:
                $$\1(z, w) = \sum_{m \in \Z} z^m w^{-m - 1}$$
                $$\1^+(z, w) = \sum_{m \in \Z_{\geq 0}} z^m w^{-m - 1}$$
            as opposed to the usual $\delta$ notation, in order to avoid confusion with Lie cobrackets.
        \end{convention}

        \begin{remark}[Total degrees of \say{Yangian} canonical elements] \label{remark: total_degrees_of_classical_yangian_R_matrices}
            One property of the R-matrix $\sfr_{\extendedtoroidal^{\positive}}$ from corollary \ref{coro: extended_toroidal_lie_bialgebras} that will help simplify some computations later on (see the proof of theorem \ref{theorem: toroidal_lie_bialgebras}) is that they are of total degree $-1$. 

            Recall that if $V := \bigoplus_{m \in \Z} V_m, W := \bigoplus_{n \in \Z} W_n$ are $\Z$-graded vector spaces then for any $d \in \Z$, we have that:
                $$(V \tensor_{\bbC} W)_d \cong \bigoplus_{m + n = d} V_m \tensor_{\bbC} W_n$$
                
            If we now take $V = W = \rmU(\toroidal)$ then the claim from above would read:
                $$\sfr_{\toroidal^{\positive}} \in ( \rmU(\toroidal^{\positive}) \tensor_{\bbC} \rmU(\toroidal^{\negative}) )_{-1}$$
            with the $\Z$-grading on $\toroidal^{\positive/\negative}$ (and hence on $\rmU(\toroidal^{\positive/\negative})$) being the one on the second variable $t$ (cf. remark \ref{remark: Z_gradings_on_toroidal_lie_algebras}), and actually, this is entirely due to:
                $$\sfr_{\g_{[2]}^{\positive}} \in ( \rmU(\g_{[2]}^{\positive}) \tensor_{\bbC} \rmU(\g_{[2]}^{\negative}) )_{-1}$$

            What this means for us is that, should we have $X \in \toroidal^{\positive}$ such that:
                $$\deg X \leq 0$$
            then it will automatically be the case that:
                $$\delta_{\toroidal^{\positive}}(X) = 0$$
        \end{remark}
        
        We are now finally able to put a Lie cobracket on the toroidal Lie algebra $\toroidal^{\positive}$, compatible with the Lie bracket thereon in a manner that produces a Lie bialgebra structure. This Lie bialgebra structure is the classical limit of the coproduct on the formal Yangian:
            $$\rmY_{\hbar}(\hat{\g}) := \rmY_{\hbar}(\hat{\g})$$
        associated to the affine Kac-Moody algebra $\hat{\g}$. 
        \begin{theorem}[Toroidal Lie bialgebras] \label{theorem: toroidal_lie_bialgebras}
            Assume the conventions laid out in subsection \ref{subsection: a_fixed_untwisted_affine_kac_moody_algebra} and let us abbreviate:
                $$\hat{\delta}^{\positive} := \hat{\delta}^{\positive}$$
            with $\hat{\delta}^{\positive}$ as in corollary \ref{coro: extended_toroidal_lie_bialgebras}. Let:
                $$\tilde{\delta}^{\positive} := \hat{\delta}^{\positive}|_{\toroidal^{\positive}}$$
            Then $(\toroidal^{\positive}, \tilde{\delta}^{\positive})$ will be a complete topological Lie sub-bialgebra of $(\extendedtoroidal^{\positive}, \hat{\delta}^{\positive})$ as given in corollary \ref{coro: extended_toroidal_lie_bialgebras}. Thanks to proposition \ref{prop: levendorskii_presentation_for_positive_toroidal_lie_algebras}, we know that it is enough to specify how $\tilde{\delta}^{\positive}$ is given on the set of generators:
                $$\{X_{i, 0}^{\pm}\}_{i \in \hat{\simpleroots}} \cup \{H_{i, r}\}_{ (i, r) \in \hat{\simpleroots} \x \{0, 1\} }$$
            and since we know that under the isomorphism in \textit{loc. cit.}, we have the following assignments, given for all $i \in \hat{\simpleroots}$:
                $$X_{i, 0}^{\pm} \mapsto x_i^{\pm}, H_{i, 0} \mapsto h_i$$
                $$
                    H_{i, 1} \mapsto
                    \begin{cases}
                        \text{$h_i t$ if $i \in \simpleroots$}
                        \\
                        \text{$\theta^{\vee} t + t c_v$ if $i = \theta$}
                    \end{cases}
                $$
            it is enough to specify the following, wherein $h \in \h$ is arbitrary:
                $$\tilde{\delta}^{\positive}(h) = 0$$
                $$\tilde{\delta}^{\positive}(ht) = [h_1 \tensor 1, \sfr_{\g} v_2 \1(v_1, v_2)]$$
                $$\tilde{\delta}^{\positive}(t c_v) = 0$$
        \end{theorem}
            \begin{proof}
                \begin{enumerate}
                    \item Since $\deg x = 0$ for all $x \in \g$, we get via remark \ref{remark: total_degrees_of_classical_yangian_R_matrices} that:
                        $$\hat{\delta}^{\positive}(x) = 0$$
                    and in particular, we have that:
                        $$\hat{\delta}^{\positive}(h) = 0$$

                    \item Let us now compute $\hat{\delta}^{\positive}(ht)$ for an arbitrary $h \in \h$. 
                    \begin{enumerate}
                        \item \textbf{($\g_{[2]}^{\positive}$-component):} Firstly, to compute:
                            $$[\bar{\Delta}(ht), \sfr_{\g_{[2]}^{\positive}}]$$
                        let us firstly note that:
                            $$\sfr_{\g_{[2]}^{\positive}} = \sfr_{\g} v_2\1(v_1, v_2) \1^+(t_1, t_2)$$
                        Let us also choose a root basis for $\g$ for writing out $\sfr_{\g}$ explicitly: this is to say that for each positive root $\alpha \in \Phi^+$, we choose corresponding basis vectors $x_{\alpha}^{\pm} \in \g_{\pm \alpha}$ normalised so that:
                            $$(x_{\alpha}^-, x_{\alpha}^+)_{\g} = 1$$
                        to get the following basis for $\g$:
                            $$\{h_i\}_{i \in \simpleroots} \cup \{x_{\alpha}^-, x_{\alpha}^+\}_{\alpha \in \Phi^+}$$
                        From this, we see that:
                            $$
                                \begin{aligned}
                                    & [\bar{\Delta}(ht), \sfr_{\g_{[2]}^{\positive}}]
                                    \\
                                    = & 
                                    \begin{aligned}
                                        & -\sum_{i \in \simpleroots} [\bar{\Delta}(ht), h_i \tensor h_i v_2\1(v_1, v_2) \1^+(t_1, t_2)]
                                        \\
                                        - & \sum_{\alpha \in \Phi^+} [\bar{\Delta}(ht), (x_{\alpha}^- \tensor x_{\alpha}^+ + x_{\alpha}^+ \tensor x_{\alpha}^-) v_2\1(v_1, v_2) \1^+(t_1, t_2)]
                                    \end{aligned}
                                \end{aligned}
                            $$

                        Now, for each $i \in \simpleroots$, observe that:
                            $$
                                \begin{aligned}
                                    & [h t_1 \tensor 1, h_i \tensor h_i v_2\1(v_1, v_2) \1^+(t_1, t_2)]
                                    \\
                                    = & \sum_{(m, p) \in \Z \x \Z_{\geq 0}} [ht_1 \tensor 1, h_i v_1^m t_1^p \tensor h_i v_2^{-m} t_2^{-p - 1}]
                                    \\
                                    = & \sum_{(m, p) \in \Z \x \Z_{\geq 0}} [ht_1, h_i v_1^m t_1^p]_{\toroidal^{\positive}} \tensor h_i v_2^{-m} t_2^{-p - 1}
                                    \\
                                    = & \sum_{(m, p) \in \Z \x \Z_{\geq 0}} (h, h_i)_{\g} v_1^m t_1^p \bar{d}t_1 \tensor h_i v_2^{-m} t_2^{-p - 1}
                                \end{aligned}
                            $$
                        and likewise, that:
                            $$[1 \tensor h t_2, h_i \tensor h_i v_2\1(v_1, v_2) \1^+(t_1, t_2)] = \sum_{(m, p) \in \Z \x \Z_{\geq 0}} h_i v_1^m t_1^p \tensor (h, h_i)_{\g} v_2^{-m} t_2^{-p - 1} \bar{d}t_2$$
                        Adding the two summands together then yields:
                            $$
                                \begin{aligned}
                                    & [\bar{\Delta}(ht), h_i \tensor h_i v_2\1(v_1, v_2) \1^+(t_1, t_2)]
                                    \\
                                    = & (h, h_i)_{\g} \sum_{(m, p) \in \Z \x \Z_{\geq 0}} \left( v_1^m t_1^p \bar{d}t_1 \tensor h_i v_2^{-m} t_2^{-p - 1} + h_i v_1^m t_1^p \tensor v_2^{-m} t_2^{-p - 1} \bar{d}t_2 \right)
                                    \\
                                    = & (h, h_i)_{\g} ( \bar{d}t_1 \tensor h_i + h_i \tensor \bar{d}t_2 ) v_2\1(v_1, v_2) \1^+(t_1, t_2)
                                \end{aligned}
                            $$
                        
                        Next, consider the following:
                            $$
                                \begin{aligned}
                                    & [ht_1 \tensor 1, x_{\alpha}^- \tensor x_{\alpha}^+ v_2\1(v_1, v_2) \1^+(t_1, t_2)]
                                    \\
                                    = & \sum_{(m, p) \in \Z \x \Z_{\geq 0}} [ht_1 \tensor 1, x_{\alpha}^- v_1^m t_1^p \tensor x_{\alpha}^+ v_2^{-m} t_2^{-p - 1}]
                                    \\
                                    = & \sum_{(m, p) \in \Z \x \Z_{\geq 0}} [ht_1, x_{\alpha}^- v_1^m t_1^p]_{\toroidal^{\positive}} \tensor x_{\alpha}^+ v_2^{-m} t_2^{-p - 1}
                                    \\
                                    = & \sum_{(m, p) \in \Z \x \Z_{\geq 0}} \left( -\alpha(h) x_{\alpha}^- v_1^m t_1^{p + 1} + (h, x_{\alpha}^-)_{\g} t_1 \bar{d}(v_1^m t_1^p) \right) \tensor x_{\alpha}^+ v_2^{-m} t_2^{-p - 1}
                                    \\
                                    = & \sum_{(m, p) \in \Z \x \Z_{\geq 0}} -\alpha(h) x_{\alpha}^- v_1^m t_1^{p + 1} \tensor x_{\alpha}^+ v_2^{-m} t_2^{-p - 1}
                                    \\
                                    & = -\alpha(h) ( x_{\alpha}^- \tensor x_{\alpha}^+ ) v_2 \1(v_1, v_2) t_1 \1^+(t_1, t_2)
                                \end{aligned}    
                            $$
                        wherein the second-to-last identity comes from the fact that\footnote{This can be proven easily by passing to the vector representation of $\g$, wherein $h$ is represented by a diagonal matrix while $x^{\pm}$ is represented by an upper/lower triangular matrix, and then using the fact that $(-, -)_{\g}$ differs from the trace form only by a non-zero constant.}:
                            $$(h, x^{\pm})_{\g} = 0$$
                        for every $h \in \h$ and every $x^{\pm} \in \n^{\pm}$. Similarly, we find that:
                            $$[ht_1 \tensor 1, x_{\alpha}^+ \tensor x_{\alpha}^- v_2\1(v_1, v_2) \1^+(t_1, t_2)] = \alpha(h) ( x_{\alpha}^+ \tensor x_{\alpha}^- ) v_2 \1(v_1, v_2) t_1 \1^+(t_1, t_2)$$
                        By putting the two together, one obtains:
                            $$[h t_1 \tensor 1, (x_{\alpha}^- \tensor x_{\alpha}^+ + x_{\alpha}^+ \tensor x_{\alpha}^-) v_2\1(v_1, v_2) \1^+(t_1, t_2)] = -\alpha(h) ( x_{\alpha}^- \tensor x_{\alpha}^+ - x_{\alpha}^+ \tensor x_{\alpha}^- ) v_2 \1(v_1, v_2) t_1 \1^+(t_1, t_2)$$
                        Likewise, we find that:
                            $$[1 \tensor h t_2, (x_{\alpha}^- \tensor x_{\alpha}^+ + x_{\alpha}^+ \tensor x_{\alpha}^-) v_2\1(v_1, v_2) \1^+(t_1, t_2)] = \alpha(h) ( x_{\alpha}^- \tensor x_{\alpha}^+ - x_{\alpha}^+ \tensor x_{\alpha}^- ) v_2 \1(v_1, v_2) t_2 \1^+(t_1, t_2)$$
                        and hence:
                            $$
                                \begin{aligned}
                                    & [\bar{\Delta}(ht), \sfr_{\g_{[2]}^{\positive}}]
                                    \\
                                    = &
                                    -\left(
                                    \begin{aligned}
                                        & \sum_{i \in \simpleroots} (h, h_i)_{\g} ( \bar{d}t_1 \tensor h_i + h_i \tensor \bar{d}t_2 )
                                        \\
                                        + & \sum_{\alpha \in \Phi^+} \alpha(h) ( x_{\alpha}^- \tensor x_{\alpha}^+ - x_{\alpha}^+ \tensor x_{\alpha}^- )(t_2 - t_1)
                                    \end{aligned}
                                    \right) v_2 \1(v_1, v_2) \1^+(t_1, t_2)
                                    \\
                                    & = -\left( \bar{d}t_1 \tensor h + h \tensor \bar{d}t_2 + [h_1 \tensor 1, \sfr_{\g}] (t_2 - t_1) \right) v_2 \1(v_1, v_2) \1^+(t_1, t_2)
                                    \\
                                    & = -\left( \bar{d}t_1 \tensor h + h \tensor \bar{d}t_2 \right) v_2 \1(v_1, v_2) \1^+(t_1, t_2) + [h_1 \tensor 1, \sfr_{\g}] v_2 \1(v_1, v_2)
                                \end{aligned}
                            $$
                        We note that the last equality holds thanks to the fact that:
                            $$(t_2 - t_1) \1^+(t_1, t_2) = (t_2 - t_1) \sum_{p \in \Z_{\geq 0}} t_1^p t_2^{-p - 1} = (t_2 - t_1) \frac{1}{t_2 - t_1} = 1$$
                            
                        \item \textbf{($\z_{[2]}^{\positive}$-component):} Recall from corollary \ref{coro: extended_toroidal_lie_bialgebras} that:
                            $$\sfr_{\z_{[2]}^{\positive}} := \sum_{(r, s) \in \Z \x \Z_{> 0}} K_{r, s} \tensor D_{r, s} + c_{v_1} \tensor D_{v_2}$$
                        and so:
                            $$
                                \begin{aligned}
                                    & [\bar{\Delta}(ht), \sfr_{\z_{[2]}^{\positive}}]
                                    \\
                                    = & \sum_{(r, s) \in \Z \x \Z_{> 0}} [\bar{\Delta}(ht), K_{r, s} \tensor D_{r, s}] + [\bar{\Delta}(ht), c_{v_1} \tensor D_{v_2}]
                                    \\
                                    = & -\sum_{(r, s) \in \Z \x \Z_{> 0}} K_{r, s} \tensor h D_{r, s}(t) - c_{v_1} \tensor h D_{v_2}(t_2)
                                    \\
                                    = & -\sum_{(r, s) \in \Z \x \Z_{> 0}} K_{r, s} \tensor r h v_2^{-r} t_2^{-s}
                                \end{aligned}
                            $$
                        where the minus sign in the third equation appeared because:
                            $$[ht, D_{r, s}] = -[D_{r, s}, ht] = -h D_{r, s}(t)$$
                            $$[ht, D_v] = -[D_v, ht] = -h D_v(t)$$
                        and the last equality is due to the fact that:
                            $$D_{r, s} = -s v^{-r + 1} t^{-s - 1} \del_v + r v^{-r} t^{-s} \del_t$$
                            $$D_v = -v t^{-1} \del_v$$
                        (for both, see lemma \ref{lemma: derivation_action_on_multiloop_algebras}). 
                        
                        \item \textbf{($\d_{[2]}^{\positive}$-component):} Recall from corollary \ref{coro: extended_toroidal_lie_bialgebras} that:
                            $$\sfr_{\z_{[2]}^{\positive}} := \sum_{(r, s) \in \Z \x \Z_{\leq 0}} D_{r, s} \tensor K_{r, s} + D_{t_1} \tensor c_{t_2}$$
                        and so:
                            $$
                                \begin{aligned}
                                    & [\bar{\Delta}(ht), \sfr_{\z_{[2]}^{\positive}}]
                                    \\
                                    = & \sum_{(r, s) \in \Z \x \Z_{\leq 0}} [\bar{\Delta}(ht), D_{r, s} \tensor K_{r, s}] + [\bar{\Delta}(ht), D_{t_1} \tensor c_{t_2}]
                                    \\
                                    = & -\sum_{(r, s) \in \Z \x \Z_{\leq 0}} h D_{r, s}(t_1) \tensor K_{r, s} - h D_{t_1}(t_1) \tensor c_{t_2}
                                    \\
                                    = & -\sum_{(r, s) \in \Z \x \Z_{\leq 0}} r h v_1^{-r} t_1^{-s} \tensor K_{r, s} + h \tensor c_{t_2}
                                \end{aligned}
                            $$
                        where the minus sign in the third equation appeared because:
                            $$[ht, D_{r, s}] = -[D_{r, s}, ht] = -h D_{r, s}(t)$$
                            $$[ht, D_t] = -[D_t, ht] = -h D_t(t)$$
                        and the the last equality is due to the fact that:
                            $$D_{r, s} = -s v^{-r + 1} t^{-s - 1} \del_v + r v^{-r} t^{-s} \del_t$$
                            $$D_t = -\del_t$$
                        (for both, see lemma \ref{lemma: derivation_action_on_multiloop_algebras}).
                    \end{enumerate}

                    Since we know that:
                        $$[\bar{\Delta}(ht), \sfr_{\g_{[2]}^{\positive}}] = -\left( \bar{d}t_1 \tensor h + h \tensor \bar{d}t_2 \right) v_2 \1(v_1, v_2) \1^+(t_1, t_2) + [h_1 \tensor 1, \sfr_{\g}] v_2 \1(v_1, v_2)$$
                    we now claim that:
                        $$[\bar{\Delta}(ht), \sfr_{\z_{[2]}^{\positive}} + \sfr_{\d_{[2]}^{\positive}}] = \left( \bar{d}t_1 \tensor h + h \tensor \bar{d}t_2 \right) v_2 \1(v_1, v_2) \1^+(t_1, t_2)$$
                    (since ultimately, we would like to show that $\hat{\delta}^{\positive}(ht) = \sfr_{\g} v_2 \1(v_1, v_2)$), and to prove that this is the case, let us first note that we now have that:
                        $$
                            \begin{aligned}
                                & [\bar{\Delta}(ht), \sfr_{\z_{[2]}^{\positive}} + \sfr_{\d_{[2]}^{\positive}}]
                                \\
                                = & -\sum_{(r, s) \in \Z \x \Z_{> 0}} \left( K_{r, s} \tensor r h v_2^{-r} t_2^{-s} + r h v_1^{-r} t_1^s \tensor K_{r, -s} \right) - \sum_{r \in \Z} r h v_1^{-r} \tensor K_{r, 0} + h \tensor c_{t_2}
                            \end{aligned}
                        $$
                    wherein the first summand corresponds to the indices $(r, 0) \in \Z \x \Z_{\leq 0}$. From example \ref{example: toroidal_lie_algebras_centres}, we know that:
                        $$
                            K_{r, s} :=
                            \begin{cases}
                                \text{$\frac1s v^{r - 1} t^s \bar{d}v$ if $(r, s) \in \Z \x (\Z \setminus \{0\})$}
                                \\
                                \text{$-\frac1r v^r t^{-1} \bar{d}(t)$ if $(r, s) \in (\Z \setminus \{0\}) \x \{0\}$}
                                \\
                                \text{$0$ if $(r, s) = (0, 0)$}
                            \end{cases}
                        $$
                    from which one infers that:
                        $$
                            \begin{aligned}
                                & -\sum_{r \in \Z} r h v_1^{-r} \tensor K_{r, 0}
                                \\
                                = & -\sum_{r \in \Z} r h v_1^{-r} \tensor \left( -\frac1r v_2^r t_2^{-1} \bar{d}t_2 \right)
                                \\
                                = & \sum_{r \in \Z} h v_1^{-r} \tensor v_2^r t_2^{-1} \bar{d}t_2
                                \\
                                = & \sum_{r \in \Z} h v_1^{-r} \tensor v_2^r t_2^{-1} \bar{d}t_2
                            \end{aligned}
                        $$
                        
                    Next, recall again from example \ref{example: toroidal_lie_algebras_centres} that:
                        $$(r, s) \in \Z^2 \implies K_{r, s} = \frac1s v^{r - 1} t^s \bar{d}v = -\frac1r v^r t^{s - 1} \bar{d}(t) \in \bar{\Omega}_{[2]}$$
                    and then consider the following:
                        $$
                            \begin{aligned}
                                & -\sum_{(r, s) \in \Z \x \Z_{> 0}} \left( K_{r, s} \tensor r h v_2^{-r} t_2^{-s} + r h v_1^{-r} t_1^s \tensor K_{r, -s} \right)
                                \\
                                = & \sum_{(r, s) \in \Z \x \Z_{> 0}} \left( v_1^r t_1^{s - 1} \bar{d}t_1 \tensor h v_2^{-r} t_2^{-s} - h v_1^{-r} t_1^s \tensor v_2^r t_2^{-s - 1} \bar{d}t_2 \right)
                            \end{aligned}
                        $$
                    wherein we note that for all $s \in \Z_{> 0}$, the summands corresponding to the indices $(0, s)$ vanish.

                    We now have that:
                        $$
                            \begin{aligned}
                                & [\bar{\Delta}(ht), \sfr_{\z_{[2]}^{\positive}} + \sfr_{\d_{[2]}^{\positive}}]
                                \\
                                = & \sum_{(r, s) \in \Z \x \Z_{> 0}} \left( K_{r, s} \tensor r h v_2^{-r} t_2^{-s} + r h v_1^{-r} t_1^s \tensor K_{r, -s} \right) - \sum_{r \in \Z} r h v_1^{-r} \tensor K_{r, 0} + h \tensor c_{t_2}
                                \\
                                = & \sum_{(r, s) \in \Z \x \Z_{> 0}} \left( v_1^r t_1^{s - 1} \bar{d}t_1 \tensor h v_2^{-r} t_2^{-s} - h v_1^{-r} t_1^s \tensor v_2^r t_2^{-s - 1} \bar{d}t_2 \right) + \sum_{r \in \Z} h v_1^{-r} \tensor v_2^r t_2^{-1} \bar{d}t_2 + h \tensor t_2^{-1} \bar{d}t_2
                                \\
                                = & \sum_{(r, s) \in \Z \x \Z_{> 0}} \left( v_1^r t_1^{s - 1} \bar{d}t_1 \tensor h v_2^{-r} t_2^{-s} + h v_1^r t_1^s \tensor v_2^{-r} t_2^{-s - 1} \bar{d}t_2 \right) + \sum_{r \in \Z} h v_1^{-r} \tensor v_2^r t_2^{-1} \bar{d}t_2
                                \\
                                = & \sum_{(r, s) \in \Z \x \Z_{> 0}} \left( v_1^r t_1^{s - 1} \bar{d}t_1 \tensor h v_2^{-r} t_2^{-s} + h v_1^{-r} t_1^s \tensor v_2^r t_2^{-s - 1} \bar{d}t_2 \right) + \sum_{r \in \Z} h v_1^r \tensor v_2^{-r} t_2^{-1} \bar{d}t_2
                                \\
                                = & ( \bar{d}t_1 \tensor h ) \sum_{(r, s) \in \Z \x \Z_{> 0}} v_1^r t_1^{s - 1} \tensor v_2^{-r} t_2^{-s} + ( h \tensor \bar{d}t_2 ) \left( \sum_{(r, s) \in \Z \x \Z_{> 0}} v_1^r t_1^s \tensor v_2^{-r} t_2^{-s - 1} + \sum_{r \in \Z} v_1^r \tensor v_2^{-r} t_2^{-1} \right)
                                \\
                                = & ( \bar{d}t_1 \tensor h ) \sum_{(r, s) \in \Z \x \Z_{\geq 0}} v_1^r t_1^s \tensor v_2^{-r} t_2^{-s - 1} + ( h \tensor \bar{d}t_2 ) \left( \sum_{(r, s) \in \Z \x \Z_{> 0}} v_1^r t_1^s \tensor v_2^{-r} t_2^{-s - 1} + \sum_{r \in \Z} v_1^r \tensor v_2^{-r} t_2^{-1} \right)
                                \\
                                = & ( \bar{d}t_1 \tensor h + h \tensor \bar{d}t_2 ) \sum_{(r, s) \in \Z \x \Z_{\geq 0}} v_1^r t_1^s \tensor v_2^{-r} t_2^{-s - 1}
                                \\
                                = & ( \bar{d}t_1 \tensor h + h \tensor \bar{d}t_2 ) v_2 \1(v_1, v_2) \1^+(t_1, t_2)
                            \end{aligned}
                        $$

                    We can now add the three components together to yield:
                        $$[\bar{\Delta}(ht), \sfr_{\extendedtoroidal^{\positive}}] = [ \bar{\Delta}(ht), \sfr_{\g_{[2]}^{\positive}} + (\sfr_{\z_{[2]}^{\positive}} + \sfr_{\d_{[2]}^{\positive}}) ] =  [h_1 \tensor 1] v_2 \1(v_1, v_2)$$
                    precisely as claimed. 
                    
                    \item Finally, in order to compute $\hat{\delta}^{\positive}(t c_v)$, let us first notice that:
                        $$t c_v = K_{0, 1}$$
                    Since $[K_{0, 1}, \g_{[2]}] = 0$, we have that:
                        $$[\bar{\Delta}(K_{0, 1}), \sfr_{\extendedtoroidal^{\positive}}] = [\bar{\Delta}(K_{0, 1}), \sfr_{\z_{[2]}^{\positive}} + \sfr_{\d_{[2]}^{\positive}}] = [1 \tensor K_{0, 1}, \sfr_{\z_{[2]}^{\positive}}] + [K_{0, 1} \tensor 1, \sfr_{\d_{[2]}^{\positive}}]$$
                    From lemma \ref{lemma: explicit_commutators_between_central_basis_elements_and_derivations}, we know that:
                        $$
                            \forall (a, b) \in \Z^2: [D, K_{a, b}] =
                            \begin{cases}
                                \text{$((b - 1)r - sa) D_{a - r, b - s - 1}$ if $D = D_{r, s}$}
                                \\
                                \text{$-r K_{a, b - 1}$ if $D = D_v$}
                                \\
                                \text{$- D_{a, b - 1}$ if $D = D_t$}
                            \end{cases}
                        $$
                        $$[\d_{[2]}, c_v]_{\extendedtoroidal} = [\d_{[2]}, c_t]_{\extendedtoroidal} = 0 = 0$$
                    and so:
                        $$[1 \tensor K_{0, 1}, \sfr_{\z_{[2]}^{\positive}}] = \sum_{(r, s) \in \Z \x \Z_{> 0}} K_{r, s} \tensor [K_{0, 1}, D_{r, s}] = sum_{(r, s) \in \Z \x \Z_{> 0}} K_{r, s} \tensor 0 = 0$$
                        $$[K_{0, 1} \tensor 1, \sfr_{\d_{[2]}^{\positive}}] = \sum_{(r, s) \in \Z \x \Z_{\leq 0}} [K_{0, 1}, D_{r, s}] \tensor K_{r, s} = \sum_{(r, s) \in \Z \x \Z_{\leq 0}} 0 \tensor K_{r, s} = 0$$
                    We are thus able to conclude that:
                        $$\tilde{\delta}^{\positive}(t c_v) = 0$$
                    as claimed.
                \end{enumerate}
            \end{proof}
    
    \subsection{Classical limits of affine Yangians}
        In this subsection, we seek to establish the identity:
            $$\frac{1}{\hbar}(\Delta_{\hbar} - \Delta_{\hbar}^{\cop}) \pmod{\hbar} \equiv \tilde{\delta}^{\positive}$$
        This allows us to realise the topological Lie bialgebra $(\toroidal^{\positive}, \tilde{\delta}^{\positive})$ from theorem \ref{theorem: toroidal_lie_bialgebras} as the classical limit of the formal Yangian $\rmY_{\hbar}(\hat{\g})$ in some sense (which, let us caution, is not exactly the same as in \cite{etingof_kazhdan_quantisation_1}).
    
        \begin{theorem}[Toroidal Lie algebras as classical limits of formal affine Yangians] \label{theorem: toroidal_lie_algebras_as_classical_limits_of_formal_affine_yangians}
           The topological Lie bialgebra $(\toroidal^{\positive}, \tilde{\delta}^{\positive})$ from theorem \ref{theorem: toroidal_topological_lie_bialgebras} (see also theorem \ref{theorem: toroidal_lie_bialgebras}) is the classical limit of the formal affine Yangian $\rmY_{\hbar}(\hat{\g})$ with the \say{coproduct} $\Delta_{\hbar}$ (as in theorem \ref{theorem: hopf_coproduct_on_formal_yangians}), in the sense that:
                $$\frac{1}{\hbar}( \Delta_{\hbar} - \Delta_{\hbar}^{\cop} ) \equiv \tilde{\delta}^{\positive} \pmod{\hbar}$$
        \end{theorem}
            \begin{proof}
                Before we begin computing, let us make the preliminary observation that $T_{i, 1}(\hbar) \in \rmY_{\hbar}(\hat{\g})$ is a lift modulo $\hbar$ of $H_{i, 1} \in \toroidal^{\positive}$:
                    $$T_{i, 1}(\hbar) := H_{i, 1} - \frac12 \hbar H_{i, 0}^2 \equiv H_{i, 1} \pmod{\hbar}$$
                Also, let us note that, we know by lemma \ref{lemma: levendorskii_presentation} that it is enough to only check the value of $\Delta_{\hbar}^{\cop}$ on the generators $H_{i, 0}, X_{i, 0}^{\pm}$, and $T_{i, 1} \equiv H_{i, 1} \pmod{\hbar}$, for all $i \in \hat{\simpleroots}$.
            
                Firstly, from theorem \ref{theorem: hopf_coproduct_on_formal_yangians}, we know that:
                    $$\forall h \in \hat{\h}: \Delta_{\hbar}(h) := \bar{\Delta}(h)$$
                    $$\forall i \in \hat{\simpleroots}: \Delta_{\hbar}(X_{i, 0}^{\pm}) := \bar{\Delta}(X_{i, 0}^{\pm})$$
                    $$\forall i \in \hat{\simpleroots}: \Delta_{\hbar}(T_{i, 0}) = \bar{\Delta}(T_{i, 0}) + [H_{i, 0} \tensor 1, \sfr_{ \hat{\g} }^-]$$
                This tells us that:
                    $$\forall h \in \hat{\h}: \Delta_{\hbar}^{\cop}(h) := \bar{\Delta}(h)$$
                    $$\forall i \in \hat{\simpleroots}: \Delta_{\hbar}^{\cop}(X_{i, 0}^{\pm}) := \bar{\Delta}(X_{i, 0}^{\pm})$$
                    $$\forall i \in \hat{\simpleroots}: \Delta_{\hbar}^{\cop}(T_{i, 0}) = \bar{\Delta}(T_{i, 0}) + [1 \tensor H_{i, 0}, \sfr_{ \hat{\g} }^+]$$
                with $\sfr_{\hat{\g}}^-$ being the Casimir tensor associated to the non-degenerate Kac-Moody pairing on $\hat{\n}^- \hattensor_k \hat{\n}^+$.

                It is then trivial that:
                    $$\frac{1}{\hbar}( \Delta_{\hbar} - \Delta_{\hbar}^{\cop} )(X) = 0$$
                for:
                    $$X \in \hat{\h} \cup \{X_{i, 0}^{\pm}\}_{i \in \hat{\simpleroots}}$$
                which implies that:
                    $$\frac{1}{\hbar}( \Delta_{\hbar} - \Delta_{\hbar}^{\cop} )(X) \equiv \tilde{\delta}^{\positive}(X) \pmod{\hbar}$$
                which is because it is known from theorem \ref{theorem: toroidal_lie_bialgebras} that:
                    $$\tilde{\delta}^{\positive}(X) = 0$$
                whenever $\deg X = 0$, which is the case here.
                
                Now, let us verify that:
                    $$\frac{1}{\hbar}(\Delta_{\hbar} - \Delta_{\hbar}^{\cop})(T_{i, 1}) \equiv \tilde{\delta}^{\positive}(H_{i, 1})$$
                It is not hard to see that\footnote{One can prove this by e.g. picking the root bases foor $\hat{\n}^{\pm}$.}:
                    $$[1 \tensor H_{i, 0}, \sfr_{ \hat{\g} }^+] = -[H_{i, 0} \tensor 1, \sfr_{ \hat{\g} }^+]$$
                which tells us that:
                    $$\frac{1}{\hbar}( \Delta_{\hbar} - \Delta_{\hbar}^{\cop} )(T_{i, 1}) = [H_{i, 0} \tensor 1, \sfr_{ \hat{\g} }^- + \sfr_{ \hat{\g} }^+]$$
                Since $H_{i, 0}$ commutes with every element of $\hat{\h}$, we can equivalently rewrite the above into:
                    $$\frac{1}{\hbar}( \Delta_{\hbar} - \Delta_{\hbar}^{\cop} )(T_{i, 1}) = [H_{i, 0} \tensor 1, \sfr_{\hat{\h}} + \sfr_{ \hat{\g} }^- + \sfr_{ \hat{\g} }^+] = [H_{i, 0} \tensor 1, \sfr_{ \hat{\g} }]$$
                wherein $\sfr_{\hat{\h}}$ is the Casimir element associated to the Kac-Moody pairing on $\hat{\h} \tensor_{\bbC} \hat{\h}$. 

                We know from theorem \ref{theorem: toroidal_lie_bialgebras} that:
                    $$\tilde{\delta}^{\positive}(H_{i, 1}) = [ H_{i, 0} \tensor 1, \sfr_{\g} v_2 \1(v_1, v_2) ]$$
                so we will be done if we can show that:
                    $$[H_{i, 0} \tensor 1, \sfr_{ \hat{\g} }] = [ H_{i, 0} \tensor 1, \sfr_{\g} v_2 \1(v_1, v_2) ]$$
                From the fact that:
                    $$\hat{\g} \cong \g[v^{\pm 1}] \oplus \bbC c_v \oplus \bbC D_{0, -1}$$
                we infer that:
                    $$\sfr_{ \hat{\g} } = \sfr_{\g} v_2 \1(v_1, v_2) + \sfr_{\z} + \sfr_{\d} = \sfr_{\g} v_2 \1(v_1, v_2) + K_{0, -1} \tensor D_{0, -1} + D_{0, -1} \tensor K_{0, -1}$$
                wherein $\sfr_{\z}, \sfr_{\d}$ respectively denote the Casimir elements corresponding to the Kac-Moody form on $\z \tensor_{\bbC} \d$ and on $\d \tensor_{\bbC} \z$ respectively. The element $K_{0, -1} \in \hat{\g} := \g \oplus \z$ is central and therefore commutes with $H_{i, 0}$, which implies that:
                    $$[H_{i, 0} \tensor 1, \sfr_{\z}] = [H_{i, 0} \tensor 1, K_{0, -1} \tensor D_{0, -1}] = 0$$
                At the same time, we also know that $D_{0, -1}$ acts as $\id_{\g} \tensor \left(-v \frac{d}{dv}\right)$ on $\g$ and hence as zero on the elements of $\g$ (i.e. degree-$0$ elements of $\g$), and so:
                    $$[H_{i, 0} \tensor 1, \sfr_{\d}] = [H_{i, 0} \tensor 1, D_{0, -1} \tensor K_{0, -1}] = 0$$
                as well. As such, we have demonstrated that:
                    $$[H_{i, 0} \tensor 1, \sfr_{ \hat{\g} }] = [ H_{i, 0} \tensor 1, \sfr_{\g} v_2 \1(v_1, v_2) ]$$
                as we sought to. As mentioned above, this allows us to conclude that:
                    $$\frac{1}{\hbar}( \Delta_{\hbar} - \Delta_{\hbar}^{\cop} )(T_{i, 1}) \equiv \tilde{\delta}^{\positive}(H_{i, 1}) \pmod{\hbar}$$
            \end{proof}

        \todo[inline]{Not written: grading-completions}

    \newpage

    \printbibliography
    \addcontentsline{toc}{chapter}{\textbf{Bibliography}}

\end{document}