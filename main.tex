\documentclass[a4paper, 12pt]{book}

%\usepackage[center]{titlesec}

\usepackage{amsfonts, amssymb, amsmath, amsthm, amsxtra}

\usepackage{foekfont}

\usepackage{MnSymbol}

\usepackage{pdfrender, xcolor}
%\pdfrender{StrokeColor=black,LineWidth=.4pt,TextRenderingMode=2}

\usepackage{minitoc}
\setcounter{tocdepth}{4}
\setcounter{minitocdepth}{4}
\setcounter{secnumdepth}{4}

\usepackage{graphicx}

\usepackage[english]{babel}
\usepackage[utf8]{inputenc}
%\usepackage{mathpazo}
%\usepackage{euler}
\usepackage{eucal}
\usepackage{bbm}
\usepackage{bm}
\usepackage{csquotes}
\usepackage[nottoc]{tocbibind}
\usepackage{appendix}
\usepackage{float}
\usepackage[T1]{fontenc}
\usepackage[
    left = \flqq{},% 
    right = \frqq{},% 
    leftsub = \flq{},% 
    rightsub = \frq{} %
]{dirtytalk}

\usepackage{imakeidx}
\makeindex

%\usepackage[dvipsnames]{xcolor}
\usepackage{hyperref}
    \hypersetup{
        colorlinks=true,
        linkcolor=teal,
        filecolor=pink,      
        urlcolor=teal,
        citecolor=magenta
    }
\usepackage{comment}

% You would set the PDF title, author, etc. with package options or
% \hypersetup.

\usepackage[backend=biber, style=alphabetic, sorting=nty]{biblatex}
    \addbibresource{bibliography.bib}

\raggedbottom

\usepackage{mathrsfs}
\usepackage{mathtools} 
\mathtoolsset{showonlyrefs}
%\usepackage{amsthm}
\renewcommand\qedsymbol{$\blacksquare$}
\usepackage{tikz-cd}
\tikzcdset{scale cd/.style={every label/.append style={scale=#1},
    cells={nodes={scale=#1}}}}
\usepackage{tikz}
\usepackage{setspace}
\usepackage[version=3]{mhchem}
\parskip=0.1in
\usepackage[margin=25mm]{geometry}

\usepackage{listings, lstautogobble}
\lstset{
	language=matlab,
	basicstyle=\scriptsize\ttfamily,
	commentstyle=\ttfamily\itshape\color{gray},
	stringstyle=\ttfamily,
	showstringspaces=false,
	breaklines=true,
	frameround=ffff,
	frame=single,
	rulecolor=\color{black},
	autogobble=true
}

\usepackage{todonotes,tocloft,xpatch,hyperref}

% This is based on classicthesis chapter definition
\let\oldsec=\section
\renewcommand*{\section}{\secdef{\Sec}{\SecS}}
\newcommand\SecS[1]{\oldsec*{#1}}%
\newcommand\Sec[2][]{\oldsec[\texorpdfstring{#1}{#1}]{#2}}%

\newcounter{istodo}[section]

% http://tex.stackexchange.com/a/61267/11984
\makeatletter
%\xapptocmd{\Sec}{\addtocontents{tdo}{\protect\todoline{\thesection}{#1}{}}}{}{}
\newcommand{\todoline}[1]{\@ifnextchar\Endoftdo{}{\@todoline{#1}}}
\newcommand{\@todoline}[3]{%
	\@ifnextchar\todoline{}
	{\contentsline{section}{\numberline{#1}#2}{#3}{}{}}%
}
\let\l@todo\l@subsection
\newcommand{\Endoftdo}{}

\AtEndDocument{\addtocontents{tdo}{\string\Endoftdo}}
\makeatother

\usepackage{lipsum}

%   Reduce the margin of the summary:
\def\changemargin#1#2{\list{}{\rightmargin#2\leftmargin#1}\item[]}
\let\endchangemargin=\endlist 

%   Generate the environment for the abstract:
\newcommand\summaryname{Abstract}
\newenvironment{abstract}%
    {\small\begin{center}%
    \bfseries{\summaryname} \end{center}}

\newtheorem{theorem}{Theorem}[section]
    \numberwithin{theorem}{subsection}
\newtheorem{proposition}{Proposition}[section]
    \numberwithin{proposition}{subsection}
\newtheorem{lemma}{Lemma}[section]
    \numberwithin{lemma}{subsection}
\newtheorem{claim}{Claim}[section]
    \numberwithin{claim}{subsection}
\newtheorem{question}{Question}[section]
    \numberwithin{question}{subsection}

\theoremstyle{definition}
    \newtheorem{definition}{Definition}[section]
        \numberwithin{definition}{subsection}

\theoremstyle{remark}
    \newtheorem{remark}{Remark}[section]
        \numberwithin{remark}{subsection}
    \newtheorem{example}{Example}[section]
        \numberwithin{example}{subsection}    
    \newtheorem{convention}{Convention}[section]
        \numberwithin{convention}{subsection}
    \newtheorem{corollary}{Corollary}[section]
        \numberwithin{corollary}{subsection}

\numberwithin{equation}{chapter}

\usepackage{fancyhdr} % Custom headers and footers
\pagestyle{fancyplain} % Makes all pages in the document conform to the custom headers and footers
\fancyhead[L]{}% Empty left header
\fancyhead[C]{} %SECTION TITLE
\fancyhead[R]{}% Empty right header
\fancyfoot[L]{}% Empty left footer
\fancyfoot[C]{\thepage}% PAGE NUMBERING
\fancyfoot[R]{}% Empty left footer

\setcounter{chapter}{0}
\setcounter{section}{0}

\renewcommand{\implies}{\Rightarrow}
\renewcommand{\cong}{\simeq}
\newcommand{\ladjoint}{\dashv}
\newcommand{\radjoint}{\vdash}
\newcommand{\<}{\langle}
\renewcommand{\>}{\rangle}
\newcommand{\ndiv}{\hspace{-2pt}\not|\hspace{5pt}}
\newcommand{\cond}{\blacktriangle}
\newcommand{\decond}{\triangle}
\newcommand{\solid}{\blacksquare}
\newcommand{\ot}{\leftarrow}
\renewcommand{\-}{\text{-}}
\renewcommand{\mapsto}{\leadsto}
\renewcommand{\leq}{\leqslant}
\renewcommand{\geq}{\geqslant}
\renewcommand{\setminus}{\smallsetminus}
\newcommand{\punc}{\overset{\circ}}
\renewcommand{\div}{\operatorname{div}}
\newcommand{\grad}{\operatorname{grad}}
\newcommand{\curl}{\operatorname{curl}}
\makeatletter
\DeclareRobustCommand{\cev}[1]{%
  {\mathpalette\do@cev{#1}}%
}
\newcommand{\do@cev}[2]{%
  \vbox{\offinterlineskip
    \sbox\z@{$\m@th#1 x$}%
    \ialign{##\cr
      \hidewidth\reflectbox{$\m@th#1\vec{}\mkern4mu$}\hidewidth\cr
      \noalign{\kern-\ht\z@}
      $\m@th#1#2$\cr
    }%
  }%
}
\makeatother

\newcommand{\N}{\mathbb{N}}
\newcommand{\Z}{\mathbb{Z}}
\newcommand{\Q}{\mathbb{Q}}
\newcommand{\R}{\mathbb{R}}
\newcommand{\bbC}{\mathbb{C}}
\NewDocumentCommand{\x}{e{_^}}{%
  \mathbin{\mathop{\times}\displaylimits
    \IfValueT{#1}{_{#1}}
    \IfValueT{#2}{^{#2}}
  }%
}
\NewDocumentCommand{\pushout}{e{_^}}{%
  \mathbin{\mathop{\sqcup}\displaylimits
    \IfValueT{#1}{_{#1}}
    \IfValueT{#2}{^{#2}}
  }%
}
\newcommand{\supp}{\operatorname{supp}}
\newcommand{\im}{\operatorname{im}}
\newcommand{\coim}{\operatorname{coim}}
\newcommand{\coker}{\operatorname{coker}}
\newcommand{\id}{\mathrm{id}}
\newcommand{\chara}{\operatorname{char}}
\newcommand{\trdeg}{\operatorname{trdeg}}
\newcommand{\rank}{\operatorname{rank}}
\newcommand{\trace}{\operatorname{tr}}
\newcommand{\length}{\operatorname{length}}
\newcommand{\height}{\operatorname{ht}}
\renewcommand{\span}{\operatorname{span}}
\newcommand{\e}{\epsilon}
\newcommand{\p}{\mathfrak{p}}
\newcommand{\q}{\mathfrak{q}}
\newcommand{\m}{\mathfrak{m}}
\newcommand{\n}{\mathfrak{n}}
\newcommand{\calF}{\mathcal{F}}
\newcommand{\calG}{\mathcal{G}}
\newcommand{\calO}{\mathcal{O}}
\newcommand{\F}{\mathbb{F}}
\DeclareMathOperator{\lcm}{lcm}
\newcommand{\gr}{\operatorname{gr}}
\newcommand{\vol}{\mathrm{vol}}
\newcommand{\ord}{\operatorname{ord}}
\newcommand{\projdim}{\operatorname{proj.dim}}
\newcommand{\injdim}{\operatorname{inj.dim}}
\newcommand{\flatdim}{\operatorname{flat.dim}}
\newcommand{\globdim}{\operatorname{glob.dim}}
\renewcommand{\Re}{\operatorname{Re}}
\renewcommand{\Im}{\operatorname{Im}}
\newcommand{\sgn}{\operatorname{sgn}}
\newcommand{\coad}{\operatorname{coad}}
\newcommand{\ch}{\operatorname{ch}} %characters of representations

\newcommand{\Ad}{\mathrm{Ad}}
\newcommand{\GL}{\mathrm{GL}}
\newcommand{\SL}{\mathrm{SL}}
\newcommand{\PGL}{\mathrm{PGL}}
\newcommand{\PSL}{\mathrm{PSL}}
\newcommand{\Sp}{\mathrm{Sp}}
\newcommand{\GSp}{\mathrm{GSp}}
\newcommand{\GSpin}{\mathrm{GSpin}}
\newcommand{\rmO}{\mathrm{O}}
\newcommand{\SO}{\mathrm{SO}}
\newcommand{\SU}{\mathrm{SU}}
\newcommand{\rmU}{\mathrm{U}}
\newcommand{\rmY}{\mathrm{Y}}
\newcommand{\rmu}{\mathrm{u}}
\newcommand{\rmV}{\mathrm{V}}
\newcommand{\gl}{\mathfrak{gl}}
\renewcommand{\sl}{\mathfrak{sl}}
\newcommand{\diag}{\mathfrak{diag}}
\newcommand{\pgl}{\mathfrak{pgl}}
\newcommand{\psl}{\mathfrak{psl}}
\newcommand{\fraksp}{\mathfrak{sp}}
\newcommand{\gsp}{\mathfrak{gsp}}
\newcommand{\gspin}{\mathfrak{gspin}}
\newcommand{\frako}{\mathfrak{o}}
\newcommand{\so}{\mathfrak{so}}
\newcommand{\su}{\mathfrak{su}}
%\newcommand{\fraku}{\mathfrak{u}}
\newcommand{\Spec}{\operatorname{Spec}}
\newcommand{\Spf}{\operatorname{Spf}}
\newcommand{\Spm}{\operatorname{Spm}}
\newcommand{\Spv}{\operatorname{Spv}}
\newcommand{\Spa}{\operatorname{Spa}}
\newcommand{\Spd}{\operatorname{Spd}}
\newcommand{\Proj}{\operatorname{Proj}}
\newcommand{\Gr}{\mathrm{Gr}}
\newcommand{\Hecke}{\mathrm{Hecke}}
\newcommand{\Sht}{\mathrm{Sht}}
\newcommand{\Quot}{\mathrm{Quot}}
\newcommand{\Hilb}{\mathrm{Hilb}}
\newcommand{\Pic}{\mathrm{Pic}}
\newcommand{\Div}{\mathrm{Div}}
\newcommand{\Jac}{\mathrm{Jac}}
\newcommand{\Alb}{\mathrm{Alb}} %albanese variety
\newcommand{\Bun}{\mathrm{Bun}}
\newcommand{\loopspace}{\mathbf{\Omega}}
\newcommand{\suspension}{\mathbf{\Sigma}}
\newcommand{\tangent}{\mathrm{T}} %tangent space
\newcommand{\Eig}{\mathrm{Eig}}
\newcommand{\Cox}{\mathrm{Cox}} %coxeter functors
\newcommand{\rmK}{\mathrm{K}} %Killing form
\newcommand{\km}{\mathfrak{km}} %kac-moody algebras
\newcommand{\Dyn}{\mathrm{Dyn}} %associated Dynkin quivers
\newcommand{\Car}{\mathrm{Car}} %cartan matrices of finite quivers
\newcommand{\uce}{\mathfrak{uce}} %universal central extension of lie algebras

\newcommand{\Ring}{\mathrm{Ring}}
\newcommand{\Cring}{\mathrm{CRing}}
\newcommand{\Alg}{\mathrm{Alg}}
\newcommand{\Leib}{\mathrm{Leib}} %leibniz algebras
\newcommand{\Fld}{\mathrm{Fld}}
\newcommand{\Sets}{\mathrm{Sets}}
\newcommand{\Equiv}{\mathrm{Equiv}} %equivalence relations
\newcommand{\Cat}{\mathrm{Cat}}
\newcommand{\Grp}{\mathrm{Grp}}
\newcommand{\Ab}{\mathrm{Ab}}
\newcommand{\Sch}{\mathrm{Sch}}
\newcommand{\Coh}{\mathrm{Coh}}
\newcommand{\QCoh}{\mathrm{QCoh}}
\newcommand{\Perf}{\mathrm{Perf}} %perfect complexes
\newcommand{\Sing}{\mathrm{Sing}} %singularity categories
\newcommand{\Desc}{\mathrm{Desc}}
\newcommand{\Sh}{\mathrm{Sh}}
\newcommand{\Psh}{\mathrm{PSh}}
\newcommand{\Fib}{\mathrm{Fib}}
\renewcommand{\mod}{\-\mathrm{mod}}
\newcommand{\comod}{\-\mathrm{comod}}
\newcommand{\bimod}{\-\mathrm{bimod}}
\newcommand{\Vect}{\mathrm{Vect}}
\newcommand{\Rep}{\mathrm{Rep}}
\newcommand{\Grpd}{\mathrm{Grpd}}
\newcommand{\Arr}{\mathrm{Arr}}
\newcommand{\Esp}{\mathrm{Esp}}
\newcommand{\Ob}{\mathrm{Ob}}
\newcommand{\Mor}{\mathrm{Mor}}
\newcommand{\Mfd}{\mathrm{Mfd}}
\newcommand{\Riem}{\mathrm{Riem}}
\newcommand{\RS}{\mathrm{RS}}
\newcommand{\LRS}{\mathrm{LRS}}
\newcommand{\TRS}{\mathrm{TRS}}
\newcommand{\TLRS}{\mathrm{TLRS}}
\newcommand{\LVRS}{\mathrm{LVRS}}
\newcommand{\LBRS}{\mathrm{LBRS}}
\newcommand{\Spc}{\mathrm{Spc}}
\newcommand{\Top}{\mathrm{Top}}
\newcommand{\Topos}{\mathrm{Topos}}
\newcommand{\Nil}{\mathfrak{nil}}
\newcommand{\J}{\mathfrak{J}}
\newcommand{\Stk}{\mathrm{Stk}}
\newcommand{\Pre}{\mathrm{Pre}}
\newcommand{\simp}{\mathbf{\Delta}}
\newcommand{\Res}{\mathrm{Res}}
\newcommand{\Ind}{\mathrm{Ind}}
\newcommand{\Pro}{\mathrm{Pro}}
\newcommand{\Mon}{\mathrm{Mon}}
\newcommand{\Comm}{\mathrm{Comm}}
\newcommand{\Fin}{\mathrm{Fin}}
\newcommand{\Assoc}{\mathrm{Assoc}}
\newcommand{\Semi}{\mathrm{Semi}}
\newcommand{\Co}{\mathrm{Co}}
\newcommand{\Loc}{\mathrm{Loc}}
\newcommand{\Ringed}{\mathrm{Ringed}}
\newcommand{\Haus}{\mathrm{Haus}} %hausdorff spaces
\newcommand{\Comp}{\mathrm{Comp}} %compact hausdorff spaces
\newcommand{\Stone}{\mathrm{Stone}} %stone spaces
\newcommand{\Extr}{\mathrm{Extr}} %extremely disconnected spaces
\newcommand{\Ouv}{\mathrm{Ouv}}
\newcommand{\Str}{\mathrm{Str}}
\newcommand{\Func}{\mathrm{Func}}
\newcommand{\Crys}{\mathrm{Crys}}
\newcommand{\LocSys}{\mathrm{LocSys}}
\newcommand{\Sieves}{\mathrm{Sieves}}
\newcommand{\pt}{\mathrm{pt}}
\newcommand{\Graphs}{\mathrm{Graphs}}
\newcommand{\Lie}{\mathrm{Lie}}
\newcommand{\Env}{\mathrm{Env}}
\newcommand{\Ho}{\mathrm{Ho}}
\newcommand{\rmD}{\mathrm{D}}
\newcommand{\Cov}{\mathrm{Cov}}
\newcommand{\Frames}{\mathrm{Frames}}
\newcommand{\Locales}{\mathrm{Locales}}
\newcommand{\Span}{\mathrm{Span}}
\newcommand{\Corr}{\mathrm{Corr}}
\newcommand{\Monad}{\mathrm{Monad}}
\newcommand{\Var}{\mathrm{Var}}
\newcommand{\sfN}{\mathrm{N}} %nerve
\newcommand{\Diam}{\mathrm{Diam}} %diamonds
\newcommand{\co}{\mathrm{co}}
\newcommand{\ev}{\mathrm{ev}}
\newcommand{\bi}{\mathrm{bi}}
\newcommand{\Nat}{\mathrm{Nat}}
\newcommand{\Hopf}{\mathrm{Hopf}}
\newcommand{\Dmod}{\mathrm{D}\mod}
\newcommand{\Perv}{\mathrm{Perv}}
\newcommand{\Sph}{\mathrm{Sph}}
\newcommand{\Moduli}{\mathrm{Moduli}}
\newcommand{\Pseudo}{\mathrm{Pseudo}}
\newcommand{\Lax}{\mathrm{Lax}}
\newcommand{\Strict}{\mathrm{Strict}}
\newcommand{\Opd}{\mathrm{Opd}} %operads
\newcommand{\Shv}{\mathrm{Shv}}
\newcommand{\Char}{\mathrm{Char}} %CharShv = character sheaves
\newcommand{\Huber}{\mathrm{Huber}}
\newcommand{\Tate}{\mathrm{Tate}}
\newcommand{\Affd}{\mathrm{Affd}} %affinoid algebras
\newcommand{\Adic}{\mathrm{Adic}} %adic spaces
\newcommand{\Rig}{\mathrm{Rig}}
\newcommand{\An}{\mathrm{An}}
\newcommand{\Perfd}{\mathrm{Perfd}} %perfectoid spaces
\newcommand{\Sub}{\mathrm{Sub}} %subobjects
\newcommand{\Ideals}{\mathrm{Ideals}}
\newcommand{\Isoc}{\mathrm{Isoc}} %isocrystals
\newcommand{\Ban}{\mathrm{Ban}} %Banach spaces
\newcommand{\Fre}{\mathrm{Fr\acute{e}}} %Frechet spaces
\newcommand{\Ch}{\mathrm{Ch}} %chain complexes
\newcommand{\Pure}{\mathrm{Pure}}
\newcommand{\Mixed}{\mathrm{Mixed}}
\newcommand{\Hodge}{\mathrm{Hodge}} %Hodge structures
\newcommand{\Mot}{\mathrm{Mot}} %motives
\newcommand{\KL}{\mathrm{KL}} %category of Kazhdan-Lusztig modules
\newcommand{\Pres}{\mathrm{Pres}} %presentable categories
\newcommand{\Noohi}{\mathrm{Noohi}} %category of Noohi groups
\newcommand{\Inf}{\mathrm{Inf}}
\newcommand{\LPar}{\mathrm{LPar}} %Langlands parameters
\newcommand{\ORig}{\mathrm{ORig}} %overconvergent sites
\newcommand{\Quiv}{\mathrm{Quiv}} %quivers
\newcommand{\Def}{\mathrm{Def}} %deformation functors
\newcommand{\Root}{\mathrm{Root}}
\newcommand{\gRep}{\mathrm{gRep}}
\newcommand{\Higgs}{\mathrm{Higgs}}
\newcommand{\BGG}{\mathrm{BGG}}
\newcommand{\Poiss}{\mathrm{Poiss}}

\newcommand{\Aut}{\mathrm{Aut}}
\newcommand{\Inn}{\mathrm{Inn}}
\newcommand{\Out}{\mathrm{Out}}
\newcommand{\der}{\mathfrak{der}} %derivations on Lie algebras
\newcommand{\frakend}{\mathfrak{end}}
\newcommand{\aut}{\mathfrak{aut}}
\newcommand{\inn}{\mathfrak{inn}} %inner derivations
\newcommand{\out}{\mathfrak{out}} %outer derivations
\newcommand{\Stab}{\mathrm{Stab}}
\newcommand{\Cent}{\mathrm{Cent}}
\newcommand{\Norm}{\mathrm{Norm}}
\newcommand{\cent}{\mathfrak{cent}}
\newcommand{\norm}{\mathfrak{norm}}
\newcommand{\Rad}{\operatorname{Rad}}
\newcommand{\Transporter}{\mathrm{Transp}} %transporter between two subsets of a group
\newcommand{\Conj}{\mathrm{Conj}}
\newcommand{\Diag}{\mathrm{Diag}}
\newcommand{\Gal}{\mathrm{Gal}}
\newcommand{\bfG}{\mathbf{G}} %absolute Galois group
\newcommand{\Frac}{\mathrm{Frac}}
\newcommand{\Ann}{\mathrm{Ann}}
\newcommand{\Val}{\mathrm{Val}}
\newcommand{\Chow}{\mathrm{Chow}}
\newcommand{\Sym}{\mathrm{Sym}}
\newcommand{\End}{\mathrm{End}}
\newcommand{\Mat}{\mathrm{Mat}}
\newcommand{\Diff}{\mathrm{Diff}}
\newcommand{\Autom}{\mathrm{Autom}}
\newcommand{\Artin}{\mathrm{Artin}} %artin maps
\newcommand{\sk}{\mathrm{sk}} %skeleton of a category
\newcommand{\eqv}{\mathrm{eqv}} %functor that maps groups $G$ to $G$-sets
\newcommand{\Inert}{\mathrm{Inert}}
\newcommand{\Fil}{\mathrm{Fil}}
\newcommand{\Prim}{\mathfrak{Prim}}
\newcommand{\Nerve}{\mathrm{N}}
\newcommand{\Hol}{\mathrm{Hol}} %holomorphic functions %holonomy groups
\newcommand{\Bi}{\mathrm{Bi}} %Bi for biholomorphic functions
\newcommand{\chev}{\operatorname{chev}}
\newcommand{\bfLie}{\mathbf{Lie}} %non-reduced lie algebra associated to generalised cartan matrices
\newcommand{\frakLie}{\mathfrak{Lie}} %reduced lie algebra associated to generalised cartan matrices
\newcommand{\frakChev}{\mathfrak{Chev}} 
\newcommand{\Rees}{\operatorname{Rees}}
\newcommand{\Dr}{\mathrm{Dr}} %Drinfeld's quantum double 
\newcommand{\frakDr}{\mathfrak{Dr}} %classical double of lie bialgebras
\newcommand{\Tot}{\operatorname{Tot}} %total complexes

\renewcommand{\projlim}{\varprojlim}
\newcommand{\indlim}{\varinjlim}
\newcommand{\colim}{\operatorname{colim}}
\renewcommand{\lim}{\operatorname{lim}}
\newcommand{\toto}{\rightrightarrows}
%\newcommand{\tensor}{\otimes}
\NewDocumentCommand{\tensor}{e{_^}}{%
  \mathbin{\mathop{\otimes}\displaylimits
    \IfValueT{#1}{_{#1}}
    \IfValueT{#2}{^{#2}}
  }%
}
\NewDocumentCommand{\singtensor}{e{_^}}{%
  \mathbin{\mathop{\odot}\displaylimits
    \IfValueT{#1}{_{#1}}
    \IfValueT{#2}{^{#2}}
  }%
}
\NewDocumentCommand{\hattensor}{e{_^}}{%
  \mathbin{\mathop{\hat{\otimes}}\displaylimits
    \IfValueT{#1}{_{#1}}
    \IfValueT{#2}{^{#2}}
  }%
}
\NewDocumentCommand{\brevetensor}{e{_^}}{%
  \mathbin{\mathop{\breve{\otimes}}\displaylimits
    \IfValueT{#1}{_{#1}}
    \IfValueT{#2}{^{#2}}
  }%
}
\NewDocumentCommand{\semidirect}{e{_^}}{%
  \mathbin{\mathop{\rtimes}\displaylimits
    \IfValueT{#1}{_{#1}}
    \IfValueT{#2}{^{#2}}
  }%
}
\newcommand{\eq}{\operatorname{eq}}
\newcommand{\coeq}{\operatorname{coeq}}
\newcommand{\Hom}{\mathrm{Hom}}
\newcommand{\Maps}{\mathrm{Maps}}
\newcommand{\Tor}{\mathrm{Tor}}
\newcommand{\Ext}{\mathrm{Ext}}
\newcommand{\Isom}{\mathrm{Isom}}
\newcommand{\stalk}{\mathrm{stalk}}
\newcommand{\RKE}{\operatorname{RKE}}
\newcommand{\LKE}{\operatorname{LKE}}
\newcommand{\oblv}{\mathrm{oblv}}
\newcommand{\const}{\mathrm{const}}
\newcommand{\free}{\mathrm{free}}
\newcommand{\adrep}{\mathrm{ad}} %adjoint representation
\newcommand{\NL}{\mathbb{NL}} %naive cotangent complex
\newcommand{\pr}{\operatorname{pr}}
\newcommand{\Der}{\mathrm{Der}}
\newcommand{\Frob}{\mathrm{Fr}} %Frobenius
\newcommand{\frob}{\mathrm{f}} %trace of Frobenius
\newcommand{\bfpt}{\mathbf{pt}}
\newcommand{\bfloc}{\mathbf{loc}}
\DeclareMathAlphabet{\mymathbb}{U}{BOONDOX-ds}{m}{n}
\newcommand{\0}{\mymathbb{0}}
\newcommand{\1}{\mathbbm{1}}
\newcommand{\2}{\mathbbm{2}}
\newcommand{\Jet}{\mathrm{Jet}}
\newcommand{\Split}{\mathrm{Split}}
\newcommand{\Sq}{\mathrm{Sq}}
\newcommand{\Zero}{\mathrm{Z}}
\newcommand{\SqZ}{\Sq\Zero}
\newcommand{\lie}{\mathfrak{lie}}
\newcommand{\y}{\mathrm{y}} %yoneda
\newcommand{\Sm}{\mathrm{Sm}}
\newcommand{\AJ}{\phi} %abel-jacobi map
\newcommand{\act}{\mathrm{act}}
\newcommand{\ram}{\mathrm{ram}} %ramification index
\newcommand{\inv}{\mathrm{inv}}
\newcommand{\Spr}{\mathrm{Spr}} %the Springer map/sheaf
\newcommand{\Refl}{\mathrm{Refl}} %reflection functor]
\newcommand{\HH}{\mathrm{HH}} %Hochschild (co)homology
\newcommand{\HC}{\mathrm{HC}} %cyclic (co)homology
\newcommand{\Poinc}{\mathrm{Poinc}}
\newcommand{\Simpson}{\mathrm{Simpson}}
\newcommand{\Section}{\operatorname{Sect}}

\newcommand{\bbU}{\mathbb{U}}
\newcommand{\V}{\mathbb{V}}
\newcommand{\W}{\mathbb{W}}
\newcommand{\calU}{\mathcal{U}}
\newcommand{\calW}{\mathcal{W}}
\newcommand{\rmI}{\mathrm{I}} %augmentation ideal
\newcommand{\bfV}{\mathbf{V}}
\newcommand{\C}{\mathcal{C}}
\newcommand{\D}{\mathcal{D}}
\newcommand{\scrD}{\mathscr{D}}
\newcommand{\T}{\mathscr{T}} %Tate modules
\newcommand{\calM}{\mathcal{M}}
\newcommand{\calN}{\mathcal{N}}
\newcommand{\calP}{\mathcal{P}}
\newcommand{\calQ}{\mathcal{Q}}
\newcommand{\A}{\mathbb{A}}
\renewcommand{\P}{\mathbb{P}}
\newcommand{\calL}{\mathcal{L}}
\newcommand{\scrL}{\mathscr{L}}
\newcommand{\E}{\mathcal{E}}
\renewcommand{\H}{\mathbf{H}}
\newcommand{\scrS}{\mathscr{S}}
\newcommand{\calX}{\mathcal{X}}
\newcommand{\calY}{\mathcal{Y}}
\newcommand{\calZ}{\mathcal{Z}}
\newcommand{\calS}{\mathcal{S}}
\newcommand{\calR}{\mathcal{R}}
\newcommand{\scrX}{\mathscr{X}}
\newcommand{\scrY}{\mathscr{Y}}
\newcommand{\scrZ}{\mathscr{Z}}
\newcommand{\calA}{\mathcal{A}}
\newcommand{\calB}{\mathcal{B}}
\renewcommand{\S}{\mathcal{S}}
\newcommand{\B}{\mathbb{B}}
\newcommand{\bbD}{\mathbb{D}}
\newcommand{\G}{\mathbb{G}}
\newcommand{\horn}{\mathbf{\Lambda}}
\renewcommand{\L}{\mathbb{L}}
\renewcommand{\a}{\mathfrak{a}}
\renewcommand{\b}{\mathfrak{b}}
\renewcommand{\c}{\mathfrak{c}}
\renewcommand{\d}{\mathfrak{d}}
\renewcommand{\t}{\mathfrak{t}}
\renewcommand{\r}{\mathfrak{r}}
\newcommand{\fraku}{\mathfrak{u}}
\newcommand{\frakv}{\mathfrak{v}}
\newcommand{\frake}{\mathfrak{e}}
\newcommand{\bbX}{\mathbb{X}}
\newcommand{\frakw}{\mathfrak{w}}
\newcommand{\frakG}{\mathfrak{G}}
\newcommand{\frakH}{\mathfrak{H}}
\newcommand{\frakE}{\mathfrak{E}}
\newcommand{\frakF}{\mathfrak{F}}
\newcommand{\g}{\mathfrak{g}}
\newcommand{\frakl}{\mathfrak{l}}
\newcommand{\h}{\mathfrak{h}}
\renewcommand{\k}{\mathfrak{k}}
\newcommand{\z}{\mathfrak{z}}
\newcommand{\fraki}{\mathfrak{i}}
\newcommand{\frakj}{\mathfrak{j}}
\newcommand{\del}{\partial}
\newcommand{\bbE}{\mathbb{E}}
\newcommand{\scrO}{\mathscr{O}}
\newcommand{\bbO}{\mathbb{O}}
\newcommand{\scrA}{\mathscr{A}}
\newcommand{\scrB}{\mathscr{B}}
\newcommand{\scrE}{\mathscr{E}}
\newcommand{\scrF}{\mathscr{F}}
\newcommand{\scrG}{\mathscr{G}}
\newcommand{\scrM}{\mathscr{M}}
\newcommand{\scrN}{\mathscr{N}}
\newcommand{\scrP}{\mathscr{P}}
\newcommand{\frakS}{\mathfrak{S}}
\newcommand{\frakT}{\mathfrak{T}}
\newcommand{\calI}{\mathcal{I}}
\newcommand{\calJ}{\mathcal{J}}
\newcommand{\scrI}{\mathscr{I}}
\newcommand{\scrJ}{\mathscr{J}}
\newcommand{\scrK}{\mathscr{K}}
\newcommand{\calK}{\mathcal{K}}
\newcommand{\scrV}{\mathscr{V}}
\newcommand{\scrW}{\mathscr{W}}
\newcommand{\bbS}{\mathbb{S}}
\newcommand{\scrH}{\mathscr{H}}
\newcommand{\bfA}{\mathbf{A}}
\newcommand{\bfB}{\mathbf{B}}
\newcommand{\bfC}{\mathbf{C}}
\renewcommand{\O}{\mathbb{O}}
\newcommand{\calV}{\mathcal{V}}
\newcommand{\scrR}{\mathscr{R}} %radical
\newcommand{\sfR}{\mathsf{R}} %quantum R-matrices
\newcommand{\sfr}{\mathsf{r}} %classical R-matrices
\newcommand{\rmZ}{\mathrm{Z}} %centre of algebra
\newcommand{\rmC}{\mathrm{C}} %centralisers in algebras
\newcommand{\bfGamma}{\mathbf{\Gamma}}
\newcommand{\scrU}{\mathscr{U}}
\newcommand{\rmW}{\mathrm{W}} %Weil group
\newcommand{\frakM}{\mathfrak{M}}
\newcommand{\frakN}{\mathfrak{N}}
\newcommand{\frakB}{\mathfrak{B}}
\newcommand{\frakX}{\mathfrak{X}}
\newcommand{\frakY}{\mathfrak{Y}}
\newcommand{\frakZ}{\mathfrak{Z}}
\newcommand{\frakU}{\mathfrak{U}}
\newcommand{\frakR}{\mathfrak{R}}
\newcommand{\frakP}{\mathfrak{P}}
\newcommand{\frakQ}{\mathfrak{Q}}
\newcommand{\sfX}{\mathsf{X}}
\newcommand{\sfY}{\mathsf{Y}}
\newcommand{\sfZ}{\mathsf{Z}}
\newcommand{\sfS}{\mathsf{S}}
\newcommand{\sfT}{\mathsf{T}}
\newcommand{\sfOmega}{\mathsf{\Omega}} %drinfeld p-adic upper-half plane
\newcommand{\rmA}{\mathrm{A}}
\newcommand{\rmB}{\mathrm{B}}
\newcommand{\calT}{\mathcal{T}}
\newcommand{\sfA}{\mathsf{A}}
\newcommand{\sfB}{\mathsf{B}}
\newcommand{\sfC}{\mathsf{C}}
\newcommand{\sfD}{\mathsf{D}}
\newcommand{\sfE}{\mathsf{E}}
\newcommand{\sfF}{\mathsf{F}}
\newcommand{\sfG}{\mathsf{G}}
\newcommand{\frakL}{\mathfrak{L}}
\newcommand{\K}{\mathrm{K}}
\newcommand{\rmT}{\mathrm{T}}
\newcommand{\bfv}{\mathbf{v}}
\newcommand{\bfg}{\mathbf{g}}
\newcommand{\frakV}{\mathfrak{V}}
\newcommand{\bfn}{\mathbf{n}}
\renewcommand{\o}{\mathfrak{o}}
\newcommand{\standard}{\Delta}
\newcommand{\costandard}{\nabla}
\newcommand{\simple}{\bar{\Delta}}
\newcommand{\cosimple}{\bar{\nabla}}

\newcommand{\aff}{\mathrm{aff}}
\newcommand{\ft}{\mathrm{ft}} %finite type
\newcommand{\fp}{\mathrm{fp}} %finite presentation
\newcommand{\fr}{\mathrm{fr}} %free
\newcommand{\tft}{\mathrm{tft}} %topologically finite type
\newcommand{\tfp}{\mathrm{tfp}} %topologically finite presentation
\newcommand{\tfr}{\mathrm{tfr}} %topologically free
\newcommand{\aft}{\mathrm{aft}}
\newcommand{\lft}{\mathrm{lft}}
\newcommand{\laft}{\mathrm{laft}}
\newcommand{\cpt}{\mathrm{cpt}}
\newcommand{\cproj}{\mathrm{cproj}}
\newcommand{\qc}{\mathrm{qc}}
\newcommand{\qs}{\mathrm{qs}}
\newcommand{\lcmpt}{\mathrm{lcmpt}}
\newcommand{\red}{\mathrm{red}}
\newcommand{\fin}{\mathrm{fin}}
\newcommand{\fd}{\mathrm{fd}} %finite-dimensional
\newcommand{\gen}{\mathrm{gen}}
\newcommand{\petit}{\mathrm{petit}}
\newcommand{\gros}{\mathrm{gros}}
\newcommand{\loc}{\mathrm{loc}}
\newcommand{\glob}{\mathrm{glob}}
%\newcommand{\ringed}{\mathrm{ringed}}
%\newcommand{\qcoh}{\mathrm{qcoh}}
\newcommand{\cl}{\mathrm{cl}}
\newcommand{\et}{\mathrm{\acute{e}t}}
\newcommand{\fet}{\mathrm{f\acute{e}t}}
\newcommand{\profet}{\mathrm{prof\acute{e}t}}
\newcommand{\proet}{\mathrm{pro\acute{e}t}}
\newcommand{\Zar}{\mathrm{Zar}}
\newcommand{\fppf}{\mathrm{fppf}}
\newcommand{\fpqc}{\mathrm{fpqc}}
\newcommand{\orig}{\mathrm{orig}} %overconvergent topology
\newcommand{\smooth}{\mathrm{sm}}
\newcommand{\sh}{\mathrm{sh}}
\newcommand{\op}{\mathrm{op}}
\newcommand{\cop}{\mathrm{cop}}
\newcommand{\open}{\mathrm{open}}
\newcommand{\closed}{\mathrm{closed}}
\newcommand{\geom}{\mathrm{geom}}
\newcommand{\alg}{\mathrm{alg}}
\newcommand{\sober}{\mathrm{sober}}
\newcommand{\dR}{\mathrm{dR}}
\newcommand{\rad}{\mathfrak{rad}}
\newcommand{\discrete}{\mathrm{discrete}}
%\newcommand{\add}{\mathrm{add}}
%\newcommand{\lin}{\mathrm{lin}}
\newcommand{\Krull}{\mathrm{Krull}}
\newcommand{\qis}{\mathrm{qis}} %quasi-isomorphism
\newcommand{\ho}{\mathrm{ho}} %homotopy equivalence
\newcommand{\sep}{\mathrm{sep}}
\newcommand{\unr}{\mathrm{unr}}
\newcommand{\tame}{\mathrm{tame}}
\newcommand{\wild}{\mathrm{wild}}
\newcommand{\nil}{\mathrm{nil}}
\newcommand{\defm}{\mathrm{defm}}
\newcommand{\Art}{\mathrm{Art}}
\newcommand{\Noeth}{\mathrm{Noeth}}
\newcommand{\affd}{\mathrm{affd}}
%\newcommand{\adic}{\mathrm{adic}}
\newcommand{\pre}{\mathrm{pre}}
\newcommand{\coperf}{\mathrm{coperf}}
\newcommand{\perf}{\mathrm{perf}}
\newcommand{\perfd}{\mathrm{perfd}}
\newcommand{\rat}{\mathrm{rat}}
\newcommand{\cont}{\mathrm{cont}}
\newcommand{\dg}{\mathrm{dg}}
\newcommand{\almost}{\mathrm{a}}
%\newcommand{\stab}{\mathrm{stab}}
\newcommand{\heart}{\heartsuit}
\newcommand{\proj}{\mathrm{proj}}
\newcommand{\qproj}{\mathrm{qproj}}
\newcommand{\pd}{\mathrm{pd}}
\newcommand{\crys}{\mathrm{crys}}
\newcommand{\prisma}{\mathrm{prisma}}
\newcommand{\FF}{\mathrm{FF}}
\newcommand{\sph}{\mathrm{sph}}
\newcommand{\lax}{\mathrm{lax}}
\newcommand{\weak}{\mathrm{weak}}
\newcommand{\strict}{\mathrm{strict}}
\newcommand{\mon}{\mathrm{mon}}
\newcommand{\sym}{\mathrm{sym}}
\newcommand{\lisse}{\mathrm{lisse}}
\newcommand{\an}{\mathrm{an}}
\newcommand{\ad}{\mathrm{ad}}
\newcommand{\sch}{\mathrm{sch}}
\newcommand{\rig}{\mathrm{rig}}
\newcommand{\pol}{\mathrm{pol}}
\newcommand{\plat}{\mathrm{flat}}
\newcommand{\proper}{\mathrm{proper}}
\newcommand{\compl}{\mathrm{compl}}
\newcommand{\non}{\mathrm{non}}
\newcommand{\access}{\mathrm{access}}
\newcommand{\comp}{\mathrm{comp}}
\newcommand{\tstructure}{\mathrm{t}} %t-structures
\newcommand{\pure}{\mathrm{pure}} %pure motives
\newcommand{\mixed}{\mathrm{mixed}} %mixed motives
\newcommand{\num}{\mathrm{num}} %numerical motives
\newcommand{\ess}{\mathrm{ess}}
\newcommand{\topological}{\mathrm{top}}
\newcommand{\convex}{\mathrm{cvx}}
\newcommand{\locconvex}{\mathrm{lcvx}}
\newcommand{\ab}{\mathrm{ab}} %abelian extensions
\newcommand{\inj}{\mathrm{inj}}
\newcommand{\surj}{\mathrm{surj}} %coverage on sets generated by surjections
\newcommand{\eff}{\mathrm{eff}} %effective Cartier divisors
\newcommand{\Weil}{\mathrm{Weil}} %weil divisors
\newcommand{\lex}{\mathrm{lex}}
\newcommand{\rex}{\mathrm{rex}}
\newcommand{\AR}{\mathrm{A\-R}}
\newcommand{\cons}{\mathrm{c}} %constructible sheaves
\newcommand{\tor}{\mathrm{tor}} %tor dimension
\newcommand{\connected}{\mathrm{connected}}
\newcommand{\cg}{\mathrm{cg}} %compactly generated
\newcommand{\nilp}{\mathrm{nilp}}
\newcommand{\isg}{\mathrm{isg}} %isogenous
\newcommand{\qisg}{\mathrm{qisg}} %quasi-isogenous
\newcommand{\irr}{\mathrm{irr}} %irreducible represenations
\newcommand{\indecomp}{\mathrm{indecomp}}
\newcommand{\preproj}{\mathrm{preproj}}
\newcommand{\preinj}{\mathrm{preinj}}
\newcommand{\reg}{\mathrm{reg}}
\newcommand{\semisimple}{\mathrm{ss}}
\newcommand{\integrable}{\mathrm{int}}
\newcommand{\s}{\mathfrak{s}}
\newcommand{\elliptic}{\mathrm{ell}}
\newcommand{\stab}{\mathrm{stab}}

%prism custom command
\usepackage{relsize}
\usepackage[bbgreekl]{mathbbol}
\usepackage{amsfonts}
\DeclareSymbolFontAlphabet{\mathbb}{AMSb} %to ensure that the meaning of \mathbb does not change
\DeclareSymbolFontAlphabet{\mathbbl}{bbold}
\newcommand{\prism}{{\mathlarger{\mathbbl{\Delta}}}}
\newcommand{\toroidal}{\t}
\newcommand{\extendedtoroidal}{\hat{\t}}
\newcommand{\simpleroots}{\mathbb{I}}
\renewcommand{\positive}{+}
\renewcommand{\negative}{-}
\newcommand{\divzero}{\der_{\gamma}(A)}

\begin{document}

    \title{
    \texorpdfstring{\Huge On a class of extended toroidal Lie algebras coming from untwisted affine Yangians}

    \todo[inline]{I decided to go with this slightly vague title, which is along the lines suggested by Dr. Weekes. After thinking about it some more, I think any insinuation that our $\gamma$-extended toroidal Lie algebras might resemble any kind of 2-variable analogue of Kac-Moody algebras might be misleading, since that might imply that we are also studying combinatorial structures like root systems and so on in this thesis.}

    \vfill

    \begin{centering}
        \normalsize A Thesis Submitted to the
        \\
        College of Graduate and Postdoctoral Studies
        \\
        in Partial Fulfillment of the Requirements
        \\
        for the Degree of Master of Science
        \\
        in the Department of Mathematics and Statistics
        \\
        University of Saskatchewan
        \\
        Saskatoon, Saskatchewan, Canada
    \end{centering}

    \vfill

    \begin{centering}
        \normalsize \textcopyright Copyright Dat Minh Ha, June, 2024. All rights reserved.
        \\
        Unless otherwise noted, copyright of the material in this thesis belongs to the author.
    \end{centering}

    \vfill

    \author{\normalsize Dat Minh Ha}
    \date{\normalsize \today}
}

\maketitle

    \newpage

    \listoftodos

    \todo[inline]{Remove to-do list before final compilation}

    \newpage

    \chapter*{Frontmatter}
        \minitoc
    
        \newpage
    
        \section*{Permission to use}
    In presenting this thesis in partial fulfillment of the requirements for a Postgraduate degree from the University of Saskatchewan, I agree that the Libraries of this University may make it freely available for inspection. I further agree that permission for copying of this thesis in any manner, in whole or in part, for scholarly purposes may be granted by the professor or professors who supervised my thesis work or, in their absence, by the Head of the Department or the Dean of the College in which my thesis work was done. It is understood that any copying or publication or use of this thesis or parts thereof for financial gain shall not be allowed without my written permission. It is also understood that due recognition shall be given to me and to the University of Saskatchewan in any scholarly use which may be made of any material in my thesis.
    
        \newpage
    
        \section*{Abstract}
    The purpose of this thesis is to construct so-called $\gamma$-extended toroidal Lie algebras. Originally, these $\gamma$-extended toroidal Lie algebras, i.e. universal central extensions (UCEs):
        $$\toroidal$$
    of Lie algebras of the form:
        $$\g[v^{\pm 1}, t^{\pm 1}] := \g \tensor_{\bbC} \bbC[v^{\pm 1}, t^{\pm 1}]$$
    where $\g$ is a finite-dimensional simple Lie algebra over $\bbC$, were created as a mean of endowing toroidal Lie algebras with Lie bialgebra structures, so that toroidal Lie bialgebras can be recognised as classical limits of so-called \say{affine Yangians} in a certain sense. Per \cite{etingof_kazhdan_quantisation_1}, this would mean construct Manin triples of the form:
        $$(\toroidal, \toroidal^{\positive}, \toroidal^{\negative})$$
    which in itself, requires us to endow $\toroidal$ with an invariant bilinear form satisfying some conditions. However, an issue is that any invariant bilinear form on a UCE is necessarily \textit{degenerate}. As such, we are motivated to enlarge toroidal Lie algebras into $\gamma$-extended Lie algebras:
        $$\extendedtoroidal$$
    and this is done in such a way that the resulting larger Lie algebras can then be endowed with \textit{invariant} symmetric bilinear forms that are also \textit{non-degenerate}; importantly, the construction of these bilinear form depends entirely on a certain linear map:
        $$\gamma: \bbC[v^{\pm 1}, t^{\pm 1}] \to \bbC$$
    (and hence the name of our Lie algebras).
    
    We shall see that the Lie algebras $\extendedtoroidal$ all arise as \say{twists} of the semi-direct product $\toroidal \rtimes \der_{\gamma}(\bbC[v^{\pm 1}, t^{\pm 1}])$ by Lie $2$-cocycles $\sigma \in Z^2_{\Lie}(\der_{\gamma}(\bbC[v^{\pm 1}, t^{\pm 1}]), \z(\toroidal))$, with $\der_{\gamma}(\bbC[v^{\pm 1}, t^{\pm 1}])$ being a certain ($\Z^2$-graded) Lie subalgebra of the Lie algebra $\der(\bbC[v^{\pm 1}, t^{\pm 1}])$ of all derivations on $\bbC[v^{\pm 1}, t^{\pm 1}]$. Moreover, we will see that there is a readily available example of such a $2$-cocycle giving rise to a $\gamma$-extended toroidal Lie algebra that is \textit{not} isomorphic to the aforementioned semi-direct product.
    
        \newpage
    
        \section*{Acknowledgements}
    I must firstly acknowledge the invaluable contributions that my advisors, Prof. Dr. Curtis Wendlandt and Prof. Dr. Alex Weekes, have made towards this thesis. Not only were they the ones to give me this problem, but the thesis simply could not have been written without their close following of my progress (and often, even the lack thereof), as well as their many suggestions and advices regarding both proof strategies and presentation. I would also like to thank them for all the mathematics that they have taught and introduced me to throughout the length of my MSc., much of which I had no prior familiarity with. I would also like to express my gratitude towards them, for teaching me how to not just mathematical contents themselves, but also how to actually \textit{do} mathematics. I was not able to appreciate many of their advices in the moment, but with the benefit of hindsight, these advices have become very precious to me.
    
    I would also like to thank Prof. Dr. Steven Rayan, who was my instructor for two of the courses that I was required to take as a part of my MSc., not only for being a wonderful instructor, but also for the many administrative help that he has provided me with over the past two years. I am also grateful for the fact that through him, I was able to become friends with some other graduate students in the Department, namely Mahmud Azam and Kuntal Banerjee. We have had many enjoyable discussions.

    \newpage

    {
      \hypersetup{} 
      \dominitoc
      \tableofcontents %sort sections alphabetically
    }

    \newpage

    \chapter{Introduction}
        \begin{abstract}
            We begin this thesis by giving an account of the contexts and motivations that lead to our construction of $\gamma$-extended toroidal Lie algebras, as well as the structure of the whole thesis itself.
        \end{abstract}
    
        \minitoc

        \newpage

        \section{How and why have we constructed \texorpdfstring{$\gamma$}{}-extended toroidal Lie algebras ?}
    \subsection{Background}
        A rather well-known story in the representation theory of Lie algebras over algebraically closed fields of characteristic $0$ (e.g. $\bbC$, which from now on will be the default underlying field) is that of finite-dimensional semi-simple Lie algebras. This theory dates back to the works of Wilhelm Killing and \'Elie Cartan in the late $19^{th}$ century and early $20^{th}$ century on the classification of finite-dimensional semi-simple Lie algebras over $\bbC$. A salient feature of, and indeed, a very important technical ingredient in this theory, is an essentially unique\footnote{I.e. unique up to non-zero scalar multiples after domain-restriction to a simple direct summand.} \textit{non-degenerate} and \textit{invariant} symmetric bilinear form that any finite-dimensional semi-simple Lie algebra can be endowed with. This is the famous \textbf{Killing form} (named after Wilhelm Killing) and using it, one is able to more-or-less develop the entire structural and representation theory of these Lie algebras. In particular, using the Killing form, one is able to construct the so-called \textbf{Cartan matrix} (named after \'Elie Cartan), which contains within it a lot - if not even all - of the important information about finite-dimensional semi-simple Lie algebras. For instance, should one know only of a Cartan matrix, one can then write down a presentation in terms of generators and relations (which is due to Claude Chevalley and Jean-Pierre Serre) for a uniquely corresponding finite-dimensional semi-simple Lie algebra. For more details, we refer the reader to \cite{humphreys_lie_algebras} and the first half of \cite{carter_affine_lie_algebras}, and we have also given a brief account of this story in subsection \ref{subsection: finite_dimensional_simple_lie_algebras}.

        Now, one particular property that the Cartan matrix of a finite-dimensional semi-simple Lie algebra is that it is \textit{positive-definite}. However, it is also possible to construct similar matrices that are not necessarily positive-definite, which are nowadays commonly called \textbf{generalised Cartan matrices} (cf. \cite[Chapter 1]{kac_infinite_dimensional_lie_algebras}). Should we then insist nevertheless on writing down presentations \textit{\`a la} Chevalley-Serre, we would obtain certain infinite-dimensional Lie algebras that are nowadays known as \textbf{Kac-Moody algebras} (cf. \cite[Chapter 1]{kac_infinite_dimensional_lie_algebras}). It should be noted right away that generalised Cartan matrices - true to their names - admit the Cartan matrices of finite-dimensional semi-simple Lie algebras as special cases. Consequently, finite-dimensional semi-simple Lie algebras are certain instances of Kac-Moody algebras, and they are commonly referred to as \textbf{finite-type} Kac-Moody algebras when considered in this context.
        
        Amongst the Kac-Moody algebras, those of so-called \textbf{affine type} are of special interest. Specifically, as Robert Moody and Victor Kac independently discovered in the late 1960s and early 1970s, one obtains such Lie algebras by requiring that the associated Cartan matrix be \textit{positive-semi-definite}. What is rather remarkable about these affine Kac-Moody algebras is that they admit something called \textbf{loop realisations}: starting with a finite-dimensional semi-simple Lie algebra:
            $$\g$$
        with Cartan matrix $C$, one can firstly form the \textbf{loop algebra}:
            $$\g[v^{\pm 1}] := \g \tensor_{\bbC} \bbC[v^{\pm 1}]$$
        - which will always be equipped with a non-degenerate and invariant symmetric bilinear form originating naturally from the Killing form - then consider its universal central extension (UCE), which happens to be by a $1$-dimensional centre, which is to say that:
            $$\uce(\g[v^{\pm 1}]) \cong \g[v^{\pm 1}] \oplus \bbC c_{\aff}$$
        where $c_{\aff} \in \uce(\g[v^{\pm 1}])$ is central, and then finally adding on a Lie derivation:
            $$D_{\aff} \in \der( \uce(\g[v^{\pm 1}]) )$$
        with the purpose of it all being so that at the end of the process, one shall obtain a Lie algebra:
            $$\hat{\g} := \uce(\g[v^{\pm 1}]) \rtimes \bbC D_{\aff} \cong (\g[v^{\pm 1}] \oplus \bbC c_{\aff}) \rtimes \bbC D_{\aff}$$
        that is isomorphic to the Kac-Moody algebra whose Cartan matrix is obtained from $C$ by adding one extra row and one extra column in a certain manner (for more details, see \cite[Chapter 7]{kac_infinite_dimensional_lie_algebras}). The idea here is that, because any invariant symmetric bilinear form on $\uce(\g[v^{\pm 1}])$ is necessarily degenerate and with radical containing at least the non-zero central element $c_{\aff}$, one introduces the extra element $D_{\aff}$ to pair non-trivially with $c_{\aff}$, thereby fixing the issue of degeneracy. In other words, $D_{\aff}$ is to be dual to $c_{\aff}$ to begin with. The fact that $D_{\aff}$ is a Lie derivation on $\g[v^{\pm 1}] \oplus \bbC c_{\aff}$ is actually a consequence of this construction.

    \subsection{What is done in this thesis ?}
        Our starting point is not the single-loop algebra $\g[v^{\pm 1}]$ as above, but rather the \textbf{double-loop algebra}\footnote{Which are \textit{not} instances of Kac-Moody algebras!}:
            $$\g[v^{\pm 1}, t^{\pm 1}]$$
        which, like above, will also be equipped with a non-degenerate and invariant symmetric bilinear form originating naturally from the Killing form and depending on a distinguished linear map, a kind of modified formal residue map:
            $$\gamma: \bbC[v^{\pm 1}, t^{\pm 1}] \to \bbC$$
        on which our constructions will depend crucially (see subsection \ref{subsection: definition_of_yangian_extended_toroidal_lie_algebras}). We then again consider the UCE:
            $$\toroidal := \uce(\g[v^{\pm 1}, t^{\pm 1}])$$
        but a large difference in contrast to the affine Kac-Moody situation is that now, the centre $\z(\toroidal)$ is \textit{infinite-dimensional}; luckily though, it is graded, and the graded components are all finite-dimensional. Regardless, the issue whereby invariant symmetric bilinear forms on the UCE must be degenerate persists, which leads us to consider the vector space:
            $$\extendedtoroidal := \toroidal \oplus \z(\toroidal)^{\star}$$
        If we can somehow endow $\extendedtoroidal$ with a Lie algebra structure, then we will be able to speak of a \textit{non-degenerate} and invariant symmetric bilinear form:
            $$(-, -)_{\extendedtoroidal}$$
        that pairs elements of $\z(\toroidal)$ - which were causing the degeneracy issue - with those of its graded dual $\z(\toroidal)^{\star}$ non-trivially. The steps that we will be taking in order to realise this goal are then, in sequence, as follows.
        \begin{enumerate}
            \item Just like above, we shall prove that $\z(\toroidal)^{\star}$ is graded-isomorphic as a vector space to a certain Lie subalgebra $\der_{\gamma}(\bbC[v^{\pm 1}, t^{\pm 1}])$ of the Lie algebra $\der(\bbC[v^{\pm 1}, t^{\pm 1}])$ of derivations on $\bbC[v^{\pm 1}, t^{\pm 1}]$ (where the Lie structure is given by commutators). See definition \ref{def: yangian_div_zero_vector_fields} and proposition \ref{prop: yangian_div_zero_vector_fields_are_graded_dual_to_toroidal_centre}. This allows us to endow any vector space that is isomorphic to $\extendedtoroidal$ with a natural Lie algebra structure, coming from those on $\toroidal$ and on $\der_{\gamma}(\bbC[v^{\pm 1}, t^{\pm 1}])$. 
            \item A \textbf{$\gamma$-extended toroidal Lie algebra} structure shall then be a Lie bracket on $\extendedtoroidal$, with respect to which the non-degenerate bilinear form $(-, -)_{\extendedtoroidal}$ is \textit{invariant} and the UCE $\toroidal$ becomes a Lie subalgebra of $\extendedtoroidal$ (cf. definition \ref{def: yangian_extended_toroidal_lie_algebras}). 
            \item Finally, because $\divzero$ canonically acts on $\toroidal$ (see corollary \ref{coro: a_fixed_yangian_div_zero_vector_field_action}), we can postulate that $\gamma$-extended toroidal Lie algebras ought to arise as certain extensions of the form:
                $$0 \to \toroidal \to \extendedtoroidal \to \divzero \to 0$$
        \end{enumerate}
        
        With all of these ingredients in place, we will be able to establish and subsequently prove the first main theorem of the thesis.
        \begin{theorem}[Imprecise version of theorem \ref{theorem: yangian_extended_toroidal_lie_algebras_main_theorem}]
            A given Lie algebra will be a $\gamma$-extended toroidal Lie algebra if and only if it is isomorphic to a twisted semi-direct product:
                $$\toroidal \rtimes^{\sigma} \der_{\gamma}(\bbC[v^{\pm 1}, t^{\pm 1}])$$
            whose corresponding Lie $2$-cocycle $\sigma \in Z^2_{\Lie}(\der_{\gamma}(\bbC[v^{\pm 1}, t^{\pm 1}]), \toroidal)$ satisfies a certain invariance property that depends on $\gamma$.
        \end{theorem}

        The most important example of a $\gamma$-extended toroidal Lie algebra (at least for us) is the semi-direct product $\toroidal \rtimes \der_{\gamma}(\bbC[v^{\pm 1}, t^{\pm 1}])$, but we will also be able to provide an example of a $\gamma$-extended toroidal Lie algebra that is not isomorphic to this semi-direct product, which will be our second main result (see theorem \ref{theorem: billig_cocycle_main_theorem}) by analysing a particular Lie $2$-cocycle of $\der(\bbC[v^{\pm 1}, t^{\pm 1}])$ with values in $\z(\toroidal)$, which has been known since a paper of Robert Moody and Senapathi Eswara Rao from 1990, namely \cite{moody_rao_n_toroidal_vertex_representations}.

    \subsection{History} \label{subsection: history}
        Even though the above relationship between our construction of $\gamma$-extended toroidal Lie algebras and the older loop realisations of untwisted affine Kac-Moody algebras can indeed serve as a motivation for the main topic of this thesis all on its own, this was actually not how we were lead to considering these $\gamma$-extended toroidal Lie algebras. Originally, we were motivated to consider $\gamma$-extended toroidal Lie algebras because they would help us realise a certain Lie bialgebra structure on:
            $$\toroidal^{\positive} := \uce(\g[v^{\pm 1}, t])$$
        as the classical limit of a quantum group:
            $$\rmY_{\hbar}(\hat{\g})$$
        known as the \textbf{Yangian} associated to the affine Kac-Moody algebra $\hat{\g}$, or simple \say{the} \textbf{affine Yangian} for short, when $\g$ is fixed.
        
        First of all, what is the affine Yangian $\rmY_{\hbar}(\hat{\g})$ ? This is a topological bialgebra over $\bbC[\hbar]$ which deforms the universal enveloping algebra $\rmU(\toroidal^{\positive})$, in the sense that:
            $$\rmY_{\hbar}(\hat{\g})/\hbar \cong \rmU(\toroidal^{\positive})$$
        Per the general theory of quantisastions (as in \cite{etingof_kazhdan_quantisation_1}), the above tells us that $\rmY_{\hbar}(\hat{\g})$ is to quantise a topological \textbf{Lie bialgebra} structure:
            $$\delta^{\positive}: \extendedtoroidal^{\positive} \to \extendedtoroidal^{\positive} \hattensor_{\bbC} \extendedtoroidal^{\positive}$$
        on $\toroidal^{\positive}$; here, $\hattensor_{\bbC}$ denotes a suitable topological completion of the algebraic tensor product, which is necessary due to the appearance of certain infinite sums. Now, although it is certainly possible to endow $\toroidal^{\positive}$ with the Lie bialgebra structure coming directly from a topological bialgebra structure constructed in \cite{guay_nakajima_wendlandt_affine_yangian_coproduct}. However, the problem with this approach is that by the end of the process, it will not be clear whether or not the resulting Lie cobracket $\delta^{\positive}$ originates from the structure of $\toroidal^{\positive}$ itself. In turn, this will lead us towards difficulties in say, identifying the classical $\sfr$-matrix of $\toroidal^{\positive}$ corresponding to the quantum $\sfR$-matrix of the affine Yangian (which has been found in \cite{appel_gautam_wendlandt_R_matrices_of_affine_yangians}).
        
        Now, let us recall that under suitably mild finiteness assumptions, there is a bijective correspondence between so-called \textbf{Manin triples}, which are triples of Lie algebras $(\p, \p^+, \p^-)$ subjected to certain conditions, and isomorphism classes of Lie bialgebra structures on $\p^+$ (and indeed, on $\p$ and $\p^-$) as well (see \cite{etingof_kazhdan_quantisation_1} for more details). One key detail here is that in defining such Manin triples, one is required to supply a \textit{non-degenerate} and \textit{invariant} symmetric bilinear form on $\p$ (sastisfying certain conditions). As elaborated on earlier, any invariant symmetric bilinear form on a UCE such as $\toroidal$ is necessarily degenerate, so we ought not to attempt to directly construct a Manin triple of the form $(\toroidal, \toroidal^{\positive}, \toroidal^{\negative})$. We do, however, now have an extension of $\toroidal$ on which there is a non-degenerate and invariant symmetric bilinear form, namely $\extendedtoroidal$ equipped with $(-, -)_{\extendedtoroidal}$ as constructed above, and so an alternative strategy is to construct a Manin triple of the form:
            $$(\extendedtoroidal, \extendedtoroidal^{\positive}, \extendedtoroidal^{\negative})$$
        (wherein $\extendedtoroidal^{\positive} := \toroidal^{\positive} \rtimes \der_{\gamma}(\bbC[v^{\pm 1}, t])$), which helps us construct the Lie cobracket $\delta^{\positive}$. Specifically, we can rely on the fact that $\toroidal \subset \extendedtoroidal$ is not just a Lie ideal but also a Lie coideal, and hence a Lie sub-bialgebra. 

        As an aside, let us note that the story that we have just outlined above runs in analogy with the more classical story of Drinfeld's construction of the Yangian $\rmY_{\hbar}(\g)$ of a finite-dimensional simple Lie algebra $\g$, as was done in \cite{drinfeld_original_yangian_paper}. This is the bialgebra\footnote{In fact, $\rmY_{\hbar}(\g)$ is a Hopf algebra. Also, even though quantisations are generally not unique up to isomorphism, this feature does hold in the case of $\rmY_{\hbar}(\g)$.} quantising the Lie bialgebra structure on $\g[t]$ that is specified by the Manin triple $(\g[t^{\pm 1}], \g[t], t^{-1}\g[t^{-1}])$. Let us also remark that like in the affine setting where the classical limit is a UCE, $\g[t]$ is in fact also a UCE, namely the trivial UCE of itself (cf. example \ref{example: affine_lie_algebras_centres}).

    \subsection{What have we \textit{not} done ?}
        There are many natural questions that can be posed by the end of this thesis. We have chosen to highlight the following, which is a question that pertains directly to the latter parts of the thesis, particular to theorem \ref{theorem: billig_cocycle_main_theorem}.
        \begin{question}
            How many \say{$\gamma$-invariant} Lie $2$-cocycles (cf. definition \ref{def: yangian_toroidal_cocycles}) are there ? And for that matter, what is $H^2_{\Lie}(\der_{\gamma}(\bbC[v^{\pm 1}, t^{\pm 1}]), \z(\toroidal))$ ?
        \end{question}
        Some inspiration and guidance can perhaps be taken from \cite{billig_neeb_vector_field_cyclic_cohomology_parallelisable_manifolds}, where the authors have investigated the Lie algebra cohomology ($H^2_{\Lie}$, in particular) of the Lie algebra of all $C^{\infty}$-vector fields on a parallelisable smooth compact manifold $M$ with coefficients in the global section of $\Omega^p_M/d( \Omega^{p - 1}_M )$ (as we will see via theorem \ref{theorem: kassel_realisation}, $\z(\toroidal) \cong \Omega^1_{ \bbC[v^{\pm 1}, t^{\pm 1}]/\bbC }/d \bbC[v^{\pm 1}, t^{\pm 1}]$). That said, there is still much work to be done.

        Otherwise, we can continue pursuing the original goal of computing the classical limit of affine Yangians immediately after this thesis as well, and even though we have a somewhat detailed sketch of the proof at this point, there remain some difficulties at present. For instance, usually if one would like to prove that a topological bialgebra $(Y, \Delta)$ quantises a topological Lie bialgebra $(\fraku, \delta)$, then one would verify that the following equation holds true for all $x \in \fraku$ and any lift $\tilde{x} \in Y$ thereof:
            $$\delta(x) \equiv \frac{1}{\hbar}(\Delta - \Delta^{\cop})(\tilde{x}) \pmod{\hbar}$$
        However, it is still not known whether or not $\rmY_{\hbar}(\hat{\g})$ is torsion-free as a $\bbC[\hbar]$-module for all $\g$, even for those $\g$ for which $\rmY_{\hbar}(\hat{\g})$ is known to carry a topological bialgebra structure (see \cite[Section 5]{guay_nakajima_wendlandt_affine_yangian_coproduct}), meaning that the equation above might not make sense for all $\g$. When $\g$ is simply-laced, though, $\rmY_{\hbar}(\hat{\g})$ is known to be torsion-free (see \cite[Section 6]{guay_regelskis_wendlandt_affine_yangian_vertex_representations_and_PBW}). Also, for representation-theoretic purposes, one usually would also want $\rmY_{\hbar}(\hat{\g})$ to admit some sort of PBW basis. This is so that the isomorphism:
            $$\rmY_{\hbar}(\hat{\g})/\hbar \cong \rmU(\toroidal^{\positive})$$
        to be one between \say{PBW algebras} (cf. e.g. \cite[Example 0.6]{braverman_gaitsgory_PBW_for_koszul_quadratic_algebras}).

        That said, we do now believe that there remain no serious technical barriers preventing the verification of the following slightly different identity:
            $$\delta(x) \equiv \frac{1}{\hbar}(\Delta - \Delta^{\cop})(\tilde{x}) \pmod{\hbar^2}$$
        for all $x \in \toroidal^{\positive}$ and all lifts $\tilde{x} \in \rmY_{\hbar}(\hat{\g})$ thereof (now, this is a lift modulo $\hbar^2$ instead of modulo $\hbar$). 

        Finally, let us make some very brief comments on why we have chosen to approach $\gamma$-extended toroidal Lie algebras form the loop point of view, and how this perspective of ours compares to the construction of \textbf{extended affine Lie algebras} (or \textbf{EALAs} for short) of Neher et al. (see e.g. \cite{neher_lectures_on_EALAs}).

        EALAs in the sense of Neher et al. are defined in a style similar to how Kac-Moody algebras attached to a Cartan matrix are defined in say, \cite[Chapters 1 and 2]{kac_infinite_dimensional_lie_algebras}. Slightly more specifically, they are designed to be Lie algebras supporting a notion of Cartan subalgebras and subsequent notions of root space decompositions, root systems, etc. (see \cite[Subsection 2.1]{neher_lectures_on_EALAs}), and one then proves that under some hypotheses, such algebras can be realised as certain extensions of Lie algebras of derivations by UCEs of multiloop algebras (see \cite{allison_berman_faulkner_pianzola_multiloop_realisation_of_EALAs}). We believe that our notion of $\gamma$-extended toroidal Lie algebras and the notion of EALAs of so-called \say{nullity} $2$ that were considered in the aforementioned work do not coincide, ultimately because the bilinear forms on the former are not of total degree $0$ (by construction), while those on the latter are (also by construction). Some further effort will have to be spent on analysing the structure of $\gamma$-extended toroidal Lie algebras, e.g. we should inquire into whether or not they may admit root space decompositions, and if there is a reasonable notion of Cartan subalgebras in this setting, etc.
        
        In any event, loop realisations have historically shown themselves to be very useful in practice, both when one seeks an understanding the structures of infinite-dimensional Lie algebras, as well as when one seeks applications of said Lie algebras (often to physics). For instance, they are useful for showing that all affine Kac-Moody algebras arise as \say{twists}\footnote{In the sense of \cite[Chapter 8]{kac_infinite_dimensional_lie_algebras}.} of the untwisted ones, and how these twists can be classified in terms of Galois cohomology (see \cite{pianzola_vanishing_of_H1_of_dedekind_rings} and its sequels\footnote{... and our thanks to A. Pianzola for letting us know of such results!}). Another example of how loop realisations are useful, this time in mathematical physics, is how they allow us to write down so-called \say{vertex representations}, i.e. representations with \say{vertex algebra} structures, which have proven themselves to be very valuable for studying infinite-dimensional Lie algebras; see, for example, \cite[Chapter 14]{kac_infinite_dimensional_lie_algebras} and \cite{berman_billig_szmigielski_VOAs_and_toroidal_lie_algebras}, and also \cite{guay_regelskis_wendlandt_affine_yangian_vertex_representations_and_PBW} for an example of how these vertex representations might still be useful even if one is only concerned with the original motivation of studying affine Yangians. This is a difficult topic, so for the sake of brevity and conciseness, we shall only be making this brief mention of it.
        
        For such reasons, we believe that the objects of main interest in this thesis, which arise from a double-loop realisation, are worthy of attention for reasons beyond their original motivation that was discussed in subsection \ref{subsection: history} above.

        \newpage

        \section{The structure of the thesis}
    We would also like to provide the reader with a short reading guide for the thesis.
    
    In chapter \ref{chapter: kassel_UCEs}, we begin by recalling some background information on perfect Lie algebras and on Lie algebra extensions, and then we will study UCEs in some detail, recalling in particular a realisation of Kassel, whereby centres of UCEs of current algebras (in the sense of definition \ref{def: current_algebras}) can be described in terms of algebraic differential $1$-forms modulo exact forms (cf. theorem \ref{theorem: kassel_realisation}). We will then end the chapter by analysing, using the aforementioned theorem, particular UCEs that will be of interest to us for the rest of the thesis (see, in particular, example \ref{example: toroidal_lie_algebras_centres}).

    Chapter \ref{chapter: yangian_EALAs} will begin with the construction of the technical ingredients necessary for defining $\gamma$-extended toroidal Lie algebras, and then of those Lie algebras themselves; the procedure will be as outlined in the previous section, and will be outlined in further details in the sections in chapter \ref{chapter: yangian_EALAs}; the main result of this portion of the chapter will be theorem \ref{theorem: yangian_extended_toroidal_lie_algebras_main_theorem}. Afterwards, we will be making some quick remarks about some structural features of $\gamma$-extended toroidal Lie algebras, such as an identification of its centre and intriguingly, a homomorphic image of the Witt algebra inside $\der_{\gamma}(\bbC[v^{\pm 1}, t^{\pm 1}])$ (see propositions \ref{prop: centres_of_yangian_extended_toroidal_lie_algebras} and \ref{prop: a_copy_of_the_witt_algebra_inside_the_lie_algebra_of_yangian_div_zero_vector_fields}, respectively). Finally, we will attempt to provide an example of a $\gamma$-extended toroidal Lie algebra that is \textit{not} isomorphic to $\toroidal \rtimes \der_{\gamma}(\bbC[v^{\pm 1}, t^{\pm 1}])$ by analysing a particular Lie $2$-cocycle known from \cite{billig_energy_momentum_tensor}; this will be done through theorem \ref{theorem: billig_cocycle_main_theorem}.

    \newpage
    
    \chapter{Kassel's characterisation of UCEs of current algebras}
        \begin{abstract}
            The purpose of this chapter is to recall some relevant features of the theory of universal central extensions (UCEs) of so-called \say{perfect Lie algebras}. Of special interest to us is Kassel's realisation of UCEs of Lie algebras of the kind $\g \tensor_{\bbC} A$, where $\g$ is a finite-dimensional simple Lie algebra over $\bbC$, and $A$ is a commutative $\bbC$-algebra (cf. theorem \ref{theorem: kassel_realisation}). Along the way, we will also recall some features of the structures of finite-dimensional simple Lie algebras, as well as some generalities about extensions of Lie algebras.
            
            We require this theory in order to be able to explicitly compute bases for the centre of the UCE of $\g[v^{\pm 1}, t^{\pm 1}]$, which is necessary for constructing of \say{$\gamma$-extended toroidal Lie algebras} in chapter \ref{chapter: yangian_EALAs}. 
        \end{abstract}

        \minitoc

        \newpage
    
        \section{Some generalities on Lie algebras}
    \subsection{Structure of finite-dimensional simple Lie algebras}
        As a precursor to our main discussion, let us recall some features of the theory of finite-dimensional simple Lie algebras, particularly about their structure.

        \begin{definition}[Simple Lie algebras]
            A Lie algebra over an arbitrary commutative ring $k$ is said to be \textbf{simple} if and only if it admits no non-zero Lie ideals. 
        \end{definition}

        Over a field $k$ that is algebraically closed and of characteristic $0$, much is known about the structure of a simple Lie algebra $\g$ that is finite-dimensional when regarded as a $k$-vector space. The bulk of the content presented above is discussed in further details in any standard textbook on Lie algebras (cf. e.g. \cite{humphreys_lie_algebras} or the first half of \cite{carter_affine_lie_algebras}). Let us give a very brief recap of this theory.

        Firsly, one of the most important features of finite-dimensional simple Lie algebras (henceforth implicitly understood to be defined over a characteristic-$0$ and algebraically closed field $k$) is that each such Lie algebra, say $\g$, posses an invariant and non-degenerate $k$-bilinear form:
            $$(-, -)_{\g}$$
        which is unique up to $k^{\x}$-multiples. The canonical choice is the so-called Killing form, given by $\kappa(x, y) := \trace(\ad(x) \circ \ad(y))$ for all $x, y \in \g$, but in various other context, other natural choices such as the trace form $\trace(xy)$ are also very useful. What is important to us is that the Killing form is essentially unique: if $\kappa'$ is any invariant and non-degenerate symmetric $k$-bilinear form on $\g$ then there will exist a \textit{unique} $c \in k^{\x}$ such that $\kappa' = c \kappa$.

        Now, such a bilinear form $(-, -)_{\g}$ allows us to construct a natural grading of any simple Lie algebra $\g$ by its \say{root lattice} (to be defined shortly). One begins this process by choosing a \textbf{Cartan subalgebra} $\h$ for $\g$, which is a maximal abelian Lie subalgebra\footnote{... and it is well-known that all Cartan subalgebras of $\g$ are conjugate to one another, so this choice does not matter.} of $\g$. The restriction of $(-, -)_{\g}$ to $\h$ remains non-degenerate, so we can canonically identify $\h \xrightarrow[]{\cong} \h^*$ via said bilinear form. By picking a basis for our choice of a Catan subalgebra $\h$ and hence a dual basis:
            $$\simpleroots := \{\alpha_i\}_{i \in \simpleroots}$$
        for $\h^*$, which we shall regard as a choice of a set of \textbf{simple roots}, and writing our bilinear form $(-, -)_{\g}$ in terms of that matrix, we shall get the \textbf{Cartan matrix} of $\g$:
            $$C := (c_{ij})_{i, j \in \simpleroots}$$
        The aforementioned \textbf{root lattice} is then given by:
            $$Q := \Z \simpleroots$$
        Given an element:
            $$\mu := \sum_{i \in \simpleroots} m_i \alpha_i \in Q$$
        we define its \textbf{height} to be the sum of the coefficients:
            $$\height \mu := \sum_{i \in \simpleroots} m_i$$
        It can also be shown that $2\id - C$ is the adjacency matrix of an undirected graph without loops, called the \textbf{Dynkin diagram} of $\g$, and the \textbf{roots} of $\g$ are the roots of this Dynkin diagram.
            
        Let $V$ be a $\g$-module. Then, one can abstractly define the vector subspace of $V$ consisting of elements of \textbf{weight} $\mu \in \h^*$ to be:
            $$V_{\mu} := \{v \in V \mid \forall h \in \h: h \cdot v = \mu(h) v\}$$
        If we have a direct sum decomposition of $\h$-module:
            $$V \cong \bigoplus_{\mu \in \h^*} V_{\mu}$$
        then we will say that $V$ is a \textbf{weight module} for $\g$. Interestingly, elements of $\g_{\alpha}$ (with $\g$ acting on itself by the adjoint action) act by raising/lowering the weights of elements of $\g$-modules $V$ in the sense that:
            $$\g_{\alpha} \cdot V_{\mu} \subseteq V_{\mu + \alpha}$$
        for all weights $\alpha, \mu \in \h^*$. 
        
        As it turns out, the $Q$-grading of $\g$ that was mentioned earlier is actually a weight space decomposition for $\g$, regarded as a module over itself via the adjoint representation.
        \begin{theorem}[Root space decomposition for finite-dimensional simple Lie algebras] \label{theorem: root_space_decomposition_for_finite_dimensional_simple_lie_algebras}
            Let $\g$ be a module over itself via the adjoint representation.
            \begin{enumerate}
                \item $\g$ is a weight module over itself. As a particular consequence, $\g$ is graded by the abelian group $Q$: for every $x \in \g_{\alpha}$, one has that $\deg x = \height \alpha$.
                \item The weight space $\g_0$ is nothing but the Cartan subalgebra $\h$.
                \item For each non-zero weight $\alpha$ of this $\g$-module, $\dim_k \g_{\alpha} = 1$.
            \end{enumerate}
        \end{theorem}
        Typically, the non-zero weights $\alpha$ of the adjoint representations of $\g$ such that $\g_{\alpha} \not \cong 0$ are called \textbf{roots}. The set of roots is usually denoted by $\Phi$. These are the same as the roots that can be constructed from the Cartan matrix of $\g$.

        Elements of $Q^+ := \Z_{\geq 0} \simpleroots$ are typically regarded as being \textbf{positive} (and in particular, the simple roots are positive by convention) and conversely, elements of $Q^- := \Z_{\leq 0} \simpleroots$ are typically said to be \textbf{negative}. One can subsequently construct the sets of positive/negative roots as:
            $$\Phi^{\pm} := \Phi \cap Q^{\pm}$$
        and note that $\Phi = \Phi^+ \cup \Phi^-$.

        From theorem \ref{theorem: root_space_decomposition_for_finite_dimensional_simple_lie_algebras}, we see that for any given positive root $\alpha \in \Phi^+$ and corresponding choice of root vectors\footnote{Choices of which are unique up to non-zero scalar multiples, since subspaces of non-zero weights are equally $1$-dimensional.} $x_{\pm\alpha} \in \g_{\pm\alpha}$, one has that:
            $$[x_{\alpha}, x_{-\alpha}] = \frac12 (x_{\alpha}, x_{-\alpha})_{\g} \check{\alpha}$$
        where $\check{\alpha} \in \h$ is such that $(\alpha, \check{\alpha})_{\g} = 2$. For simplicity, the root vectors $x_{\pm \alpha}$ are typically chosen so that $(x_{\alpha}, x_{-\alpha})_{\g} = 2$.
        
        The next result is a fundamental theorem in the study of finite-dimensional simple Lie algebras over algebraically closed fields of characteristic $0$. It essentially asserts that to give such a Lie algebra via a presentation by generators and relations is the same as to specify its Cartan matrix. The result is not only practically useful, but also is the mean by which one approaches Kac-Moody algebras, where the Cartan matrix is no longer required to be positive-definite (cf. \cite[Chapters 1-5]{kac_infinite_dimensional_lie_algebras}). 
        \begin{theorem}[Serre's Theorem]
            $\g$ is isomorphic to the Lie algebra generated by the set:
                $$\{h_i, x_i^{\pm}\}_{1 \leq i \leq l}$$
            whose elements are subjected to the following relations, given for all $1 \leq i, j \leq l$:
                $$[h_i, h_j] = 0$$
                $$[h_i, x_j^{\pm}] = \pm c_{ij} x_j^{\pm}, [x_i^+, x_j^-] = \delta_{ij} h_i$$
                $$\ad(x_i^{\pm})^{1 - c_{ij}}(x_j^{\pm}) = 0$$
            This is usually referred to as the \textbf{Chevalley-Serre} presentation for $\g$; final set of relations is usually known as the \textbf{Serre relations}.
        \end{theorem}

        We end this subsection with a brief analysis of the easiest possible example of a finite-dimensional simple Lie algebra. 
        \begin{example}[$\sl_2$]
            Recall that $\sl_2(k)$ is the kernel of the trace map:
                $$\trace: \gl_2(k) \to k$$
            i.e. it is the Lie algebra of trace-zero $2 \x 2$-matrices whose Lie bracket is the usual commutator of matrices. It is of dimension $\dim_k \gl_2(k) - \dim_k k = 4 - 1 = 3$, and happens to be also generated by a set of cardinality $3$ (though this is a coincidence, due entirely to how \say{degenerate} of an example $\sl_2(k)$ is):
                $$\{h, x^+, x^-\}$$
            and the elements of this set are subjected to the relations:
                $$[h, x^{\pm}] = \pm 2 x^{\pm}, [x^+, x^-] = h$$
            One proves both of these statements by showing firstly that a basis for $\sl_2(k)$ is:
                $$\left\{ \begin{pmatrix} 1 & 0 \\ 0 & -1 \end{pmatrix}, \begin{pmatrix} 0 & 1 \\ 0 & 0 \end{pmatrix}, \begin{pmatrix} 0 & 0 \\ 1 & 0 \end{pmatrix} \right\}$$
            and then seeing that any Cartan subalgebra of $\sl_2(k)$ must therefore be isomorphic to $k \begin{pmatrix} 1 & 0 \\ 0 & -1 \end{pmatrix}$; the rest then follows. In particular, the Cartan matrix is just:
                $$\begin{pmatrix} 2 \end{pmatrix}$$
            and the Dynkin diagram consists of only a single vertex and no edges:
                $$\bullet$$
        \end{example}

    \subsection{Perfect Lie algebras and their central extensions}
        \begin{definition}[Extensions of Lie algebras]
            Let $k$ be a commutative ring and $\a$ be a Lie algebra over $k$. An \textbf{extension} of $\a$ by another Lie algebra $\z$ over $k$ is an extension of $k$-modules:
                $$0 \to \z \to \frake \xrightarrow[]{\pi} \a \to 0$$
            such that $\frake$ is also a Lie algebra over $k$. A morphism of extensions of a Lie algebra $\a$ by another Lie algebra $\z$ is a morphism of short exact sequences of Lie algebras:
                $$
                    \begin{tikzcd}
                	&& {\frake'} \\
                	0 & \z & \frake & \a & 0
                	\arrow["\varphi", from=1-3, to=2-3]
                	\arrow["{\pi'}", two heads, from=1-3, to=2-4]
                	\arrow["\pi", two heads, from=2-3, to=2-4]
                	\arrow[from=2-2, to=2-3]
                	\arrow[from=2-2, to=1-3]
                	\arrow[from=2-1, to=2-2]
                	\arrow[from=2-4, to=2-5]
                    \end{tikzcd}
                $$
            An isomorphism of extension occurs if $\varphi: \frake' \to \frake$ is a Lie algebra isomorphism. An extension:
                $$0 \to \z \to \tilde{\a} \to \a \to 0$$
            of $\a$ by $\z$ is said to be \textbf{universal} if it is initial amongst all such Lie algebra extensions, in the sense that for every other extension:
                $$0 \to \z \to \frake \to \a \to 0$$
            there must exist a unique Lie algebra homomorphism $\tilde{\a} \to \frake$ fitting into the following morphism of extensions:
                $$
                    \begin{tikzcd}
                	&& {\tilde{\a}} \\
                	0 & \z & \frake & \a & 0
                	\arrow[dashed, from=1-3, to=2-3]
                	\arrow[two heads, from=1-3, to=2-4]
                	\arrow[two heads, from=2-3, to=2-4]
                	\arrow[from=2-2, to=2-3]
                	\arrow[from=2-2, to=1-3]
                	\arrow[from=2-1, to=2-2]
                	\arrow[from=2-4, to=2-5]
                    \end{tikzcd}
                $$
        \end{definition}
        \begin{remark}
            Universal extensions are unique up to unique isomorphisms. 
        \end{remark}
        \begin{remark}[Lie brackets on extensions]
            Suppose that:
                $$0 \to \z \to \frake \to \a \to 0$$
            is an extension of Lie algebras. Then, up to a choice of so-called \say{$2$-cocycle} $\sigma$, which can be regarded as a $k$-linear map:
                $$\sigma: \bigwedge^2 \a \to \z$$:
            satisfying a certain condition, the Lie bracket on $\frake$ will be given by:
                $$\forall X, Y \in \a: \forall K, K' \in \z: [ (X, K), (Y, K') ]_{\frake} := [X, Y]_{\a} + ( K \cdot Y - K' \cdot Y + \sigma(X, Y) )$$
            The aforementioned condition is that the bracket operation $[-, -]_{\frake}$ satisfies the Jacobi identity (note that it is already bilinear and skew-symmetric by construction).
        \end{remark}
        \begin{example}[Semi-direct products of Lie algebras]
            Let $\d$ be a Lie algebra acting on another Lie algebra $\z$, i.e. let $\t$ be a $\d$-module that happens also to be a Lie algebra over $k$. The canonical extension of $\d$ by $\z$ corresponding to the $2$-cocycle:
                $$\sigma = 0$$
            is known as the \textbf{semi-direct product} of $\d$ by $\z$, and commonly denoted by:
                $$\z \rtimes \d$$
            For the sake of completeness, let us note that the Lie bracket here is given by:
                $$\forall D, D' \in \d: \forall K, K' \in \z: [ (D, K), (D', K') ]_{\z \rtimes \d} := [D, D']_{\d} + ( D \cdot K' - D' \cdot K )$$
        \end{example}
        \begin{definition}[Central extensions of Lie algebras]
            A Lie algebra extension:
                $$0 \to \z \to \frake \to \a \to 0$$
            is \textbf{central} if the elements of $\z$ are central in $\frake$. A \textbf{universal central extension} (UCE) of a Lie algebra $\a$ by another Lie algebra $\z$, typically denoted by $\uce(\a)$, is an extension that is initial amongst all such central extensions.
        \end{definition}

        A particular class of Lie algebras that happen to admit UCEs are the so-called \say{perfect} ones.
        \begin{definition}[Perfect Lie algebras]
            A Lie algebra over a commutative ring is said to be \textbf{perfect} if it is equal to its derived subalgebra. 
        \end{definition}
        \begin{example}
            Since simple Lie algebras lack non-zero ideals by definition, they are perfect. 
        \end{example}
        \begin{example}
            Let $A$ be any commutative algebra over some base commutative ring $k$, and let $\g$ be a simple Lie algebra over $k$. Endow the $k$-module $\g_A := \g \tensor_k A$ with the Lie bracket:
                $$\forall x, y \in \g: \forall f, g \in A: [x f, y g]_{\g_A} := [x, y]_{\g} fg$$
            Then, $\g_A$ will be perfect when regraded as a Lie algebra over $k$, precisely because $\g$ is simple.
        \end{example}
        \begin{example}
            Counter-examples include nilpotent and abelian Lie algebras. For the former, their derived subalgebras are always proper Lie subalgebras, while for the latter, their derived subalgebras are zero. 
        \end{example}
        \begin{proposition}[Perfect Lie algebras admit UCEs] \label{prop: perfect_lie_algebras_admit_UCEs}
            \cite[Lemma 1.10]{garland_arithmetics_of_loop_groups} If $k$ is a field and $\a$ is a perfect Lie algebra over $k$, then it will admit a UCE.
        \end{proposition}
        
        \newpage
        
        \section{The Kassel realisation of UCEs of current Lie algebras}
    \subsection{A recollection of K\"ahler differentials}
        In \cite{kassel_universal_central_extensions_of_lie_algebras}, Kassel showed that centres of UCEs of current algebras (i.e. Lie algebras of the form $\g \tensor_k A$ for some commutative $k$-algebra $A$) are identified with spaces of algebraic differential $1$-forms modulo exact forms, so before proceeding, let us take the time to recall some relevant details about algebraic differential forms. 

        There are many approaches to algebraic differential forms. Ultimately, however, we will be relying on the description of modules of differential forms by generators and relations\footnote{One reason is that this makes it clear how, should a commutative $R$-algebra $S$ be graded by some abelian group $Z$, then that grading will induce a $Z$-grading on $\Omega^1_{S/R}$ as well (see remark \ref{remark: Z_gradings_on_toroidal_lie_algebras}).}, so let us define them that way.
        \begin{definition}[Modules of K\"ahler differentials] \label{def: kahler_differentials}
            Let $R$ be a base commutative ring and let $S$ be a commutative $R$-algebra. The $S$-module of K\"ahler differentials $\Omega^1_{S/R}$ relative to the ring map $R \to S$ is then the quotient of the $S$-module $S \tensor_R S$ by the $S$-submodule generated by the relations:
                $$ss' \tensor 1 - s' \tensor s - s \tensor s'$$
            given for all $s, s' \in S$
        \end{definition}
        \begin{remark}[Diffentials and derivations] \label{remark: differentials_and_derivations}
            The definition suggests to us that K\"ahler differentials might have something to do with derivations, and indeed they do. In fact, this relationship between algebraic $1$-forms and derivations comes from a universal property that the $S$-module $\Omega^1_{S/R}$ enjoys. Namely, for any $R$-module $M$, there exists a natural isomorphism of $S$-modules\footnote{The LHS is the $S$-module of $R$-linear derivations from $S$ into $M$.}:
                $$\Der_R(S, M) \cong \Hom_S(\Omega^1_{S/R}, M)$$
            (cf. \cite[\href{https://stacks.math.columbia.edu/tag/00RO}{Tag 00RO}]{stacks}). From this universal property, one infers that $\Omega^1_{S/R}$ is isomorphic to the $S$-module generated by the set:
                $$\{ds\}_{s \in S}$$
            whose elements are constrained by the relations:
                $$d(ss') = s' ds + s ds'$$
            given for all $s, s' \in S$. The isomorphism in question is given by:
                $$ds \mapsto s \tensor 1$$
            for all $s \in S$.
            
            A particular instance of this phenomenon is that $R$-linear derivations from $S$ to itself are dual to differential $1$-forms relative to $R \to S$, in the sense that there is an $S$-module isomorphism:
                $$\Der_R(S) := \Der_R(S, S) \cong \Hom_A(\Omega^1_{S/R}, A)$$
        \end{remark}
        \begin{remark}
            If $\Omega^1_{S/R}$ is finite free of rank $n$ over $A$, e.g.:
                $$\Omega^1_{S/R} \cong \bigoplus_{1 \leq i \leq n} A dv_i$$
            then we can identify:
                $$\Der_R(S) \cong \bigoplus_{1 \leq i \leq n} A \del_{v_i}$$
            where $\del_{v_i} \in \Der_R(S)$ are the preimages under the isomorphism $\Der_R(S) \xrightarrow[]{\cong} \Hom_S(\Omega^1_{S/R}, S)$ of the $S$-linear duals of the generators $dv_i \in \Omega^1_{S/R}$. 
        \end{remark}
        
        The following well-known lemmas are very useful. Proofs can be be found in any standard reference on general commutative algebra (e.g. \cite[\href{https://stacks.math.columbia.edu/tag/00AO}{Tag 00AO}]{stacks}).
        \begin{lemma}[$1$-forms over polynomial algebras] \label{lemma: 1_forms_over_polynomial_algebras}
            \cite[\href{https://stacks.math.columbia.edu/tag/00RX}{Tag 00RX}]{stacks} Let $R$ be a commutative ring and fix some $n \in \Z_{\geq 0}$. In this case, $\Omega^1_{R[v_1, ..., v_n]/R}$ will be free and of finite rank $n$ as an $R[v_1, ..., v_n]$-module; in particular, it admits the set $\{dv_1, ..., dv_n\}$ as a $R[v_1, ..., v_n]$-linear basis.
        \end{lemma}
        \begin{lemma}[Localisation of $1$-forms] \label{lemma: localisation_1_forms}
            \cite[\href{https://stacks.math.columbia.edu/tag/031G}{Tag 031G}]{stacks} Let $k$ be a field\footnote{... so that the only prime ideal of $k$ would be $(0)$.} and fix some $n \in \Z_{\geq 0}$, and consider the canonical ring homomorphism $k \to k[v_1, ..., v_n]$. Then, for any $1 \leq i \leq n$, there will be a $k[v_1, ..., v_n][v_i^{-1}]$-module isomorphism:
                $$\Omega^1_{k[v_1, ..., v_n][v_i^{-1}]/k} \cong \Omega^1_{k[v_1, ..., v_n]/k}[v_i^{-1}]$$
        \end{lemma}

        Let us end this subsection with the following examples, which will be useful for what comes later on.
        \begin{example}
            Let $k$ be a field.
        
            Per lemma \ref{lemma: 1_forms_over_polynomial_algebras}, we know that:
                $$\Omega^1_{k[v, t]/k} \cong k[v, t] dv \oplus k[v, t] dv$$
            Using lemma \ref{lemma: localisation_1_forms}, we then see that:
                $$\Omega^1_{k[v^{\pm 1}, t^{\pm 1}]/k} \cong k[v^{\pm 1}, t^{\pm 1}] dv \oplus k[v^{\pm 1}, t^{\pm 1}] dt$$
                $$\Omega^1_{k[v^{\pm 1}, t]/k} \cong k[v^{\pm 1}, t] dv \oplus k[v^{\pm 1}, t] dt$$
        \end{example}

    \subsection{UCEs of current Lie algebras}
        \begin{convention} \label{conv: a_fixed_finite_dimensional_simple_lie_algebra}
            From now on, we fix a finite-dimensional simple Lie algebra $\g$ over an algebraically closed field $k$ of characteristic $0$, equipped with a symmetric and non-degenerate invariant $k$-bilinear form $(-, -)_{\g}$. We fix also a Cartan subalgebra $\h$ of $\g$, along with all the accompanying data (e.g. root system, Cartan matrix, etc.) as in subsection \ref{subsection: finite_dimensional_simple_lie_algebras}. 
        \end{convention}

        For the sake of establishing the terminology, let us make the following definition:
        \begin{definition}[Current algebras] \label{def: current_algebras}
            Let $A$ be a commutative algebra over $k$. The vector space:
                $$\g \tensor_k A$$
            with the following Lie bracket:
                $$[x f, y g]_{\g \tensor_k A} := [x, y]_{\g} \tensor fg$$
            (given for all $x, y \in \g$ and all $f, g \in A$) shall then be referred to as a \textbf{current algebra}. 
                
            Also, we will be abbreviating:
                $$xf := x \tensor f$$
            for $x \in \g$ and $f \in A$.
        \end{definition}
        \begin{remark}[(Multi)loop algebras]
            When $A \cong k[v_1^{\pm 1}, ..., v_n^{\pm 1}]$, it is common to refer to $\g \tensor_k A$ as a \textbf{multiloop algebra}. When $n = 1$, we will only be saying \textbf{loop algebra}.
        \end{remark}

        \begin{convention}
            Let $R \to S$ be a homomorphism of commutative rings. Then, let us write:
                $$\bar{\Omega}^1_{S/R} := \Omega^1_{S/R}/dS$$
            Note that this is only an $R$-module, not an $S$-module. Let us also write $\bar{d}: S \to \bar{\Omega}^1_{S/R}$ for the canonical composition:
                $$
                    \begin{tikzcd}
                	S & {\Omega^1_{S/R}} \\
                	& {\bar{\Omega}^1_{S/R}}
                	\arrow["d", from=1-1, to=1-2]
                	\arrow[two heads, from=1-2, to=2-2]
                	\arrow["{\bar{d}}"', from=1-1, to=2-2]
                    \end{tikzcd}
                $$
        \end{convention}

        In order to characterise centres of UCE Kassel constructed in the proof of \cite[Theorem 3.3(iii)]{kassel_universal_central_extensions_of_lie_algebras} a $k$-linear map:
            $$\e: \bigwedge^2 (\g \tensor_k A) \to \bar{\Omega}^1_{A/k}$$
        by the formula\footnote{One can also take $\e(x f, y g) := -(x, y)_{\g} f \bar{d}g$, since $-f \bar{d}g \equiv g \bar{d}f \pmod{d(A)}$. This results in an isomorphic Lie algebra. We will switch back and forth between these choices depending on necessity.}:
            $$\e(x f, y g) := (x, y)_{\g} g \bar{d}f$$
        for all $x, y \in \g$ and for all $f, g \in A$. This can be shown - relying on the $\g$-invariance of the bilinear form $(-, -)_{\g}$ - to be a $2$-cocycle of $\g \tensor_k A$ with coefficients in $\bar{\Omega}^1_{A/k}$\footnote{... i.e. a representative of an isomorphism class $[\e] \in H^2_{\Lie}(\g \tensor_k A, \bar{\Omega}^1_{A/k})$.} and hence gives a central extension $\fraku$ of $\g \tensor_k A$ by $\bar{\Omega}_{A/k}^1$ (cf. proposition \ref{prop: lie_brackets_on_extensions}), whose underlying $k$-vector space is:
            $$(\g \tensor_k A) \oplus \bar{\Omega}_{A/k}^1$$
        and whose Lie bracket is:
            $$[-, -]_{\fraku} := [-, -]_{\g \tensor_k A} + \e$$
        \begin{lemma} \label{lemma: lie_brackets_on_UCEs_of_current_algebras}
            $[-, -]_{\fraku}$ as constructed above is a well-defined Lie bracket.
        \end{lemma}
            \begin{proof}
                By construction, it is already bilinear and skew-symmetric, so the only thing to show is that it satisfies the Jacobi identity. To this end, pick $x, y, z \in \g$ and $f, g, h \in A$ and then consider the following\footnote{We chose $\e(xf, yg) := (x, y)_{\g} g \bar{d}f$ because it makes the verification that $\e$ satisfies the Jacobi identity easier.} computations in $\bar{\Omega}^1_{A/k}$:
                    $$
                        \begin{aligned}
                            & [xf, [yg, zh]_{\fraku}]_{\fraku} + [yg, [zh, xf]_{\fraku}]_{\fraku} + [zh, [xf, yg]_{\fraku}]_{\fraku}
                            \\
                            & = [xf, [y, z]_{\g} gh + \e(yg, zh)]_{\fraku} + [yg, [z, x]_{\g} hf + \e(zh, xf)]_{\fraku} + [zh, [x, y]_{\g} fg + \e(xf, yg)]_{\fraku}
                            \\
                            & = [xf, [y, z]_{\g} gh]_{\fraku} + [yg, [z, x]_{\g} hf]_{\fraku} + [zh, [x, y]_{\g} fg]_{\fraku}
                            \\
                            & = 
                            \begin{aligned}
                                & \left( [x, [y, z]_{\g}]_{\g} fgh + \e(xf, [y, z]_{\g} gh) \right)
                                \\
                                + & \left( [y, [z, x]_{\g}]_{\g} ghf + \e(yg, [z, x]_{\g} hf) \right)
                                \\
                                + & \left( [z, [x, y]_{\g}]_{\g} hfg + \e(zh, [x, y]_{\g} fg) \right)
                            \end{aligned}
                            \\
                            & = \e(xf, [y, z]_{\g} gh) + \e(yg, [z, x]_{\g} hf) + \e(zh, [x, y]_{\g} fg)
                            \\
                            & = (x, [y, z]_{\g})_{\g} gh \bar{d}f + (y, [z, x]_{\g})_{\g} hf \bar{d}g + (z, [x, y]_{\g})_{\g} fg \bar{d}h
                            \\
                            & = (x, [y, z]_{\g})_{\g} ( gh \bar{d}f + hf \bar{d}g + fg\bar{d}h )
                            \\
                            & = 0
                        \end{aligned}
                    $$
                The fourth equality comes from the fact that $\g$ is a Lie algebra, and hence any triple of elements $x, y, z \in \g$ therein satisfies the Jacobi identity:
                    $$[x, [y, z]_{\g}]_{\g} + [y, [z, x]_{\g}]_{\g} + [z, [x, y]_{\g}]_{\g} = 0$$
                The last equality came from the fact that:
                    $$gh df + hf dg + fg dh = d(fgh)$$
                in $\Omega^1_{A/k}$, per definition \ref{def: kahler_differentials} (see also remark \ref{remark: differentials_and_derivations}), which hence implies that:
                    $$gh \bar{d}f + hf \bar{d}g + fg \bar{d}h \equiv 0 \pmod{d(A)}$$
                in $\bar{\Omega}^1_{A/k}$.
            \end{proof}
        Kassel then showed that the Lie algebra $\fraku$ as above is a UCE of $\g \tensor_k A$. 
        \begin{theorem}[The Kassel realisation] \label{theorem: kassel_realisation}
            \cite[Corollary 3.5]{kassel_universal_central_extensions_of_lie_algebras} For the (perfect) Lie $k$-algebra $\g \tensor_k A$, we have that:
                $$\uce(\g \tensor_k A) \cong (\g \tensor_k A) \oplus \bar{\Omega}^1_{A/k}$$
            with Lie bracket as in lemma \ref{lemma: lie_brackets_on_UCEs_of_current_algebras}.
        \end{theorem}
        
        \newpage
        
        \section{Some useful examples of UCEs}
    This section is for analysing some of the instances of UCEs of current algebras that are particularly useful to our purposes. Namely, we are interested in the UCEs of the Lie algebras:
        $$\g, \g[v^{\pm 1}], \g[v^{\pm 1}, t^{\pm 1}]$$
    (corresponding to $A$ being isomorphic to $k, \bbC[v^{\pm 1}]$, and $\bbC[v^{\pm 1}, t^{\pm 1}]$ respectively). 

    \subsection{Finite-dimensional simple Lie algebras}
        Consider, firstly, the case:
            $$A := \bbC$$
        It is trivial to see that:
            $$\dim_{\bbC} \bar{\Omega}^1_{\bbC/\bbC} \cong 0$$
        from which one sees that:
            $$\uce(\g) \cong \g$$
        i.e. $\g$ is its own universal central extension, and hence every central extension of $\g$ is trivial. Of course, there are other more conventional ways to see that $\g$ admits no non-trivial central extensions, but we thought we would include this example as a particularly degenerate case of Kassel's realisation of UCEs of current algebras.

    \subsection{Affine Lie algebras}
        Now, consider:
            $$A := \bbC[v^{\pm 1}]$$
    
        \begin{example}[Affine Lie algebras] \label{example: affine_lie_algebras_centres}
            Let us compute the UCE of $\g[v^{\pm 1}]$. From this, we can construct the so-called \say{untwisted affine Kac-Moody algebra} attached to $\g$ (cf. \cite[Chapter 7]{kac_infinite_dimensional_lie_algebras}). 

            To this end, let us firstly compute the underlying vector space of the centre of $\uce(\g[v^{\pm 1}])$. Abstractly, we know that it is isomorphic to $\bar{\Omega}^1_{\bbC[v^{\pm 1}]/\bbC}$, and it is also known that:
                $$\Omega^1_{\bbC[v^{\pm 1}]/\bbC} \cong \bbC[v^{\pm 1}] dv \cong \bigoplus_{m \in \Z} \bbC  v^m dv$$
            so the only non-trivial computation to make is that of $d( \bbC[v^{\pm 1}] )$. For this, let us consider how $d$ acts on the basis elements $v^m \in \bbC[v]$:
                $$d(v^m) = m v^{m - 1} dv$$
            We see that $d(v^m) = 0$ if and only if $m = 0$, and since the set $\{v^m\}_{m \in \Z}$ is a $\bbC$-linear basis for $\bbC[v^{\pm 1}]$, the set:
                $$\{m v^{m - 1} dv\}_{m \in \Z \setminus \{0\}}$$
            therefore spans $d(\bbC[v^{\pm 1}])$. It is also easy to see this subset of $d(\bbC[v^{\pm 1}])$ is $\bbC$-linearly independent and hence is a basis for $d(\bbC[v^{\pm 1}])$. This then tells us that:
                $$\bar{\Omega}^1_{\bbC[v^{\pm 1}]/\bbC} \cong \bbC  v^{-1} \bar{d}v$$
            The underlying vector space of $\uce(\g[v^{\pm 1}])$ is thus isomorphic to:
                $$\g[v^{\pm 1}] \oplus \bbC  v^{-1} \bar{d}v$$

            We know that the Lie bracket on $\uce(\g[v^{\pm 1}])$ is given by:
                $$[x f, y g]_{\g[v^{\pm 1}]} = [x, y]_{\g} fg + (x, y)_{\g} g \bar{d}f$$
            for all $x, y \in \g$ and all $f, g \in \bbC[v^{\pm 1}]$. Since:
                $$g \bar{d}f \in \bbC  v^{-1} \bar{d}v$$
            necessarily, the bracket can be given simplier as:
                $$[x f, y g]_{\g[v^{\pm 1}]} = [x, y]_{\g} fg + (x, y)_{\g} c(f, g) v^{-1} \bar{d}v$$
            for a uniquely determined scalar $c(f, g) \in k$, which can be computed explicitly. To do this, if suffices to perform the computation for basis elements of $\bbC[v^{\pm 1}]$, i.e. we can pick $f := v^m, g := v^n$ for some $m, n \in \Z$ and then consider the following:
                $$g \bar{d}f = v^n \bar{d}(v^m) = m v^{n + m} v^{-1} \bar{d}v$$
            This expression vanishes if and only if $m + n = 0$, so we have that:
                $$c(v^m, v^n) = m \delta_{n + m, 0}$$
            For general $f, g \in \bbC[v^{\pm 1}]$, we can write this more succinctly as:
                $$c(f, g) = \Res_{v = 0}( gdf )$$
            The Lie bracket on $\uce(\g[v^{\pm 1}])$ then takes the form:
                $$[x f, y g]_{\g[v^{\pm 1}]} = [x, y]_{\g} fg + (x, y)_{\g} \Res_{v = 0}(g df) v^{-1} \bar{d}v$$
                
            Note also, that there is a non-degenerate and invariant\footnote{Because $(-, -)_{\g}$ is invariant.} symmetric bilinear form on $\g[v^{\pm 1}]$ given by:
                $$(xf, yg)_{\g[v^{\pm 1}]} := (x, y)_{\g} \Res_{v = 0}(g df)$$
            for all $x, y \in \g$ and all $f, g \in \bbC[v^{\pm 1}]$. By invariance, the extension of this bilinear form to $\uce(\g[v^{\pm 1}])$ is necessarily invariant and degenerate (see lemma \ref{lemma: extending_bilinear_forms_to_central_extensions}).

            Let us also note that unlike $\uce(\g[v^{\pm 1}])$, the UCE of the perfect Lie algebra $\g[v]$ is trivial, since:
                $$\forall f, g \in \bbC[v]: \Res_{v = 0}(g df) = 0$$
        \end{example}

    \subsection{Toroidal Lie algebras}
        \begin{example}[Toroidal Lie algebras] \label{example: toroidal_lie_algebras_centres}
            Next, let us compute the UCE of $\g[v^{\pm 1}, t^{\pm 1}]$. For this, $\bbC[v^{\pm 1}, t^{\pm 1}]$ shall be endowed with the $\Z^2$-grading given by:
                $$\deg v^r t^s := (r, s)$$
            for all $(r, s) \in \Z^2$ so that remarks \ref{remark: gradings_on_1_forms} and \ref{remark: induced_gradings_on_UCEs} can be applied. 
            
            Firstly, let us compute its underlying vector space. Thanks to remark \ref{remark: induced_gradings_on_UCEs}, we know that the only non-trivial computation to make is that of (a basis of) the $\bbC$-vector space:
                $$\bar{\Omega}^1_{\bbC[v^{\pm 1}, t^{\pm 1}]/\bbC} := \Omega^1_{\bbC[v^{\pm 1}, t^{\pm 1}]/\bbC}/d\bbC[v^{\pm 1}, t^{\pm 1}]$$
            From remark \ref{remark: gradings_on_1_forms}, it can be inferred that:
                $$( \Omega^1_{\bbC[v^{\pm 1}, t^{\pm 1}]/\bbC} )_{(r, s)} \cong \bbC v^{r - 1} t^s dv \oplus \bbC v^r t^{s - 1} dt$$
                $$( d\bbC[v^{\pm 1}, t^{\pm 1}] )_{(r, s)} = d( \bbC[v^{\pm 1}, t^{\pm 1}]_{(r, s)} ) = d( \bbC v^r t^s ) = \bbC d(v^r t^s)$$
            which hold for all $(r, s) \in \Z^2$. We thus get:
                $$\bar{\Omega}^1_{\bbC[v^{\pm 1}, t^{\pm 1}]/\bbC} \cong \bigoplus_{(r, s) \in \Z^2} \frac{ \bbC v^{r - 1} t^s dv \oplus \bbC v^r t^{s - 1} dt }{ \bbC d(v^r t^s) }$$
            Observe that when $(r, s) = (0, 0)$, we have that:
                $$\bbC v^{0 - 1} t^0 dv \oplus \bbC v^0 t^{0 - 1} dt = \bbC v^{-1} dv \oplus \bbC t^{-1} dt$$
                $$\bbC d(v^0 t^0) = \bbC d(1) = \bbC 0 = 0$$
            and so:
                $$\bar{\Omega}^1_{\bbC[v^{\pm 1}, t^{\pm 1}]/\bbC} \cong \bigoplus_{(r, s) \in \Z^2 \setminus \{(0, 0)\}} ( \bbC v^{r - 1} t^s dv \oplus \bbC v^r t^{s - 1} dt ) \oplus ( \bbC v^{-1} dv \oplus \bbC t^{-1} dt )$$
            Finally, since:
                $$v^{r - 1} t^s dv, v^r t^{s - 1} dt, v^{-1} dv, t^{-1} dt \not \in d\bbC[v^{\pm 1}, t^{\pm 1}]$$
            meaning that $v^{r - 1} t^s dv \equiv v^{r - 1} t^s \bar{d}v \pmod{d\bbC[v^{\pm 1}, t^{\pm 1}]}$ etc., we can write:
                $$\bar{\Omega}^1_{\bbC[v^{\pm 1}, t^{\pm 1}]/\bbC} \cong \bigoplus_{(r, s) \in \Z^2 \setminus \{(0, 0)\}} ( \bbC v^{r - 1} t^s \bar{d}v \oplus \bbC v^r t^{s - 1} \bar{d}t ) \oplus ( \bbC v^{-1} \bar{d}v \oplus \bbC t^{-1} \bar{d}t )$$
            
            That said, for the indexing to be slightly more suited to our purposes\footnote{Ultimately, this is a matter of personal taste. The elements $K_{r, s}, c_v, c_t$ can be regarded as idiosyncrasies that the author personally find to be visually clear (note, for instance, that our indexing is slightly different from that on \cite[p. 35]{wendlandt_formal_shift_operators_on_yangian_doubles}), should the reader feel puzzled by this re-indexing. For instance, we find $K_{0, 0} = 0$ to be good notation.}, we shall be writing $\bar{\Omega}^1_{\bbC[v^{\pm 1}, t^{\pm 1}]/\bbC}$ as a direct sum as follows:
                $$\bar{\Omega}^1_{\bbC[v^{\pm 1}, t^{\pm 1}]/\bbC} \cong ( \bigoplus_{(r, s) \in \Z^2} \bbC K_{r, s}) \oplus \bbC c_v \oplus \bbC c_t$$
            wherein:
                $$
                    K_{r, s} :=
                    \begin{cases}
                        \text{$\frac1s v^{r - 1} t^s \bar{d}v$ if $(r, s) \in \Z \x (\Z \setminus \{0\})$}
                        \\
                        \text{$-\frac1r v^r t^{-1} \bar{d}t$ if $(r, s) \in (\Z \setminus \{0\}) \x \{0\}$}
                        \\
                        \text{$0$ if $(r, s) = (0, 0)$}
                    \end{cases}
                $$
                $$c_v := v^{-1} \bar{d}v, c_t := t^{-1} \bar{d}t$$
            In particular, the coefficients $\frac1r, \frac1s$ have been included as visual aids for keeping track of whether or not the indices $r$ or $s$ may be allowed to be $0$ in $K_{r, s}$. In fact, any element of the form:
                $$v^n t^q \bar{d}(v^m t^p) \in \bar{\Omega}^1_{\bbC[v^{\pm 1}, t^{\pm 1}]/\bbC}$$
            can be written in terms of the basis vectors $K_{r, s}, c_v, c_t$ in the following manner:
                $$v^n t^q \bar{d}(v^m t^p) = (mq - np) K_{m + n, p + q} + \delta_{(m, p) + (n, q), (0, 0)} ( m c_v + p c_t )$$
            (cf. \cite[p. 35]{wendlandt_formal_shift_operators_on_yangian_doubles}).

            Finally, let us note that the Lie bracket on $\uce(\g[v^{\pm 1}, t^{\pm 1}])$ is given by:
                $$
                    \begin{aligned}
                        & [x v^m t^p, y v^n t^q]_{\uce(\g[v^{\pm 1}, t^{\pm 1}])}
                        \\
                        = & [x, y]_{\g} v^{m + n} t^{p + q} + (x, y)_{\g} v^n t^q \bar{d}(v^m t^p)
                        \\
                        = & [x, y]_{\g} v^{m + n} t^{p + q} + (x, y)_{\g} \left( (mq - np) K_{m + n, p + q} + \delta_{(m, p) + (n, q), (0, 0)} ( m c_v + p c_t ) \right)
                    \end{aligned}
                $$
            for all $x, y \in \g$ and all $(m, p), (n, q) \in \Z^2$. As a side note, let us note that interestingly, unlike how $\g[v]$ admits only the trivial UCE, $\g[v^{\pm 1}, t]$ admits a non-trivial UCE, on which the Lie bracket is given by:
                $$[x v^m t^p, y v^n t^q]_{\uce(\g[v^{\pm 1}, t])} = [x, y]_{\g} v^{m + n} t^{p + q} + (x, y)_{\g} \left( (mq - np) K_{m + n, p + q} + \delta_{m + n, 0} m c_v \right)$$
            for all $x, y \in \g$ and all $(m, p), (n, q) \in \Z \x \Z_{\geq 0}$. This also demonstrates that the vector subspace $\uce(\g[v^{\pm 1}, t])$ is closed under the Lie bracket on the larger vector space $\uce(\g[v^{\pm 1}, t^{\pm 1}])$, and therefore is a Lie subalgebra thereof.
        \end{example}

    \newpage

    \chapter{\texorpdfstring{$\gamma$}{}-extended toroidal Lie algebras} \label{chapter: yangian_EALAs}
        \begin{abstract}
            In this chapter, we attempt to construct \say{$\gamma$-extended toroidal Lie algebras}, which are to be extensions $\extendedtoroidal$ of a certain Lie algebra of derivations by the toroidal Lie algebra $\toroidal := \uce(\g[v^{\pm 1}, t^{\pm 1}])$. This is to rectify the problem whereby any invariant symmetric bilinear form on $\toroidal$ is necessarily degenerate.
        \end{abstract}

        \minitoc

        \newpage
    
        \section{Setup and overview of results}
    Again, $\g$ is a finite-dimensional simple Lie algebra over $\bbC$. It is accompanied by all the data listed in subsection \ref{subsection: finite_dimensional_simple_lie_algebras}. 

    \subsection{Conventions for toroidal Lie algebras} \label{subsection: toroidal_lie_algebra_conventions}
        For our own convenience, we will also be adopting the following abbreviations:
            $$A := \bbC[v^{\pm 1}, t^{\pm 1}], \bar{\Omega}_{[2]} := \bar{\Omega}^1_{A/\bbC}$$
        and also that:
            $$\g_{[2]} := \g \tensor_{\bbC} A$$
        being understood as current algebras (in the sense of definition \ref{def: current_algebras}).
    
        We will then be interested in the Lie algebra:
            $$\toroidal := \uce(\g_{[2]})$$
        which, respectively, shall be referred to as the \textbf{toroidal Lie algebra} associated to $\g$. In example \ref{example: toroidal_lie_algebras_centres} and remark \ref{remark: Z_gradings_on_toroidal_lie_algebras}, the structures of the underlying ($\Z$-graded) vector spaces of these Lie algebras have already been described, and we refer the reader there for the details (in particular, the construction of a canonical basis for the underlying vector spaces of their centres). So that our notations would be suggestive, we shall be writing:
            $$\z_{[2]} := \z(\toroidal)$$
        from now on; often, we might refer to these as the \textbf{(positive) toroidal centres}.
        
        Now, instead of being equipped with the usual residue bilinear form of degree $(0, 0)$, the Lie algebra $\g_{[2]}$ will be equipped with the residue bilinear form of degree $(0, -1)$:
            $$(x v^m t^p, y v^n t^q)_{\g_{[2]}} := -(x, y)_{\g} \delta_{(m, p) + (n, q), (0, -1)}$$
        given for all $x, y \in \g$ and all $(m, p), (n, q) \in \Z^2$. It is easy to see that the bilinear form:
            $$(-, -)_{\g_{[2]}}$$
        is symmetric, non-degenerate, and invariant. Because $\toroidal$ has a non-trivial centre, any invariant (symmetric) bilinear form thereon (so in particular, any extension of $(-, -)_{\g_{[2]}}$ to $\toroidal$) is necessarily degenerate (see remark \ref{remark: extending_bilinear_forms_to_central_extensions}). The purpose of constructing \say{Yangian extended toroidal Lie algebras} is to remedy such degeneracy.

    \subsection{Outline of the construction and classification of Yangian extended toroidal Lie algebras}
        \begin{definition}[Formal residue] \label{def: formal_residues}
            The \textbf{formal residue} of any element $f(v_1, ..., v_n) \in \bbC[v_1^{\pm 1}, ..., v_n^{\pm 1}]$ is then given by:
                $$\Res_{(v_1, ..., v_n) = (0, ..., 0)}(f) := ( \Res_{v_n = 0} \circ ... \circ \Res_{v_1 = 0} )( f(v_1, ..., v_n) )$$
            Each map:
                $$\Res_{v_i = 0}: \bbC[v_i, ..., v_n] \to \bbC[v_{i + 1}, ..., v_n]$$
            is takes as input an element $f(v_i, ..., v_n) \in \bbC[v_{i + 1}, ..., v_n]$ and outputs the $\bbC[v_{i + 1}, ..., v_n]$-coefficient of the term with $v_i^{-1}$, or $0$ if there is no such term. 
        \end{definition}
        \begin{remark}[Taking formal residues is linear]
            It is not hard to see that:
                $$\Res_{(v_1, ..., v_n) = (0, ..., 0)}( v_1^{m_1} ... v_n^{m_n} ) = \delta_{(m_1, ..., m_n), (-1, ..., -1)}$$
            From this, one sees that:
                $$\Res_{(v_1, ..., v_n) = (0, ..., 0)}: \bbC[v_1, ..., v_n] \to k$$
            is a linear map.
        \end{remark}
    
        \begin{convention}[The Yangian residue] \label{conv: yangian_residue}
            For each $m \in \Z$, define a $\bbC$-linear map:
                $$\gamma_m: A \to \bbC$$
            given by:
                $$\gamma_m(f) := -\Res_{(v, t) = (0, 0)}( v^{-m} f )$$
            We will only be interested in the case $m = 1$, when we will be writing:
                $$\gamma := \gamma_1$$
            and refer to the map as the \textbf{Yangian residue map}. We will be making all of our constructions relative to this choice of linear map. All the definitions and results can be worked out similarly for a general $\gamma_m$ to what we shall be writing down in the sequel for $\gamma_1$.
        \end{convention}
        \begin{remark}
            One sees thus that the bilinear form on $\g_{[2]}$ that was defined in subsection \ref{subsection: toroidal_lie_algebra_conventions} can now be given more compactly as:
                $$(xf, yg)_{\g_{[2]}} := (x, y)_{\g} \gamma(fg)$$
            which is why we refer to $\gamma$ as the Yangian residue map.
        \end{remark}

        Next, let us fix an extension (in the sense of remark \ref{remark: extending_bilinear_forms_to_central_extensions}):
            $$(-, -)_{\toroidal}$$
        of the bilinear form $(-, -)_{\g_{[2]}}$ to $\toroidal$; \textit{a priori}, its radical is $\z_{[2]}$. A \textit{non-degenerate} symmetric $\bbC$-bilinear form:
            $$(-, -)_{\extendedtoroidal}$$
        on the vector space:
            $$\extendedtoroidal := \toroidal \oplus \z_{[2]}^{\star}$$
        can thus be constructed as an extension of the aforementioned bilinear form $(-, -)_{\toroidal}$: namely, it pairs elements of $\z_{[2]}$ and those of $\z_{[2]}^{\star}$ in the canonical manner. What we shall aim to do in the next section is to endow the vector space $\extendedtoroidal$ with a Lie algebra structure, with respect to which the form $(-, -)_{\extendedtoroidal}$ will be \textit{invariant}. These Lie algebra structures shall be known as \textbf{Yangian extended toroidal Lie algebras} or \textbf{$\gamma$-extended toroidal Lie algebras}, should the dependence on $\gamma$ need emphasising.

        These Lie algebra structures shall arise as extensions of $\z_{[2]}^{\star}$ by $\toroidal$, but for this to even make sense, a Lie algebra structure must be put onto $\z_{[2]}^{\star}$. For this, we shall identify $\z_{[2]}^{\star}$ with a certain $\Z^2$-graded vector subspace:
            $$\divzero$$
        of $\der(A)$ (which shall always be endowed with the usual commutator bracket). Not only does this gives a natural Lie algebra structure on $\z_{[2]}^{\star}$, but because we know how to compute commutators of derivations, explicit commutators between elements of $\divzero$ (and hence between those of $\z_{[2]}^{\star}$) can be computed (cf. lemma \ref{lemma: commutators_of_yangian_div_zero_vector_fields}).

        \textit{A priori}, the following definition can be stated:
        \begin{definition}[Yangian extended toroidal Lie algebras] \label{def: yangian_extended_toroidal_lie_algebras}
            A \textbf{Yangian extended toroidal Lie algebra} or \textbf{$\gamma$-extended toroidal Lie algebra} is a Lie algebra:
                $$(\fraky, [-, -]_{\fraky})$$
            along with the accompanying data of:
            \begin{itemize}
                \item an isomorphism of vector spaces:
                    $$\mu: \fraky \xrightarrow[]{\cong} \toroidal \oplus \z_{[2]}^{\star}$$
                \item an invariant non-degenerate symmetric bilinear form\footnote{... and we note the subtlety that the construction of the bilinear form $(-, -)_{\extendedtoroidal}$ depends on $\gamma$ (which is not reflected in our notations).}:
                    $$(-, -)_{\fraky} := (-, -)_{\extendedtoroidal} \circ \mu^{\tensor 2}$$
                \item a Lie algebra monomorphism:
                    $$\iota: \toroidal \hookrightarrow \fraky$$
            \end{itemize}
        \end{definition}

        Our main result concerning these Lie algebras is as follows:
        \begin{theorem} \label{theorem: yangian_extended_toroidal_lie_algebras_preliminary_version}
            A Lie algebra $\fraky$ is a Yangian toroidal Lie algebra if and only if it is isomorphic to some twisted semi-direct product (cf. definition \ref{def: twisted_semi_direct_products}):
                $$\toroidal \rtimes^{\sigma} \divzero$$
            wherein $\sigma \in Z^2_{\Lie}(\divzero, \z_{[2]})$ is such that:
                $$(\sigma(D, D'), D'')_{\fraky(\sigma)} = (D, \sigma(D', D''))_{\fraky(\sigma)}$$
            for any triple of elements $D, D', D'' \in \divzero$.
        \end{theorem}
        A trivial - but nevertheless very important to us - corollary of this theorem is as follows.
        \begin{corollary}
            The semi-direct product:
                $$\toroidal \rtimes \divzero$$
            corresponding to $\sigma = 0$, is a Yangian extended toroidal Lie algebra.
        \end{corollary}
        

        \newpage

        \section{Construction of \texorpdfstring{$\gamma$}{}-extended toroidal Lie algebras}
    \subsection{\texorpdfstring{$\gamma$}{}-divergence-zero vector fields} \label{subsection: yangian_div_zero_vector_fields}
        We begin the process of constructing $\gamma$-extended toroidal Lie algebras in the sense of definition \ref{def: yangian_extended_toroidal_lie_algebras} by firstly defining so-called \say{$\gamma$-divergence-zero vector fields} (cf. definition \ref{def: yangian_div_zero_vector_fields}), with the first and foremost goal in doing so being that, we would like to be able to endow the underlying vector space $\toroidal \oplus \z(\toroidal)^{\star}$ of any $\gamma$-extended toroidal Lie algebra with a natural Lie algebra structure somehow (cf. corollary \ref{coro: lie_bracket_on_graded_dual_of_toroidal_centres}). We will be accomplishing this task by recognising that, the aforementioned $\gamma$-divergence-zero vector fields form a $\Z^2$-graded Lie subalgebra $\divzero$ of $\der(A)$ (cf. lemma \ref{lemma: yangian_div_zero_vector_fields_basic_properties} and corollary \ref{coro: yangian_div_zero_vector_fields_are_graded}) that is - as a $\Z^2$-graded vector space - isomorphic to $\z(\toroidal)^{\star}$ in a rather canonical manner (cf. proposition \ref{prop: yangian_div_zero_vector_fields_are_graded_dual_to_toroidal_centre}). Secondly, knowing that elements of $\z(\toroidal)^{\star}$ are identifiable with certain derivations will also help us have a concrete understanding how elements of $\z(\toroidal)^{\star}$ must \say{bracket} with those of $\toroidal$ with respect to a given $\gamma$-extended toroidal Lie algebra structure on $\toroidal \oplus \z(\toroidal)^{\star}$. In fact, this is necessary for even formulating the main theorem in this section, i.e. theorem \ref{theorem: yangian_extended_toroidal_lie_algebras_main_theorem}, where we would like to compare these bracket relations with a certain action of $\divzero$ on $\toroidal$, thereby realising that any $\gamma$-extended toroidal Lie algebra must be isomorphic to a twisted semi-direct product $\toroidal \rtimes^{\sigma} \divzero$ (cf. definition \ref{def: twisted_semi_direct_products}).
    
        \begin{definition}[$\gamma$-divergence-zero vector fields] \label{def: yangian_div_zero_vector_fields}
            Let $\divzero$ be the vector subspace of $\der(A)$ defined as follows:
                $$\divzero := \{ D \in \der(A) \mid \forall f \in A: \gamma(D(f)) = 0 \}$$
            Elements of this vector space shall be referred to as \textbf{$\gamma$-divergence-zero vector fields}.
        \end{definition}
        \begin{remark}
            The name is because if we were to consider instead the linear map $A \to \bbC$ given by $f \mapsto -\Res(f)$ then we would recover the notion of algebraically divergence-zero vector fields on the (smooth) affine scheme $\Spec \bbC[v^{\pm 1}, t^{\pm 1}]$ (cf. e.g. \cite[Exercise 1.7c]{neher_lectures_on_EALAs} and \cite{billig_talboom_category_J_for_div_zero_vector_fields_on_tori})
        \end{remark}

        \begin{lemma}[Basic properties of $\gamma$-divergence-zero vector fields] \label{lemma: yangian_div_zero_vector_fields_basic_properties}
            \begin{enumerate}
                \item $\divzero$ is a Lie subalgebra of $\der(A)$ (with the usual commutator bracket).
                \item $\divzero$ admits the following subset as a basis:
                    $$\{D_{r, s}\}_{(r, s) \in \Z^2} \cup \{D_v, D_t\}$$
                Its elements are given in terms of the partial derivatives $\del_v := \frac{\del}{\del v}$ and $\del_t := \frac{\del}{\del t}$ by:
                    $$D_{r, s} := -s v^{-r + 1} t^{-s - 1} \del_v + r v^{-r} t^{-s} \del_t$$
                    $$D_v := -v t^{-1} \del_v$$
                    $$D_t := -\del_t$$
                \item The basis elements of $\divzero$ satify the following commutation relations:
                    $$[D_v, D_t] = -D_{0, 1}$$
                    $$[D_v, D_{r, s}] = r D_{r, s + 1}$$
                    $$[D_t, D_{r, s}] = s D_{r, s + 1}$$
                    $$[D_{a, b}, D_{r, s}] = (br - as) D_{a + r, b + s + 1}$$
            \end{enumerate}
        \end{lemma}
            \begin{proof}
                \begin{enumerate}
                    \item To show that $\divzero$ is a Lie subalgebra of $\der(A)$ with respect to the usual commutator bracket, simply consider the following, where $D, D' \in \divzero$ and $f \in A$ are arbitrary:
                        $$\gamma( [D, D'](f) ) = \gamma( D(D'(f)) - \gamma( D'(D(f)) ) = 0$$
                    which holds per the very definition of $\divzero$ itself.
                    \item Any element $D \in \divzero$ is, of course, an element of $\der(A) \cong A \del_v \oplus A \del_t$, and hence can be written as:
                        $$D := \sum_{(a, b) \in \Z^2} \left( \lambda_{a, b} v^a t^b \del_v + \mu_{a, b} v^a t^b \del_t \right)$$
                    for some $\lambda_{a, b}, \mu_{a, b} \in \bbC$. Consider then the following, where $f \in A$ is arbitrary:
                        $$0 =\gamma(D(f)) = \sum_{(a, b) \in \Z^2} \left( \lambda_{a, b} \gamma(v^r t^s \del_v f) + \mu_{a, b} \gamma(v^a t^b \del_t f) \right)$$
                    Without any loss of generality, we can take $f \in A$ to be a basis element, i.e. $f := v^m t^p$ for some $(m, p) \in \Z^2$. Doing so yields:
                        $$
                            \begin{aligned}
                                0 & = \sum_{(a, b) \in \Z^2} \left( \lambda_{a, b} \gamma(v^a t^b \del_v(v^m t^p)) + \mu_{a, b} \gamma(v^a t^b \del_t(v^m t^p)) \right)
                                \\
                                & = \sum_{(a, b) \in \Z^2} \left( m\lambda_{a, b} \gamma(v^{a + m - 1} t^{b + p}) + p \mu_{a, b} \gamma( v^{a + m} t^{b + p - 1} ) \right)
                                \\
                                & = -\sum_{(a, b) \in \Z^2} \left( m \lambda_{a, b} \delta_{(a + m - 1, b + p), (0, -1)} + p \mu_{a, b} \delta_{(a + m, b + p - 1), (0, -1)} \right)
                                \\
                                & = -\sum_{(a, b) \in \Z^2} \left( m \lambda_{a, b} \delta_{(a + m, b + p), (1, -1)} + p \mu_{a, b} \delta_{(a + m, b + p), (0, 0)} \right)
                                \\
                                & = -\left( m \lambda_{-m + 1, -p - 1} + p \mu_{-m, -p} \right)
                            \end{aligned} 
                        $$
                    for all $(m, p) \in \Z^2$. From this, one sees that:
                        $$D = \sum_{(r, s) \in \Z^2} \lambda_{r, s} D_{r, s} + \lambda_v D_v + \lambda_t D_t$$
                    for some $\lambda_{r, s}, \lambda_v, \lambda_t \in \bbC$, where:
                        $$D_{r, s} := -s v^{-r + 1} t^{-s - 1} \del_v + r v^{-r} t^{-s} \del_t$$
                        $$D_v := -v t^{-1} \del_v$$
                        $$D_t := -\del_t$$
                    These elements are clearly linearly independent, so we are done.
                    \item Next, let us compute the commutation relations satisfied by the basis elements of $\divzero$.
                    \begin{enumerate}
                        \item Since we know that:
                            $$D_v = -vt^{-1} \del_v, D_t = -\del_t$$
                        we can perform the following computation:
                            $$
                                \begin{aligned}
                                    & [D_v, D_t](v^m t^p)
                                    \\
                                    = & v t^{-1} \del_v( \del_t (v^m t^p) ) - \del_t ( v t^{-1} \del_v ( v^m t^p ) )
                                    \\
                                    = & v t^{-1} \del_v( p v^m t^{p - 2} ) - \del_t ( mv^m t^{p - 1} )
                                    \\
                                    = & p m v^m t^{p - 2} - (p - 1) m v^m t^{p - 2}
                                    \\
                                    = & m v^m t^{p - 2}
                                    \\
                                    = & v t^{-2} \del_v( v^m t^p ) 
                                \end{aligned}
                            $$
                        which tells us that:
                            $$[D_v, D_t] = v t^{-2} \del_v = -D_{0, 1}$$
                        \item Next, observe that:
                            $$D_v(v^m t^p) = -m v^m t^{p - 1}$$
                            $$D_{r, s}(v^m t^p) = ( rp - ms ) v^{m - r} t^{p - s - 1}$$
                        which can be easily shown via directly calculating the expressions. From this, we infer that:
                            $$
                                \begin{aligned}
                                    & [D_v, D_{r, s}](v^m t^p)
                                    \\
                                    = & D_v( D_{r, s}(v^m t^p) ) - D_{r, s}( D_v(v^m t^p) )
                                    \\
                                    = & (rp - ms) D_v( v^{m - r} t^{p - s - 1} ) + m D_{r, s}( v^m t^{p - 1} )
                                    \\
                                    = & -(m - r)(rp - ms) v^{m - r} t^{p - s - 2} + (r(p - 1) - ms) m v^{m - r} t^{p - s - 2}
                                    \\
                                    = & r(rp - m(s + 1)) v^{m - r} t^{p - (s + 1) - 1}
                                    \\
                                    = & r D_{r, s + 1}(v^m t^p)
                                \end{aligned}
                            $$
                        and hence:
                            $$[D_v, D_{r, s}] = r D_{r, s + 1}$$
                        \item Likewise, we have that:
                            $$[D_t, D_{r, s}] = s D_{r, s + 1}$$
                        \item Using the same method, we can show that:
                            $$[D_{a, b}, D_{r, s}] = (br - as) D_{a + r, b + s + 1}$$
                    \end{enumerate}
                \end{enumerate}
            \end{proof}
        \begin{corollary}[$\Z^2$-grading on $\gamma$-divergence-zero vector fields] \label{coro: yangian_div_zero_vector_fields_are_graded}
            $\divzero$ is a $\Z^2$-graded Lie subalgebra of the $\Z^2$-graded Lie algebra $\der(A)$ (see remark \ref{remark: gradings_on_derivations} for a description of the standard $\Z^2$-grading on $\der(A)$ coming from the one on $A$). Namely, the grading on $\divzero$ is given by:
                $$\forall (r, s) \in \Z^2 \setminus \{(0, 0)\}: \deg D_{r, s} = (-r, -s - 1)$$
                $$\deg D_v = \deg D_t = (0, -1)$$
            The reason for this choice of grading will become clear after proposition \ref{prop: yangian_div_zero_vector_fields_are_graded_dual_to_toroidal_centre}, which asserts that $\divzero \cong \z(\toroidal)^{\star}$ as $\Z^2$-graded vector spaces; recall that the latter has a natural $\Z^2$-grading given by:
                $$\deg K_{r, s} = (r, s)$$
                $$\deg c_v = \deg c_t = 0$$
        \end{corollary}

        \begin{proposition}[$\gamma$-divergence-zero vector fields are graded-dual to toroidal centre] \label{prop: yangian_div_zero_vector_fields_are_graded_dual_to_toroidal_centre}
            There is a $\Z^2$-graded vector space isomorphism:
                $$\varphi: \divzero \xrightarrow[]{\cong} \z(\toroidal)^{\star}$$
            determined by:
                $$\varphi(D)( f\bar{d}g ) := \gamma( f D(g) )$$
            for all $D \in \divzero$. This identifies the basis $\{D_{r, s}\}_{(r, s) \in \Z^2} \cup \{D_v, D_t\}$ of $\divzero$ as being $\Z^2$-graded dual to the basis $\{K_{r, s}\}_{(r, s) \in \Z^2} \cup \{c_v, c_t\}$ of $\z(\toroidal)$.
        \end{proposition}
            \begin{proof}
                First of all, let us prove that $\varphi: \divzero \xrightarrow[]{\cong} \z(\toroidal)^{\star}$ as given is graded, and it is enough to check this on the basis elements: we claim that, because:
                    $$\deg D_{r, s} = (-r, -s - 1)$$
                    $$\deg D_v = \deg D_t = (0, 0)$$
                (cf. corollary \ref{coro: yangian_div_zero_vector_fields_are_graded}) and because:
                    $$\deg K_{a, b} = (a, b)$$
                    $$\deg c_v = \deg c_t = (0, 0)$$
                (see example \ref{example: toroidal_lie_algebras_centres}), we ought to have that:
                    $$\varphi(D_{r, s})(K_{a, b}) = \delta_{(a - r, b - s - 1), (-1, -1)}$$
                    $$\varphi(D_v)(c_v) = \varphi(D_t)(c_t) = 1$$
                for $D_{r, s}, D_v, D_t$ to be identified - respectively - as $\Z^2$-graded dual basis elements corresponding to $K_{r, s}, c_v, c_t$; a straightforward dimension argument will then show that $\varphi$ must be a vector space isomorphism. Indeed, we have that:
                    $$
                        \begin{aligned}
                            \varphi(D_{r, s})(K_{a, b}) & = 
                            \begin{cases}
                                \text{$\gamma\left( \frac1b v^{a - 1} t^b D_{r, s}(v) \right)$ if $(a, b) \in \Z \x (\Z \setminus \{0\})$}
                                \\
                                \text{$\gamma\left( -\frac1a v^a t^{-1} \bar{d}t \right)$ if $(a, b) \in (\Z \setminus \{0\}) \x \{0\}$}
                                \\
                                \text{$0$ if $(a, b) = (0, 0)$}
                            \end{cases}
                            \\
                            & = 
                            \begin{cases}
                                \text{$\gamma\left( -\frac{s}{b} v^{a - r} t^{b - s - 1} \right)$ if $(a, b) \in \Z \x (\Z \setminus \{0\})$}
                                \\
                                \text{$\gamma\left( -\frac{r}{a} v^{a - r} t^{-s - 1} \right)$ if $(a, b) \in (\Z \setminus \{0\}) \x \{0\}$}
                                \\
                                \text{$0$ if $(a, b) = (0, 0)$}
                            \end{cases}
                            \\
                            & = \delta_{(a - r, b - s - 1), (-1, -1)}
                        \end{aligned}
                    $$
                as well as:
                    $$\varphi(D_v)(c_v) = \gamma(v^{-1} D_v(v)) = \gamma( -t^{-1} ) = 1$$
                    $$\varphi(D_t)(c_t) = \gamma(t^{-1} D_t(t)) = \gamma( -t^{-1} ) = 1$$
                Since:
                    $$\divzero \cong \bigoplus_{(r, s) \in \Z^2} \bbC D_{r, s} \oplus \bbC D_v \oplus \bbC D_t$$
                    $$\z(\toroidal)^{\star} \cong \bigoplus_{(r, s) \in \Z^2} (\bbC K_{r, s})^* \oplus (\bbC c_v)^* \oplus (\bbC c_t)^*$$
                the computations above are enough to show that:
                    $$\varphi: \divzero \to \z(\toroidal)^{\star}$$
                as given is a vector space isomorphism.
            \end{proof}
        \begin{corollary}[Lie brackets on graded-duals of the toroidal centres] \label{coro: lie_bracket_on_graded_dual_of_toroidal_centres}
            $\z(\toroidal)^{\star}$ is naturally a Lie algebra via the vector space isomorphism $\varphi: \divzero \xrightarrow[]{\cong} \z(\toroidal)^{\star}$.
        \end{corollary}
        \begin{corollary}[Non-degenerate bilinear forms on $\z(\toroidal) \oplus \divzero$ and on $\toroidal \oplus \divzero$] \label{coro: pairing_yangian_div_zero_vector_fields_and_cyclic_1_forms}
            There is a non-degenerate symmetric bilinear form $(-, -)_{\varphi}$ on the vector space $\z(\toroidal) \oplus \divzero$, given by:
                $$(K, D)_{\varphi} := \varphi(D)(K)$$
                $$(K, K')_{\varphi} = (D, D')_{\divzero} := 0$$
            for all $K, K' \in \z(\toroidal), D, D' \in \divzero$.

            This extends to a non-degenerate and symmetric bilinear form $(-, -)_{\toroidal \oplus \divzero}$ on $\toroidal \oplus \divzero$, given by:
                $$
                    \begin{aligned}
                        & ( X + D, Y + D' )_{\toroidal \oplus \divzero}
                        \\
                        := & (X, Y)_{\toroidal} + ( \pi_{\z(\toroidal)}(X), D' )_{\varphi} + ( \pi_{\z(\toroidal)}(Y), D )_{\varphi}
                        \\
                        = & (X, Y)_{\toroidal} + \varphi(D')( \pi_{\z(\toroidal)}(X) ) + \varphi(D)( \pi_{\z(\toroidal)}(Y) )
                    \end{aligned}
                $$
            for all $X, Y \in \toroidal$ and all $D, D' \in \divzero$, where $\pi_{\z(\toroidal)}: \toroidal \to \z(\toroidal)$ denotes the canonical projection. This bilinear form itself extends the bilinear form $(-, -)_{\toroidal}$ on $\toroidal$, and it is related to the bilinear form $(-, -)_{\toroidal \oplus \z(\toroidal)^{\star}}$ from lemma \ref{lemma: extended_toroidal_bilinear_form} by:
                $$(-, -)_{\toroidal \oplus \divzero} = (-, -)_{\toroidal \oplus \z(\toroidal)^{\star}} \circ ( \id_{\toroidal} \oplus \varphi )^{\tensor 2}$$
        \end{corollary}
            \begin{proof}
                This is a direct consequence of lemma \ref{lemma: extended_toroidal_bilinear_form}.
            \end{proof}
        \begin{remark}
            Note that the pairing $(-, -)_{\varphi}$ as in corollary \ref{coro: pairing_yangian_div_zero_vector_fields_and_cyclic_1_forms}, when regarded as an element of $\z(\toroidal) \tensor_{\bbC} \z(\toroidal)^{\star}$, which has a canonical $\Z^2$-grading coming from those on $\z(\toroidal)$ and $\z(\toroidal)^{\star}$, has total degree $-1$ due to the choice of $\Z^2$-grading on $\divzero$ (and hence on $\z(\toroidal)^{\star}$, thanks to proposition \ref{prop: yangian_div_zero_vector_fields_are_graded_dual_to_toroidal_centre}) that was made in corollary \ref{coro: yangian_div_zero_vector_fields_are_graded}. Because $(-, -)_{\g_{[2]}}$ is also of total degree $-1$ (by construction), the bilinear form $(-, -)_{\toroidal \oplus \divzero}$ is also of total degree $-1$.
        \end{remark}

    \subsection{Statement of the main theorem and proof outline}
        \begin{lemma}[A $\der(A)$-action on $\toroidal$] \label{lemma: vector_field_action_on_toroidal_lie_algebras} 
            There is a $\der(A)$-module structure on $\toroidal$:
                $$\rho: \der(A) \to \der(\toroidal)$$
            by Lie derivations (cf. definition \ref{def: lie_derivations}), given explicitly on the generators $xf \in \g_{[2]}$ (for any $x \in \g$ and $f \in A$) and $g \bar{d}f \in \z(\toroidal)$ (for any $f, g \in A$) by:
                $$\rho(D)( xf ) := x D(f)$$
                $$\rho( g\bar{d}f ) := D(g) \bar{d}f + g \bar{d}(D(f))$$
            for all $D \in \der(A)$.
        \end{lemma}
            \begin{proof}
                The first assertion is clear from the discussion in example \ref{example: derivations_on_current_algebras}.
            
                To prove the second assertion, recall from example \ref{example: lie_derivatives} that there is a Lie algebra action by Lie derivations:
                    $$L: \der(A) \to \der(\Omega^1_{A/\bbC})$$
                given by Lie derivatives, i.e.:
                    $$L(D)( g df ) := D(g) df + g d(D(f))$$
                for all $D \in \der(A)$ and all $f, g \in A$. Therefore, to show that it suffices to only show that $d(A)$ is a $\der(A)$-submodule of $\Omega^1_{A/\bbC}$ in order for us to show that the vector space $\z(\toroidal) \cong \Omega^1_{A/\bbC}/d(A)$ is a $\der(A)$-module. To this end, simply consider the following, where $D \in \der(A)$ and $f \in A$ are arbitrarily chosen:
                    $$L(D)( df ) = d( D(f) )$$
                Since $D(f) \in A$, the above shows that:
                    $$L(D)( df ) \in d(A)$$
                and hence we have what we need.

                We have therefore constructed a Lie algebra action:
                    $$\rho: \der(A) \to \gl(\toroidal)$$
                given by:
                    $$\rho(D)( xf ) := x D(f)$$
                    $$\rho( g\bar{d}f ) := D(g) \bar{d}f + g \bar{d}(D(f))$$
                for all $x \in \g$ and all $f, g \in A$, so it remains to check that the codomain of $\rho$ is actually $\der(\toroidal)$, i.e. for every $D \in \der(A)$, the operator:
                    $$\rho(D) \in \gl(\toroidal)$$
                is a Lie derivation in the sense of definition \ref{def: lie_derivations} (for which we shall have to regard $\toroidal$ as a module over itself via the adjoint action). For this, pick arbitrary elements $X, Y \in \g_{[2]}, K, K' \in \z(\toroidal)$ - and without loss of generality, we can take $X := xf$ and $Y := yg$ for some $x, y \in \g$ and some $f, g \in A$ - along with an element $D \in \der(A)$ and then consider the following:
                    $$
                        \begin{aligned}
                            & \rho(D)( [X + K, Y + K']_{\toroidal} )
                            \\
                            = & \rho([x, y]_{\g} fg) + (x, y)_{\g} \rho(D)(g \bar{d}f)
                            \\
                            = & [x, y]_{\g} D(fg) + (x, y)_{\g} ( D(g) \bar{d}f + g \bar{d}(D(f)) )
                            \\
                            = & [x, y]_{\g} ( f D(g) + D(f) g ) + (x, y)_{\g} ( D(g) \bar{d}f + g \bar{d}(D(f)) )
                            \\
                            = & [ xf, y D(g) ]_{\toroidal} + [ x D(f), yg ]_{\toroidal}
                            \\
                            = &  [ xf, \rho(yg) ]_{\toroidal} + [ \rho(xf), yg ]_{\toroidal}
                            \\
                            = & [ X, \rho(Y) ]_{\toroidal} + [ \rho(X), Y ]_{\toroidal}
                            \\
                            = & [ X + K', \rho(Y + K') ]_{\toroidal} + [ \rho(X + K), Y + K' ]_{\toroidal}
                        \end{aligned}
                    $$
                where the last equality is true because $K, K'$ as well as $\rho(K), \rho(K')$ are central in $\toroidal$. From this, it is clear that the Leibniz rule is satisfied, so we are done.
            \end{proof}
        \begin{remark}
            Lemma \ref{lemma: vector_field_action_on_toroidal_lie_algebras} remains true when we replace our $A := \bbC[v^{\pm 1}, t^{\pm 1}]$ by \textit{any} commutative $\bbC$-algebra.
        \end{remark}
        \begin{corollary}[A $\divzero$-action on $\toroidal$] \label{coro: a_fixed_yangian_div_zero_vector_field_action} 
            Because $\divzero$ is a Lie subalgebra of $\der(A)$ (cf. lemma \ref{lemma: yangian_div_zero_vector_fields_basic_properties}), the $\der(A)$-action $\rho: \der(A) \to \der(\toroidal)$ as in lemma \ref{lemma: vector_field_action_on_toroidal_lie_algebras} above defines an action:
                $$\rho: \divzero \to \der(\toroidal)$$
            of $\divzero$ on $\toroidal$ by the same formulae, and hence via Lie derivations, as well.
        \end{corollary}
            
        Corollary \ref{coro: a_fixed_yangian_div_zero_vector_field_action} allows us to form the semi-direct product:
            $$\toroidal \rtimes \divzero$$
        (cf. example \ref{example: lie_algebra_semi_direct_products}), which prompts the content of our main result concerning $\gamma$-extended toroidal Lie algebras. The statement is as follows:
        \begin{theorem} \label{theorem: yangian_extended_toroidal_lie_algebras_main_theorem}
            A Lie algebra $(\extendedtoroidal, [-, -]_{\extendedtoroidal})$ is a \textbf{$\gamma$-extended toroidal Lie algebra} in the sense of definition \ref{def: yangian_extended_toroidal_lie_algebras} if and only if there exists a Lie algebra isomorphism:
                $$\nu: \toroidal \rtimes^{\sigma} \divzero \xrightarrow[]{\cong} \extendedtoroidal$$
            for some Lie $2$-cocycle\footnote{See definitions \ref{def: twisted_semi_direct_products} and \ref{def: lie_cocycles_and_coboundaries}.} $\sigma \in Z^2_{\Lie}(\divzero, \z(\toroidal))$ such that the following \textbf{$\gamma$-invariance} property is satisfied:
                $$\left( \sigma(D, D'), D'' \right)_{\toroidal \oplus \divzero} = \left( D, \sigma(D', D'') \right)_{\toroidal \oplus \divzero}$$
                
            Furthermore, if $\extendedtoroidal$ is a $\gamma$-extended toroidal Lie algebra and:
                $$\mu: \extendedtoroidal \xrightarrow[]{\cong} \toroidal \oplus \z(\toroidal)^{\star}$$
            is a vector space isomorphism as in definition \ref{def: yangian_extended_toroidal_lie_algebras}, then:
            \begin{enumerate}
                \item a particular Lie algebra isomorphism $\nu: \toroidal \rtimes^{\sigma} \divzero \xrightarrow[]{\cong} \extendedtoroidal$ will be given by:
                    $$\nu = \mu^{-1} \circ (\id_{\toroidal} \oplus \varphi)$$
                with $\varphi$ as in proposition \ref{prop: yangian_div_zero_vector_fields_are_graded_dual_to_toroidal_centre}, and
                \item $\mu^{-1}(\toroidal) \subset \extendedtoroidal$ will be a Lie algebra ideal, and
                \item the action of $\divzero$ on $\toroidal$ is indepdendent of $\sigma$ and given by $\rho: \divzero \to \der(\toroidal)$ (cf. corollary \ref{coro: a_fixed_yangian_div_zero_vector_field_action}).
            \end{enumerate}
        \end{theorem}
        \begin{remark}
            The last consequential statement in theorem \ref{theorem: yangian_extended_toroidal_lie_algebras_main_theorem} can be rephrased equivalently as follows: if $\extendedtoroidal$ is a $\gamma$-extended toroidal Lie algebra there is an isomorphism of Lie algebras:
                $$\extendedtoroidal/\toroidal \xrightarrow[]{\cong} \divzero$$
            that is independent of the corresponding $\gamma$-invariant $2$-cocycle $\sigma$ (such that $\toroidal \rtimes^{\sigma} \divzero \cong \extendedtoroidal$).
        \end{remark}
        
        \begin{convention}[The underlying vector space of $\gamma$-extended toroidal Lie algebras]
            Henceforth, we will be writing:
                $$\extendedtoroidal := \toroidal \oplus \divzero$$
            The non-degenerate bilinear form on this vector space that was constructed in corollary \ref{coro: pairing_yangian_div_zero_vector_fields_and_cyclic_1_forms} shall therefore be denoted by:
                $$(-, -)_{\extendedtoroidal}$$
            and as a small abuse of notation, for any isomorphism of vector spaces:
                $$\mu: \extendedtoroidal \xrightarrow[]{\cong} \fraky$$
            we shall be using the same notation to speak of the bilinear form $\mu \circ (-, -)_{\extendedtoroidal} \circ (\mu^{-1})^{\tensor 2}$ on $\fraky$. 
        \end{convention}

        \begin{proof}[Outline of the proof of theorem \ref{theorem: yangian_extended_toroidal_lie_algebras_main_theorem}]
            Our proof of theorem \ref{theorem: yangian_extended_toroidal_lie_algebras_main_theorem} shall be split into two main steps, namely the proofs of the \say{if} and the \say{only if} implications therein, respectively.
            \begin{itemize}
                \item Let us discuss firstly the proof of the \say{if} direction, the content of subsection \ref{subsection: which_twisted_semi_direct_products_are_yangian_extended_toroidal_lie_algebras}.
                \begin{enumerate}
                    \item After some trivial reduction, one shall see that in fact, the most non-trivial thing to prove is that the semi-direct product:
                        $$\toroidal \rtimes \divzero$$
                    (well-defined now, thanks to corollary \ref{coro: a_fixed_yangian_div_zero_vector_field_action}) is an instance of a $\gamma$-extended toroidal Lie algebra, and this will be achieved through lemma \ref{lemma: semi_direct_product_of_toroidal_lie_algebras_with_div_zero_vector_fields_are_yangian_extended_toroidal_lie_algebras}. \item Using this, we will then prove in proposition \ref{proposition: twisted_semi_direct_products_are_yangian_extended_toroidal_lie_algebras} that a given Lie $2$-cocycle:
                        $$\sigma \in Z^2_{\Lie}(\divzero, \z(\toroidal))$$
                    must sastisfy:
                        $$( \sigma(D, D'), D'' )_{\extendedtoroidal} = ( D, \sigma(D', D'') )_{\extendedtoroidal}$$
                    so that the correpsonding twisted semi-direct product:
                        $$\toroidal \rtimes^{\sigma} \divzero$$
                    would a $\gamma$-extended toroidal Lie algebra. This will conclude the proof.
                \end{enumerate}
                \item In the \say{only if} direction, the proof - which shall occupy subsection \ref{subsection: yangian_extended_toroidal_lie_algebras_are_twisted_semi_direct_products} and culminate in proposition \ref{prop: yangian_extended_toroidal_lie_algebras_are_twisted_semi_direct_products} - is more involved and somehow much less \say{routine}. Broadly, our strategy is to make use of proposition \ref{prop: twisted_semi_direct_product_criterion}, which requires two inputs:
                \begin{enumerate}
                    \item Firstly, given any $\gamma$-extended toroidal Lie algebra structure $[-, -]_{\extendedtoroidal}$ on $\extendedtoroidal$, the $\divzero$-action $\rho: \divzero \to \der(\toroidal)$ constructed in corollary \ref{coro: a_fixed_yangian_div_zero_vector_field_action} shall then satisfy:
                        $$[D, X]_{\extendedtoroidal} = \rho(D)(X)$$
                    for all $D \in \divzero$ and all $X \in \toroidal$. We will be verifying this requirement via lemmas \ref{lemma: derivation_action_on_multiloop_algebras}, \ref{lemma: derivation_action_on_toroidal_centres}, and \ref{lemma: yangian_extended_toroidal_lie_algebras_are_extensions} in that sequence.
                    \item Secondly, in equipping the vector space $\extendedtoroidal := \toroidal \oplus \divzero$ with the Lie bracket $[-, -]_{\extendedtoroidal}$ as above, one shall obtain a Lie algebra extension:
                        $$0 \to \toroidal \to \extendedtoroidal \to \divzero \to 0$$
                    This will be checked in lemma \ref{lemma: yangian_extended_toroidal_lie_algebras_are_extensions}.
                \end{enumerate}
                Let us also note that, in verifying that these requirements are satisfied, we will be relying heavily on the fact that $(-, -)_{\extendedtoroidal}$ is invariant with respect to $[-, -]_{\extendedtoroidal}$.
            \end{itemize}
        \end{proof}

    \subsection{Which twisted semi-direct products are \texorpdfstring{$\gamma$}{}-extended toroidal Lie algebras ?} \label{subsection: which_twisted_semi_direct_products_are_yangian_extended_toroidal_lie_algebras}
        In this subsection, we prove the \say{if} direction of theorem \ref{theorem: yangian_extended_toroidal_lie_algebras_main_theorem}. To this end, let us firstly fix a twisted semi-direct product:
            $$\toroidal \rtimes^{\sigma} \divzero$$
        (with corresponding Lie $2$-cocycle $\sigma \in Z^2_{\Lie}(\divzero, \z(\toroidal))$) such that:
            $$[D, X]_{\sigma} = \rho(D) \cdot X$$
        (for all $X \in \toroidal$ and $D \in \divzero$, and with $\rho: \divzero \to \der(\toroidal)$ being as in corollary \ref{coro: a_fixed_yangian_div_zero_vector_field_action}). Let us also denote the Lie bracket on this twisted semi-direct product by:
            $$[-, -]_{\sigma}$$

        \begin{remark}[What do we need to show ?] \label{remark: yangian_extended_toroidal_lie_algebras_main_theorem_if_direction_proof_outline}
            Obviously, there is a vector space isomorphism between $\toroidal \rtimes^{\sigma} \divzero$ and $\toroidal \oplus \z(\toroidal)^{\star}$, namely:
                $$\id_{\toroidal} \oplus \varphi: \toroidal \rtimes^{\sigma} \divzero \to \toroidal \oplus \z(\toroidal)^{\star}$$
            with $\varphi: \divzero \to \z(\toroidal)$ as in proposition \ref{prop: yangian_div_zero_vector_fields_are_graded_dual_to_toroidal_centre}. It is also clear (per definition \ref{def: twisted_semi_direct_products}), that the canonical inclusion of vector spaces $\toroidal \subset \toroidal \rtimes^{\sigma} \divzero$ is a Lie algebra monomorphism. As such, the only thing to prove is that the non-degenerate symmetric bilinear form given by:
                $$(-, -)_{\extendedtoroidal} \circ (\id_{\toroidal} \oplus \varphi)^{\tensor 2}$$
            is invariant with respect to the Lie bracket $[-, -]_{\sigma}$.

            To prove that $(-, -)_{\extendedtoroidal}$ is invariant with respect to $[-, -]_{\sigma}$, one must check that for all:
                $$X, Y, Z \in \extendedtoroidal$$
            the following identity holds:
                $$( [X, Y]_{\sigma}, Z )_{\extendedtoroidal} = ( X, [Y, Z]_{\sigma} )_{\extendedtoroidal}$$
            Because of how the bilinear form $(-, -)_{\extendedtoroidal}$ is constructed (cf. lemma \ref{lemma: extended_toroidal_bilinear_form} and corollary \ref{coro: pairing_yangian_div_zero_vector_fields_and_cyclic_1_forms}), there are $10$ separate cases to check, namely wherein $(X, Y, Z)$ - as an \textit{ordered} tuple - is an element of the following sets, respectively:
                $$\g_{[2]} \x \z(\toroidal) \x \g_{[2]}$$
                $$\g_{[2]} \x \z(\toroidal) \x \z(\toroidal)$$
                $$\g_{[2]} \x \z(\toroidal) \x \divzero$$
                $$\g_{[2]} \x \divzero \x \z(\toroidal)$$
                $$\g_{[2]} \x \divzero \x \divzero$$
                $$\z(\toroidal) \x \divzero \x \z(\toroidal)$$
                
                $$\g_{[2]} \x \g_{[2]} \x \g_{[2]}$$
                $$\g_{[2]} \x \g_{[2]} \x \divzero$$
                $$\divzero \x \divzero \x \z(\toroidal)$$
                $$\divzero \x \divzero \x \divzero$$
            However, the only the last four cases are non-trivial, which simplifies our task significantly. Furthermore, even though the last case is not trivial for a general $\sigma$. it is trivial when $\sigma = 0$, since $(\divzero, \divzero)_{\extendedtoroidal} = 0$; therefore, we will deal with the case where $\sigma = 0$ first.
        \end{remark}

        In accordance with the notations above, let us write $[-, -]_0$ for the bracket on the semi-direct product $\toroidal \rtimes \divzero$.
        \begin{lemma} \label{lemma: semi_direct_product_of_toroidal_lie_algebras_with_div_zero_vector_fields_are_yangian_extended_toroidal_lie_algebras}
            Let $\divzero$ act on $\toroidal$ via $\rho$ as in corollary \ref{coro: a_fixed_yangian_div_zero_vector_field_action}. The semi-direct product $\toroidal \rtimes \divzero$ - corresponding to the $2$-cocycle:
                $$0 \in Z^2_{\Lie}(\divzero, \z(\toroidal))$$
            - is a $\gamma$-extended toroidal Lie algebra in the sense of definition \ref{def: yangian_extended_toroidal_lie_algebras}.
        \end{lemma}
            \begin{proof}
                Again, as stated in remark \ref{remark: yangian_extended_toroidal_lie_algebras_main_theorem_if_direction_proof_outline}, one must check that for all:
                    $$X, Y, Z \in \extendedtoroidal$$
                the following identity holds:
                    $$( [X, Y]_0, Z )_{\extendedtoroidal} = ( X, [Y, Z]_0 )_{\extendedtoroidal}$$
                and this can be done by verifying the following cases. Recall also, from example \ref{example: lie_algebra_semi_direct_products}, that the semi-direct product Lie bracket $[-, -]_0$ is as follows for all $X, Y \in \toroidal$ and all $D, D' \in \divzero$:
                    $$[X + D, Y + D']_0 := [X, Y]_{\toroidal} + \rho(D)(Y) - \rho(D')(X) + [D, D']$$
                \begin{enumerate}
                    \item Assume firstly that $X, Y, Z \in \g_{[2]}$. Because we have that:
                        $$(\toroidal, \toroidal)_{\extendedtoroidal} := (\toroidal, \toroidal)_{\toroidal}$$
                    per the construction of $(-, -)_{\extendedtoroidal}$ (cf. lemma \ref{lemma: extended_toroidal_bilinear_form} and corollary \ref{coro: pairing_yangian_div_zero_vector_fields_and_cyclic_1_forms}) and because $(-, -)_{\toroidal}$ is necessarily invariant with respect to $[-, -]_{\toroidal}$, we have that:
                        $$( [X, Y]_0, Z )_{\extendedtoroidal} = ( [X, Y]_{\toroidal}, Z )_{\toroidal} = ( X, [Y, Z]_{\toroidal} )_{\toroidal} = ( X, [Y, Z]_0 )_{\extendedtoroidal}$$
                    which gives the desired invariance property of $(-, -)_{\extendedtoroidal}$ on $X, Y, Z \in \g_{[2]}$.
                    \item Next, consider $X, Y \in \g_{[2]}$ but $Z \in \divzero$. Without any loss of generality, suppose that:
                        $$X := xf, Y := yg$$
                    for some $x, y \in \g$ and some $f, g \in A$. Due to the fact that $(\g_{[2]}, \divzero)_{\extendedtoroidal} = 0$ (cf. lemma \ref{lemma: extended_toroidal_bilinear_form} and corollary \ref{coro: pairing_yangian_div_zero_vector_fields_and_cyclic_1_forms}), we shall have that:
                        $$( [X, Y]_0, Z )_{\extendedtoroidal} = ( [X, Y]_{\toroidal}, Z )_{\toroidal} = (x, y)_{\g} ( g \bar{d}f, Z )_{\extendedtoroidal}$$
                    and likewise, that:
                        $$
                            \begin{aligned}
                                & ( X, [Y, Z]_0 )_{\extendedtoroidal}
                                \\
                                = & ( X, [Y, Z]_{\toroidal} )_{\extendedtoroidal}
                                \\
                                = & -( xf, \rho(Z)(yg))_{\extendedtoroidal}
                                \\
                                = & -(x, y)_{\g} (f, Z(g))_{\extendedtoroidal}
                                \\
                                = & -(x, y)_{\g} \gamma(f Z(g))
                                \\
                                = & -(x, y)_{\g} (f \bar{d}g, Z)_{\extendedtoroidal}
                            \end{aligned}
                        $$
                    Now, since $-f \bar{d}g = g\bar{d}f$, per the fact that $\z(\toroidal) \cong \bar{\Omega}^1_{A/\bbC}$ (cf. example \ref{example: affine_lie_algebras_centres}), we are done. 
                    \item Finally, consider $X, Y \in \divzero$ and $Z \in \z(\toroidal)$. Without any loss of generality, assume that:
                        $$Z := g\bar{d}f$$
                    for some $f, g \in A$. In this case, we have:
                        $$( [X, Y]_0, Z )_{\extendedtoroidal} = ( [X, Y], g\bar{d}f )_{\extendedtoroidal} = ( XY, g\bar{d}f )_{\extendedtoroidal} - ( YX, g\bar{d}f )_{\extendedtoroidal}$$
                    Using the fact that $b\bar{d}a = -a\bar{d}b$ for all $a, b \in A$, we can furthermore rewrite the above into:
                        $$
                            \begin{aligned}
                                & ( [X, Y]_0, Z )_{\extendedtoroidal}
                                \\
                                = & ( XY, g\bar{d}f )_{\extendedtoroidal} - ( YX, g\bar{d}f )_{\extendedtoroidal}
                                \\
                                = & ( XY, g\bar{d}f )_{\extendedtoroidal} + ( YX, f\bar{d}g )_{\extendedtoroidal}
                                \\
                                = & \gamma( g X(Y(f)) + f Y(X(g)) )
                                \\
                                = & ( X, g \bar{d}(Y(f)) )_{\extendedtoroidal} + ( Y, f \bar{d}(X(g) ) )_{\extendedtoroidal}
                            \end{aligned}
                        $$
                    At the same time, we have that:
                        $$
                            \begin{aligned}
                                & ( X, [Y, Z]_0 )_{\extendedtoroidal}
                                \\
                                = & ( X, \rho(Y)(g \bar{d}f) )_{\extendedtoroidal}
                                \\
                                = & ( X, Y(g) \bar{d}f + g \bar{d}(Y(f)) )_{\extendedtoroidal}
                            \end{aligned}
                        $$
                    It therefore remains to show that:
                        $$( Y, f \bar{d}(X(g) ) )_{\extendedtoroidal} = ( X, Y(g) \bar{d}f )_{\extendedtoroidal}$$
                    For this, consider the following:
                        $$( Y, f \bar{d}(X(g) ) )_{\extendedtoroidal} = \gamma( f Y(X(g)) )$$
                    while at the same time, we have that:
                        $$( X, Y(g) \bar{d}f )_{\extendedtoroidal} = \gamma( Y(g) X(f) ) = ( Y, X(f) \bar{d}g )_{\extendedtoroidal} = -(Y, g \bar{d}(X(f)) )_{\extendedtoroidal} = -\gamma( g Y(X(f)) )$$
                    From these two observations, we shall have that:
                        $$( Y, f \bar{d}(X(g) ) )_{\extendedtoroidal} - ( X, Y(g) \bar{d}f )_{\extendedtoroidal} = \gamma( f Y(X(g)) + g Y(X(f)) ) = ( YX, f \bar{d}g + g \bar{d}f )_{\extendedtoroidal} = 0$$
                    where the last equality holds because $f \bar{d}g + g \bar{d}f = 0$. We have therefore shown that $( Y, f \bar{d}(X(g) ) )_{\extendedtoroidal} = ( X, Y(g) \bar{d}f )_{\extendedtoroidal}$, as needed, so we are now done.
                \end{enumerate}
            \end{proof}
            
        \begin{proposition}[A $\gamma$-invariance criterion for toroidal $2$-cocycles] \label{proposition: twisted_semi_direct_products_are_yangian_extended_toroidal_lie_algebras}
            Consider a Lie $2$-cocycle $\sigma \in Z^2_{\Lie}(\divzero, \z(\toroidal))$. The corresponding twisted semi-direct product:
                $$\toroidal \rtimes^{\sigma} \divzero$$
            will then be a $\gamma$-extended toroidal Lie algebra if:
                $$(\sigma(D, D'), D'')_{\extendedtoroidal} = (D, \sigma(D', D''))_{\extendedtoroidal}$$
            for all $D, D', D'' \in \divzero$.
        \end{proposition}
            \begin{proof}
                Pick arbitrary elements $D, D', D'' \in \divzero$ and then consider the following:
                    $$
                        \begin{aligned}
                            ([D', D'']_{\sigma}, D'')_{\extendedtoroidal} & = ([D, D']_0 + \sigma(D, D'))_{\extendedtoroidal}
                            \\
                            & = ([D, D']_0, D'')_{\extendedtoroidal} + (\sigma(D, D'), D'')_{\extendedtoroidal}
                            \\
                            & = (D, [D', D'']_0)_{\extendedtoroidal} + (\sigma(D, D'), D'')_{\extendedtoroidal}
                            \\
                            & = (D, [D', D'']_0)_{\extendedtoroidal} + (\sigma(D, D'), D'')_{\extendedtoroidal}
                        \end{aligned}
                    $$
                where the last equality is because $\sigma(D, D') \in \z(\toroidal)$. Similarly, we have that:
                    $$(D', [D'', D'']_{\sigma})_{\extendedtoroidal} = ([D, D']_0, D'')_{\extendedtoroidal} + (D, \sigma(D', D''))_{\extendedtoroidal}$$
                As lemma \ref{lemma: semi_direct_product_of_toroidal_lie_algebras_with_div_zero_vector_fields_are_yangian_extended_toroidal_lie_algebras} shows, the semi-direct product $\toroidal \rtimes \divzero$ is a $\gamma$-extended toroidal Lie algebra, so we have that:
                    $$(D, [D', D'']_0)_{\extendedtoroidal} = ([D, D']_0, D'')_{\extendedtoroidal}$$
                We have also assumed:
                    $$(\sigma(D, D'), D'')_{\extendedtoroidal} = (D, \sigma(D', D''))_{\extendedtoroidal}$$
                Together, these two facts imply that:
                    $$([D', D'']_{\sigma}, D'')_{\extendedtoroidal} = (D', [D'', D'']_{\sigma})_{\extendedtoroidal}$$
                which is precisely what we need so we are done.
            \end{proof}

        We have thus proven the \say{if} implication in theorem \ref{theorem: yangian_extended_toroidal_lie_algebras_main_theorem}.

    \subsection{\texorpdfstring{$\gamma$}{}-extended toroidal Lie algebras are twisted semi-direct products} \label{subsection: yangian_extended_toroidal_lie_algebras_are_twisted_semi_direct_products}
        Let us fix a Lie bracket:
            $$[-, -]_{\extendedtoroidal}: \bigwedge^2 \extendedtoroidal \to \extendedtoroidal$$
        on the vector space $\extendedtoroidal := \toroidal \oplus \divzero$. Since we are now proving the \say{only if} direction of theorem \ref{theorem: yangian_extended_toroidal_lie_algebras_main_theorem}, let us assume that $(\extendedtoroidal, [-, -]_{\extendedtoroidal})$ is a $\gamma$-extended toroidal Lie algebra. In particular, let us recall from definition \ref{def: yangian_extended_toroidal_lie_algebras} that the bilinear form:
            $$(-, -)_{\extendedtoroidal}$$
        is invariant with respect to $[-, -]_{\extendedtoroidal}$.

        \begin{lemma}[$\divzero$ acts on $\g_{[2]}$ by derivations] \label{lemma: derivation_action_on_multiloop_algebras}
            For any $D \in \divzero$ and any $x \in \g, f \in A$, we have that:
                $$[D, xf]_{\extendedtoroidal} = \rho(D)(xf) + K( D, xf ) = x D(f) + K( D, xf )$$
            for some yet-unknown $K( D, xf ) \in \z(\toroidal)$ and with $\rho: \divzero \to \der(\toroidal)$ as in corollary \ref{coro: a_fixed_yangian_div_zero_vector_field_action}. As such, we have that:
                $$[\divzero, \g_{[2]}]_{\extendedtoroidal} \subseteq \g_{[2]} \oplus \z(\toroidal)$$
            as of now\footnote{In lemma \ref{lemma: no_polynomial_terms_for_derivation_action_on_multiloop_algebras}, it will be shown that in fact, the $\z(\toroidal)$-summand $K( D, xf )$ vanishes.}.
        \end{lemma}
            \begin{proof}
                Consider firstly the following, for any $D \in \divzero$ and any $x, y \in \g, f, g \in A$:
                    $$( [D, xf]_{\extendedtoroidal}, yg )_{\extendedtoroidal} = ( D, [xy, fg]_{\toroidal} )_{\extendedtoroidal} = (x, y)_{\g} ( D, g\bar{d}f )_{\extendedtoroidal} = (x, y)_{\g} \gamma( g D(f) )$$
                At the same time, we have that:
                    $$( x D(f), yg )_{\extendedtoroidal} = (x, y)_{\g} \gamma( g D(f) )$$
                Clearly, then, we have that:
                    $$( [D, xf]_{\extendedtoroidal}, yg )_{\extendedtoroidal} = ( x D(f), yg )_{\extendedtoroidal}$$
                Since $yg \in \g_{[2]}$ is arbitrary and since:
                    $$(\z(\toroidal), \g_{[2]})_{\extendedtoroidal} = (\divzero, \g_{[2]})_{\extendedtoroidal} = 0$$
                per the construction of $(-, -)_{\extendedtoroidal}$, the above implies via the non-degeneracy of $(-, -)_{\extendedtoroidal}$ that there exists some $K(D, xf) \in \z(\toroidal)$ and some $\xi(D, xf) \in \divzero$ such that:
                    $$[D, xf]_{\extendedtoroidal} = x D(f) + K(D, xf) + \xi(D, xf) = \rho(D)(xf) + K(D, xf) + \xi(D, xf)$$
                with $\rho: \divzero \to \der(\toroidal)$ as in corollary \ref{coro: a_fixed_yangian_div_zero_vector_field_action}.

                We claim now that we actually also have that:
                    $$\xi(D, xf) = 0$$
                for all $D \in \divzero$ and all $x \in \g, f \in A$. To see why this is true, pick an arbitrary element $K \in \z(\toroidal)$ and then consider the following, which holds once again thanks to the invariance of $(-, -)_{\extendedtoroidal}$ and also, due to the fact that $(\toroidal, \toroidal)_{\extendedtoroidal} = (\toroidal, \toroidal)_{\toroidal}$ per the construction of $(-, -)_{\extendedtoroidal}$:
                    $$0 = ( D, [xf, K]_{\toroidal} ) = ( [D, xf]_{\extendedtoroidal}, K )_{\extendedtoroidal} = ( \xi(D, f), K )$$
                Via non-degeneracy again, and because $K$ was chosen arbitrarily, we have that:
                    $$\xi(D, f) = 0$$
                necessarily.

                In conclusion, we have shown that:
                    $$[D, xf]_{\extendedtoroidal} = x D(f) + K(D, xf)$$
                for all $D \in \divzero$ and all $x \in \g, f \in A$, and with $K(D, xf) \in \z(\toroidal)$ not yet determined.
            \end{proof}

        \begin{lemma}[$\divzero$ acts on $\z(\toroidal)$ by Lie derivatives] \label{lemma: derivation_action_on_toroidal_centres}
            One has that\footnote{We will be referring to this fact as $\divzero$ (and in fact, $\der(A)$ as well) acting on $\z(\toroidal)$ via Lie derivatives.}:
                $$[D, K]_{\extendedtoroidal} = \rho(D)(K)$$
            for all $D \in \divzero$ and all $K \in \z(\toroidal)$, and with $\rho: \divzero \to \der(\toroidal)$ being the action of $\divzero$ on $\toroidal$ as in corollary \ref{coro: a_fixed_yangian_div_zero_vector_field_action}. In particular, this means that:
                $$[\divzero, \z(\toroidal)]_{\extendedtoroidal} \subseteq \z(\toroidal)$$
        \end{lemma}
            \begin{proof}
                Let $f, g \in A$. Since elements of the form $f \bar{d}g$ span $\z(\toroidal)$ as a vector space (cf. theorem \ref{theorem: kassel_realisation}), it shall suffice to prove that:
                    $$[D, g \bar{d}f]_{\extendedtoroidal} = \rho(D)( g\bar{d}f ) = D(g) \bar{d}f + g \bar{d}(D(f))$$
                for an arbitrary $D \in \divzero$. To this end, let us firstly recall from theorem \ref{theorem: kassel_realisation} that for any $h, h \in \g$ and any $f, g \in A$, one has that:
                    $$[h f, h' g]_{\toroidal} = [h, h']_{\g} fg + (h, h')_{\g} g \bar{d}f$$
                From this, one infers that:
                    $$[D, g \bar{d}f]_{\extendedtoroidal} = [ D, [hf, h' g]_{\toroidal} ]_{\extendedtoroidal}$$
                for any choices of $h, h \in \h$ - which gives $[h, h']_{\g} = 0$, since the Cartan subalgebra $\h \subseteq \g$ is abelian (cf. subsection \ref{subsection: finite_dimensional_simple_lie_algebras}) - such that $(h, h')_{\g} = 1$. Then, by exploiting the Jacobi identity, we shall get the following equalities for all $D \in \divzero$:
                    $$
                        \begin{aligned}
                            & [D, g \bar{d}f]_{\extendedtoroidal}
                            \\
                            = & [ D, [h f, h' g]_{\toroidal} ]_{\extendedtoroidal}
                            \\
                            = & [ h f, [D, h' g]_{\extendedtoroidal} ]_{\toroidal} + [ [D, h f]_{\extendedtoroidal}, h' g ]_{\toroidal}
                            \\
                            = & [ h f, h' D( g ) ]_{\toroidal} + [ h D( f ), h' g ]_{\toroidal}
                            \\
                            = & D(g) \bar{d}f + g \bar{d}(D(f))
                            \\
                            = & \rho(D)( g \bar{d}f )
                        \end{aligned}
                    $$
                which is precisely as we wanted, so we are done. 
            \end{proof}
        \begin{corollary}[Toroidal Lie algebras are ideals] \label{coro: toroidal_lie_algebras_are_ideals}
            With respect to the bracket $[-, -]_{\extendedtoroidal}$, the vector subspace $\toroidal$ is actually a Lie ideal of $\extendedtoroidal$.
        \end{corollary}

        Finally, let us investigate how the brackets of the form:
            $$[D, D']_{\extendedtoroidal}$$
        are given, for all $D, D' \in \divzero$. Because we would like to show that:
            $$\extendedtoroidal \cong \toroidal \rtimes^{\sigma} \divzero$$
        we shall need to show that there exists some $\sigma \in Z^2_{\Lie}(\divzero, \z(\toroidal))$ such that:
            $$[D, D']_{\extendedtoroidal} = [D, D'] + \sigma(D, D')$$
        where $[D, D'] := DD' - D'D$ is the usual commutator Lie bracket (cf. definition \ref{def: twisted_semi_direct_products}). This shall involve showing, firstly that the $\divzero$-component of any bracket of the form $[D, D']_{\extendedtoroidal}$ is necessarily $[D, D']$ and secondly, that $[D, D']_{\extendedtoroidal} \in \divzero \oplus \z(\toroidal)$ for all $D, D' \in \divzero$.
        \begin{lemma}[$\gamma$-extended toroidal Lie algebras are extensions] \label{lemma: yangian_extended_toroidal_lie_algebras_are_extensions}
            The canonical projection of vector spaces:
                $$\pi: \extendedtoroidal \to \divzero$$
            is a Lie algebra homomorphism. Since $\toroidal \subset \extendedtoroidal$ is a Lie subalgebra by definition (cf. definition \ref{def: yangian_extended_toroidal_lie_algebras}), the Lie bracket $[-, -]_{\extendedtoroidal}$ thus necessarily arises as an extension:
                $$0 \to \toroidal \to \extendedtoroidal \xrightarrow[]{\pi} \divzero \to 0$$
        \end{lemma}
            \begin{proof}
                Fix arbitrary elements $D, D' \in \divzero$ and suppose that:
                    $$[D, D']_{\extendedtoroidal} = X(D, D') + \xi(D, D')$$
                for some $X(D, D') \in \toroidal$ and some $\xi(D, D') \in \divzero$. Next, pick arbitrary elements $f, g \in A$, and then consider the following, which holds thanks to the fact that $(\toroidal, \toroidal)_{\extendedtoroidal} = (\toroidal, \toroidal)_{\toroidal}$ (note that $g \bar{d}f \in \toroidal$ is central):
                    $$( [D, D']_{\extendedtoroidal}, g \bar{d}f )_{\extendedtoroidal} = ( \xi(D, D'), g \bar{d}f )_{\extendedtoroidal}$$
                At the same time, by the invariance of $(-, -)_{\extendedtoroidal}$ with respect to $[-, -]_{\extendedtoroidal}$, we have that:
                    $$
                        \begin{aligned}
                            & ( [D, D']_{\extendedtoroidal}, g \bar{d}f )_{\extendedtoroidal}
                            \\
                            = & ( D, [D', g \bar{d}f]_{\extendedtoroidal} )_{\extendedtoroidal}
                            \\
                            = & ( D, D'(g) \bar{d}f + g \bar{d}(D(f)) )_{\extendedtoroidal}
                            \\
                            = & \gamma( D'(g) D(f) ) + \gamma( g D(D'(f)) )
                            \\
                            = & ( D', D(f) \bar{d}g )_{\extendedtoroidal} + \gamma( g D(D'(f)) )
                        \end{aligned}
                    $$
                Now, using the fact that $b\bar{d}a = -a \bar{d}b$ for all $a, b \in A$, one obtains that:
                    $$( D', D(f) \bar{d}g )_{\extendedtoroidal} = -( D', g \bar{d}(D(f)) )_{\extendedtoroidal} = -\gamma( g D'(D(f)) )$$
                and hence:
                    $$( [D, D']_{\extendedtoroidal}, g \bar{d}f )_{\extendedtoroidal} = \gamma( -g D'(D(f)) + g D(D'(f)) ) = \gamma( g[D, D'](f) ) = ([D, D'], g\bar{d}f)_{\extendedtoroidal}$$
                In turn, this implies that:
                    $$( \xi(D, D'), g \bar{d}f )_{\extendedtoroidal} = ([D, D'], g\bar{d}f)_{\extendedtoroidal}$$
                Via the non-degeneracy of the bilinear form $(-, -)_{\extendedtoroidal}$ and the arbitarity of our choices of $f, g \in A$, we thus get that:
                    $$\xi(D, D') = [D, D']$$
                The lemma follows suite.
            \end{proof}
        It now remains to show that the brackets of the form $[D, D']_{\extendedtoroidal}$ (for any $D, D' \in \divzero$) have no $\g_{[2]}$-summands.
        \begin{proposition}[Appearance of $2$-cocycles $\sigma \in Z^2_{\Lie}(\divzero, \z(\toroidal))$] \label{prop: appearance_of_toroidal_cocycles}
            For any $D, D' \in \divzero$, one has that:
                $$[D, D']_{\extendedtoroidal} \equiv [D, D'] \pmod{\z(\toroidal)}$$
            (where $[D, D'] := DD' - D'D$ is the usual commutator).
        \end{proposition}
            \begin{proof}
                Pick arbitrary elements $D, D' \in \divzero$. From lemma \ref{lemma: yangian_extended_toroidal_lie_algebras_are_extensions}, we know that:
                    $$[D, D']_{\extendedtoroidal} := X(D, D') + Z(D, D') + [D, D']$$
                for some $X(D, D') \in \g_{[2]}, Z(D, D') \in \z(\toroidal)$ depending on $D, D'$. Pick also a test element:
                    $$Y := yg \in \g_{[2]}$$
                for some arbitrary $y \in \g$ and $g \in A$ and set:
                    $$[D, y g]_{\extendedtoroidal} := y D( g ) + K(D, Y)$$
                    $$[D', y g]_{\extendedtoroidal} := y D'( g ) + K(D', Y)$$
                for some $K(D, Y) \in \z(\toroidal)$ depending on $Y$ (cf. lemma \ref{lemma: derivation_action_on_multiloop_algebras}).
                
                Via the Jacobi identity, we get that:
                    $$
                        \begin{aligned}
                            & [ [D, D']_{\extendedtoroidal}, y g ]_{\extendedtoroidal}
                            \\
                            = & [ D, [ D', y g ]_{\extendedtoroidal} ]_{\extendedtoroidal} + [ D', [ y g, D ]_{\extendedtoroidal} ]_{\extendedtoroidal}
                            \\
                            = & [ D, y D'( g ) + K(D', Y) ]_{\extendedtoroidal} - [ D', y D( g ) + K(D, Y) ]_{\extendedtoroidal}
                            \\
                            = & \left( y D( D'(g) ) + K(DD', Y) + [ D, K(D', Y) ]_{\extendedtoroidal} \right) - \left( y D'( D(g) ) + K(D'D, Y) + [ D', K(D, Y) ]_{\extendedtoroidal} \right)
                            \\
                            = & y (DD' - D'D)( g ) + ( K(DD', Y) - K(D'D, Y) ) + ( [ D, K(D', Y) ]_{\extendedtoroidal} - [ D', K(D, Y) ]_{\extendedtoroidal} )
                        \end{aligned}
                    $$
                for some $K(DD', Y), K(D'D, Y) \in \z(\toroidal)$ such that:
                    $$[ D, y D'( g ) ]_{\extendedtoroidal} := y D( D'( g ) ) + K(DD', Y)$$
                    $$[ D', y D( g ) ]_{\extendedtoroidal} := y D( D'( g ) ) + K(D'D, Y)$$
                At the same time, we have that:
                    $$
                        \begin{aligned}
                            & [ [D, D']_{\extendedtoroidal}, y g ]_{\extendedtoroidal}
                            \\
                            = & [ X(D, D') + Z(D, D') + [D, D'] , y g ]_{\extendedtoroidal}
                            \\
                            = & [ X(D, D') + [D, D'] , y g ]_{\extendedtoroidal}
                            \\
                            = & [ X(D, D') , y g ]_{\extendedtoroidal} + \left( y [D, D'](g) + K([D, D'], Y) \right)
                        \end{aligned}
                    $$
                wherein the second equality holds thanks to the fact that $[\z(\toroidal), \g_{[2]}]_{\extendedtoroidal} = 0$, and $K([D, D'], Y) \in \z(\toroidal)$ is some element (cf. lemma \ref{lemma: derivation_action_on_multiloop_algebras}). Combining the two observations together then yields:
                    $$
                        \begin{aligned}
                            & [ X(D, D') , y g ]_{\extendedtoroidal} + \left( y [D, D'](g) + K([D, D'], Y) \right)
                            \\
                            = & y (DD' - D'D)( g ) + ( K(DD', Y) - K(D'D, Y) ) + ( [ D, K(D', Y) ]_{\extendedtoroidal} - [ D', K(D, Y) ]_{\extendedtoroidal} )
                        \end{aligned}
                    $$
                which can be simplified into:
                    $$
                        \begin{aligned}
                            & [ X(D, D') , y g ]_{\extendedtoroidal} + K([D, D'], Y)
                            \\
                            = & ( K(DD', Y) - K(D'D, Y) ) + ( [ D, K(D', Y) ]_{\extendedtoroidal} - [ D', K(D, Y) ]_{\extendedtoroidal} )
                        \end{aligned}
                    $$
                From lemma \ref{lemma: derivation_action_on_multiloop_algebras}, we know that there exists $K( X(D, D'), Y ) \in \z(\toroidal)$ such that:
                    $$[ X(D, D') , y g ]_{\extendedtoroidal} = [ X(D, D') , Y ]_{\extendedtoroidal} = [X(D, D'), Y]_{\g_{[2]}} + K( X(D, D'), Y )$$
                using which we can write:
                    $$
                        \begin{aligned}
                            & [X(D, D'), Y]_{\g_{[2]}}
                            \\
                            = & \left( [ D, K(D', Y) ]_{\extendedtoroidal} - [ D', K(D, Y) ]_{\extendedtoroidal} \right) - \left( K( X(D, D'), Y ) + K([D, D'], Y) \right)
                        \end{aligned}
                    $$
                    
                We note right away that the LHS lies entirely in $\g_{[2]}$, whereas the RHS is an element of $\z(\toroidal)$ due to the fact that $[\divzero, \z(\toroidal)]_{\extendedtoroidal} \subseteq \z(\toroidal)$ (cf. lemma \ref{lemma: derivation_action_on_toroidal_centres}), which tells us that:
                    $$[ D, K(D', Y) ]_{\extendedtoroidal}, [ D', K(D, Y) ]_{\extendedtoroidal} \in \z(\toroidal)$$
                in particular. Because $\g_{[2]}$ is centre-less (as $\g$ is simple and the Lie bracket on $\g_{[2]}$ is given by extension of scalars), this observation subsequently implies that the LHS must vanish, i.e.:
                    $$[X(D, D'), Y]_{\g_{[2]}} = 0$$
                We remind the reader now that $Y \in \g_{[2]}$ was chosen arbitrarily, and again, because $\g_{[2]}$ is centre-less, we now have that:
                    $$X(D, D') = 0$$
                necessarily. In other words, the brackets of the form $[D, D']_{\extendedtoroidal}$ have no $\g_{[2]}$-summand for any $D, D' \in \divzero$, and hence:
                    $$[D, D']_{\extendedtoroidal} \equiv [D, D'] \pmod{\z(\toroidal)}$$
            \end{proof}

        \begin{lemma}[\texorpdfstring{$\z(\toroidal)$}{}-summands of elements of \texorpdfstring{$[\divzero, \g_{[2]}]_{\extendedtoroidal}$}{}] \label{lemma: no_polynomial_terms_for_derivation_action_on_multiloop_algebras}
            For any $x \in \g, f \in A$ and any $D \in \divzero$, we have that:
                $$[D, xf]_{\extendedtoroidal} = x D(f) = \rho(D)(xf)$$
            with $\rho: \divzero \to \der(\toroidal)$ as in corollary \ref{coro: a_fixed_yangian_div_zero_vector_field_action}. Consequently one has that:
                $$[\divzero, \g_{[2]}]_{\extendedtoroidal} \subseteq \g_{[2]}$$
        \end{lemma}
            \begin{proof}
                From lemma \ref{lemma: derivation_action_on_multiloop_algebras}, we know that given some $D \in \divzero$ and some $x \in \g$ and $f \in A$, there shall exist $K(D, xf) \in \z(\toroidal)$ (depending on the choices of $D$ and $x, f$) such that:
                    $$[D, xf]_{\extendedtoroidal} = x D(f) + K(D, xf)$$
                Next, consider the following:
                    $$( \divzero, x D(f) + K(D, xf) )_{\extendedtoroidal} = ( \divzero, [D, xf]_{\extendedtoroidal} )_{\extendedtoroidal} = ( [\divzero, D]_{\extendedtoroidal}, xf )_{\extendedtoroidal} = 0$$
                where the second equality holds thanks to invariance, and the third equality holds due to a combination of the fact that $[\divzero, \divzero]_{\extendedtoroidal} \subset \z(\toroidal) \oplus \divzero$ (cf. proposition \ref{prop: appearance_of_toroidal_cocycles}) and the fact that $(\z(\toroidal) \oplus \divzero, \g_{[2]})_{\extendedtoroidal} = 0$ per the construction of the bilinear form $(-, -)_{\extendedtoroidal}$. We also have the following, again per the construction of the bilinear form $(-, -)_{\extendedtoroidal}$:
                    $$( \divzero, x D(f) + K(D, xf) )_{\extendedtoroidal} = ( \divzero, K(D, xf) )_{\extendedtoroidal}$$
                This implies that:
                    $$( \divzero, K(D, xf) )_{\extendedtoroidal} = 0$$
                for all $D \in \divzero$ and all $x \in \g, f \in A$. The fact that $( \divzero, \z(\toroidal) )_{\extendedtoroidal} \not = 0$ along with the non-degeneracy of $(-, -)_{\extendedtoroidal}$ then imply together that:
                    $$K(D, xf) = 0$$
                necessarily. This means that, indeed, we have that:
                    $$[D, xf]_{\extendedtoroidal} = x D(f)$$
                for all $D \in \divzero$ and all $x \in \g$ and all $f \in A$. Since $\g_{[2]}$ is generated by elements of the form $xf$, this implies that:
                    $$[\divzero, \g_{[2]}]_{\extendedtoroidal} \subseteq \g_{[2]}$$
                as claimed. 
            \end{proof}

        We have now yielded the following intermediate conclusion:
        \begin{proposition}[$\toroidal$ as a $\divzero$-module] \label{prop: toroidal_lie_algebras_as_modules_over_div_zero_vector_field_lie_algebras}
            The $\divzero$-module structure:
                $$\rho: \divzero \to \der(\toroidal)$$
            satisfies:
                $$[D, X]_{\extendedtoroidal} = \rho(D)(X)$$
            for all $D \in \divzero$ and all $X \in \toroidal$, regardless of our choice of the Lie bracket $[-, -]_{\extendedtoroidal}$ on the vector space $\extendedtoroidal$.
        \end{proposition}
            \begin{proof}
                Combine lemmas \ref{lemma: derivation_action_on_multiloop_algebras} and \ref{lemma: no_polynomial_terms_for_derivation_action_on_multiloop_algebras}, and lemma \ref{lemma: derivation_action_on_toroidal_centres}, with corollary \ref{coro: a_fixed_yangian_div_zero_vector_field_action}.
            \end{proof}
            
        All the requirements for applying proposition \ref{prop: twisted_semi_direct_product_criterion} are now available, and the following proposition concludes our proof of the \say{only if} direction of the main theorem.
        \begin{proposition}[$\gamma$-extended toroidal Lie algebras are twisted semi-direct products] \label{prop: yangian_extended_toroidal_lie_algebras_are_twisted_semi_direct_products}
            Let $\extendedtoroidal$ be a $\gamma$-extended toroidal Lie algebra in the sense of definition \ref{def: yangian_extended_toroidal_lie_algebras}. Then:
            \begin{enumerate}
                \item there will exist some Lie $2$-cocycle $\sigma \in Z^2_{\Lie}(\divzero, \z(\toroidal))$ so that:
                    $$\toroidal \rtimes^{\sigma} \divzero \xrightarrow[\cong]{\id_{\toroidal} \oplus \varphi} \extendedtoroidal$$
                where $\varphi: \divzero \xrightarrow[]{\cong} \z(\toroidal)^{\star}$ being as in proposition \ref{prop: yangian_div_zero_vector_fields_are_graded_dual_to_toroidal_centre}, and
                \item furthermore, the Lie $2$-cocycle $\sigma$ as above satisfies $\gamma$-invariance, i.e. that:
                    $$(\sigma(D, D'), D'')_{\extendedtoroidal} = (D, \sigma(D', D''))_{\extendedtoroidal}$$
                for all $D, D', D'' \in \divzero$.
            \end{enumerate}
        \end{proposition}
            \begin{proof}
                \begin{enumerate}
                    \item Combine proposition \ref{prop: toroidal_lie_algebras_as_modules_over_div_zero_vector_field_lie_algebras} with lemma \ref{lemma: yangian_extended_toroidal_lie_algebras_are_extensions}, and then apply proposition \ref{prop: twisted_semi_direct_product_criterion}.
                    \item Per definition \ref{def: twisted_semi_direct_products}, we know that:
                        $$\sigma(\xi, \xi') = [\xi, \xi']_{\extendedtoroidal} - [\xi, \xi]$$
                    for all $\xi, \xi' \in \divzero$, meaning that we have that:
                        $$
                            \begin{aligned}
                                & (\sigma(D, D'), D'')_{\extendedtoroidal}
                                \\
                                = & ([D, D]_{\extendedtoroidal} - [D, D'], D'')_{\extendedtoroidal}
                                \\
                                = & ([D, D]_{\extendedtoroidal}, D'')_{\extendedtoroidal} 
                                \\
                                = & (D, [D', D'']_{\extendedtoroidal})
                                \\
                                = & (D, [D', D'']_{\extendedtoroidal} - [D', D''])_{\extendedtoroidal}
                                \\
                                = & (D, \sigma(D', D''))_{\extendedtoroidal}
                            \end{aligned}
                        $$
                    wherein the first and second-to-last equalities hold because $(\divzero, \divzero)_{\extendedtoroidal} := 0$ per the construction of the bilinear form $(-, -)_{\extendedtoroidal}$. This is precisely as needed, so we are done.
                \end{enumerate}
            \end{proof}

        \newpage

        \section{Structure of \texorpdfstring{$\gamma$}{}-extended toroidal Lie algebras}
    \subsection{Centres of \texorpdfstring{$\gamma$}{}-extended toroidal Lie algebras}
        Fix an arbitrary $\gamma$-extended toroidal Lie algebra $\extendedtoroidal$.
    
        Using lemma \ref{lemma: yangian_div_zero_vector_fields_basic_properties} in conjunction with lemma \ref{lemma: derivation_action_on_toroidal_centres}, we can now also explicitly compute the commutation relations between the basis elements of $\divzero$ and $\z(\toroidal)$.
        \begin{lemma}[Explicit commutators between basis elements of $\divzero$ and $\z(\toroidal)$] \label{lemma: explicit_commutators_between_central_basis_elements_and_derivations}
            In the Lie algebra $\extendedtoroidal$, one has the following commutation relations between elements of $\z(\toroidal)$ and those of $\divzero$. Namely, for all $D \in \divzero$, the following relations hold:
                $$
                    \forall (a, b) \in \Z^2: [D, K_{a, b}]_{\extendedtoroidal} =
                    \begin{cases}
                        \text{$((b - 1)r - sa) K_{a - r, b - s - 1} + \delta_{(r, s + 1), (a, b)} \left( r c_v + s c_t \right)$ if $D = D_{r, s}$}
                        \\
                        \text{$a K_{a, b - 1}$ if $D_v$}
                        \\
                        \text{$b K_{a, b - 1}$ if $D_t$}
                    \end{cases}
                $$
                $$[D, c_v]_{\extendedtoroidal} = [D, c_t]_{\extendedtoroidal} = 0$$
        \end{lemma}
            \begin{proof}
                For this, we shall be making use of invariance again, namely:
                    $$(D, [D', K]_{\extendedtoroidal})_{\extendedtoroidal} = ([D, D']_{\extendedtoroidal}, K)_{\extendedtoroidal}$$
                for all $D, D' \in \divzero$ and all $K \in \z(\toroidal)$, and how the brackets $[D, D']_{\extendedtoroidal}$ are given explicitly (cf. lemma \ref{lemma: yangian_div_zero_vector_fields_basic_properties}) as well as how basis elements $K \in \z(\toroidal)$ pair with basis elements of $\divzero$ in the construction of $(-, -)_{\extendedtoroidal}$. Without any loss of generality, we can assume that $D, D' \in \divzero$ and $K \in \z(\toroidal)$ are basis elements, i.e.:
                    $$D, D' \in \{D_{r, s}\}_{(r, s) \in \Z^2} \cup \{D_v, D_t\}$$
                    $$K \in \{K_{a, b}\}_{(a, b) \in \Z^2} \cup \{c_v, c_t\}$$
                and then perform the computations case-by-case, for which we shall recall from example \ref{example: toroidal_lie_algebras_centres} that:
                    $$
                        K_{a, b} :=
                        \begin{cases}
                            \text{$\frac1b v^{a - 1} t^b \bar{d}v$ if $(a, b) \in \Z \x (\Z \setminus \{0\})$}
                            \\
                            \text{$-\frac1a v^a t^{-1} \bar{d}t$ if $(a, b) \in (\Z \setminus \{0\}) \x \{0\}$}
                            \\
                            \text{$0$ if $(a, b) = (0, 0)$}
                        \end{cases}
                    $$
                    $$c_v := v^{-1} \bar{d}v, c_t := t^{-1} \bar{d}t$$
                and from lemma \ref{lemma: yangian_div_zero_vector_fields_basic_properties}, that:
                    $$[D_v, D_t] = 0$$
                    $$[D_v, D_{r, s}] = r D_{r, s + 1}$$
                    $$[D_t, D_{r, s}] = D_{r, s + 1}$$
                    $$[D_{\alpha, \beta}, D_{r, s}] = (\beta r - s \alpha) D_{\alpha + r, \beta + s + 1}$$
                For what follows, let:
                    $$D := \sum_{(\alpha, \beta) \in \Z^2} \lambda_{\alpha, \beta} D_{\alpha, \beta} + \lambda_v D_v + \lambda_t D_t$$
                for some $\lambda_{\alpha, \beta}, \lambda_v, \lambda_t \in \bbC$.
                \begin{enumerate}
                    \item Assume firstly that $K = K_{a, b}$.
                    \begin{enumerate}
                        \item If $D' = D_{r, s}$, then we shall have that:
                            $$
                                \begin{aligned}
                                    & ( D, [D_{r, s}, K_{a, b}]_{\extendedtoroidal} )_{\extendedtoroidal}
                                    \\
                                    = & ( [D, D_{r, s}]_{\extendedtoroidal}, K_{a, b} )_{\extendedtoroidal}
                                    \\
                                    = & \sum_{(\alpha, \beta) \in \Z^2} (\beta r - s \alpha) \lambda_{\alpha, \beta} \delta_{(\alpha + r, \beta + s + 1), (a, b)} + \delta_{(r, s + 1), (a, b)} \left( r\lambda_v + s\lambda_t \right)
                                    \\
                                    = & ((b - s - 1) r - s (a - r)) \lambda_{a - r, b - s - 1} + \delta_{(r, s + 1), (a, b)} \left( r\lambda_v + s\lambda_t \right)
                                    \\
                                    = & ((b - 1)r - sa) \lambda_{a - r, b - s - 1} + \delta_{(r, s + 1), (a, b)} \left( r\lambda_v + s\lambda_t \right)
                                \end{aligned}
                            $$
                        from which we are able to conclude that:
                            $$[D_{r, s}, K_{a, b}]_{\extendedtoroidal} = ((b - 1)r - sa) K_{a - r, b - s - 1} + \delta_{(r, s + 1), (a, b)} \left( r c_v + s c_t \right)$$
                        \item If $D' = D_v$, then:
                            $$
                                \begin{aligned}
                                    & ( D, [D_v, K_{a, b}]_{\extendedtoroidal} )_{\extendedtoroidal}
                                    \\
                                    = & ( [D, D_v]_{\extendedtoroidal}, K_{a, b} )_{\extendedtoroidal}
                                    \\
                                    = & \sum_{(\alpha, \beta) \in \Z^2} \alpha \lambda_{\alpha, \beta} \delta_{(\alpha, \beta + 1), (a, b)} 
                                    \\
                                    = & a \lambda_{a, b - 1}
                                \end{aligned}
                            $$
                        from which we are able to conclude that:
                            $$[D_v, K_{a, b}]_{\extendedtoroidal} = aK_{a, b - 1}$$   
                        \item Finally, if $D' = D_t$, then:
                            $$
                                \begin{aligned}
                                    & ( D, [D_t, K_{a, b}]_{\extendedtoroidal} )_{\extendedtoroidal}
                                    \\
                                    = & ( [D, D_t]_{\extendedtoroidal}, K_{a, b} )_{\extendedtoroidal}
                                    \\
                                    = & \sum_{(\alpha, \beta) \in \Z^2} \beta \lambda_{\alpha, \beta} \delta_{(\alpha, \beta + 1), (a, b)} 
                                    \\
                                    = & b \lambda_{a, b - 1}
                                \end{aligned}
                            $$
                        from which we are able to conclude that:
                            $$[D_t, K_{a, b}]_{\extendedtoroidal} = b K_{a, b - 1}$$
                    \end{enumerate}
                    \item If $K = c_v$ or $K = c_t$ then simply note that because:
                        $$[D, D']_{\extendedtoroidal} \in \bigoplus_{(\alpha, \beta) \in \Z^2} \bbC D_{\alpha, \beta}$$
                    for all $D' \in \divzero$, we shall have that:
                        $$[D, K]_{\extendedtoroidal} = 0$$
                    for all $D := \sum_{(\alpha, \beta) \in \Z^2} \lambda_{\alpha, \beta} D_{\alpha, \beta} + \lambda_v D_v + \lambda_t D_t \in \divzero$.
                \end{enumerate}
            \end{proof}

        \begin{proposition}[Centres of $\gamma$-extended toroidal Lie algebras] \label{prop: centres_of_yangian_extended_toroidal_lie_algebras}
            The centre of $\extendedtoroidal$ is given by:
                $$\z(\extendedtoroidal) = \bbC c_v \oplus \bbC c_t$$
        \end{proposition}
            \begin{proof}
                From example \ref{example: toroidal_lie_algebras_centres} and lemma \ref{lemma: explicit_commutators_between_central_basis_elements_and_derivations}, one sees that:
                    $$c_v, c_t \in \z(\extendedtoroidal)$$
                and hence:
                    $$\z(\extendedtoroidal) \supset \bbC c_v \oplus \bbC c_t$$
                To show the other containment, it shall suffice - due to lemma \ref{lemma: explicit_commutators_between_central_basis_elements_and_derivations} - to demonstrate that $\z(\extendedtoroidal) \subset \z(\toroidal)$; in fact, because $[\toroidal, \z(\toroidal)]_{\extendedtoroidal} = [\toroidal, \z(\toroidal)]_{\toroidal} = 0$ and because $\g_{[2]}$ is centre-less, it shall suffice to show that there does \textit{not} exist $D \in \divzero$ such that:
                    $$[D, \g_{[2]}]_{\extendedtoroidal} = 0$$
                To this end, recall from lemma \ref{lemma: no_polynomial_terms_for_derivation_action_on_multiloop_algebras} that because $\extendedtoroidal$ is a $\gamma$-extended toroidal Lie algebra, we have that:
                    $$[D, \g_{[2]}]_{\extendedtoroidal} = \rho(D) \cdot \g_{[2]}$$
                with $\rho: \divzero \to \der(\toroidal)$ as in corollary \ref{coro: a_fixed_yangian_div_zero_vector_field_action}. Because of this, all we need to do now is to prove that there does not exist $D \in \divzero$ so that:
                    $$\forall f \in A: D(f) = 0$$
                Suppose for the sake of deriving a contradiction that such an element $D \in \divzero$ does exist. This would imply that for all $f, g \in A$, one would have that:
                    $$(D, g \bar{d}f)_{\extendedtoroidal} = \gamma( g D(f) ) = 0$$
                meaning that such an element $D \in \divzero$ must be orthogonal to every element of $\z(\toroidal)$ (since $\z(\toroidal)$ is spanned by elements of the form $g \bar{d}f$). But this is clearly false, because of the construction of the bilinear form $(-, -)_{\extendedtoroidal}$ (cf. corollary \ref{coro: pairing_yangian_div_zero_vector_fields_and_cyclic_1_forms}), and so we have a contradiction. Therefore:
                    $$\z(\extendedtoroidal) \subset \bbC c_v \oplus \bbC c_t$$
                as well, and hence:
                    $$\z(\extendedtoroidal) = \bbC c_v \oplus \bbC c_t$$
                as claimed.
            \end{proof}
        \begin{remark}
            It is rather interesting that:
                $$\z(\extendedtoroidal) \cong \bbC c_v \oplus \bbC c_t$$
            as this is in good analogy with the affine Kac-Moody case, where the centre of $\hat{\g}$ is $1$-dimensional, namely spanned by $c_v$ (cf. example \ref{example: affine_lie_algebras_centres}).
        \end{remark}

    \subsection{The Witt algebra is spanned by \texorpdfstring{$\gamma$}{}-divergence-zero vector fields}
        Recall from lemma \ref{lemma: yangian_div_zero_vector_fields_basic_properties} that:
            $$\forall (r, s) \in \Z^2: D_{r, s} = s v^{-r + 1} t^{-s - 1} \del_v - r v^{-r} t^{-s} \del_t$$
            $$D_v = -v t^{-1} \del_v$$
            $$D_t = -\del_t$$
        and from lemma \ref{lemma: yangian_div_zero_vector_fields_basic_properties} that the commutation relations that these basis elements of $\divzero$ satisfy are:
            $$[D_v, D_t] = 0$$
            $$[D_v, D_{r, s}] = r D_{r, s + 1}$$
            $$[D_t, D_{r, s}] = D_{r, s + 1}$$
            $$[D_{a, b}, D_{r, s}] = (br - sa) D_{a + r, b + s + 1}$$
        (given for all $(r, s), (a, b) \in \Z^2$). With these information in mind, one sees that the following vector subspace of $\divzero$:
            $$\frakw := \bigoplus_{r \in \Z} \bbC D_{r, -1}$$
        is actually a Lie subalgebra, as the basis elements satisfy the following commutators:
            $$[D_{a, -1}, D_{r, -1}] = (a - r) D_{a + r, -1}$$
        given for all $a, r \in \Z$; note also that we have $D_v, D_t \not \in \frakw$ because:
            $$[D_v, D_{r, -1}] = r D_{r, 0} \not \in \frakw$$
            $$[D_t, D_{r, -1}] = D_{r, 0} \not \in \frakw$$    
        for all $r \in \Z$. Interestingly, these are precisely the commutation relations satisfied by the elements of the following basis of the Lie algebra $\der(\bbC[v^{\pm 1}])$:
            $$\left\{ d_r := -z^r \cdot z \frac{d}{dz} \right\}_{r \in \Z}$$
        and in light of this, we make the following observation:
        \begin{proposition}[A copy of the Witt algebra inside $\divzero$] \label{prop: a_copy_of_the_witt_algebra_inside_the_lie_algebra_of_yangian_div_zero_vector_fields}
            There is an isomorphism of Lie algebras:
                $$\der(\bbC[z^{\pm 1}]) \xrightarrow[]{\cong} \frakw$$
            given by:
                $$d_r \mapsto D_{r, -1}$$
            This identifies a copy of $\der(\bbC[z^{\pm 1}])$ inside $\divzero$ as a Lie subalgebra. 
        \end{proposition}

        As such, the titular statement of this subsection, namely that the Witt algebra is spanned by \texorpdfstring{$\gamma$}{}-divergence-zero vector fields, is actually true in a literal sense.

        \newpage

        \section{Examples of \texorpdfstring{$\gamma$}{}-extended toroidal Lie algebras}
    \subsection{Statement of the main theorem}
        For convenience, let us fix the following terminologies (which have already been eluded to in the statement of theorem \ref{theorem: yangian_extended_toroidal_lie_algebras_main_theorem}).
        \begin{definition}[$\gamma$-invariant toroidal $2$-cocycles] \label{def: yangian_toroidal_cocycles}
            Any $2$-cocyle $\sigma \in Z^2_{\Lie}(\divzero, \z(\toroidal))$ shall be referred to as a \textbf{toroidal $2$-cocycle}.
            
            Any toroidal $2$-cocycle $\sigma$ such that $\toroidal \rtimes^{\sigma} \divzero$ is a $\gamma$-extended toroidal Lie algebra shall be called a \textbf{$\gamma$-invariant toroidal $2$-cocycle}.
        \end{definition}
        
        In this section, we seek to produce examples of $\gamma$-extended toroidal Lie algebras beyond the semi-direct product $\toroidal \rtimes \divzero$ (which we have shown to be a $\gamma$-extended Lie algebra in lemma \ref{lemma: semi_direct_product_of_toroidal_lie_algebras_with_div_zero_vector_fields_are_yangian_extended_toroidal_lie_algebras}). We will do this through examination of some toroidal $2$-cocycles that have been known from \cite{billig_energy_momentum_tensor} (see also: \cite{billig_a_module_category_over_toroidal_EALAs}).

        Let us begin by introducing the two toroidal $2$-cocycles we will be working with. In \cite[p. 5, below Equation 1.3]{billig_energy_momentum_tensor}, it was noted that there are at least $2$-cocyles:
            $$\sigma_1, \sigma_2 \in Z^2_{\Lie}( \der(A), \z(\toroidal) )$$
        These are given in terms of the following basis of $\der(A)$:
            $$\{ v^{r_v} t^{r_t} \cdot v \del_v \}_{(r_v, r_t) \in \Z^2} \cup \{ v^{m_v} t^{m_t} \cdot t \del_t \}_{(m_v, m_t) \in \Z^2}$$
        by the following formulae:
            \begin{equation} \label{equation: billig_toroidal_cocycles}
                \begin{aligned}
                    & \sigma_1( v^{r_v} t^{r_t} \cdot x \del_x, v^{m_v} t^{m_t} \cdot y \del_y ) := r_y m_x \cdot v^{r_v} t^{r_t} \bar{d}( v^{m_v} t^{m_t} )
                    \\
                    & \sigma_2( v^{r_v} t^{r_t} \cdot x \del_x, v^{m_v} t^{m_t} \cdot y \del_y ) := r_x m_y \cdot v^{r_v} t^{r_t} \bar{d}( v^{m_v} t^{m_t} )
                \end{aligned}
            \end{equation}
        where $x, y \in \{v, t\}$ are symbolic placeholders.
        
        As a quick aside, let us note that the first cocycle $\sigma_1$ was known as far back as \cite{moody_rao_n_toroidal_vertex_representations}, but we are not aware of the history of $\sigma_2$ beyond its appearance in first \cite{billig_energy_momentum_tensor}.

        \begin{remark}
            Before we state the main theorem of this section, let us remind the reader of a few relevant notions from Lie algebra cohomology, and for more details, we refer the reader to appendix \ref{section: lie_algebra_cohomology_appendix}.
            
            If $\a$ is a Lie algebra and $M$ is an $\a$-module then into the vector space of Lie $2$-cocycles $Z^2_{\Lie}(\a, M)$ of $\a$ with values in $M$, there will be a linear map:
                $$d_1^M: \Hom_{\bbC}(\a, M) \to Z^2_{\Lie}(\a, M)$$
            called the \textbf{differential}, given by:
                $$d_1^M(\tau)(X, Y) = X \cdot \tau(Y) - Y \cdot \tau(X) - \tau([X, Y])$$
            for all linear maps $\tau \in \Hom_{\bbC}(\a, M)$ and all $X, Y \in \a$. The $2$-cocycles that are in the image of $d_1^M$ are called \textbf{$2$-coboundaries}. Also, one typically writes:
                $$B^2_{\Lie}(\a, M) := \im d_1^M$$
            as well as
                $$H^2_{\Lie}(\a, M) := Z^2_{\Lie}(\a, M)/B^2_{\Lie}(\a, M)$$
            This latter vector space is called the \textbf{$2^{nd}$ Lie algebra cohomology} of $\a$ with coefficients in $M$, and note that a $2$-cocycle is $2$-coboundary if and only if its image under the canonical projection $Z^2_{\Lie}(\a, M) \to H^2_{\Lie}(\a, M)$ is zero; in that case, the cocycle might also be be said to be \textbf{cohomologus} to $0$.
        \end{remark}

        Our main result concerning $\sigma_1$ and $\sigma_2$ is as below:
        \begin{theorem}[Non-trivial (counter-)examples of $\gamma$-extended toroidal Lie algebras] \label{theorem: billig_cocycle_main_theorem}
            Consider the $2$-cocycles:
                $$\sigma_1, \sigma_2 \in Z^2_{\Lie}(\divzero, \z(\toroidal))$$
            as in equation \eqref{equation: billig_toroidal_cocycles}.
            \begin{enumerate}
                \item $\sigma_1$ is $\gamma$-invariant in the sense of definition \ref{def: yangian_toroidal_cocycles}, and has non-zero image under the canonical projection $Z^2_{\Lie}(\divzero, \z(\toroidal)) \to H^2_{\Lie}(\divzero, \z(\toroidal))$, and hence $\toroidal \rtimes^{\sigma_1} \divzero$ is a $\gamma$-extended toroidal Lie algebra that is \textit{not} isomorphic to $\toroidal \rtimes \divzero$.
                \item On the other hand, $\sigma_2$ is \textit{not} $\gamma$-invariant, and has zero image under the canonical projection $Z^2_{\Lie}(\divzero, \z(\toroidal)) \to H^2_{\Lie}(\divzero, \z(\toroidal))$, and hence $\toroidal \rtimes^{\sigma_2} \divzero$ is not a $\gamma$-extended toroidal Lie algebra.

                Moreover, there is a one-parameter family $\{\tau_2^{\kappa}\}_{\kappa \in \bbC}$ of graded linear maps $\tau_2^{\kappa} \in C_1(\divzero, \z(\toroidal))$ of $\sigma_2$ under $d_1^{\z(\toroidal)}: C_1(\divzero, \z(\toroidal)) \to C_2(\divzero, \z(\toroidal))$\footnote{See remark \ref{remark: simplified_chevalley_eilenberg_complexes} and definition \ref{def: lie_algebra_cohomology} for notations. In particular, we have that $C_i(\a, M) := \Hom_{\bbC}\left(\bigwedge^i \a, M \right)$ for all Lie algebras $\a$ and all $\a$-modules $M$.}, whose elements are given by:
                    $$\tau_2^{\kappa}(D_{r, s}) = \left( \frac12 r^2 + r\kappa \right) K_{-r, -s - 2} + \delta_{(r, s), (0, -1)} \kappa c_t$$
                    $$\tau_2^{\kappa}(D_v) = 0$$
                    $$\tau_2^{\kappa}(D_t) = \kappa K_{0, -1}$$
            \end{enumerate}
        \end{theorem}
            \begin{proof}[Proof outline]
                Our proof of theorem \ref{theorem: billig_cocycle_main_theorem} will end up being rather long and computational, and therefore will be split up into multiple sections, culminating particularly in propositions \ref{prop: invariance_of_billig_toroidal_cocycles}, \ref{prop: sigma_1_is_not_coboundary}, and \ref{prop: cohomological_non_triviality_of_billig_toroidal_cocycles}. Below is an outline of our strategy.
                \begin{enumerate}
                    \item Firstly, we need to give descriptions of $\sigma_1, \sigma_2$ as elements of $Z^2_{\Lie}(\divzero, \z(\toroidal))$ instead of elements of $Z^2_{\Lie}(\der(A), \z(\toroidal))$, which is the same as computing the domain restrictions of these two cocycles from $\bigwedge^2 \der(A)$ down to the vector subspace $\bigwedge^2 \divzero$. This is done in lemma \ref{lemma: billig_toroidal_cocycles_on_yangian_div_zero_vector_fields}.
                    \item Using the $\gamma$-invariance criterion given in proposition \ref{proposition: twisted_semi_direct_products_are_yangian_extended_toroidal_lie_algebras}, we can then check that $\sigma_1$ is $\gamma$-invariant while $\sigma_2$ is not by directly performing computations.
                    \item Checking whether or not the images of $\sigma_1$ and of $\sigma_2$ might be zero under the canonical projection $Z^2_{\Lie}(\divzero, \z(\toroidal)) \to H^2_{\Lie}(\divzero, \z(\toroidal))$ (cf. definition \ref{def: lie_algebra_cohomology}) requires the following two steps, to be carried out in sequence. Let $i \in \{1, 2\}$.
                    \begin{enumerate}
                        \item Firstly, we shall argue that should $\sigma_i$ be $2$-coboundary, then an element:
                            $$\tau_i \in C_1(\divzero, \z(\toroidal))$$
                        (cf. remark \ref{remark: simplified_chevalley_eilenberg_complexes}) such that:
                            $$d_1^{\z(\toroidal)}(\tau_i) = \sigma_i$$
                        (cf. definition \ref{def: lie_algebra_cohomology}) will have to be graded\footnote{Recall that both $\divzero$ and $\z(\toroidal)$ are $\Z^2$-graded vector spaces (cf. example \ref{example: toroidal_lie_algebras_centres} and ccorollary \ref{coro: yangian_div_zero_vector_fields_are_graded}).} and be necessarily given by a certain formula, in which the only freedom lies in a parameter $\kappa \in \bbC$; incidentally, this is why we have written $\tau_2^{\kappa}$ for the pre-images along $d_1^{\z(\toroidal)}$ of $\sigma_2$ in theorem \ref{theorem: billig_cocycle_main_theorem}. This is the content of proposition \ref{prop: sigma_1_is_not_coboundary}. During the process, we will see that $\sigma_1$ is actually \textit{not} $2$-coboundary, and hence has non-zero image in $H^2_{\Lie}(\divzero, \z(\toroidal))$.
                        \item Secondly, to verify that $\sigma_2$ is $2$-coboundary, we will need to check that it satisfies:
                            $$\tau_2^{\kappa}([D, D']) = [ D, \tau_2^{\kappa}(D') ]_{\extendedtoroidal} - [ D', \tau_2^{\kappa}(D) ]_{\extendedtoroidal} - \sigma_2(D, D')$$
                        for all $D, D' \in \divzero$ (cf. example \ref{example: low_degree_lie_coboundaries_with_non_trivial_coefficients}). This will be done in proposition \ref{prop: cohomological_non_triviality_of_billig_toroidal_cocycles}, which concludes our proof.
                    \end{enumerate}
                \end{enumerate}
            \end{proof}

        Along the way (cf. remark \ref{remark: billig_toroidal_cocycles_on_the_witt_algebra}), we will also make note of a curious phenomenon, whereby after being restricted down to the homomorphic image $\frakw \subset \divzero$ of the Witt algebra $\der(\bbC[z^{\pm 1}])$ inside $\divzero$ (cf. proposition \ref{prop: a_copy_of_the_witt_algebra_inside_the_lie_algebra_of_yangian_div_zero_vector_fields}), both $\sigma_1$ and $\sigma_2$ will become equal to a $2$-cocycle of $\frakw$ with values in the trivial module $\bbC c_v$ that is \textit{not} cohomologous to $0$. The $2$-cocycles $\sigma_1$ and $\sigma_2$ therefore give rise to the unique non-trivial central extension of the Witt algebra (which happens also to be its UCE), i.e. the Virasoro algebra.
    
    \subsection{\texorpdfstring{$\gamma$}{}-invariance}
        Since lemma \ref{lemma: yangian_div_zero_vector_fields_basic_properties}, we have known how the elements of the basis:
            $$\{D_{r, s}\}_{(r, s) \in \Z^2} \cup \{D_v, D_t\}$$
        of $\divzero$ are given in terms of the partial derivatives $\del_v, \del_t$ (or more accurately, in terms of the aforementioned basis of $\der(A)$) by:
            $$D_{r, s} = -s v^{-r + 1} t^{-s - 1} \del_v + r v^{-r} t^{-s} \del_t = -s v^{-r} t^{-s - 1} \cdot v\del_v + r v^{-r} t^{-s - 1} \cdot t\del_t$$
            $$D_v = -v t^{-1} \del_v = -t^{-1} \cdot v\del_v$$
            $$D_t = -\del_t = -t^{-1} \cdot t\del_t$$
        (to rewrite the expressions slightly in terms of the currently employed basis for $\der(A)$). Knowing this allows us to compute the values of $\sigma_1$ and $\sigma_2$ on pairs of these basis elements.
        \begin{lemma}[Values of $\sigma_1$ and $\sigma_2$ on pairs of basis elements of $\divzero$] \label{lemma: billig_toroidal_cocycles_on_yangian_div_zero_vector_fields}
            In what follows, let $i \in \{1, 2\}$.
            \begin{enumerate}
                \item The values of the toroidal $2$-cocycles $\sigma_1, \sigma_2$ as in equation \eqref{equation: billig_toroidal_cocycles} on pairs of elements of the basis $\{D_{r, s}\}_{(r, s) \in \Z^2} \cup \{D_v, D_t\}$ of $\divzero$ (cf. lemma \ref{lemma: yangian_div_zero_vector_fields_basic_properties}) are:
                $$\sigma_i(D_{r, s}, D_{a, b}) = N_i(r, s, a, b) v^{-r} t^{-s - 1} \bar{d}( v^{-a} t^{-b - 1} )$$
                where:
                    \begin{equation} \label{equation: billig_cocycles_coefficient}
                        N_i(r, s, a, b) =
                        \begin{cases}
                            \text{$ra(2sb + s + b + 1) - ( (sa)^2 + s a^2 ) - ( (rb)^2 + r^2 b )$ if $i = 1$}
                            \\
                            \text{$ra$ if $i = 2$}
                        \end{cases}
                    \end{equation}
                and:
                    $$\sigma_i(D_{r, s}, D_v) = -\delta_{i, 1} r^2 v^{-r} t^{-s - 1} \bar{d}(t^{-1})$$
                    $$\sigma_i(D_{r, s}, D_t) = -\delta_{i, 1} rs v^{-r} t^{-s - 1} \bar{d}(t^{-1})$$
                    $$\sigma_i(D_v, D_t) = 0$$
                \item Furthermore, the restriction of both $\sigma_1$ and $\sigma_2$ to the Lie subalgebra $\frakw \cong \der(\bbC[z^{\pm 1}])$ (cf. proposition \ref{prop: a_copy_of_the_witt_algebra_inside_the_lie_algebra_of_yangian_div_zero_vector_fields}) of $\divzero$ coincides with the $2$-cocycle $\sigma_{\frakv} \in Z^2_{\Lie}(\frakw, \bbC c_v)$ given by:
                    $$\sigma_{\frakv}(D_{r, -1}, D_{a, -1}) := \delta_{r + a, 0} r^3 c_v$$
                for every $r, a \in \Z$.
            \end{enumerate}
        \end{lemma}
            \begin{proof}
                Even though the second statement will turn out to be more-or-less a corollary of the first identity in the first statement, let us nevertheless prove the two claims separately, for the sake of clarity.
                \begin{enumerate}
                    \item
                    \begin{enumerate}
                        \item Firstly, let us compute $\sigma_i(D_{r, s}, D_{a, b})$. For this, consider the following:
                            $$
                                \begin{aligned}
                                    & \sigma_i(D_{r, s}, D_{a, b})
                                    \\
                                    = & \sigma_i( -s v^{-r + 1} t^{-s - 1} \del_v + r v^{-r} t^{-s} \del_t, -b v^{-a + 1} t^{-b - 1} \del_v + a v^{-a} t^{-b} \del_t )
                                    \\
                                    = & \sigma_i( s v^{-r} t^{-s - 1} \cdot v\del_v - r v^{-r} t^{-s - 1} \cdot t \del_t, b v^{-a} t^{-b - 1} \cdot v\del_v - a v^{-a} t^{-b - 1} \cdot t \del_t )
                                    \\
                                    = &
                                        s \sigma_i( v^{-r} t^{-s - 1} \cdot v\del_v, b v^{-a} t^{-b - 1} \cdot v\del_v - a v^{-a} t^{-b - 1} \cdot t \del_t )
                                        \\
                                        & \qquad - r \sigma_i( v^{-r} t^{-s - 1} \cdot t \del_t, b v^{-a} t^{-b - 1} \cdot v\del_v - a v^{-a} t^{-b - 1} \cdot t \del_t )
                                    \\
                                    = &
                                        s b \cdot \sigma_i( v^{-r} t^{-s - 1} \cdot v\del_v, v^{-a} t^{-b - 1} \cdot v\del_v )
                                        \\
                                        & \qquad - s a \cdot \sigma_i( v^{-r} t^{-s - 1} \cdot v\del_v, v^{-a} t^{-b - 1} \cdot t \del_t )
                                        \\
                                        & \qquad - r b \cdot \sigma_i( v^{-r} t^{-s - 1} \cdot t \del_t, v^{-a} t^{-b - 1} \cdot v\del_v )
                                        \\
                                        & \qquad + r a \cdot \sigma_i( v^{-r} t^{-s - 1} \cdot t \del_t, v^{-a} t^{-b - 1} \cdot t \del_t )
                                    \\
                                    = & N_i(r, s, a, b) v^{-r} t^{-s - 1} \bar{d}( v^{-a} t^{-b - 1} )
                                \end{aligned}
                            $$
                        where:
                            $$
                                \begin{aligned}
                                    & N_i(r, s, a, b)
                                    \\
                                    = & 
                                    sbra
                                    - sa \left( \delta_{i, 1} a(s + 1) + \delta_{i, 2} (b + 1) r \right) 
                                    - rb \left( \delta_{i, 1} (b + 1) r + \delta_{i, 2} a (s + 1) \right)
                                    + r a (s + 1) (b + 1)
                                    \\
                                    = & 
                                    \begin{cases}
                                        \text{$
                                            sbra
                                            - s a^2 (s + 1) 
                                            - r^2 b (b + 1)
                                            + r a (s + 1) (b + 1)
                                        $if $i = 1$}
                                        \\
                                        \text{$
                                            sbra
                                            - sa (b + 1) r
                                            - rb a (s + 1)
                                            + r a (s + 1) (b + 1)
                                        $ if $i = 2$}
                                    \end{cases}
                                    \\
                                    = & 
                                    \begin{cases}
                                        \text{$
                                            sbra
                                            - ( (sa)^2 + s a^2 ) 
                                            - ( (rb)^2 + r^2 b ) 
                                            + rasb + rsa + rab + ra
                                        $ if $i = 1$}
                                        \\
                                        \text{$
                                            sbra
                                            - (sabr + sar)
                                            - (rbas + rba)
                                            + rasb + rsa + rab + ra
                                        $ if $i = 2$}
                                    \end{cases}
                                    \\
                                    = & 
                                    \begin{cases}
                                        \text{$2 rsab - ( (sa)^2 + s a^2 ) - ( (rb)^2 + r^2 b ) + rsa + rab + ra$ if $i = 1$}
                                        \\
                                        \text{$ra$ if $i = 2$}
                                    \end{cases}
                                    \\
                                    = &
                                    \begin{cases}
                                        \text{$ra(2sb + s + b + 1) - ( (sa)^2 + s a^2 ) - ( (rb)^2 + r^2 b )$ if $i = 1$}
                                        \\
                                        \text{$ra$ if $i = 2$}
                                    \end{cases}
                                \end{aligned}
                            $$
                        \item Secondly, we shall be computing $\sigma_i(D_v, D_{r, s})$. Consider the following:
                            $$
                                \begin{aligned}
                                    & \sigma_i(D_{r, s}, D_v)
                                    \\
                                    = & \sigma_i( -s v^{-r + 1} t^{-s - 1} \del_v + r v^{-r} t^{-s} \del_t, -v t^{-1} \del_v )
                                    \\
                                    = & \sigma_i( s v^{-r} t^{-s - 1} \cdot v \del_v - r v^{-r} t^{-s - 1} t \del_t, t^{-1} \cdot v \del_v )
                                    \\
                                    = & s \sigma_i( v^{-r} t^{-s - 1} \cdot v \del_v, t^{-1} \cdot v \del_v ) - r\sigma_i( v^{-r} t^{-s - 1} \cdot t \del_t, t^{-1} \cdot v \del_v )
                                    \\
                                    = & \left( s \left( \delta_{i, 1} (-r) \cdot 0 + \delta_{i, 2} (-r) \cdot 0 \right) - r\left( \delta_{i, 1} (-r) \cdot (-1) + \delta_{i, 2} (-s - 1) \cdot 0 \right) \right) v^{-r} t^{-s - 1} \bar{d}(t^{-1}) 
                                    \\
                                    = & -\delta_{i, 1} r^2 v^{-r} t^{-s - 1} \bar{d}(t^{-1})
                                \end{aligned}
                            $$
                        \item Using similar methods as in the previous case, we shall get that:
                            $$\sigma_i(D_{r, s}, D_t) = -\delta_{i, 1} rs v^{-r} t^{-s - 1} \bar{d}(t^{-1})$$
                        \item Finally, we have that:
                            $$\sigma_i(D_v, D_t) = \sigma_i(-v t^{-1} \del_v, -\del_t) = \sigma_i(t^{-1} \cdot v \del_v, t^{-1} t \del_t) = 0$$
                    \end{enumerate}
                    \item Using the fact that any element:
                        $$v^n t^q \bar{d}(v^m t^p) \in \z(\toroidal)$$
                    can be written in terms of the basis elements of $\z(\toroidal)$ in the following manner:
                        $$v^n t^q \bar{d}(v^m t^p) = \delta_{(m, p) + (n, q), (0, 0)} ( n c_v + q c_t ) + (np - mq) K_{m + n, p + q}$$
                    we can rewrite:
                        $$
                            \begin{aligned}
                                & \sigma_i(D_{r, s}, D_{a, b})
                                \\
                                = & N_i(r, s, a, b) \left( ( r(b + 1) - a(s + 1) )K_{-r - a, -s - b - 2} - \delta_{ (-a - r, -b - s - 2), (0, 0) } (r c_v + (s + 1) c_t) \right)
                            \end{aligned}
                        $$
                    Setting $s = b = -1$ in the equation above then yields:
                        $$\sigma_i(D_{r, -1}, D_{a, -1}) = -N_i(r, -1, a, -1) \delta_{r + a, 0} r c_v$$
                    where now, we have that:
                        $$N_i(r, -1, a, -1) = ra$$
                    for both $i = 1$ and $i = 2$, and hence:
                        $$\sigma_i(D_{r, -1}, D_{a, -1}) = \delta_{r + a, 0} r^3 c_v$$
                    as claimed.
                \end{enumerate}
            \end{proof}
        Aside from being a prerequisite for verifying if $\sigma_1$ and/or $\sigma_2$ might be $\gamma$-invariant, lemma \ref{lemma: billig_toroidal_cocycles_on_yangian_div_zero_vector_fields} has the following trivial but important consequence:
        \begin{corollary}[$\sigma_1$ and $\sigma_2$ are $\Z^2$-graded] \label{coro: billig_toroidal_cocycles_are_graded}
            When regarded as linear maps from $\bigwedge^2 \divzero$ to $\z(\toroidal)$, both with their $\Z^2$-gradings (cf. example \ref{example: toroidal_lie_algebras_centres} and corollary \ref{coro: yangian_div_zero_vector_fields_are_graded}), both $\sigma_1$ and $\sigma_2$ as in equation \eqref{equation: billig_toroidal_cocycles} are graded. 
        \end{corollary}
        \begin{remark}
            Despite the way that lemma \ref{lemma: billig_toroidal_cocycles_on_yangian_div_zero_vector_fields} has been stated, which was for the sake of conciseness, the form of the non-trivial values of $\sigma_1$ and $\sigma_2$ on pairs of basis elements of $\divzero$ that we will actually be using in our proofs are as follows:
                $$
                    \begin{aligned}
                        & \sigma_i(D_{r, s}, D_{a, b})
                        \\
                        = & N_i(r, s, a, b) \left( ( r(b + 1) - a(s + 1) )K_{-r - a, -s - b - 2} - \delta_{ (-a - r, -b - s - 2), (0, 0) } (r c_v + (s + 1) c_t) \right)
                    \end{aligned}
                $$
                $$\sigma_i(D_v, D_{r, s}) = \delta_{i, 1} r^3 K_{-r, -s - 2}$$
                $$\sigma_i(D_t, D_{r, s}) = \delta_{i, 1} r^2 s K_{-r, -s - 2}$$
            To prove that these identities hold true, we only need to recall from example \ref{example: toroidal_lie_algebras_centres} that any element:
                $$v^n t^q \bar{d}(v^m t^p) \in \z(\toroidal)$$
            can be written in terms of the basis elements of $\z(\toroidal)$ in the following manner:
                $$v^n t^q \bar{d}(v^m t^p) = \delta_{(m, p) + (n, q), (0, 0)} ( n c_v + q c_t ) + (np - mq) K_{m + n, p + q}$$
        \end{remark}
        \begin{remark}[An appearance of the Virasoro algebra] \label{remark: billig_toroidal_cocycles_on_the_witt_algebra}
            A rather well-known fact (cf. e.g. \cite[Proposition 1.3]{kac_raina_rozhkovskaya_bombay_lectures_on_highest_weight_modules_of_infinite_dimensional_lie_algebras}) is that:
                $$H^2_{\Lie}(\der(\bbC[z^{\pm 1}], \bbC)) \cong \bbC$$
            and hence the Witt algebra admits a non-trivial UCE by a $1$-dimensional centre (cf. theorem \ref{theorem: H^2_of_lie_algebras_and_abelian_extensions}), typically called the \textbf{Virasoro algebra}\footnote{... and hence the notation $\sigma_{\frakv}$.}. This UCE corresponds precisely to the $2$-cocycle $\sigma_{\frakv} \in Z^2_{\Lie}(\frakw, c_v)$ as in lemma \ref{lemma: billig_toroidal_cocycles_on_yangian_div_zero_vector_fields}. 
        \end{remark}

        One can now use the criterion given in proposition \ref{proposition: twisted_semi_direct_products_are_yangian_extended_toroidal_lie_algebras} (see also theorem \ref{theorem: yangian_extended_toroidal_lie_algebras_main_theorem}) to verify whether or not the cocycles $\sigma_1, \sigma_2$ are $\gamma$-invariant in the sense of definition \ref{def: yangian_toroidal_cocycles}.
        \begin{proposition}[$\gamma$-invariance of Billig's toroidal $2$-cocycles] \label{prop: invariance_of_billig_toroidal_cocycles}
            Of the two toroidal $2$-cocycles:
                $$\sigma_1, \sigma_2 \in Z^2_{\Lie}(\divzero, \z(\toroidal))$$
            as in equation \eqref{equation: billig_toroidal_cocycles}, $\sigma_1$ is $\gamma$-invariant in the sense of definition \ref{def: yangian_toroidal_cocycles}, while $\sigma_2$ fails to be so.
        \end{proposition}
            \begin{proof}
                \begin{enumerate}
                    \item Firstly, using the fact that:
                        $$
                            \begin{aligned}
                                & \sigma_i(D_{r, s}, D_{a, b})
                                \\
                                = & N_i(r, s, a, b) \left( ( r(b + 1) - a(s + 1) )K_{-r - a, -s - b - 2} - \delta_{ (-a - r, -b - s - 2), (0, 0) } (r c_v + (s + 1) c_t) \right)
                            \end{aligned}
                        $$
                    we shall get that:
                        $$
                            \left( \sigma_i(D_{r, s}, D_{a, b}), D \right)_{\extendedtoroidal} =
                            \begin{cases}
                                \text{$N_i(r, s, a, b) ( r(b + 1) - a(s + 1) ) \delta_{(-r - a, -s - b - 2), (\alpha, \beta)}$ if $D = D_{\alpha, \beta}$}
                                \\
                                \text{$-N_i(r, s, a, b) \delta_{(r, s), -(a, b)} r$ if $D = D_v$}
                                \\
                                \text{$-N_i(r, s, a, b) \delta_{(r, s), -(a, b)} (s + 1)$ if $D = D_t$}
                            \end{cases}
                        $$
                    At the same time, using the fact that:
                        $$\sigma_i(D_{a, b}, D_v) = -\delta_{i, 1} a^3 K_{-a, -b - 2}$$
                        $$\sigma_i(D_{a, b}, D_t) = - \delta_{i, 1} a^2b K_{-a, -b - 2}$$
                    we have that:
                        $$
                            \begin{aligned}
                                \left( D_{r, s}, \sigma_i(D_{a, b}, D) \right)_{\extendedtoroidal} =
                                \begin{cases}
                                    \text{$N_i(r, s, \alpha, \beta) ( r(\beta + 1) - \alpha(s + 1) ) \delta_{(r, s), (-a - \alpha, -b - \beta - 2)}$ if $D = D_{\alpha, \beta}$}
                                    \\
                                    \text{$-\delta_{i, 1} a^3 \delta_{(r, s), (-a, -b - 2)}$ if $D = D_v$}
                                    \\
                                    \text{$-\delta_{i, 1} a^2 b \delta_{(r, s), (-a, -b - 2)}$ if $D = D_t$}
                                \end{cases}
                            \end{aligned}
                        $$
                    We can thus conclude immediately that $\sigma_2$ is \textit{not} invariant, as:
                        $$\left( \sigma_2(D_{r, s}, D_{a, b}), D \right)_{\extendedtoroidal} \not = \left( D_{r, s}, \sigma_2(D_{a, b}, D) \right)_{\extendedtoroidal}$$
                    when $D \in \{D_v, D_t\}$. As such, let us focus on $\sigma_1$ from now on, for which we now have:
                        $$\left( \sigma_1(D_{r, s}, D_{a, b}), D \right)_{\extendedtoroidal} \not = \left( D_{r, s}, \sigma_1(D_{a, b}, D) \right)_{\extendedtoroidal}$$
                    for all $D \in \divzero$.
                    \item Secondly, using the fact that:
                        $$\sigma_1(D_{r, s}, D_v) = r^3 K_{-r, -s - 2}$$
                    we shall get that:
                        $$
                            \left( \sigma_1(D_{r, s}, D_v), D \right)_{\extendedtoroidal} =
                            \begin{cases}
                                \text{$r^3 \delta_{(-r, -s - 2), (\alpha, \beta)}$ if $D = D_{\alpha, \beta}$}
                                \\
                                \text{$0$ if $D = D_v$}
                                \\
                                \text{$0$ if $D = D_t$}
                            \end{cases}
                        $$
                    At the same time, knowing that:
                        $$\sigma_1(D_v, D_t) = 0$$
                    we see that:
                        $$
                            \begin{aligned}
                                \left( D_{r, s}, \sigma_1(D_v, D) \right)_{\extendedtoroidal} =
                                \begin{cases}
                                    \text{$-\alpha^3 \delta_{(r, s), (-\alpha, -\beta - 2)}$ if $D = D_{\alpha, \beta}$}
                                    \\
                                    \text{$0$ if $D = D_v$}
                                    \\
                                    \text{$0$ if $D = D_t$}
                                \end{cases}
                            \end{aligned}
                        $$
                    We thus have:
                        $$\left( \sigma_1(D_{r, s}, D_v), D \right)_{\extendedtoroidal} = \left( D_{r, s}, \sigma_1(D_v, D) \right)_{\extendedtoroidal}$$
                    for all $D \in \divzero$.
                    \item Next, by using the fact that:
                        $$\sigma_1(D_{r, s}, D_t) = r^2 s K_{-r, -s - 2}$$
                    we shall get that:
                        $$
                            \left( \sigma_1(D_{r, s}, D_t), D \right)_{\extendedtoroidal} =
                            \begin{cases}
                                \text{$r^2 s \delta_{(-r, -s - 2), (\alpha, \beta)}$ if $D = D_{\alpha, \beta}$}
                                \\
                                \text{$0$ if $D = D_v$}
                                \\
                                \text{$0$ if $D = D_t$}
                            \end{cases}
                        $$
                    At the same time, we have that:
                        $$
                            \begin{aligned}
                                \left( D_{r, s}, \sigma_1(D_t, D) \right)_{\extendedtoroidal} =
                                \begin{cases}
                                    \text{$-\alpha^2 \beta \delta_{(r, s), (-\alpha, -\beta - 2)}$ if $D = D_{\alpha, \beta}$}
                                    \\
                                    \text{$0$ if $D = D_v$}
                                    \\
                                    \text{$0$ if $D = D_t$}
                                \end{cases}
                            \end{aligned}
                        $$
                    By combining these two observations, one is able to conclude furthermore that:
                        $$\left( \sigma_1(D_{r, s}, D_t), D \right)_{\extendedtoroidal} = \left( D_{r, s}, \sigma_1(D_t, D) \right)_{\extendedtoroidal}$$
                    for all $D \in \divzero$.
                    \item Lastly, since:
                        $$\sigma_1(D_v, D_t) = 0$$
                    we automatically have that:
                        $$( \sigma_1(D_v, D_t), D )_{\extendedtoroidal} = ( D_v, \sigma_1(D_t, D) )_{\extendedtoroidal}$$
                    for all $D \in \divzero$.
                \end{enumerate}
                We have therefore shown that $\sigma_1$ is $\gamma$-invariant in the sense of definition \ref{def: yangian_toroidal_cocycles}. 
            \end{proof}

    \subsection{Cohomological non-triviality}
        Whether or not the cocycles:
            $$\sigma_1, \sigma_2 \in Z^2_{\Lie}(\divzero, \z(\toroidal))$$
        might be cohomologous to $0$ (cf. definition \ref{def: lie_algebra_cohomology}) - and hence whether or not they might give rise to extensions that are isomorphic to the semi-direct product $\toroidal \rtimes \divzero$ - is a much subtler issue. Remark \ref{remark: billig_toroidal_cocycles_on_the_witt_algebra} in fact does not imply anything about whether or not the images in $H^2_{\Lie}(\divzero, \z(\toroidal))$ of $\sigma_1$ and of $\sigma_2$ might vanish. The subtlety here is that because $\z(\toroidal)$ is non-trivial as a module over $\divzero$ (and likewise, over the Lie subalgebra $\frakw \subset \divzero$), unlike $\bbC c_v$, one would have to actually check whether or not the restricted toroidal $2$-cocycle:
            $$\sigma_i|_{\bigwedge^2 \frakw}: \bigwedge^2 \divzero \to \z(\toroidal)$$
        is $2$-coboundary.

        Now, although in order to check if $\sigma_i$ is $2$-coboundary, it suffices to prove the existence of an element:
            $$\tau_i \in C_1(\divzero, \z(\toroidal))$$
        (i.e. a linear map; cf. remark \ref{remark: simplified_chevalley_eilenberg_complexes}) such that:
            $$\tau_i([D, D']) = [D, \tau_i(D')]_{\extendedtoroidal} - [D', \tau_i(D)]_{\extendedtoroidal} - \sigma_i(D, D')$$
        we shall prove a stronger statement, namely the existence of $\tau_i \in C_1(\divzero, \z(\toroidal))$ that is \textit{graded} with respect to the $\Z^2$-gradings on $\divzero$ and $\z(\toroidal)$. We will do this by firstly assuming that such a graded $\tau_i$ exists, deriving an explicit closed-form formula for it, and then verifying whether or not $d_1^{\z(\toroidal)}(\tau_i) = \sigma_i$. Our final result, proposition \ref{prop: cohomological_non_triviality_of_billig_toroidal_cocycles} will as such be proven firstly through the use of proposition \ref{prop: sigma_1_is_not_coboundary} below.
        \begin{remark}[Some recollection]
            The proofs of propositions \ref{prop: sigma_1_is_not_coboundary} and \ref{prop: cohomological_non_triviality_of_billig_toroidal_cocycles} down below are somewhat computation-heavy, so for the sake of clarity and convenience, let us recall for the proof the following facts that have been known since earlier in this chapter.
            \begin{itemize}
                \item Firstly, recall from example \ref{example: toroidal_lie_algebras_centres} and corollary \ref{coro: yangian_div_zero_vector_fields_are_graded} that:
                    $$\deg D_{-a, -b - 1} = \deg K_{a, b} = (a, b)$$
                    $$\deg D_v = \deg D_t = (0, -1)$$
                    $$\deg c_v = \deg c_t = (0, 0)$$
                \item Secondly, let us recall from lemma \ref{lemma: yangian_div_zero_vector_fields_basic_properties} that:
                    $$[D_v, D_{r, s}] = r D_{r, s + 1}$$
                    $$[D_t, D_{r, s}] = s D_{r, s + 1}$$
                    $$[D_{a, b}, D_{r, s}] = (br - sa) D_{a + r, b + s + 1}$$
                \item Thirdly, from lemma \ref{lemma: explicit_commutators_between_central_basis_elements_and_derivations}, we know that:
                    $$[D, K_{\alpha, \beta}]_{\extendedtoroidal} =
                        \begin{cases}
                            \text{$((\beta - 1)a - b\alpha) K_{\alpha - a, \beta - b - 1} + \delta_{(a, b + 1), (\alpha, \beta)} \left( a c_v + b c_t \right)$ if $D = D_{a, b}$}
                            \\
                            \text{$\alpha K_{\alpha, \beta - 1}$ if $D = D_v$}
                            \\
                            \text{$\beta K_{\alpha, \beta - 1}$ if $D = D_t$}
                        \end{cases}
                    $$
                    $$[D, c_v]_{\extendedtoroidal} = [D, c_t]_{\extendedtoroidal} = 0$$
                for all $D \in \divzero$.
                \item Finally, we will need to know how $\sigma_1$ and $\sigma_2$ are given explicitly on the basis elements of $\divzero$. For this, we refer the reader to lemma \ref{lemma: billig_toroidal_cocycles_on_yangian_div_zero_vector_fields} at the top of this section.
            \end{itemize}
        \end{remark}
        \begin{proposition} \label{prop: sigma_1_is_not_coboundary}
            The toroidal $2$-cocycle $\sigma_1 \in Z^2_{\Lie}(\divzero, \z(\toroidal) )$ as in equation \eqref{equation: billig_toroidal_cocycles} is \textit{not} $2$-coboundary, i.e. its image under the canonical projection $Z^2_{\Lie}(\divzero, \z(\toroidal) ) \to H^2_{\Lie}(\divzero, \z(\toroidal) )$ is non-zero.
        \end{proposition}
            \begin{proof}
                Let us suppose that $\sigma_i$ (for either $i = 1$ or $i = 2$, or perhaps even neither) is $2$-coboundary, say:
                    $$d_1^{\z(\toroidal)}(\tau_i) = \sigma_i$$
                for some $\tau_i \in C_1(\divzero, \z(\toroidal))$. Recall firstly, from example \ref{example: toroidal_lie_algebras_centres} and corollary \ref{coro: yangian_div_zero_vector_fields_are_graded}, respectively, that there are $\Z^2$-gradings on $\z(\toroidal)$ and $\divzero$. Secondly, recall from corollary \ref{coro: billig_toroidal_cocycles_are_graded} that $\sigma_i$ is graded with respect to these $\Z^2$-gradings. Thirdly, recall from lemma \ref{lemma: explicit_commutators_between_central_basis_elements_and_derivations} (cf. corollary \ref{coro: a_fixed_yangian_div_zero_vector_field_action}) that the $\divzero$ acts homogeneously on $\z(\toroidal)$. We can therefore apply lemma \ref{lemma: graded_2_coboundaries}, which tells us that we can assume without any loss of generality that $\tau_i$ is a graded linear map (with respect to the $\Z^2$-gradings on $\divzero$ and on $\z(\toroidal)$).
            
                Assume thus that $\tau_i$ is graded. Firstly, this assumption  implies that there are coefficients:
                    $$\lambda_{r, s}, \mu, \kappa \in \bbC$$
                such that:
                    $$\tau_i(D_{r, s}) = \lambda_{r, s} K_{-r, -s - 1} + \delta_{(r, s), (0, -1)} ( \mu c_v + \kappa c_t )$$
                whenever $(r, s) \not = (0, 0)$, where the second equality holds because $K_{0, 0} = 0$ (cf. example \ref{example: toroidal_lie_algebras_centres}). Likewise, there shall be coefficients:
                    $$\nu_v, \nu_t \in \bbC$$
                such that:
                    $$\tau_i(D_v) = \nu_v K_{0, -1}$$
                    $$\tau_i(D_t) = \nu_t K_{0, -1}$$

                We attempt first of all to compute $\tau_i(D_{r, s})$. For this, we refer the reader to lemma \ref{lemma: billig_toroidal_cocycles_on_yangian_div_zero_vector_fields} for the following fact:
                    $$\sigma_i(D_v, D_{r, s}) = -\delta_{i, 1} r^3 K_{-r, -s - 2}$$
                This gives the following:
                    \begin{equation} \label{equation: tau_i_on_commutator_of_D_v_and_D_rs}
                        \begin{aligned}
                            & r \tau_i(D_{r, s + 1})
                            \\
                            = & \tau_i([D_v, D_{r, s}])
                            \\
                            = & [D_v, \tau_i(D_{r, s})]_{\extendedtoroidal} - [D_{r, s}, \tau_i(D_v)]_{\extendedtoroidal} - \sigma_i(D_v, D_{r, s})
                            \\
                            = & [D_v, \lambda_{r, s} K_{-r, -s - 1}]_{\extendedtoroidal} - [D_{r, s}, \nu_v K_{0, -1}]_{\extendedtoroidal} + \delta_{i, 1} r^3 K_{-r, -s - 2}
                            \\
                            = & r \lambda_{r, s} K_{-r, -s - 2} + \nu_v \left( -2r K_{-r, -s - 2} + \delta_{(r, s + 1), (0, -1)} \left( r c_v + s c_t \right) \right) + \delta_{i, 1} r^3 K_{-r, -s - 2}
                            \\
                            = & \left( r \lambda_{r, s} - 2r \nu_v + \delta_{i, 1} r^3 \right) K_{-r, -s - 2} + \nu_v \delta_{(r, s + 2), (0, 0)} \left( r c_v + s c_t \right)
                        \end{aligned}
                    \end{equation}
                From this, we infer that when $r \not = 0$, we have that:
                    $$\tau_i(D_{r, s + 1}) = \left( \lambda_{r, s} - 2\nu_v + \delta_{i, 1} r^2 \right) K_{-r, -s - 2}$$
                but at the same time, we have the following per our initial assumption that $\tau_i$ is graded:
                    $$\tau_i(D_{r, s + 1}) = \lambda_{r, s + 1} K_{-r, -s - 2} + \delta_{(r, s + 1), (0, -1)} ( \alpha_{r, s + 1} c_v + \beta_{r, s + 1} c_t )$$
                and so we have:
                    $$\lambda_{r, s + 1} = \lambda_{r, s} - 2\nu_v + \delta_{i, 1} r^2$$
                whenever $r \not = 0$ (in which case $\delta_{(r, s + 1), (0, -1)} = 0$). This recursive formula is equivalent to the following, valid for all $(r, s) \in \Z^2 \setminus ( \{0\} \x \Z )$:
                    \begin{equation} \label{equation: lambda_rs_coefficients_recursion}
                        \lambda_{r, s} = \lambda_{r, 0} + (-2\nu_v + \delta_{i, 1} r^2) s
                    \end{equation}
                From this, one infers that in order to determined $\lambda_{r, s}$ when $r \not = 0$, it suffices to only determine $\lambda_{r, 0}$.
                
                Next, consider the following:
                    $$
                        \begin{aligned}
                            & (br - sa) \tau_i(D_{a + r, b + s + 1})
                            \\
                            = & \tau_i( [D_{a, b}, D_{r, s}] )
                            \\
                            = & [D_{a, b}, \tau_i(D_{r, s})]_{\extendedtoroidal} - [D_{r, s}, \tau_i(D_{a, b})]_{\extendedtoroidal} - \sigma_i(D_{a, b}, D_{r, s})
                            \\
                            = & \lambda_{r, s} [D_{a, b}, K_{-r, -s - 1}]_{\extendedtoroidal} - \lambda_{a, b} [D_{r, s}, K_{-a, -b - 1}]_{\extendedtoroidal} + \sigma_i(D_{r, s}, D_{a, b})
                        \end{aligned}
                    $$
                Using lemma \ref{lemma: explicit_commutators_between_central_basis_elements_and_derivations}, we get that:
                    $$[D_{a, b}, K_{-r, -s - 1}]_{\extendedtoroidal} = -\left( a(s + 2) - br \right) K_{-a - r, -b - s - 2} + \delta_{(a + r, b + s + 2), (0, 0)} \left( a c_v + b c_t \right)$$
                    $$[D_{r, s}, K_{-a, -b - 1}]_{\extendedtoroidal} = -\left( (b + 2) r - as \right) K_{-a - r, -b - s - 2} + \delta_{(a + r, b + s + 2), (0, 0)} \left( r c_v + s c_t \right)$$
                and from lemma \ref{lemma: billig_toroidal_cocycles_on_yangian_div_zero_vector_fields}, we know that:
                    $$\sigma_i(D_{r, s}, D_{a, b}) = N_i(r, s, a, b) \left( ( r(b + 1) - a(s + 1) )K_{-r - a, -s - b - 2} - \delta_{ (a + r, b + s + 2), (0, 0) } (r c_v + (s + 1) c_t) \right)$$
                where $N_i(r, s, a, b)$ is as in \textit{loc. cit.} Simultaneously, these fact imply that:
                    \begin{equation} \label{equation: coboundary_equation_D_rs_D_ab}
                        \begin{aligned}
                            & (br - sa) \tau_i(D_{a + r, b + s + 1})
                            \\
                            = & (br - sa) \left( \lambda_{a + r, b + s + 1} K_{-a - r, -b - s - 2} + \delta_{(a + r, b + s + 1), (0, -1)}( \mu c_v + \kappa c_t ) \right)
                            \\
                            = &
                                \left( -\lambda_{r, s} \left( a(s + 2) - br \right) + \lambda_{a, b} \left( (b + 2) r - as \right) + N_i(r, s, a, b)\left( r(b + 1) - a(s + 1) \right) \right) K_{-a - r, -b - s - 2}
                                \\
                                & \qquad + \delta_{(a + r, b + s + 2), (0, 0)} \left( \lambda_{r, s} a - \lambda_{a, b} r - N_i(r, s, a, b) r \right) c_v
                                \\
                                & \qquad + \delta_{(a + r, b + s + 2), (0, 0)} \left( \lambda_{r, s} b - \lambda_{a, b} s - N_i(r, s, a, b) (s + 1) \right) c_t
                            \\
                            = &
                                \left( -\lambda_{r, s} \left( a(s + 2) - br \right) + \lambda_{a, b} \left( (b + 2) r - as \right) + N_i(r, s, a, b)\left( r(b + 1) - a(s + 1) \right) \right) K_{-a - r, -b - s - 2}
                                \\
                                & \qquad - \delta_{(a + r, b + s + 2), (0, 0)} r \left( \lambda_{r, s} + \lambda_{-r, -s - 2} + N_i(r, s, -r, -s - 2) \right) c_v
                                \\
                                & \qquad - \delta_{(a + r, b + s + 2), (0, 0)} \left( \lambda_{r, s} (s + 2) + \lambda_{-r, -s - 2} s + N_i(r, s, -r, -s - 2) (s + 1) \right) c_t
                        \end{aligned}
                    \end{equation}
                from which it can be inferred - by comparing the coefficients of $K_{-a - r, -b - s - 2}$ - that:
                    \begin{equation} \label{equation: lambda_rs_coefficients}
                        \begin{aligned}
                            & (br - sa) \lambda_{a + r, b + s + 1}
                            \\
                            = & -\lambda_{r, s} \left( a(s + 2) - br \right) + \lambda_{a, b} \left( (b + 2) r - as \right) + N_i(r, s, a, b)\left( r(b + 1) - a(s + 1) \right)
                        \end{aligned}
                    \end{equation}
                As mentioned above (cf. equation \eqref{equation: lambda_rs_coefficients_recursion}), we would like to determine $\lambda_{r, 0}$ when $r \not = 0$. To this end, let us evaluate equation \eqref{equation: lambda_rs_coefficients} at:
                    $$b = 0, s = -1$$
                Doing so yields:
                    $$
                        \begin{aligned}
                            & a \lambda_{a + r, 0}
                            \\
                            = & -\lambda_{r, -1} a + \lambda_{a, 0} \left( 2r + a \right) + N_i(r, -1, a, 0) r
                            \\
                            = & -(\lambda_{r, 0} - \delta_{i, 1} r^2) a + \lambda_{a, 0} \left( 2r + a \right) + N_i(r, -1, a, 0) r
                        \end{aligned}
                    $$
                When $r + a = 0$, this will turn into:
                    $$0 = -r \lambda_{0, 0} = (\lambda_{r, 0} - \delta_{i, 1} r^2) r + \lambda_{-r, 0} r + N_i(r, -1, -r, 0) r$$
                and since the current assumption is that $r \not = 0$, the above is equivalent to:
                    $$
                        \begin{aligned}
                            & 0
                            \\
                            = & (\lambda_{r, 0} - \delta_{i, 1} r^2) + \lambda_{-r, 0} + N_i(r, -1, -r, 0)
                            \\
                            = & (\lambda_{r, 0} + \lambda_{-r, 0}) - \delta_{i, 1} r^2 + N_i(r, -1, -r, 0)
                        \end{aligned}
                    $$
                and hence:
                    $$
                        \begin{aligned}
                            & \lambda_{r, 0} + \lambda_{-r, 0}
                            \\
                            = & \delta_{i, 1} r^2 - N_i(r, -1, -r, 0)
                            \\
                            = & \delta_{i, 1} r^2 + \delta_{i, 2} r^2
                        \end{aligned}
                    $$
                (to see why the second equality holds, see equation \eqref{equation: billig_cocycles_coefficient}, from which one sees that $N_1(r, -1, -r, 0) = 0$ and $N_2(r, -1, -r, 0) = -r^2$). Using equation \eqref{equation: lambda_rs_coefficients_recursion}, we then get that:
                    $$
                        \begin{aligned}
                            & \lambda_{r, -1} + \lambda_{-r, -1}
                            \\
                            = & ( \lambda_{r, 0} + \lambda_{-r, 0} ) - 2(-2\nu_v + \delta_{i, 1} r^2)
                            \\
                            = & ( \delta_{i, 1} r^2 + \delta_{i, 2} r^2 ) - 2\delta_{i, 1} r^2 + 4\nu_v
                            \\
                            = & \delta_{i, 1} r^2 - \delta_{i, 2} r^2 + 4\nu_v
                            \\
                            = & (-1)^{\delta_{i, 1}} r^2 + 4\nu_v
                        \end{aligned}
                    $$
                Next, let us compute $\lambda_{r, -1} - \lambda_{-r, -1}$ (and during the process, we will also be able to compute the coefficients $\mu, \kappa$), so that we can establish a linear system in the variables $\lambda_{\pm r, -1}$. To this end, let us evaluate equation \eqref{equation: coboundary_equation_D_rs_D_ab} when $r + a = 0$ and $b = s = -1$. Doing so yields:
                    $$
                        \begin{aligned}
                            & 2r \tau_i(D_{0, -1})
                            \\
                            = & r\left( \lambda_{r, -1} + \lambda_{-r, -1} + N_i(r, -1, -r, -1) \right) c_v + \left( \lambda_{r, -1} - \lambda_{-r, -1} \right) c_t
                            \\
                            = & r\left( \lambda_{r, -1} + \lambda_{-r, -1} - r^2 \right) c_v + \left( \lambda_{r, -1} - \lambda_{-r, -1} \right) c_t
                            \\
                            = & r\left( (-1)^{\delta_{i, 1}} r^2 + 4\nu_v - r^2 \right) c_v + \left( \lambda_{r, -1} - \lambda_{-r, -1} \right) c_t
                            \\
                            = & r\left( ((-1)^{\delta_{i, 1}} - 1) r^2 + 4\nu_v \right) c_v + \left( \lambda_{r, -1} - \lambda_{-r, -1} \right) c_t
                        \end{aligned}
                    $$
                (and note that from equation \eqref{equation: billig_cocycles_coefficient}, it can be inferred that $N_1(r, -1, -r, -1) = N_2(r, -1, -r, -1) = -r^2$, which gives the second equality). At the same time, we have per the graded-ness assumption on $\tau_i$ that:
                    $$\tau_i(D_{0, -1}) = \mu c_v + \kappa c_t$$
                Neither $\mu, \kappa \in \bbC$ depends on $r$, so we must have the following, by simplying comparing the coefficients of $c_v$ and of $c_t$:
                    $$\lambda_{r, -1} + \lambda_{-r, -1} - r^2 = (-1)^{\delta_{i, 2}} r^2 + 4\nu_v = 2\mu$$
                    $$\lambda_{r, -1} - \lambda_{-r, -1} = 2r\kappa$$
                Observe now that:
                    $$(-1)^{\delta_{i, 1}} - 1 = -2\delta_{i, 1}$$
                meaning that only when $i \not = 1$ (so $i = 2$ for us) is the expression $((-1)^{\delta_{i, 1}} - 1) r^2$ \textit{independent} of $r$, which is what we would like to have, since $\mu$ is not dependent on $r$. As such, we can immediately rule out the case $i = 1$. This proves our claim that $\sigma_1 \in Z^2_{\Lie}(\divzero, \z(\toroidal) )$ is \textit{not} $2$-coboundary.
            \end{proof}
        \begin{corollary}[A non-trivial $\gamma$-extended toroidal Lie algebra]
            The twisted semi-direct product $\toroidal \rtimes^{\sigma_1} \divzero$ is an example of a $\gamma$-extended toroidal Lie algebra that is \textit{not} isomorphic to the semi-direct product $\toroidal \rtimes \divzero$.
        \end{corollary}
            \begin{proof}
                Combine proposition \ref{prop: sigma_1_is_not_coboundary} with the fact that $\sigma_1$ is $\gamma$-invariant, shown in proposition \ref{prop: invariance_of_billig_toroidal_cocycles}.
            \end{proof}

        The following result is our final conclusion regarding whether or not the toroidal $2$-cocycles $\sigma_1$ and $\sigma_2$ are cohomologous to $0$.
        \begin{proposition} \label{prop: cohomological_non_triviality_of_billig_toroidal_cocycles}
            The toroidal $2$-cocycle $\sigma_2 \in Z^2_{\Lie}(\divzero, \z(\toroidal) )$ as in equation \eqref{equation: billig_toroidal_cocycles} is $2$-coboundary. Furthermore, there exists a one-parameter family $\{\tau_2^{\kappa}\}_{\kappa \in \bbC}$ of graded linear maps $\tau_2^{\kappa}: \divzero \to \z(\toroidal)$ such that $d_1^{\z(\toroidal)}(\tau_2^{\kappa}) = \sigma_2$. Each $\tau_2^{\kappa}$ is given by:
                $$\tau_2^{\kappa}(D_{r, s}) = \left( \frac12 r^2 + r\kappa \right) K_{-r, -s - 2} + \delta_{(r, s), (0, -1)} \kappa c_t$$
                $$\tau_2^{\kappa}(D_v) = 0$$
                $$\tau_2^{\kappa}(D_t) = \kappa K_{0, -1}$$
        \end{proposition}
            \begin{proof}
                We will prove this proposition by firstly pre-supposing that $\sigma_2$ is $2$-coboundary and then identifying explicit formulae for $1$-cochains $\tau_2^{\kappa} \in C_1(\divzero, \z(\toroidal))$ such that $d_1^{\z(\toroidal)}(\tau_2^{\kappa}) = \sigma_2$; along the way, we will also see that such linear maps $\tau_2^{\kappa}$ are indeed parametrised by $\kappa \in \bbC$, hence the notation. Afterwards, we will check that such $1$-cochains indeed give rise to a Lie $2$-coboundary that is equal to $\sigma_2$ by verifying that equation \eqref{equation: coboundary_verification_tau_2} holds.
                
                To this end, we begin by letting:
                    $$\tau_2 \in C_1(\divzero, \z(\toroidal))$$
                be graded, as in the proof of proposition \ref{prop: sigma_1_is_not_coboundary} (and we refer the reader there for an explanation of why $\tau_2$ can be assumed to be graded; see also lemma \ref{lemma: graded_2_coboundaries}), and by letting:
                    $$\lambda_{r, s}, \mu, \kappa, \nu_v, \nu_t \in \bbC$$
                also be as in the same proof. Recall also from \textit{loc. cit.} that we have arrived at the equation:
                    $$\lambda_{r, -1} + \lambda_{-r, -1} - r^2 = (-1)^{\delta_{i, 2}} r^2 + 4\nu_v = 2\mu$$
                which gives:
                    $$\mu = 2\nu_v$$
                when $i = 2$, which is what we are currerntly working with.
                
                We thus have the following linear system:
                    $$
                        \begin{cases}
                            \lambda_{r, -1} + \lambda_{-r, -1} = r^2 + 2\mu = r^2 + 2 \nu_v
                            \\
                            \lambda_{r, -1} - \lambda_{-r, -1} = 2r\kappa
                        \end{cases}
                    $$
                which can be solved to yield:
                    $$\lambda_{r, -1} = \frac12 r^2 + r\kappa + \mu = \frac12 r^2 + r\kappa + 2\nu_v$$
                Plugging this into equation \eqref{equation: lambda_rs_coefficients_recursion} then yields the following whenever $r \not = 0$:
                    $$
                        \begin{aligned}
                            & \lambda_{r, s}
                            \\
                            = & \lambda_{r, 0} - 2\nu_v s
                            \\
                            = & \lambda_{r, -1} - 2 \nu_v (s + 1)
                            \\
                            = & \left( \frac12 r^2 + r\kappa + 2\nu_v \right) - 2 \nu_v (s + 1)
                            \\
                            = & \frac12 r^2 + r\kappa - 2s \nu_v
                        \end{aligned} 
                    $$
                with no constrains on $\kappa \in \bbC$. 

                Next, let us verify whether there might be any constraint on either $\nu_v$ or $\nu_t$. Observe next that when $r = 0$, we have that:
                    $$0 = r \tau_2(D_{r, s + 1}) = \nu_v \delta_{s, -2} sc_t$$
                (cf. equation \eqref{equation: tau_i_on_commutator_of_D_v_and_D_rs}) and hence it is necessary that:
                    $$\nu_v = 0$$
                To determine $\nu_t$, recall from lemma \ref{lemma: billig_toroidal_cocycles_on_yangian_div_zero_vector_fields} that:
                    $$\sigma_2(D_t, D_{r, s}) = 0$$
                and hence:
                    $$
                        \begin{aligned}
                            & s \tau_2(D_{r, s + 1})
                            \\
                            = & \tau_2([D_t, D_{r, s}])
                            \\
                            = & [D_t, \tau_2(D_{r, s})]_{\extendedtoroidal} - [D_{r, s}, \tau_2(D_t)]_{\extendedtoroidal} 
                            \\
                            = & [D_t, \lambda_{r, s} K_{-r, -s - 1}]_{\extendedtoroidal} - [D_{r, s}, \nu_t K_{0, -1}]_{\extendedtoroidal}
                            \\
                            = & (s + 1) \lambda_{r, s} K_{-r, -s - 2} + \nu_t \left( -2r K_{-r, -s - 2} + \delta_{(r, s + 1), (0, -1)} \left( r c_v + s c_t \right) \right)
                            \\
                            = & \left( (s + 1) \lambda_{r, s} - 2r \nu_t \right) K_{-r, -s - 2} + \nu_t \delta_{(r, s + 2), (0, 0)} \left( r c_v + s c_t \right)
                        \end{aligned}
                    $$
                If $(r, s) = (0, -2)$ then the above will imply that:
                    $$-2\tau_2(D_{0, -1}) = -2\nu_t c_t$$
                But at the same time, we have that:
                    $$\tau_2(D_{0, -1}) = \mu c_v + \kappa c_t = -2\nu_v c_v + \kappa c_t = \kappa c_t$$
                and hence:
                    $$\nu_t = \kappa$$
                    
                Therefore, we can conclude that:
                    \begin{equation} \label{equation: lambda_rs_formula}
                        \lambda_{r, s} = \frac12 r^2 + r\kappa
                    \end{equation}
                    $$\nu_v = 0$$
                    $$\nu_t = \kappa$$
                and hence we have found a one-parameter family $\{\tau_2^{\kappa}\}_{\kappa \in \bbC}$ of graded linear maps $\tau_2^{\kappa}: \divzero \to \z(\toroidal)$ such that $d_1^{\z(\toroidal)}(\tau_2^{\kappa}) = \sigma_2$.
            
                Now, it remains to check whether or not the $2$-boundary equation:
                    \begin{equation} \label{equation: coboundary_verification_tau_2}
                        \tau_2^{\kappa}([D, D']) = [D, \tau_2^{\kappa}(D')]_{\extendedtoroidal} - [D', \tau_2^{\kappa}(D)]_{\extendedtoroidal} - \sigma_2(D, D')
                    \end{equation}
                is satisfied for all $D, D' \in \divzero$. Without loss of generality, we can assume again that $D, D' \in \divzero$ are basis elements.
                \begin{enumerate}
                    \item Firstly, let us verify if it is true that:
                        \begin{equation} \label{equation: coboundary_equation_D_rs_D_ab_verification}
                            \tau_2^{\kappa}([D_{a, b}, D_{r, s}]) = [D_{a, b}, \tau_2^{\kappa}(D_{r, s})]_{\extendedtoroidal} - [D_{r, s}, \tau_2^{\kappa}(D_{a, b})]_{\extendedtoroidal} - \sigma_2(D_{a, b}, D_{r, s})
                        \end{equation}
                    We will perform a case-by-case verification, as in each of these cases, certain terms in equation \eqref{equation: coboundary_equation_D_rs_D_ab} (and hence in equation \eqref{equation: coboundary_equation_D_rs_D_ab_verification}) would vanish.
                    \begin{enumerate}
                        \item To begin, not that when $a = r = 0$, the LHS and the RHS of equation \eqref{equation: coboundary_equation_D_rs_D_ab_verification} will both be $0$ (cf. equation \eqref{equation: coboundary_equation_D_rs_D_ab}), and hence equal to one another.
                        \item When $a + r \not = 0$ and $a, r \not = 0$, the LHS of equation \eqref{equation: coboundary_equation_D_rs_D_ab_verification} will become:
                            $$
                                \begin{aligned}
                                    & \tau_2^{\kappa}([D_{a, b}, D_{r, s}])
                                    \\
                                    = & (br - sa) \tau_2^{\kappa}(D_{a + r, b + s + 1})
                                    \\
                                    = & (br - sa) \left( \frac12 (a + r)^2 + (a + r)\kappa \right) K_{-a - r, -b - s - 1}
                                \end{aligned}
                            $$
                        while the RHS will become:
                            $$
                                \begin{aligned}
                                    & [D_{a, b}, \tau_2^{\kappa}(D_{r, s})]_{\extendedtoroidal} - [D_{r, s}, \tau_2^{\kappa}(D_{a, b})]_{\extendedtoroidal} - \sigma_2(D_{a, b}, D_{r, s})
                                    \\
                                    = & \left( -\lambda_{r, s} \left( a(s + 2) - br \right) + \lambda_{a, b} \left( (b + 2) r - as \right) + N_2(r, s, a, b)\left( r(b + 1) - a(s + 1) \right) \right) K_{-a - r, -b - s - 2}
                                    \\
                                    = &
                                        -\left( \frac12 r^2 + r\kappa \right) \left( a(s + 2) - br \right) K_{-a - r, -b - s - 2}
                                        \\
                                        & \qquad + \left( \frac12 a^2 + a\kappa \right) \left( (b + 2) r - as \right) K_{-a - r, -b - s - 2} 
                                        \\
                                        & \qquad + ra\left( r(b + 1) - a(s + 1) \right) K_{-a - r, -b - s - 2}
                                \end{aligned}
                            $$
                        which can now be easily checked to be equal to the LHS.
                        \item When $r + a = 0$ but $b + s + 2 \not = 0$, the LHS of equation \eqref{equation: coboundary_equation_D_rs_D_ab_verification} will be equal to:
                            $$
                                \begin{aligned}
                                    & \tau_2^{\kappa}([D_{-r, b}, D_{r, s}])
                                    \\
                                    = & \left( \lambda_{r, s} \left( r(s + 2) + br \right) + \lambda_{-r, b} \left( (b + 2) r + rs \right) + N_2(r, s, -r, b)\left( r(b + 1) + r(s + 1) \right) \right) K_{0, -b - s - 2}
                                    \\
                                    = & r (b + s + 2) \left( (\lambda_{r, s} + \lambda_{r, b}) - r^2 \right) K_{0, -b - s - 2}
                                \end{aligned}
                            $$
                        while the RHS will become:
                            $$
                                \begin{aligned}
                                    & [D_{-r, b}, \tau_2^{\kappa}(D_{r, s})]_{\extendedtoroidal} - [D_{r, s}, \tau_2^{\kappa}(D_{-r, b})]_{\extendedtoroidal} - \sigma_2(D_{-r, b}, D_{r, s})
                                    \\
                                    = & r (b + s + 2) \left( r^2 - \kappa + N_2(r, s, -r, b) \right) K_{0, -b - s - 2}
                                    \\
                                    = & r (b + s + 2) \left( r^2 - \kappa - r^2 \right) K_{0, -b - s - 2}
                                    \\
                                    = & -r (b + s + 2) \kappa K_{0, -b - s - 2}
                                \end{aligned}
                            $$
                        For equation \eqref{equation: coboundary_equation_D_rs_D_ab_verification} to be true, we must then have that:
                            $$\lambda_{r, s} + \lambda_{r, b} - r^2 = -\kappa$$
                        Because we have covered the case $a = r = 0$ (which does give $a + r = 0$), let us assume now that $r \not = 0$. In that case, the equation above will become:
                            $$r^2 - \kappa - r^2 = -\kappa$$
                        which is clearly true.
                        \item Lastly, consider the case where $a + r = 0$ and $b + s + 2 = 0$. In this case, the LHS of equation \eqref{equation: coboundary_equation_D_rs_D_ab_verification} will become:
                            $$-2r \tau_2^{\kappa}(D_{0, -1}) = -2r \kappa c_t$$
                        while the RHS will become:
                            $$
                                \begin{aligned}
                                    & [D_{-r, -s - 2}, \tau_2^{\kappa}(D_{r, s})]_{\extendedtoroidal} - [D_{r, s}, \tau_2^{\kappa}(D_{-r, -s - 2})]_{\extendedtoroidal} - \sigma_2(D_{-r, -s - 2}, D_{r, s})
                                    \\
                                    = &
                                        -r\left( \lambda_{r, s} + \lambda_{-r, -s - 2} + N_2(r, s, -r, -s - 2) \right) c_v
                                        \\
                                        & \qquad - \left( \lambda_{r, s} (s + 2) + \lambda_{-r, -s - 2} s + N_2(r, s, -r, -s - 2) (s + 1) \right) c_t
                                    \\
                                    = &
                                        -r \left( \left( \frac12 r^2 + r \kappa \right) + \left( \frac12 r^2 - r \kappa \right) - r^2\right) c_v
                                        \\
                                        & \qquad - \left( \left( \frac12 r^2 + r \kappa \right) (s + 2) + \left( \frac12 r^2 - r \kappa \right) s - r^2 (s + 1) \right) c_t
                                    \\
                                    = & -2r \kappa c_t
                                \end{aligned}
                            $$
                        so clearly the LHS and RHS are equal to one another.
                    \end{enumerate}
                    \item Next, recall from lemma \ref{lemma: billig_toroidal_cocycles_on_yangian_div_zero_vector_fields} that:
                        $$\sigma_2(D_v, D_{r, s}) = 0$$
                    and then consider the following for $r \not = 0$ (the case $r = 0$ is trivial):
                        $$
                            \begin{aligned}
                                & \tau_2^{\kappa}([D_v, D_{r, s}])
                                \\
                                = & r\tau_2^{\kappa}(D_{r, s + 1})
                                \\
                                = & r \left(\frac12 r^2 + r \kappa \right) K_{-r, -s - 2}
                            \end{aligned}
                        $$
                    along with the following (also for $r \not = 0$):
                        $$
                            \begin{aligned}
                                & [D_v, \tau_2^{\kappa}(D_{r, s})]_{\extendedtoroidal} - [D_{r, s}, \tau_2^{\kappa}(D_v)]_{\extendedtoroidal}
                                \\
                                = & \left(\frac12 r^2 + r \kappa \right)[D_v, K_{-r, -s - 1}]_{\extendedtoroidal}
                                \\
                                = & r \left(\frac12 r^2 + r \kappa \right) K_{-r, -s - 2}
                            \end{aligned}
                        $$
                    and hence the LHS and RHS of equation \eqref{equation: coboundary_verification_tau_2} are indeed equal when $D = D_v$ and $D' = D_{r, s}$.
                    \item The same method as above can be used to show that the LHS and RHS of equation \eqref{equation: coboundary_verification_tau_2} are indeed equal when $D = D_t$ and $D' = D_{r, s}$.
                    \item Finally, let us verify that the LHS and RHS of equation \eqref{equation: coboundary_verification_tau_2} are indeed equal when $D = D_v$ and $D' = D_t$. Since:
                        $$[D_v, D_t] = -D_{0, 1}$$
                    the LHS of said equation will vanish:
                        $$\tau_2^{\kappa}([D_v, D_t]) = -\tau_2^{\kappa}(D_{0, 1}) = 0$$
                    Recall also from lemma \ref{lemma: billig_toroidal_cocycles_on_yangian_div_zero_vector_fields} that:
                        $$\sigma_2(D_v, D_t) = 0$$
                    and therefore it suffices to check if it is true that:
                        $$[D_v, \tau_2^{\kappa}(D_t)]_{\extendedtoroidal} - [D_t, \tau_2^{\kappa}(D_v)]_{\extendedtoroidal} = 0$$
                    From proposition \ref{prop: sigma_1_is_not_coboundary}, we know that:
                        $$\tau_2^{\kappa}(D_v) = 0, \tau_2^{\kappa}(D_t) = \kappa K_{0, -1}$$
                    so the LHS of the equation above reduces down to:
                        $$\kappa [D_v, K_{0, -1}]_{\extendedtoroidal}$$
                    Since $[D_v, K_{0, -1}]_{\extendedtoroidal} = 0 \cdot K_{0, -2} = 0$, the above is certainly equal to $0$, as needed.
                \end{enumerate}
                In conclusion, we have shown that $\sigma_2$ is indeed $2$-coboundary and hence cohomologous to $0$.
            \end{proof}
        \begin{remark}
            Through proposition \ref{prop: invariance_of_billig_toroidal_cocycles}, we have seen that the toroidal $2$-cocycle:
                $$\sigma_2$$
            is \textit{not} $\gamma$-invariant in the sense of definition \ref{def: yangian_toroidal_cocycles}. However, in light of proposition \ref{prop: cohomological_non_triviality_of_billig_toroidal_cocycles}, we now see that this non-$\gamma$-invariance of $\sigma_2$ holds \textit{despite} the fact that it is cohomologous to $0$. Therefore, one can \textit{not} conclude from observing that a toroidal $2$-cocycle $\sigma$ is \textit{cohomologous} to $0$ that $\sigma$ is $\gamma$-invariant, but when the \textit{equality} $\sigma = 0$ holds in $Z^2_{\Lie}(\divzero, \z(\toroidal))$, then it will indeed be true that $\sigma$ is $\gamma$-invariant, as we know that the semi-direct product $\toroidal \rtimes \divzero$ is an instance of a $\gamma$-extended toroidal Lie algebra.
        \end{remark}

    \newpage

    \begin{appendices}
        \chapter{Technical appendices}
            \section{Lie algebra cohomology} \label{section: lie_algebra_cohomology_appendix}
    \subsection{A brief account of the generalities}
        Let us work over a fixed field $k$.
    
        Even though it is true that every twisted semi-direct product:
            $$0 \to \t \to \t \rtimes^{\sigma} \d \to \d \to 0$$
        gives rise to a $2$-cocycle $\sigma: \bigwedge^2 \d \to \t$, the converse statement is slightly more subtle. Namely, it is possible that two unequal $2$-cocycles:
            $$\sigma \not = \sigma'$$
        would give rise to two twisted semi-direct products that are isomorphic as extensions of $\d$ (see e.g. proposition \ref{prop: cohomological_non_triviality_of_billig_toroidal_cocycles}), and in order to capture this fact, we will need to introduce the language of Lie algebra cohomology, which also necessitates a brief discussion of homological algebra in general. We refer the reader to \cite{hilton_stammbach_homological_algebra} as a general reference on homological algebra, and particularly, Chapter VII therein as a reference on the generalities of Lie algebra cohomology. For brevity, our account of Lie algebra cohomology will have to be a bit \textit{ad hoc}. Also, in discussing homological algebra, we will have to make use of some categorical language, but we will keep this to a minimum, only mentioning what is absolutely needed. We refer the reader to \cite{maclane}, particularly Chapter VIII therein, for more details.
        
        We take for granted the fact that for any associative ring $R$, left-$R$-modules form an abelian category (cf. \cite[Section VIII.3, p. 198]{maclane}) - denoted by ${}^lR\mod$ - with all limits and colimits. We will only be working with left-modules, but everything works identically for right-modules.

        \begin{definition}[Complexes and cohomology] \label{def: complexes_and_cohomology}
            (Cf. \cite[Section IV.1]{hilton_stammbach_homological_algebra}) Let $R$ be an associative ring. A \textbf{complex} or \textbf{chain complex} of left-$R$-modules is a diagram of left-$R$-modules:
                $$\{P_i, \del_i\}_{i \in \Z} := \{ ... \xrightarrow[]{\del_{-3}} P_{-2} \xrightarrow[]{\del_{-2}} P_{-1} \xrightarrow[]{\del_{-1}} P_0 \xrightarrow[]{\del_0} P_1 \xrightarrow[]{\del_1} P_2 \xrightarrow[]{\del_2} ... \}$$
            (sometimes also abbreviated by $(P_{\bullet}, \del_{\bullet})$) such that for each $i \in \Z$, one has that:
                $$\del_i \circ \del_{i - 1} = 0$$
            The connecting maps $\del_i$ are typically called \textbf{differentials} or \textbf{coboundary maps}.
            
            For each $i \in \Z$, the \textbf{$i^{th}$ cohomology} of such a complex is given by:
                $$H^i(P_{\bullet}, \del_{\bullet}) := \frac{ \ker \del_i }{ \im \del_{i - 1} }$$
            (typically this is written as just $H^i(P_{\bullet})$, with the differentials being implicitly understood). Elements of $\ker \del_i$ are usually called \textbf{$i$-cocycles} and those of $\im \del_{i - 1}$ are usually called \textbf{$i$-coboundaries}; elements of $H^i(P_{\bullet}, \del_{\bullet})$ are usually called \textbf{$i^{th}$ cohomology classes}, and cocycles which are representatives of the same cohomology class are said to be \textbf{cohomologous}.
            
            If:
                $$H^i(P_{\bullet}, \del_{\bullet}) \cong 0$$
            for all $i \in \Z$ then we shall say that the complex $(P_{\bullet}, \del_{\bullet})$ is \textbf{exact}.
        \end{definition}
        \begin{example}
            A \textbf{short exact sequence} (cf. e.g. definition \ref{def: lie_algebra_extensions}) is nothing but an exact complex with only 3 terms. 
        \end{example}
        
        If $R$ is any associative ring and $M$ is any left-$R$-module, then the functor:
            $$\Hom_R(-, M): {}^lR\mod^{\op} \to \Z\mod$$
        is generally only left-exact, which for us shall mean that it maps kernels and finite direct sums in ${}^lR\mod^{\op}$ (i.e. cokernels and finite direct sums in ${}^lR\mod$) to kernels and finite direct sums in $\Z\mod$. Consequently, any projective resolution\footnote{It is a fact from homological algebra that any choice of projective resolution would return the same cohomologies, so the non-trivial task is to choose a resolution that would make computations convenient or in fact, even feasible in the first place. In modern terminologies, all the projective resolutions of the same module are \say{quasi-isomorphic}.} of a given left-$R$-module $P$:
            $$(P_{\bullet}, \del_{\bullet}) := \{ ... \xrightarrow[]{\del_{-3}} P_{-2} \xrightarrow[]{\del_{-2}} P_{-1} \xrightarrow[]{\del_{-1}} P_0 \xrightarrow[]{\del_0} P \to 0 \}$$
        (i.e. a chain complex of left-$R$-modules wherein each term is projective\footnote{A left-$R$-module $P$ is projective if and only if the functor $\Hom_R(P, -): {}^lR\mod \to \Z\mod$ is exact, or equivalently (because ${}^lR\mod$ is an abelian category), if and only if it preserves monomorphisms.} as a left-$R$-module) is mapped by $\Hom_R(-, M)$ to a diagram of $\Z$-modules as follows:
            $$\Hom_R(P_{\bullet}, M) := \{ 0 \to \Hom_R(P, M) \Hom_R(P^0, M) \to \Hom_R(P^{-1}, M) \to \Hom_R(P^{-2}, M) \to ... \}$$
        \begin{definition}[$\Ext$-groups] \label{def: Ext_groups}
            With notations as above, we can define:
                $$\Ext^i_R(P, M) := H^i(\Hom_R(P_{\bullet}, M))$$
            for each $i \in \Z_{\geq 0}$. By construction, these are $\Z$-modules. 
        \end{definition}
        \begin{remark}
            If $R$ is an associative algebra defined over a field $k$, then the $\Z$-modules $\Ext^i_R(P, M)$ will actually carry $k$-vector space structures as well.
        \end{remark}
        \begin{definition}[Lie algebra cohomology and the Chevalley-Eilenberg resolution] \label{def: lie_algebra_cohomology}
            If $\a$ is a Lie algebra over $k$, then its \textbf{(abelian) Lie algebra cohomology} with \textbf{coefficients} in some left-$\rmU(\a)$-module $M$ (which is typically referred to simply as an $\a$-module) shall be given by:
                $$H^i_{\Lie}(\a, M) := \Ext^i_{\rmU(\a)}(k, M) := H^i( \Hom_{\rmU(\a)}(k_{\bullet}, M) )$$
            where $k$ is regarded as a trivial left-$\rmU(\a)$-module, equipped with some projective resolution of left-$\rmU(\a)$-modules $k_{\bullet}$.

            The standard projective resolution for the left-$\rmU(\a)$-module is the \textbf{Chevalley-Eilenberg resolution} (sometimes referred to as the \textbf{Chevalley-Eilenberg complex}):
                $$\left\{ k_{-i}(\a) := \rmU(\a) \tensor_k \bigwedge^i \a \right\}_{i \in \Z_{\geq 0}}$$
            and for each $i \in \Z_{\geq 0}$, the corresponding differential/coboundary map:
                $$\del_{-(i + 1)}: k_{-(i + 1)}(\a) \to k_{-i}(\a)$$
            is given on pure tensors by:
                $$
                    \begin{aligned}
                        & \del_{-(i + 1)}\left( a \tensor (x_1 \wedge ... \wedge x_{i + 1}) \right)
                        \\
                        := &
                            \sum_{1 \leq p \leq i + 1} (-1)^p a x_p \tensor (x_1 \wedge ... \wedge \not{x_p} \wedge ... \wedge x_{i + 1})
                            \\
                            & \qquad + \sum_{1 \leq p < q \leq i + 1} (-1)^{p + q} a \tensor ( [x_p, x_q]_{\a} \wedge x_1 \wedge ... \wedge \not{x_p} \wedge ... \wedge \not{x_q} \wedge ... \wedge x_{i + 1})
                    \end{aligned}
                $$
            for all $a \in \rmU(\a)$ and all $x_1, ..., x_{i + 1} \in \a$ (cf. \cite[Section VII.4]{hilton_stammbach_homological_algebra}).
        \end{definition}
        \begin{remark}[A simplified expression of Chevalley-Eilenberg complexes] \label{remark: simplified_chevalley_eilenberg_complexes}
            Let $\a$ be a Lie algebra over $k$ and let $M$ be an arbitrary $\a$-module; let us also regard $M$ as an abelian Lie algebra. Next, consider the following complex of $k$-vector spaces\footnote{Which is sometimes also called the Chevalley-Eilenberg complex of $\a$ (with coefficients in $M$).}:
                $$\{ C_i(\a, M) := \Hom_{\rmU(\a)}(k_{-i}(\a), M) \}_{i \in \Z_{\geq 0}}$$
            and observe that its terms $C_{-i}(\a, M)$ can be described as follows:
                $$
                    \begin{aligned}
                        & C_i(\a, M)
                        \\
                        := & \Hom_{\rmU(\a)}(k_{-i}(\a), M)
                        \\
                        \cong & \Hom_{\rmU(\a)}( \rmU(\a) \tensor_k \bigwedge^i \a, M ) \ni \left( a \tensor (x_1 \wedge ... \wedge x_i) \mapsto \varphi( a \tensor (x_1 \wedge ... \wedge x_i) ) \right)
                        \\
                        \cong & \Hom_{\rmU(\a)}( \rmU(\a), \Hom_k( \bigwedge^i \a, M ) ) \ni \left( a \mapsto \varphi(a \tensor -) \right)
                        \\
                        \cong & \Hom_k( \bigwedge^i \a, M ) \ni \varphi(1 \tensor -)
                    \end{aligned}
                $$
            for all $a \in \rmU(\a)$, all $x_1, ..., x_i \in \a$, and all $\varphi \in \Hom_{\rmU(\a)}( \rmU(\a) \tensor_k \bigwedge^i \a, M )$. This permits us to think of Lie cocycles and coboundaries as certain alternating multi-linear maps (see definition \ref{def: lie_cocycles_and_coboundaries} below).
        \end{remark}
        \begin{definition}[Lie cocycles and coboundaries] \label{def: lie_cocycles_and_coboundaries}
            Let us denote by:
                $$d_i^M: C_i(\a, M) \to C_{i + 1}(\a, M)$$
            the differentials of the complex of $k$-vector spaces $\{C_i(\a, M)\}_{i \in \Z_{\geq 0}}$. These are given by:
                $$d_i^M := \Hom_{\rmU(\a)}(\del_{-(i + 1)}, M)$$
            When $M$ is understood from the context, the superscripts may be omitted. For each $i \in \Z_{\geq 0}$, we define \textbf{Lie $i$-cocycles} and \textbf{Lie $i$-coboundaries} of $\a$ with values in $M$, respectively, to be elements of:
                $$Z^i_{\Lie}(\a, M) := \ker d_i^M$$
                $$B^i_{\Lie}(\a, M) := \im d_{i - 1}^M$$
        \end{definition}
        \begin{remark}
            From remark \ref{remark: simplified_chevalley_eilenberg_complexes}, one sees that - for any $i \in \Z_{\geq 0}$, a Lie $i$-cocycle of $\a$ with values in some $\a$-module $M$ is nothing but an alternating $k$-linear map:
                $$\sigma: \bigwedge^i \a \to M$$
            such that:
                $$d_i^M(\sigma) = 0$$
            In particular, using the fact that:
                $$d_i^M := \Hom_{\rmU(\a)}(\del_{-(i + 1)}, M)$$
            by construction, together with the construction of the maps $\del_{-(i + 1)}$ as in definition \ref{def: lie_algebra_cohomology}, one then sees that indeed, when $i = 2$, one recovers the notion of Lie $2$-cocycles of $\a$ with values in $M$ as in definition \ref{def: twisted_semi_direct_products} as linear maps $\sigma: \bigwedge^2 \a \to M$ satisfying the Jacobi identity in the sense stated in \textit{loc. cit.}
        \end{remark}

        Let us now see how Lie cocycles and Lie coboundaries are given explicitly, particularly in low degrees. 

        We begin with the case of cocycles and coboundaries with values in trivial modules. 
        \begin{example}[Lie cocycles and coboundaries with trivial coefficients] \label{example: lie_cocycles_and_coboundaries_with_trivial_coefficients}
            Let $\a$ be a Lie algebra over $k$ and regard $k$ itself as a trivial $\a$-module. For each $i \in \Z_{\geq 0}$, we then have - per the discussions in remark \ref{remark: simplified_chevalley_eilenberg_complexes} that:
                $$C_i(\a, k) \cong \Hom_k(\bigwedge^i \a, k) =: (\bigwedge^i \a)^*$$
            and the differentials:
                $$d_i^k: C_i(\a, k) \to C_{i + 1}(\a, k)$$
            are then nothing but:
                $$\del_{-(i + 1)}^* := \Hom_k(\del_{-(i + 1)}, k)$$
            \begin{itemize}
                \item A Lie $i$-cocycle of $\a$ with coefficients in $k$ is then - by definition - an element of $\ker \del_{-(i + 1)}^*$, i.e. a linear map:
                    $$\sigma: \bigwedge^i \a \to k$$
                such that:
                    $$
                        \begin{aligned}
                            & 0
                            \\
                            = & \del_{-(i + 1)}^*(\sigma)( (x_1 \wedge ... \wedge x_{i + 1}) )
                            \\
                            = &
                                \sum_{1 \leq p \leq i + 1} (-1)^p \sigma(x_1 \wedge ... \wedge \not{x_p} \wedge ... \wedge x_{i + 1})
                                \\
                                & \qquad + \sum_{1 \leq p < q \leq i + 1} (-1)^{p + q} \sigma( [x_p, x_q]_{\a} \wedge x_1 \wedge ... \wedge \not{x_p} \wedge ... \wedge \not{x_q} \wedge ... \wedge x_{i + 1})
                        \end{aligned}
                    $$
                for all $x_1, ..., x_i \in \a$.
                \item A Lie $i$-coboundary is an element of $\im \del_{-i}^*$, i.e. a linear map:
                    $$\beta: \bigwedge^i \a \to k$$
                for which there exists another linear map:
                    $$\tilde{\beta}: \bigwedge^{i - 1} \a \to k$$
                i.e. a $(i - 1)$-cocycle $\tilde{\beta}$, such that:
                    $$
                        \begin{aligned}
                            & \beta(x_1 \wedge ... \wedge x_i)
                            \\
                            = & \del_{-i}^*(\tilde{\beta})(x_1 \wedge ... \wedge x_i)
                            \\
                            = &
                                \sum_{1 \leq p \leq i} (-1)^p \sigma(x_1 \wedge ... \wedge \not{x_p} \wedge ... \wedge x_i)
                                \\
                                & \qquad + \sum_{1 \leq p < q \leq i} (-1)^{p + q} \sigma( [x_p, x_q]_{\a} \wedge x_1 \wedge ... \wedge \not{x_p} \wedge ... \wedge \not{x_q} \wedge ... \wedge x_i)
                        \end{aligned}
                    $$
            \end{itemize}
        \end{example}
        \begin{example}[Lie $2$-cocycles and $2$-coboundaries with trivial coefficients] \label{example: low_degree_lie_cocycles_and_coboundaries_with_trivial_coefficients}
            Of particular usefulness to us are Lie $2$-cocycles and $2$-coboundaries with trivial coefficients. When $i = 2$, the conditions in example \ref{example: lie_cocycles_and_coboundaries_with_trivial_coefficients} reduce to the following.
            \begin{itemize}
                \item A Lie $2$-cocycle of $\a$ with coefficients in the trivial $\a$-module $k$ is a linear map $\sigma: \bigwedge^2 \a \to k$ satisfying the Jacobi identity, in the sense of definition \ref{def: twisted_semi_direct_products}.
                \item A Lie $2$-coboundary of $\a$ with coefficients in $k$ is a linear map $\beta: \bigwedge^2 \a \to k$ for which there exist an element $\tilde{\beta} \in C_1(\a, k)$ (i.e. a linear map $\tilde{\beta}: \a \to k$) such that:
                    $$\beta(x \wedge y) = \tilde{\beta}([x, y])$$
                Equivalently, this is saying that:
                    $$\beta = d_1^k(\tilde{\beta})$$
                Much like Lie $2$-cocycles, the value of a Lie $2$-boundaries $\beta$ at some $x \wedge y \in \bigwedge^2 \a$ is usually denotes by $\beta(x, y)$, i.e. we tend to think of Lie $2$-coboundaries as certain skew-symmetric bilinear maps $\a \x \a \to k$.
            \end{itemize}
        \end{example}
        \begin{example}[Lie $2$-coboundaries with \textit{non-trivial} coefficients] \label{example: low_degree_lie_coboundaries_with_non_trivial_coefficients}
            Let $\d$ be a Lie algebra. Let $\Omega$ be an $\d$-module defined by a Lie algebra action
                $$\rho: \d \to \gl(\Omega)$$
            By construction, we have that:
                $$d_i^\Omega := \Hom_{\rmU(\d)}(\del_{-(i + 1)}, \Omega)$$
            for all $i \in \Z_{\geq 0}$. By using remark \ref{remark: simplified_chevalley_eilenberg_complexes}, we shall see that a Lie $2$-coboundary of $\d$ with values in $\Omega$ shall be a linear map\footnote{Again, let us regard maps out of exterior powers as alternating multi-linear maps.} $\beta: \bigwedge^2 \d \to \Omega$ for which there is an element $\tilde{\beta} \in C_1(\d, \Omega)$ (i.e. a linear map $\tilde{\beta}: \d \to \Omega$) satisfying the following property:
                $$\beta(x, y) = \left( \rho(x) \cdot \tilde{\beta}(y) - \rho(y) \cdot \tilde{\beta}(x) \right) - \tilde{\beta}([x, y])$$
            (cf. \cite[Equation 1.6, p. 421]{kassel_quantum_groups}). In other words, $2$-coboundaries are error terms measuring how far a linear map $\d \to \Omega$ is from being a derivation of $\d$ with values in $\Omega$.
        \end{example}

        Finally, let us investigate a particular property enjoyed by graded Lie $2$-coboundaries, namely that they admit \say{lifts} along the coboundary map $d_1$ which are \textit{graded} linear maps. This is a technical preparation in service of the proof of proposition \ref{prop: sigma_1_is_not_coboundary}.
        \begin{remark}[Graded vector spaces and graded linear maps] \label{remark: grading_projections}
            For what follows, recall that if $V := \bigoplus_{n \in Z} V_n, W := \bigoplus_{n \in Z} W_n$ are vector spaces graded by some abelian group $Z$, and:
                $$\phi: V \to W$$
            is any linear map, then one can always find a \textit{graded} linear map:
                $$\bar{\phi}: V \to W$$
            given by:
                $$\bar{\phi} := \bigoplus_{n \in Z} \phi_n$$
            where $\phi_n: V_n \to W_n$ is the linear map - uniquely defined using $\phi$ for every $n \in Z$ - fitting into the following commutative diagram of vector spaces and linear maps, where the left vertical arrow is the canonical inclusion and the right vertical arrow is the canonical projection:
                $$
                    \begin{tikzcd}
                	V & W \\
                	{V_n} & {W_n}
                	\arrow["{\pr_n}", from=1-2, to=2-2]
                	\arrow["{\iota_n}", from=2-1, to=1-1]
                	\arrow["\phi_n", dashed, from=2-1, to=2-2]
                	\arrow["\phi", from=1-1, to=1-2]
                    \end{tikzcd}
                $$
            Clearly, if $\phi$ was graded to begin with, then $\bar{\phi} = \phi$.

            We caution the reader that even though $\bar{\phi}$ is given as a potentially infinite direct sum of maps, any evaluation $\bar{\phi}(v)$ at any vector $v \in V$ is actually just a finite sum of elements of $V$. This is because elements of $V := \bigoplus_{n \in Z} V_n$ are of the form $(v_n)_{n \in Z}$, with $v_n = 0$ for all but finitely many $n \in Z$.
        \end{remark}
        \begin{lemma}[Graded Lie $2$-coboundaries] \label{lemma: graded_2_coboundaries}
            Let $Z$ be an abelian group. Let $\d := \bigoplus_{n \in Z} \d_n$ be a $Z$-graded Lie algebra and $\Omega := \bigoplus_{n \in Z} \Omega_n$ be a $\d$-module, that is $Z$-graded as a vector space. Suppose also that the $\d$-action on $\Omega$ is homogeneous, i.e.:
                $$D \cdot \Omega_n \subseteq \Omega_{n + \deg D}$$
            for all $D \in \d$. 

            Next, let $\tau \in C_1(\d, \Omega)$. Then, $d_1^{\Omega}(\tau) \in C_2(\d, \Omega)$ will be a graded linear map from $\bigwedge^2 \d$ to $\Omega$ if and only if there exists some \textit{graded} linear map $\bar{\tau} \in C_1(\d, \Omega)$ such that:
                $$d_1^{\Omega}(\bar{\tau}) = d_1^{\Omega}(\tau)$$
        \end{lemma}
            \begin{proof}
                To prove the \say{if} direction, firstly take $\bar{\tau} = \tau$. Since $\tau$ is graded, the assertion is then clear from the equation defining Lie $2$-coboundaries, which is:
                    \begin{equation} \label{equation: graded_2_coboundaries}
                        d_1^{\Omega}(\tau)(D, D') = D \cdot \tau(D') - D' \cdot \tau(D) - \tau([D, D'])
                    \end{equation}
                given for all $D, D' \in \d$ (cf. example \ref{example: low_degree_lie_coboundaries_with_non_trivial_coefficients}).
            
                Let us now prove the \say{only if} direction. We claim that the sought-for $\bar{\tau}$ is nothing but:
                    $$\bar{\tau} := \bigoplus_{n \in Z} \tau_n$$
                where $\tau_n := \pr_n \circ \tau \circ \iota_n$ for each $n \in Z$ as in remark \ref{remark: grading_projections}. To prove this claim, let us firstly pick homogeneous elements $D, D' \in \d$ and for convenience, let us set:
                    $$m := \deg D, n := \deg D'$$
                Also, for each $d \in Z$, let us set:
                    $$d_1^{\Omega}(\tau)_d := \pr_d \circ d_1^{\Omega}(\tau) \circ \iota_d$$
                    $$\overline{d_1^{\Omega}(\tau)} := \bigoplus_{(m, n) \in Z^{\oplus 2} } d_1^{\Omega}(\tau)_{m + n}$$
                with notations as in remark \ref{remark: grading_projections}. Next, let us apply $\pr_{m + n}$ to both sides of equation \eqref{equation: graded_2_coboundaries}, which yields:
                    $$
                        \begin{aligned}
                            & d_1^{\Omega}(\tau)_{m + n}(D, D')
                            \\
                            = & \pr_{m + n}( d_1^{\Omega}(\tau)(D, D') )
                            \\
                            = & \pr_{m + n}\left( D \cdot \tau(D') - D' \cdot \tau(D) - \tau([D, D']) \right)
                            \\
                            = & \pr_{m + n}(D \cdot \tau(D')) - \pr_{m + n}(D' \cdot \tau(D)) - \pr_{m + n}(\tau([D, D']))
                            \\
                            = & D \cdot \pr_m( \tau(D') ) - D' \cdot \pr_n( \tau(D) ) - \pr_{m + n}(\tau([D, D']))
                            \\
                            = & D \cdot \tau_m(D') - D' \cdot \tau_n(D) - \tau_{m + n}([D, D'])
                        \end{aligned}
                    $$
                and note that the fourth and fifth equality result from the fact that $\d$ acts homogeneously on $\Omega$, as well as the fact that $\deg [D, D'] = m + n$, coming from our assumption that $\d$ is a $Z$-graded Lie algebra. From the above, we get that:
                    $$
                        \begin{aligned}
                            & \overline{d_1^{\Omega}(\tau)}(D, D')
                            \\
                            = & \bigoplus_{(m, n) \in Z^{\oplus 2} } d_1^{\Omega}(\tau)_{m + n}(D, D')
                            \\
                            = & \bigoplus_{(m, n) \in Z^{\oplus 2} } \left( D \cdot \tau_m(D') - D' \cdot \tau_n(D) - \tau_{m + n}([D, D']) \right)
                            \\
                            = & \bigoplus_{m \in Z} D \cdot \tau_m(D') - \bigoplus_{n \in Z} D' \cdot \tau_n(D) - \bigoplus_{(m, n) \in Z^{\oplus 2} } \tau_{m + n}([D, D'])
                            \\
                            = & D \cdot \bigoplus_{m \in Z} \tau_m(D') - D' \cdot \bigoplus_{n \in Z} \tau_n(D) - \bigoplus_{(m, n) \in Z^{\oplus 2} } \tau_{m + n}([D, D'])
                            \\
                            = & D \cdot \bar{\tau}(D') - D' \cdot \bar{\tau}(D) - \bar{\tau}([D, D'])
                        \end{aligned}
                    $$
                in which, to go from the third equality to the fourth one, we have made use of the assumption that, because $D, D' \in \d$ are homogeneous elements and because the $\d$-action on $\Omega$ is homogeneous, we have as a result that:
                    $$\bigoplus_{m \in Z} D \cdot \tau_m(D') = D \cdot \sum_{m \in Z} \tau_m(D') = D \cdot \bigoplus_{m \in Z} \tau_m(D')$$
                    $$\bigoplus_{n \in Z} D' \cdot \tau_n(D) = D' \cdot \sum_{n \in Z} \tau_n(D) = D' \cdot \bigoplus_{n \in Z} \tau_m(D)$$
                Now, since $d_1^{\Omega}(\tau) \in C_2(\d, \Omega)$ has been assumed to be graded to begin with, we have that:
                    $$\overline{d_1^{\Omega}(\tau)} = d_1^{\Omega}(\tau)$$
                and hence:
                    $$d_1^{\Omega}(\tau)(D, D') = D \cdot \bar{\tau}(D') - D' \cdot \bar{\tau}(D) - \bar{\tau}([D, D'])$$
                for every pair of homogeneous elements $D, D' \in \d$. 
            \end{proof}

    \subsection{Interpretations of \texorpdfstring{$H^1_{\Lie}$}{} and \texorpdfstring{$H^2_{\Lie}$}{}}
        We conclude this subsection by discussing interpretations of the two cases of low-dimensional Lie algebra cohomology that are of interest to us, namely $H^1_{\Lie}$ and $H^2_{\Lie}$. Respectively, these parametrise so-called \say{outer derivations} and abelian extensions. We will be needing these notions in the discussions leading up to proposition \ref{prop: cohomological_non_triviality_of_billig_toroidal_cocycles}, in order to be able to discern whether or not certain unequal $2$-cocycles might be cohomologous to one another. More specifically, we will be making use of the fact that isomorphism classes of abelian extensions of a given Lie algebra are in bijection with its $2^{nd}$ cohomology (with suitable coefficients, of course).
        
        \begin{definition}[Lie derivations] \label{def: lie_derivations}
            (Cf. \cite[Section VII.2, Equation 2.2, p. 234]{hilton_stammbach_homological_algebra}) Let $\d$ be a Lie algebra and let $\Omega$ be an $\d$-module defined by a Lie algebra action:
                $$\rho: \d \to \gl(\Omega)$$
            A \textbf{derivation} of $\d$ with values in $\Omega$ is a linear map:
                $$L: \d \to \Omega$$
            satisfying the following property for all $X, Y \in \d$:
                $$L( [X, Y] ) = \rho(X) \cdot L(Y) - \rho(Y) \cdot L(X)$$
            Such derivations form a vector space, for which we shall write $\der(\d, \Omega)$. For every $\omega \in \Omega$, one can define an \textbf{inner derivation} $L_{\omega}$, specified by:
                $$L_{\omega} = \rho(-) \cdot \omega$$
            Inner derivations form a vector subspace of $\der(\d, \Omega)$, which is denoted by $\inn(\d, \Omega)$. Derivations that are not inner are said to be \textbf{outer}, and the vector space of outer derivations is identified as:
                $$\out(\d, \Omega) := \der(\d, \Omega)/\inn(\d, \Omega)$$
        \end{definition}
        \begin{example}[The adjoint action]
            Let $\d$ be a Lie algebra and let $\d$ also be considered as a module over itself via the adjoint action. Then, for any $X \in \d$, the map:
                $$\ad(X): \d \to \d$$
            will be an inner derivation of $\d$ with values in itself.
        \end{example}
        \begin{example}[Derivations on current algebras] \label{example: derivations_on_current_algebras}
            Let $\a$ be a Lie algebra over $k$ and $A$ be a commutative $k$-algebra. On the current algebra $\a \tensor_k A$ (cf. definition \ref{def: current_algebras}), one can construct a natural action of $\der(A)$ on $\a \tensor_k A$ by:
                $$D \cdot xf := x D(f)$$
            for all $D \in \der(A)$ and all $x \in \a, f \in A$. This action is by Lie derivations, since the following holds for all $D \in \der(A)$ and all $x, y \in \g, f, g \in A$, merely because $D$ is a derivation on $A$:
                $$D \cdot [xf, yg]_{\a \tensor_k A} = D \cdot [x, y] fg = [x, y] ( D(f)g + f D(g) )$$
        \end{example}
        \begin{example}[Lie derivatives] \label{example: lie_derivatives}
            Another prominent example of Lie derivations is the operation of Lie derivatives. For us, knowing that this operation is well-defined will be useful for proving of lemma \ref{lemma: vector_field_action_on_toroidal_lie_algebras}. 

            To construct these derivations, consider firstly a commutative $k$-algebra $A$ and the standard action:
                $$L: \der(A) \to \gl(A)$$
            of $\der(A)$ on $A$, via evaluations of derivations $D \in \der(A)$ on \say{functions} $f \in A$, i.e.:
                $$L(D)(f) := D(f)$$
            It is clear that $L$ as above is a Lie algebra homomorphism, so the action is well-defined. Our goal is to somehow induce an action of $\der(A)$ on $\Omega^1_{A/k}$ using $L$ as above; the claim is that the sought-for action shall be given by so-called \textbf{Lie derivatives}, i.e.:
                $$D \cdot g df := D(g) df + g d(D(f))$$
            for all $D \in \der(A)$ and all $f, g \in A$.

            Now, recall that if $U$ is any bialgebra over $k$ (cf. \cite[Section III.2]{kassel_quantum_groups}) with comultiplication $\Delta: U \to U^{\tensor 2}$ and $V, W$ are $U$-modules determined by algebra homomorphisms $\pi: U \to \End_k(V)$ and $\rho: U \to \End_k(W)$, then $V \tensor_k W$ will be a $U^{\tensor 2}$-module \textit{a priori}, determined by an algebra homomorphism:
                $$\pi \tensor \rho: U^{\tensor 2} \to \End_k(V \tensor_k W)$$
            given by:
                $$(\pi \tensor \rho)(u \tensor u')(v \tensor w) := uv \tensor u' w$$
            for all $u, u' \in U$ and $v \in V, w \in W$. Then, using the fact that $\Delta$ is an algebra homomorphism per the definition of bialgebras, one can form the composite algebra homomorphism:
                $$U \xrightarrow[]{\Delta} U^{\tensor 2} \xrightarrow[]{\pi \tensor \rho} \End_k(V \tensor W)$$
            and thus can endow $V \tensor_k W$ with the structure of a $U$-module, given by:
                $$u \cdot (v \tensor w) := (\pi \tensor \rho)(\Delta(u))( v \tensor w )$$
            for all $u \in U$ and all $v \in V, w \in W$.
            
            Because the universal enveloping algebra of $\der(A)$ (or for that matter, any Lie algebra) is a Hopf algebra - and hence a bialgebra - with comultiplication given by:
                $$\Delta(X) := X \tensor 1 + 1 \tensor X$$
            for all $X \in \rmU(\der(A))$ (cf. \cite[Section III.3]{kassel_quantum_groups}), the abstract nonsense above about tensor products of bialgebra modules can be applied to the case:
                $$U := \rmU(\der(A))$$
            The aforementioned action of $\der(A)$ on $A$ thus automatically extends to $A^{\tensor 2}$ by:
                $$
                    \begin{aligned}
                        & D \cdot ( f \tensor g )
                        \\
                        := & L^{\tensor 2}( \Delta(D) )( f \tensor g )
                        \\
                        = & ( L(D) \tensor 1 + 1 \tensor L(D) )(f \tensor g)
                        \\
                        = & D(f) \tensor g + f \tensor D(g)
                    \end{aligned}
                $$
            Finally, recall from definition \ref{def: kahler_differentials} that as an $A$-module, $\Omega^1_{A/k}$ is the quotient of the $A$-module $A^{\tensor 2}$ by the $A$-submodule $I$ generated by elements of the form $fg \tensor 1 - f \tensor g - g \tensor f$, for all $f, g \in A$. We have that:
                $$
                    \begin{aligned}
                        & D \cdot ( fg \tensor 1 - f \tensor g - g \tensor f )
                        \\
                        = & D(fg) \tensor 1 - ( D(f) \tensor g + f \tensor D(g) ) - ( D(g) \tensor f + g \tensor D(f) )
                        \\
                        = & (D(f) g + f D(g)) \tensor 1 - ( D(f) \tensor g + f \tensor D(g) ) - ( D(g) \tensor f + g \tensor D(f) )
                        \\
                        = & ( D(f) g \tensor 1 - D(f) \tensor g - g \tensor D(f) ) + ( D(g) f \tensor 1 - D(g) \tensor f - f \tensor D(g) ) 
                        \in I
                    \end{aligned}
                $$
            which shows that $I$ is a $\der(A)$-submodule of $A^{\tensor 2}$. From this, we see that there is a $\der(A)$-action on $\Omega^1_{A/k}$ given by:
                $$D \cdot g df := D(g) df + g d(D(f))$$
            as claimed.

            It is also clear that Lie derivatives satsify the Leibniz rule, so they are indeed Lie derivations.
        \end{example}
        \begin{theorem}[$H^1_{\Lie}$ and derivations of Lie algebras]
            \cite[Theorem 2.1 and Proposition 2.2]{hilton_stammbach_homological_algebra} Let $\d$ be a Lie algebra and $\Omega$ be an $\d$-module. Then:
                $$Z^1_{\Lie}(\d, \Omega) \cong \der(\d, \Omega)$$
                $$B^1_{\Lie}(\d, \Omega) \cong \inn(\d, \Omega)$$
            and hence:
                $$H^1_{\Lie}(\d, \Omega) \cong \out(\d, \Omega)$$
        \end{theorem}

        \begin{theorem}[$H^2_{\Lie}$ and abelian extensions] \label{theorem: H^2_of_lie_algebras_and_abelian_extensions}
            (Cf. \cite[Theorem VII.3.3]{hilton_stammbach_homological_algebra}) Let $\d$ be a Lie algebra over $k$ and $\Omega$ be an $\d$-module, equipped with the abelian Lie algebra structure. There is then a bijection:
                $$H^2_{\Lie}(\d, \Omega) \xrightarrow[]{\cong} \{ \text{isomorphism classes of extensions of $\d$ by $\Omega$} \}$$
                $$\sigma \mapsto \Omega \rtimes^{\sigma} \d$$
        \end{theorem}
        
        \begin{remark}[Non-abelian Lie algebra cohomology ?]
            There is also a variant of the construction given in definition \ref{def: lie_algebra_cohomology}, called \textbf{non-abelian Lie algebra cohomology}, which ostensibly is for the purpose of classifying Lie algebra extensions:
                $$0 \to \t \to \frake \to \d \to 0$$
            where the kernel $\t$ is not necessarily abelian. This is harder to define and in fact, is unnecessary for our purposes, so we will make no further mention of it.
        \end{remark}
    \end{appendices}

    \newpage

    \printbibliography
    \addcontentsline{toc}{chapter}{\textbf{Bibliography}}

\end{document}