\input{thesis preambles}

\setcounter{chapter}{0}
\setcounter{section}{0}

\input{commands}
\newcommand{\toroidal}{\t}
\newcommand{\extendedtoroidal}{\hat{\t}}
\newcommand{\simpleroots}{\mathbb{I}}
\renewcommand{\positive}{+}
\renewcommand{\negative}{-}
\newcommand{\divzero}{\der_{\gamma}(A)}

\begin{document}

    \title{
    \texorpdfstring{\Huge On a two-variable analogue of untwisted affine Kac-Moody algebras coming from affine Yangians}

    \vfill

    \begin{centering}
        \normalsize A Thesis Submitted to the
        \\
        College of Graduate and Postdoctoral Studies
        \\
        in Partial Fulfillment of the Requirements
        \\
        for the Degree of Master of Science
        \\
        in the Department of Mathematics and Statistics
        \\
        University of Saskatchewan
        \\
        Saskatoon, Saskatchewan, Canada
    \end{centering}

    \vfill

    \begin{centering}
        \normalsize \textcopyright Copyright Dat Minh Ha, June, 2024. All rights reserved.
        \\
        Unless otherwise noted, copyright of the material in this thesis belongs to the author.
    \end{centering}

    \vfill

    \author{\normalsize Dat Minh Ha}
    \date{\normalsize \today}
}

\maketitle

    \newpage

    \listoftodos

    \todo[inline]{Remove to-do list before final compilation}

    \newpage

    \chapter*{Frontmatter}
        \minitoc
    
        \newpage
    
        \section*{Permission to use}
    In presenting this thesis in partial fulfillment of the requirements for a Postgraduate degree from the University of Saskatchewan, I agree that the Libraries of this University may make it freely available for inspection. I further agree that permission for copying of this thesis in any manner, in whole or in part, for scholarly purposes may be granted by the professor or professors who supervised my thesis work or, in their absence, by the Head of the Department or the Dean of the College in which my thesis work was done. It is understood that any copying or publication or use of this thesis or parts thereof for financial gain shall not be allowed without my written permission. It is also understood that due recognition shall be given to me and to the University of Saskatchewan in any scholarly use which may be made of any material in my thesis.
    
        \newpage
    
        \section*{Abstract}
    The purpose of this thesis is to construct so-called $\gamma$-extended toroidal Lie algebras. Originally, these $\gamma$-extended toroidal Lie algebras, i.e. universal central extensions (UCEs):
        $$\toroidal$$
    of Lie algebras of the form:
        $$\g[v^{\pm 1}, t^{\pm 1}] := \g \tensor_{\bbC} \bbC[v^{\pm 1}, t^{\pm 1}]$$
    where $\g$ is a finite-dimensional simple Lie algebra over $\bbC$, were created as a mean of endowing toroidal Lie algebras with Lie bialgebra structures, so that toroidal Lie bialgebras can be recognised as classical limits of so-called \say{affine Yangians} in a certain sense. Per \cite{etingof_kazhdan_quantisation_1}, this would mean construct Manin triples of the form:
        $$(\toroidal, \toroidal^{\positive}, \toroidal^{\negative})$$
    which in itself, requires us to endow $\toroidal$ with an invariant bilinear form satisfying some conditions. However, an issue is that any invariant bilinear form on a UCE is necessarily \textit{degenerate}. As such, we are motivated to enlarge toroidal Lie algebras into $\gamma$-extended Lie algebras:
        $$\extendedtoroidal$$
    and this is done in such a way that the resulting larger Lie algebras can then be endowed with \textit{invariant} symmetric bilinear forms that are also \textit{non-degenerate}; importantly, the construction of these bilinear form depends entirely on a certain linear map:
        $$\gamma: \bbC[v^{\pm 1}, t^{\pm 1}] \to \bbC$$
    (and hence the name of our Lie algebras).
    
    We shall see that the Lie algebras $\extendedtoroidal$ all arise as \say{twists} of the semi-direct product $\toroidal \rtimes \der_{\gamma}(\bbC[v^{\pm 1}, t^{\pm 1}])$ by Lie $2$-cocycles $\sigma \in Z^2_{\Lie}(\der_{\gamma}(\bbC[v^{\pm 1}, t^{\pm 1}]), \z(\toroidal))$, with $\der_{\gamma}(\bbC[v^{\pm 1}, t^{\pm 1}])$ being a certain ($\Z^2$-graded) Lie subalgebra of the Lie algebra $\der(\bbC[v^{\pm 1}, t^{\pm 1}])$ of all derivations on $\bbC[v^{\pm 1}, t^{\pm 1}]$. Moreover, we will see that there is a readily available example of such a $2$-cocycle giving rise to a $\gamma$-extended toroidal Lie algebra that is \textit{not} isomorphic to the aforementioned semi-direct product.
    
        \newpage
    
        \section*{Acknowledgements}
    I must firstly acknowledge the invaluable contributions that my advisors, Prof. Dr. Curtis Wendlandt and Prof. Dr. Alex Weekes, have made towards this thesis. Not only were they the ones to give me this problem, but the thesis simply could not have been written without their close following of my progress (and often, even the lack thereof), as well as their many suggestions and advices regarding both proof strategies and presentation. I would also like to thank them for all the mathematics that they have taught and introduced me to throughout the length of my MSc., much of which I had no prior familiarity with. I would also like to express my gratitude towards them, for teaching me how to not just mathematical contents themselves, but also how to actually \textit{do} mathematics. I was not able to appreciate many of their advices in the moment, but with the benefit of hindsight, these advices have become very precious to me.
    
    I would also like to thank Prof. Dr. Steven Rayan, who was my instructor for two of the courses that I was required to take as a part of my MSc., not only for being a wonderful instructor, but also for the many administrative help that he has provided me with over the past two years. I am also grateful for the fact that through him, I was able to become friends with some other graduate students in the Department, namely Mahmud Azam and Kuntal Banerjee. We have had many enjoyable discussions.

    \newpage

    {
      \hypersetup{} 
      \dominitoc
      \tableofcontents %sort sections alphabetically
    }

    \newpage

    \chapter{Introduction}
        \begin{abstract}
            We begin this thesis by giving an account of the contexts and motivations that lead to our construction of $\gamma$-extended toroidal Lie algebras, as well as the structure of the whole thesis itself.
        \end{abstract}
    
        \minitoc

        \newpage

        \section{How and why have we constructed \texorpdfstring{$\gamma$}{}-extended toroidal Lie algebras ?}
    \subsection{Background}
        A rather well-known story in the representation theory of Lie algebras over algebraically closed fields of characteristic $0$ (e.g. $\bbC$, which from now on will be the default underlying field) is that of finite-dimensional semi-simple Lie algebras. This theory dates back to the works of Wilhelm Killing and \'Elie Cartan in the late $19^{th}$ century and early $20^{th}$ century on the classification of finite-dimensional semi-simple Lie algebras over $\bbC$. A salient feature of, and indeed, a very important technical ingredient in this theory, is an essentially unique\footnote{I.e. unique up to non-zero scalar multiples after domain-restriction to a simple direct summand.} \textit{non-degenerate} and \textit{invariant} symmetric bilinear form that any finite-dimensional semi-simple Lie algebra can be endowed with. This is the famous \textbf{Killing form} (named after Wilhelm Killing) and using it, one is able to more-or-less develop the entire structural and representation theory of these Lie algebras. In particular, using the Killing form, one is able to construct the so-called \textbf{Cartan matrix} (named after \'Elie Cartan), which contains within it a lot - if not even all - of the important information about finite-dimensional semi-simple Lie algebras. For instance, should one know only of a Cartan matrix, one can then write down a presentation in terms of generators and relations (which is due to Claude Chevalley and Jean-Pierre Serre) for a uniquely corresponding finite-dimensional semi-simple Lie algebra. For more details, we refer the reader to \cite{humphreys_lie_algebras} and the first half of \cite{carter_affine_lie_algebras}, and we have also given a brief account of this story in subsection \ref{subsection: finite_dimensional_simple_lie_algebras}.

        Now, one particular property that the Cartan matrix of a finite-dimensional semi-simple Lie algebra is that it is \textit{positive-definite}. However, it is also possible to construct similar matrices that are not necessarily positive-definite, which are nowadays commonly called \textbf{generalised Cartan matrices} (cf. \cite[Chapter 1]{kac_infinite_dimensional_lie_algebras}). Should we then insist nevertheless on writing down presentations \textit{\`a la} Chevalley-Serre, we would obtain certain infinite-dimensional Lie algebras that are nowadays known as \textbf{Kac-Moody algebras} (cf. \cite[Chapter 1]{kac_infinite_dimensional_lie_algebras}). It should be noted right away that generalised Cartan matrices - true to their names - admit the Cartan matrices of finite-dimensional semi-simple Lie algebras as special cases. Consequently, finite-dimensional semi-simple Lie algebras are certain instances of Kac-Moody algebras, and they are commonly referred to as \textbf{finite-type} Kac-Moody algebras when considered in this context.
        
        Amongst the Kac-Moody algebras, those of so-called \textbf{affine type} are of special interest. Specifically, as Robert Moody and Victor Kac independently discovered in the late 1960s and early 1970s, one obtains such Lie algebras by requiring that the associated Cartan matrix be \textit{positive-semi-definite}. What is rather remarkable about these affine Kac-Moody algebras is that they admit something called \textbf{loop realisations}: starting with a finite-dimensional semi-simple Lie algebra:
            $$\g$$
        with Cartan matrix $C$, one can firstly form the \textbf{loop algebra}:
            $$\g[v^{\pm 1}] := \g \tensor_{\bbC} \bbC[v^{\pm 1}]$$
        - which will always be equipped with a non-degenerate and invariant symmetric bilinear form originating naturally from the Killing form - then consider its universal central extension (UCE), which happens to be by a $1$-dimensional centre, which is to say that:
            $$\uce(\g[v^{\pm 1}]) \cong \g[v^{\pm 1}] \oplus \bbC c_{\aff}$$
        where $c_{\aff} \in \uce(\g[v^{\pm 1}])$ is central, and then finally adding on a Lie derivation:
            $$D_{\aff} \in \der( \uce(\g[v^{\pm 1}]) )$$
        with the purpose of it all being so that at the end of the process, one shall obtain a Lie algebra:
            $$\hat{\g} := \uce(\g[v^{\pm 1}]) \rtimes \bbC D_{\aff} \cong (\g[v^{\pm 1}] \oplus \bbC c_{\aff}) \rtimes \bbC D_{\aff}$$
        that is isomorphic to the Kac-Moody algebra whose generalised Cartan matrix is obtained from $C$ by adding one extra row and one extra column in a certain manner (for more details, see \cite[Chapter 7]{kac_infinite_dimensional_lie_algebras}). The idea here is that, because any invariant symmetric bilinear form on $\uce(\g[v^{\pm 1}])$ is necessarily degenerate and with radical containing at least the non-zero central element $c_{\aff}$, one introduces the extra element $D_{\aff}$ to pair non-trivially with $c_{\aff}$, thereby fixing the issue of degeneracy. In other words, $D_{\aff}$ is to be dual to $c_{\aff}$ to begin with. The fact that $D_{\aff}$ is a Lie derivation on $\g[v^{\pm 1}] \oplus \bbC c_{\aff}$ is actually a consequence of this construction.

    \subsection{What is done in this thesis ?}
        Our starting point is not the single-loop algebra $\g[v^{\pm 1}]$ as above, but rather the \textbf{double-loop algebra}\footnote{Which are \textit{not} instances of Kac-Moody algebras!}:
            $$\g[v^{\pm 1}, t^{\pm 1}]$$
        which, like above, will also be equipped with a non-degenerate and invariant symmetric bilinear form originating naturally from the Killing form and depending on a distinguished linear map, a kind of modified formal residue map:
            $$\gamma: \bbC[v^{\pm 1}, t^{\pm 1}] \to \bbC$$
        on which our constructions will depend crucially (see subsection \ref{subsection: definition_of_yangian_extended_toroidal_lie_algebras}). We then again consider the UCE:
            $$\toroidal := \uce(\g[v^{\pm 1}, t^{\pm 1}])$$
        but a large difference in contrast to the affine Kac-Moody situation is that now, the centre $\z(\toroidal)$ is \textit{infinite-dimensional}; luckily though, it is graded, and the graded components are all finite-dimensional. Regardless, the issue whereby invariant symmetric bilinear forms on the UCE must be degenerate persists, which leads us to consider the \textit{vector space}:
            $$\extendedtoroidal := \toroidal \oplus \z(\toroidal)^{\star}$$
        which admits a canonically defined \textit{non-degenerate} symmetric bilinear form:
            $$(-, -)_{\extendedtoroidal}$$
        extending the bilinear form on $\g[v^{\pm 1}, t^{\pm 1}]$ defined using $\gamma$ (see corollary \ref{coro: pairing_yangian_div_zero_vector_fields_and_cyclic_1_forms}).
            
        A \textbf{$\gamma$-extended toroidal Lie algebra} structure shall then be a Lie bracket on $\extendedtoroidal$, with respect to which the non-degenerate bilinear form $(-, -)_{\extendedtoroidal}$ is \textit{invariant} and the UCE $\toroidal$ becomes a Lie subalgebra of $\extendedtoroidal$ (cf. definition \ref{def: yangian_extended_toroidal_lie_algebras}). \textit{Our goal for the thesis is to classify such Lie algebras}, and in order to achieve this goal, we will be proceeding in the following steps:
        \begin{enumerate}
            \item Just like above, we shall prove that $\z(\toroidal)^{\star}$ is graded-isomorphic as a vector space to a certain Lie subalgebra $\der_{\gamma}(\bbC[v^{\pm 1}, t^{\pm 1}])$ of the Lie algebra $\der(\bbC[v^{\pm 1}, t^{\pm 1}])$ of derivations on $\bbC[v^{\pm 1}, t^{\pm 1}]$ (where the Lie structure is given by commutators). See definition \ref{def: yangian_div_zero_vector_fields} and proposition \ref{prop: yangian_div_zero_vector_fields_are_graded_dual_to_toroidal_centre}. This allows us to endow any vector space that is isomorphic to $\extendedtoroidal$ with a natural Lie algebra structure, coming from those on $\toroidal$ and on $\der_{\gamma}(\bbC[v^{\pm 1}, t^{\pm 1}])$. 
            \item Because $\der_{\gamma}(\bbC[ v^{\pm 1}, t^{\pm 1} ])$ canonically acts on $\toroidal$ (see corollary \ref{coro: a_fixed_yangian_div_zero_vector_field_action}), we can form the semi-direct product:
                $$\toroidal \rtimes \der_{\gamma}(\bbC[ v^{\pm 1}, t^{\pm 1} ])$$
            The main theorem (theorem \ref{theorem: yangian_extended_toroidal_lie_algebras_main_theorem}) will then demonstrate that in fact, every $\gamma$-extended toroidal Lie algebra is no more than a \say{twist} of $\toroidal \rtimes \der_{\gamma}(\bbC[ v^{\pm 1}, t^{\pm 1} ])$ by some Lie $2$-cocycle $\sigma \in Z^2_{\Lie}(\der_{\gamma}(\bbC[ v^{\pm 1}, t^{\pm 1} ]), \z(\toroidal))$.
        \end{enumerate}
        A slightly imprecise version of theorem \ref{theorem: yangian_extended_toroidal_lie_algebras_main_theorem} is as follows:
        \begin{theorem}
            A given Lie algebra will be a $\gamma$-extended toroidal Lie algebra if and only if it is isomorphic to a twisted semi-direct product:
                $$\toroidal \rtimes^{\sigma} \der_{\gamma}(\bbC[v^{\pm 1}, t^{\pm 1}])$$
            whose corresponding Lie $2$-cocycle $\sigma \in Z^2_{\Lie}(\der_{\gamma}(\bbC[v^{\pm 1}, t^{\pm 1}]), \toroidal)$ satisfies a certain invariance property that depends on $\gamma$.
        \end{theorem}

        The most important example of a $\gamma$-extended toroidal Lie algebra (at least for us) is the semi-direct product $\toroidal \rtimes \der_{\gamma}(\bbC[v^{\pm 1}, t^{\pm 1}])$, but we will also be able to provide an example of a $\gamma$-extended toroidal Lie algebra that is not isomorphic to this semi-direct product, which will be our second main result (see theorem \ref{theorem: billig_cocycle_main_theorem}) by analysing a particular Lie $2$-cocycle of $\der(\bbC[v^{\pm 1}, t^{\pm 1}])$ with values in $\z(\toroidal)$, which has been known since a paper of Robert Moody and Senapathi Eswara Rao from 1990, namely \cite{moody_rao_n_toroidal_vertex_representations}.

        Finally, let us make some comments on how our $\gamma$-extended toroidal Lie algebras compare to the \textbf{extended affine Lie algebras} (or \textbf{EALAs} for short) of Neher et al. (see e.g. \cite{neher_lectures_on_EALAs}). EALAs are defined in a style similar to how Kac-Moody algebras attached to a Cartan matrix are defined in say, \cite[Chapters 1 and 2]{kac_infinite_dimensional_lie_algebras}. Slightly more specifically, they are designed to be Lie algebras supporting a notion of Cartan subalgebras and subsequent notions of root space decompositions, root systems, etc. (see \cite[Subsection 2.1]{neher_lectures_on_EALAs}), and one then proves that under some hypotheses, such algebras can also be written as twisted semi-direct products of the form:
            $$\uce( \g[v_1^{\pm 1}, ..., v_n^{\pm 1}] ) \rtimes^{\sigma} \d$$
        wherein $\d$ can vary between certain Lie subalgebras of $\der(\bbC[v_1^{\pm 1}, ..., v_n^{\pm 1}])$. Even though this is very similar to our main theorem concerning $\gamma$-extended toroidal Lie algebras, we believe that our notion of $\gamma$-extended toroidal Lie algebras and the notion of EALAs of so-called \say{nullity} $2$ that were considered in the aforementioned work do not coincide. Ultimately, this is because the bilinear forms on the former are not of total degree $0$ (by construction), while those on the latter are (also by construction). That said, these similarities between EALAs and our $\gamma$-extended toroidal Lie algebras are still rather remarkable and curious. For more details, we refer the reader to \cite[Subsections 6.11, 6.13, and 6.14]{neher_EALA_survey} (Theorem 6.14 in particular), as well as \cite{neher_structure_theory_of_EALAs} and \cite{allison_berman_faulkner_pianzola_multiloop_realisation_of_EALAs}.
        
        We should also mention, that our goal in constructing $\gamma$-extended toroidal Lie algebras and the goal of Neher and others in constructing their EALAs differ somewhat. At the end of the day, we are only interested in classifying enlargements of the Lie algebra $\toroidal$ satisfying certain conditions, while the other authors were interested in defining, classifying, and studying the structures of certain natural multi-variable generalisations of (affine) Kac-Moody algebras.

    \subsection{History} \label{subsection: history}
        Even though the above relationship between our construction of $\gamma$-extended toroidal Lie algebras and the older loop realisations of untwisted affine Kac-Moody algebras can indeed serve as a motivation for the main topic of this thesis all on its own, this was actually not how we were lead to considering these $\gamma$-extended toroidal Lie algebras. Originally, we were motivated to consider $\gamma$-extended toroidal Lie algebras because they would help us realise a certain Lie bialgebra structure on:
            $$\toroidal^{\positive} := \uce(\g[v^{\pm 1}, t])$$
        as the \say{classical limit} of a topological bialgebra:
            $$\rmY_{\hbar}(\hat{\g})$$
        known as the \textbf{Yangian} associated to the affine Kac-Moody algebra $\hat{\g}$, or simple \say{the} \textbf{affine Yangian} for short, when $\g$ is fixed.

        Before we can expand more on this, let us briefly recall some general terminologies. A \textbf{quantisation} of a (topological) Lie bialgebra $(\fraku, \delta)$ is a topological $\bbC[\hbar]$-bialgebra:
            $$(Y, \Delta)$$
        which satisfies the following properties:
        \begin{itemize}
            \item $Y$ is torsion-free as a $\bbC[\hbar]$-module,
            \item $Y/\hbar \cong \rmU(\a)$ as $\bbC$-bialgebras, and
            \item if $x \in \fraku$ is any element and $\tilde{x} \in Y$ is any lift modulo $\hbar$ of it (i.e. $\tilde{x} \equiv x \pmod{\hbar}$), then:
                $$\hbar\delta(x) \equiv (\Delta - \Delta^{\cop})(\tilde{x}) \pmod{\hbar^2}$$
            wherein $\Delta^{\cop}$ is the comultiplication of opposite order.
        \end{itemize}
        We will also be referring to the Lie bialgebra $(\a, \delta)$ as the \textbf{classical limit} of $(Y, \Delta)$. If $(\a, \delta)$ is a graded Lie bialgebra (e.g. if $\a = \toroidal^{\positive}$) then $\rmU(\a)$ will carry an induced grading, and if $Y/\hbar \cong \rmU(\a)$ is a graded isomorphism, then we will call $(Y, \delta)$ a \textbf{graded quantisation} of $(\a, \delta)$.
        
        Now, what is the affine Yangian $\rmY_{\hbar}(\hat{\g})$ ? This is a topological bialgebra over $\bbC[\hbar]$ that is supposed to be a graded quantisation of a certain topological Lie bialgebra structure\footnote{Here, $\hattensor_{\bbC}$ denotes a suitable topological completion of the algebraic tensor product, which is necessary due to the appearance of certain infinite sums.}:
            $$\delta: \extendedtoroidal^{\positive} \to \extendedtoroidal^{\positive} \hattensor_{\bbC} \extendedtoroidal^{\positive}$$
        Now, it is certainly possible to let $\delta$ be the Lie bialgebra structure coming directly from a topological bialgebra structure $\Delta$ constructed in \cite{guay_nakajima_wendlandt_affine_yangian_coproduct} via the equation:
            $$\hbar\delta(x) \equiv (\Delta - \Delta^{\cop})(\tilde{x}) \pmod{\hbar^2}$$
        (given for all $x \in \toroidal^{\positive}$ and all lifts $\tilde{x} \in \rmY_{\hbar}(\hat{\g})$ thereof), and one can then simply say that this Lie bialgebra structure on $\toroidal^{\positive}$ is the classical limit of the affine Yangian. However, the problem with this approach is that by the end of the process, it will not be clear whether or not the resulting Lie cobracket $\delta$ will have originated from the structure of $\toroidal^{\positive}$ itself. In turn, this will lead us towards difficulties in say, identifying the classical $\sfr$-matrix of $\toroidal^{\positive}$ corresponding to the quantum $\sfR$-matrix of the affine Yangian (which has already been found in \cite{appel_gautam_wendlandt_R_matrices_of_affine_yangians}).
        
        Now, let us recall that there is a bijective correspondence between so-called \textbf{Manin triples}, which are triples of Lie algebras $(\p, \p^+, \p^-)$ subjected to certain conditions, and Lie bialgebra structures on $\p^+$ (and indeed, on $\p$ and $\p^-$ as well; see \cite[Subsections 1.2 and 7.4]{etingof_kazhdan_quantisation_1} and \cite[Section 6.1]{etingof_schiffmann_lectures_on_quantum_groups} for more details). One key detail here is that in defining such Manin triples, one is required to supply a \textit{non-degenerate} and \textit{invariant} symmetric bilinear form on $\p$ (sastisfying certain conditions). As elaborated on earlier, any invariant symmetric bilinear form on a UCE such as $\toroidal$ is necessarily degenerate, so we ought not to attempt to directly construct a Manin triple of the form $(\toroidal, \toroidal^{\positive}, \toroidal^{\negative})$. We do, however, now have an extension of $\toroidal$ on which there is a non-degenerate and invariant symmetric bilinear form, namely $\extendedtoroidal$ equipped with $(-, -)_{\extendedtoroidal}$ as constructed above, and so an alternative strategy is to construct a Manin triple of the form:
            $$(\extendedtoroidal, \extendedtoroidal^{\positive}, \extendedtoroidal^{\negative})$$
        (wherein $\extendedtoroidal^{\positive} := \toroidal^{\positive} \rtimes \der_{\gamma}(\bbC[v^{\pm 1}, t])$), which helps us construct the Lie cobracket $\delta$. Specifically, we can rely on the fact that $\toroidal \subset \extendedtoroidal$ is not just a Lie ideal but also a Lie coideal, and hence a Lie sub-bialgebra. 

        As an aside, let us note that the story that we have just outlined above runs in analogy with the more classical story of Drinfeld's construction of the Yangian $\rmY_{\hbar}(\g)$ of a finite-dimensional simple Lie algebra $\g$, as was done in \cite{drinfeld_original_yangian_paper}. This is the unique $\Z_{\geq 0}$-graded Hopf algebra quantising the Lie bialgebra structure on $\g[t]$ that is specified by the Manin triple $(\g[t^{\pm 1}], \g[t], t^{-1}\g[t^{-1}])$. Let us also remark that like in the affine setting where the classical limit is a UCE, $\g[t]$ is in fact also a UCE, namely the trivial UCE of itself (see example \ref{example: affine_lie_algebras_centres}).

    \subsection{What have we \textit{not} done ?}
        There are many natural questions that can be posed by the end of this thesis. We have chosen to highlight the following, which is a question that pertains directly to the latter parts of the thesis, particular to theorem \ref{theorem: billig_cocycle_main_theorem}.
        \begin{question}
            How many \say{$\gamma$-invariant} Lie $2$-cocycles (cf. definition \ref{def: yangian_toroidal_cocycles}) are there ? And for that matter, what is $H^2_{\Lie}(\der_{\gamma}(\bbC[v^{\pm 1}, t^{\pm 1}]), \z(\toroidal))$ ?
        \end{question}
        Some inspiration and guidance can perhaps be taken from \cite{billig_neeb_vector_field_cyclic_cohomology_parallelisable_manifolds}, where the authors have investigated the Lie algebra cohomology ($H^2_{\Lie}$, in particular) of the Lie algebra of all $C^{\infty}$-vector fields on a parallelisable smooth compact manifold $M$ with coefficients in the global section of $\Omega^p_M/d( \Omega^{p - 1}_M )$ (as we will see via theorem \ref{theorem: kassel_realisation}, $\z(\toroidal) \cong \Omega^1_{ \bbC[v^{\pm 1}, t^{\pm 1}]/\bbC }/d \bbC[v^{\pm 1}, t^{\pm 1}]$). That said, there is still much work to be done.

        Otherwise, we can continue pursuing the original goal of computing the classical limit of affine Yangians immediately after this thesis as well, and even though we have a somewhat detailed sketch of the proof at this point, there remain one outstanding technical difficulty. Namely, \textit{it is still not known whether or not $\rmY_{\hbar}(\hat{\g})$ is torsion-free as a $\bbC[\hbar]$-module for all $\g$}, even for those $\g$ for which $\rmY_{\hbar}(\hat{\g})$ is known to carry a topological bialgebra structure (see \cite[Section 5]{guay_nakajima_wendlandt_affine_yangian_coproduct}). When $\g$ is simply-laced, though, this is known (see \cite[Section 6]{guay_regelskis_wendlandt_affine_yangian_vertex_representations_and_PBW} and also \cite{yang_zhao_affine_yangian_PBW}), and it is expected to be true at least for all $\hat{\g} \not \cong \hat{\sl}_2(\bbC)$. Let us also remark, that whether or not the affine Yangian is torsion-free as a $\bbC[\hbar]$-module is the same as whether or not it admits a PBW basis, since torsion-freeness allows us to lift PBW bases of $\rmU(\toroidal^{\positive})$ to the affine Yangian. Once torsion-freeness over $\bbC[\hbar]$ is known for the affine Yangian, however, the rest should be more-or-less routine, save for some minor topological subtleties that one would have to pay attention to while constructing the topological Lie bialgebra structure on $\toroidal^{\positive}$. That said, for circumstantial reasons, we have decided to defer the task of fully fleshing out the schematic described in subsection \ref{subsection: history} to later works.  
        
        On a somewhat separate note, we would also like to comment, that in order to be able to precisely compare our $\gamma$-extended toroidal Lie algebras with the EALAs of Neher et al., some further effort will have to be spent on analysing the structure of $\gamma$-extended toroidal Lie algebras. For example, we may inquire into whether or not the theory of $\gamma$-extended toroidal Lie algebras can admit a reasonable notion of Cartan subalgebras, and whether or not these Lie algebras possess root space decompositions, etc.
        
        Finally, we would like to make note of the fact that historically, loop realisations have shown themselves to be very useful in practice, both when one seeks an understanding the structures of infinite-dimensional Lie algebras, as well as when one seeks applications of said Lie algebras (often to physics). For instance, they are useful for showing that all affine Kac-Moody algebras arise as \say{twists}\footnote{In the sense of \cite[Chapter 8]{kac_infinite_dimensional_lie_algebras}.} of the untwisted ones, and how these twists can be classified in terms of Galois cohomology (see \cite{pianzola_vanishing_of_H1_of_dedekind_rings} and its sequels\footnote{... and our thanks to A. Pianzola for letting us know of such results!}). Another example of how loop realisations are useful, this time in mathematical physics, is how they allow us to write down so-called \say{vertex representations}, i.e. representations with \say{vertex algebra} structures, which have proven themselves to be very valuable for studying infinite-dimensional Lie algebras; see, for example, \cite[Chapter 14]{kac_infinite_dimensional_lie_algebras} and \cite{berman_billig_szmigielski_VOAs_and_toroidal_lie_algebras}, and also \cite{guay_regelskis_wendlandt_affine_yangian_vertex_representations_and_PBW} for an example of how these vertex representations might still be useful even if one is only concerned with the original motivation of studying affine Yangians. This is a difficult topic, so for the sake of brevity and conciseness, we shall only be making this brief mention of it. In any event, for the reasons listed above, we believe that the objects of main interest in this thesis, which arise from a double-loop realisation, are worthy of attention for reasons beyond their original motivation that was discussed in subsection \ref{subsection: history} above.

        \newpage

        \section{The structure of the thesis}
    We would also like to provide the reader with a short reading guide for the thesis.
    
    In chapter \ref{chapter: kassel_UCEs}, we begin by recall some background information on perfect Lie algebras and on Lie algebra extensions, and then we will study UCEs in some detail, recalling in particular a realisation of Kassel, whereby centres of UCEs of current algebras (in the sense of definition \ref{def: current_algebras}) can be described in terms of algebraic differential $1$-forms modulo exact forms (cf. theorem \ref{theorem: kassel_realisation}); we will then end the chapter by analysing, using the aforementioned theorem, particular UCEs that will be of interest to us for the rest of the thesis (see, in particular, example \ref{example: toroidal_lie_algebras_centres}).

    Chapter \ref{chapter: yangian_EALAs} will begin with the construction of the technical ingredients necessary for defining $\gamma$-extended toroidal Lie algebras, and then of those Lie algebras themselves; the procedure will be as outlined in the previous section, and will be outlined in further details in the sections in chapter \ref{chapter: yangian_EALAs}; the main result of this portion of the chapter will be theorem \ref{theorem: yangian_extended_toroidal_lie_algebras_main_theorem}. Afterwards, we will be making some quick remarks about structure features of $\gamma$-extended toroidal Lie algebras, such as an identification of its centre and intriguingly, an homomorphic image of the Witt algebra inside $\der_{\gamma}(\bbC[v^{\pm 1}, t^{\pm 1}])$ (see propositions \ref{prop: centres_of_yangian_extended_toroidal_lie_algebras} and \ref{prop: a_copy_of_the_witt_algebra_inside_the_lie_algebra_of_yangian_div_zero_vector_fields}, respectively). Finally, we will attempt to provide an example of a $\gamma$-extended toroidal Lie algebra that is \textit{not} isomorphic to $\toroidal \rtimes \der_{\gamma}(\bbC[v^{\pm 1}, t^{\pm 1}])$ by analysing a particular Lie $2$-cocycle known from \cite{billig_energy_momentum_tensor}; this will be done through theorem \ref{theorem: billig_cocycle_main_theorem}.

    \newpage
    
    \chapter{Kassel's characterisation of UCEs of current algebras}
        \begin{abstract}
            The purpose of this chapter is to recall some relevant features of the theory of universal central extensions (UCEs) of so-called \say{perfect Lie algebras}. Of special interest to us is Kassel's realisation of UCEs of Lie algebras of the kind $\g \tensor_{\bbC} A$, where $\g$ is a finite-dimensional simple Lie algebra over $\bbC$, and $A$ is a commutative $\bbC$-algebra (cf. theorem \ref{theorem: kassel_realisation}). Along the way, we will also recall some features of the structures of finite-dimensional simple Lie algebras, as well as some generalities about extensions of Lie algebras.
            
            We require this theory in order to be able to explicitly compute bases for the centre of the UCE of $\g[v^{\pm 1}, t^{\pm 1}]$, which is necessary for constructing of \say{$\gamma$-extended toroidal Lie algebras} in chapter \ref{chapter: yangian_EALAs}. 
        \end{abstract}

        \minitoc

        \newpage
    
        \input{Chapters/Background/perfect_lie_algebras}
        
        \newpage
        
        \section{The Kassel realisation of UCEs of current Lie algebras}
    \subsection{A recollection of K\"ahler differentials}
        There are many perspectives on algebraic differential forms, but for our purposes, the following will be the easiest to use.
        \begin{definition}[Modules of K\"ahler differentials] \label{def: kahler_differentials}
            Let $k$ be a base commutative ring and let $A$ be a commutative $k$-algebra, defined by a multiplication map:
                $$\mu_{A/k}: A \tensor_k A \to A$$
            The $A$-module of K\"ahler differentials $\Omega^1_{A/k}$ relative to the ring map $k \to A$ is then given by:
                $$\Omega^1_{A/k} := I/I^2$$
            where $I := \ker \mu_{A/k}$. Elements of $\Omega^1_{A/k}$ are typically referred to as (differential) $1$-forms.
        \end{definition}
        \begin{remark}[Diffentials satify the Leibniz rule]
            Observe that the $A \tensor_k A$-ideal:
                $$I := \ker \mu_{A/k}$$
            is generated by elements of the form $1 \tensor f - f \tensor 1$, for all $f \in A$. We then see that:
                $$\Omega^1_{A/k} := I/I^2 \cong I \tensor_{A \tensor_k A} ( (A \tensor_k A)/I ) \cong I \tensor_{A \tensor_k A} A$$
            should we regard $A$ as a commutative $A \tensor_k A$-algebra; note that the last isomorphism holds thanks to the fact that the multiplication map $\mu_{A/k}: A \tensor_k A \to A$ is \textit{a priori} surjective. From this, one infers that there exists a canonical $k$-linear derivation:
                $$d: A \to \Omega^1_{A/k}$$
                $$f \mapsto 1 \tensor f - f \tensor 1$$
            Indeed, for every $f, g \in A$, the Leibniz rule is satisfied:
                $$g df + f dg = g(1 \tensor f - f \tensor 1) + f(1 \tensor g - g \tensor 1) = gf \tensor 1 - fg \tensor 1 = d(fg)$$
        \end{remark}
        The following well-known lemmas are very useful. A proof can be be found in any standard reference on general commutative algebra (cf. e.g. \cite[\href{https://stacks.math.columbia.edu/tag/00AO}{Tag 00AO}]{stacks}).
        \begin{lemma}[Universal property of modules of K\"ahler differentials]
            Let $k$ be a base commutative ring and let $A$ be a commutative $k$-algebra. Then, the $A$-module of K\"ahler differentials relative to the ring map $k \to A$ corepresents the functor of $k$-linear derivations from $A$, i.e. there is a natural isomorphism of functors $A\mod \to A\mod$ as follows:
                $$\Der_k(A, -) \cong \Hom_A(\Omega^1_{A/k}, -)$$
            Because of this, the category of $k$-linear derivations $d_M: A \to M$ admit an initial object, namely $d: A \to \Omega^1_{A/k}$. 
        \end{lemma}
        \begin{corollary}[$1$-forms are dual to derivations]
            $k$-linear derivations from $A$ to itself are dual to differential $1$-forms relative to $k \to A$ in the following manner:
                $$\Der_k(A) := \Der_k(A, A) \cong \Hom_A(\Omega^1_{A/k}, A)$$
        \end{corollary}
        \begin{lemma}[$1$-forms over polynomial algebras]
            \cite[\href{https://stacks.math.columbia.edu/tag/00RX}{Tag 00RX}]{stacks} Let $k$ be a commutative ring and fix some $n \in \Z_{\geq 0}$, and consider the canonical ring homomorphism $k \to k[v_1, ..., v_n]$. In this case, $\Omega^1_{[n]} := \Omega^1_{k[v_1, ..., v_n]/k}$ will be free and of finite rank $n$ as an $k[v_1, ..., v_n]$-module; in particular, it admits the set $\{dv_1, ..., dv_n\}$ as a $k[v_1, ..., v_n]$-linear basis.
        \end{lemma}
        \begin{corollary}[Derivations as differential operators]
            The $k$-module of $k$-linear derivations from $k[v_1, ..., v_n]$ to itself admit a particularly simple and useful description:
                $$\Der_k(k[v_1, ..., v_n]) \cong \bigoplus_{1 \leq i \leq n} k[v_1, ..., v_n] \del_{v_i}$$
        \end{corollary}

    \subsection{Centres of UCEs of current Lie algebras}
        \begin{convention} 
            From now on, we fix a finite-dimensional simple Lie algebra:
                $$\g$$
            over an algebraically closed field $k$ of characteristic $0$, equipped with a symmetric and non-degenerate invariant $k$-bilinear form $(-, -)_{\g}$. It is known that such a bilinear form is unique up to $k^{\x}$-multiples, so for all intents and purposes, it can be assumed to be the Killing form, though this assumption is not necessary. 
        \end{convention}

        \begin{convention}
            Let us fix a commutative $k$-algebra $A$.

            We shall be writing:
                $$\g_A := \g \tensor_k A$$
            and endow this $k$-vector space with the following Lie bracket:
                $$\forall x, y \in \g: \forall f, g \in A: [x f, y g]_{\g_A} := [x, y]_{\g} fg$$
        \end{convention}

        \begin{convention}
            Let $R \to S$ be a homomorphism of commutative rings. Then, let us write:
                $$\bar{\Omega}^1_{S/R} := \Omega^1_{S/R}/dS$$
            Note that this is only an $R$-module, not an $S$-module.
        \end{convention}
        
        \begin{theorem}[The Kassel realisation] \label{theorem: kassel_realisation}
            \cite[Corollary 3.5]{kassel_universal_central_extensions_of_lie_algebras} Let $k$ now an algebraically closed field of characteristic $0$ again, as in convention \ref{conv: a_fixed_finite_dimensional_simple_lie_algebra}. 

            For the perfect Lie $k$-algebra $\g_A$, we have that:
                $$\z(\uce(\g_A)) \cong \bar{\Omega}^1_{A/k}$$
        \end{theorem}
            \begin{proof}[Proof sketch]
                Kassel constructed in the proof of \cite[Theorem 3.3]{kassel_universal_central_extensions_of_lie_algebras} a $k$-linear map:
                    $$\e: \bigwedge^2 \g_A \to \bar{\Omega}^1_{A/k}$$
                by the formula:
                    $$\forall x, y \in \g, \forall f, g \in A: \e(x f, y g) := (x, y)_{\g} f \bar{d}g$$
                which can be shown - relying on the $\g$-invariance of the bilinear form $(-, -)_{\g}$ - to be an element of $H^2_{\Lie}(\g_A, k)$ and hence gives a central extension $\fraku$ of $\g_A$ by $\bar{\Omega}_{A/k}^1$, whose underlying $k$-vector space is:
                    $$\g_A \oplus \bar{\Omega}_{A/k}^1$$
                and whose Lie bracket is:
                    $$[-, -]_{\fraku} = [-, -]_{\g_A} + \e$$
                It is also not hard to see that there is a section map $\g_A \to \fraku$, implying that $\fraku$ is simply connected in the sense of definition \ref{def: simply_connected_lie_algebras}. Via proposition \ref{prop: UCEs_are_simply_connected}, we then see that:
                    $$\fraku \cong \uce(\g_A)$$
                concluding the proof.
            \end{proof}
        \begin{example}[UCEs of multiloop Lie algebras]
            Fix some $n \in \Z_{\geq 0}$.
        
            Consider the case:
                $$A := A_{[n]} := k[v_1^{\pm 1}, ..., v_n^{\pm 1}]$$
            (let $A_{[0]} := k$). Set:
                $$\g_{[n]} := \g \tensor_k A_{[n]}$$
                $$\tilde{\g}_{[n]} := \uce(\g_{[n]})$$
                $$\Omega^1_{[n]} := \Omega^1_{A_{[n]}/k}, \bar{\Omega}^1_{[n]} := \bar{\Omega}^1_{A_{[n]}/k}$$
            Much can be said about the centre $\bar{\Omega}^1_{[n]}$ of the UCE $\tilde{\g}_{[n]}$ of $\g_{[n]}$, especially in the cases where $n \leq 2$, where it is rather easy to provide an explicit basis for the $k$-vector space $\bar{\Omega}^1_{[n]}$. 

            We know that $\Omega^1_{[n]}$ is free and of rank $n$ on the set:
                $$\{dv_1, ..., dv_n\}$$
            This implies that $\bar{\Omega}^1_{[n]}$ is generated by elements:
                $$\bar{d}v_j$$
            that are subjected to the following relation:
                $$0 = \bar{d}( v_1^{m_1} ... v_n^{m_n} ) = \sum_{1 \leq j \leq n} m_j v_1^{m_1} ... v_j^{m_j - 1} ... v_n^{m_n} \bar{d}v_j$$
            From this, one infers that the elements:
                $$m_j^{-1} v_1^{m_1} ... v_j^{m_j - 1} ... v_n^{m_n} \bar{d}v_j$$
            form a basis for $\bar{\Omega}_{[n]}$ as a $k$-vector space.

            When $n = 0$ or $n = 1$, it is trivial to see that:
                $$\dim_k \bar{\Omega}^1_{[0]} \cong 0, \dim_k \bar{\Omega}^1_{[1]} = 1$$

            When $n = 2$, consider the following: $\bar{\Omega}^1_{[2]}$ now decomposes as a $k$-vector space in the following manner:
                $$\bar{\Omega}^1_{[2]} \cong ( \bigoplus_{(r, s) \in \Z^2} k K_{r, s}) \oplus k c_v \oplus k c_t$$
            wherein:
                $$
                    K_{r, s} :=
                    \begin{cases}
                        \text{$\frac1s v^{r - 1} t^s \bar{d}v$ if $(r, s) \in \Z \x (\Z \setminus \{0\})$}
                        \\
                        \text{$-\frac1r v^r t^{-1} \bar{d}t$ if $(r, s) \in (\Z \setminus \{0\}) \x \{0\}$}
                        \\
                        \text{$0$ if $(r, s) = (0, 0)$}
                    \end{cases}
                $$
                $$c_v := v^{-1} \bar{d}v, c_t := t^{-1} \bar{d}t$$
            In fact, any element of the form:
                $$v^m t^p \bar{d}(v^n t^q) \in \bar{\Omega}^1_{[2]}$$
            can be written in terms of the basis vectors $K_{r, s}, c_v, c_t$ in the following manner:
                $$v^m t^p \bar{d}(v^n t^q) = \delta_{(m, p) + (n, q), (0, 0)} ( n c_v + q c_t ) + (np - mq) K_{m + n, p + q}$$
        \end{example}
        \begin{remark}[The $\Z$-grading on $\tilde{\g}_{[2]}$] \label{remark: Z_gradings_on_toroidal_lie_algebras}
            If $k$ is an arbitrary commutative ring and $A$ is a $\Z$-graded commutative $k$-algebra, say:
                $$A := \bigoplus_{n \in \Z} A_n$$
            and if $\a$ is a perfect Lie algebra over $k$, then $\a_A$ will also be $\Z$-graded, specifically in the following manner:
                $$\a_A := \a \tensor_k A \cong \bigoplus_{n \in \Z} \a \tensor_k A_n$$
            and for convenience, let us write $\a_{A_n} := \a \tensor_k A_n$ for each $n \in \Z$. This grading on $\a_A$ actually extends to the whole of $\uce(\a_A)$. Because the $A$-module $\Omega^1_{A/k}$ is generated by the set:
                $$\{da\}_{a \in A}$$
            whose elements are subjected to the relations:
                $$\forall a, b \in A: d(ab) - a d(b) - d(a) b = 0$$
           there is an induced $\Z$-grading on $\Omega^1_{A/k}$ given by:
                $$\deg d(ab) = \deg a d(b) = \deg d(a) b = \deg a + \deg b - 1$$
            for all $a, b \in A$. Inside $\Omega^1_{A/k}$, now viewed as a $k$-module, one has the $k$-submodule $\im d$, which is also $\Z$-graded: the grading is given like above, namely:
                $$\deg d(a) = \deg a - 1$$
            This $\Z$-grading induces another one on $\bar{\Omega}^1_{A/k}$, given by:
                $$\deg \bar{d}(ab) = \deg a \bar{d}(b) = \deg \bar{d}(a) b = \deg a + \deg b - 1$$
            for all $a, b \in A$.

            Now, let us focus once more on the case:
                $$A := A_{[2]}$$
            wherein the relevant $\Z$-grading is given by:
                $$\deg v := 0, \deg t := 1$$
            Since we know that the basis elements of $\bar{\Omega}^1_{[2]}$ are given by:
                $$
                    K_{r, s} :=
                    \begin{cases}
                        \text{$\frac1s v^{r - 1} t^s \bar{d}v$ if $(r, s) \in \Z \x (\Z \setminus \{0\})$}
                        \\
                        \text{$-\frac1r v^r t^{-1} \bar{d}t$ if $(r, s) \in (\Z \setminus \{0\}) \x \{0\}$}
                        \\
                        \text{$0$ if $(r, s) = (0, 0)$}
                    \end{cases}
                $$
                $$c_v := v^{-1} \bar{d}v, c_t := t^{-1} \bar{d}t$$
            (cf. \textit{loc. cit.}) their respective degrees with respect to the $\Z$-grading on $\bar{\Omega}_{[2]} \cong \bar{\Omega}^1_{[2]}$ are:
                $$
                    \deg K_{r, s} =
                    \begin{cases}
                        \text{$s - 1$ if $(r, s) \in \Z \x (\Z \setminus \{0\})$}
                        \\
                        \text{$-1$ if $(r, s) \in (\Z \setminus \{0\}) \x \{0\}$}
                        \\
                        \text{$0$ if $(r, s) = (0, 0)$}
                    \end{cases}
                $$
                $$\deg c_v = \deg c_t = -1$$
        \end{remark}
        
        \newpage
        
        \input{Chapters/Background/examples_of_kassel_uce_centre}

    \newpage

    \chapter{\texorpdfstring{$\gamma$}{}-extended toroidal Lie algebras} \label{chapter: yangian_EALAs}
        \begin{abstract}
            In this chapter, we attempt to construct \say{$\gamma$-extended toroidal Lie algebras}, which are to be extensions $\extendedtoroidal$ of a certain Lie algebra of derivations by the toroidal Lie algebra $\toroidal := \uce(\g[v^{\pm 1}, t^{\pm 1}])$. This is to rectify the problem whereby any invariant symmetric bilinear form on $\toroidal$ is necessarily degenerate.
        \end{abstract}

        \minitoc

        \newpage
    
        \input{Chapters/Yangian EALAs/setup_yangian_EALAs_thesis_version}

        \newpage

        \input{Chapters/Yangian EALAs/construction_of_yangian_EALAs_thesis_version}

        \newpage

        \input{Chapters/Yangian EALAs/structure_of_yangian_EALAs_thesis_version}

        \newpage

        \input{Chapters/Yangian EALAs/examples_of_yangian_EALAs_thesis_version}

    \newpage

    \begin{appendices}
        \chapter{Technical appendices}
            \input{Chapters/Appendices/appendix_lie_algebra_cohomology}
    \end{appendices}

    \newpage

    \printbibliography
    \addcontentsline{toc}{chapter}{\textbf{Bibliography}}

\end{document}