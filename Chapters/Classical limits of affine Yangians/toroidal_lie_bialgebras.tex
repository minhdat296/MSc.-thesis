\section{Affine Yangians as quantisations}
    \begin{convention}
        Let $V$ be a vector space.
    
        As a shorthand, we will be writing:
            $$\bar{\Delta}(X) := X \tensor 1 + 1 \tensor X$$
        and this will be understood to be an element of $\rmT(V)^{\tensor 2}$. If we have:
            $$X := \sum_{i, j} X_i \tensor X_j \in \rmT(V)^{\tensor 2}$$
        then we will be writing:
            $$X_{12} := \sum_{i, j} X_i \tensor X_j \tensor 1 \in \rmT(V)^{\tensor 3}$$
            $$X_{23} := \sum_{i, j} 1 \tensor X_i \tensor X_j \in \rmT(V)^{\tensor 3}$$
            $$X_{13} := \sum_{i, j} X_i \tensor 1 \tensor X_j \in \rmT(V)^{\tensor 3}$$
        and likewise for the other permutations. If $V$ is a Lie algebra then instead of the tensor algebra, we will typically think of these elements as living in tensor powers of the universal enveloping algebra.
    \end{convention}

    \begin{convention}[Graded duals]
        If $Z$ is an abelian group and:
            $$V := \bigoplus_{d \in Z} V_d$$
        is a $Z$-graded vector space, then we will be writing:
            $$V^{\star} := \bigoplus_{d \in Z} V_d^*$$
        for its $Z$-graded dual.
    \end{convention}

    \begin{convention}[Casimir tensor] \label{conv: casimir_tensor}
        We will be writing $\sfr_{\g}$ to mean the element of $\g \tensor_k \g^*$ that is the preimage of $\id_{\g} \in \End_k(\g)$ under the vector space isomorphism $\g \tensor_k \g^* \xrightarrow[]{\cong} \End_k(\g)$ given by $x \tensor \varphi \mapsto (-, \varphi)_{\g} x$. The element $\sfr_{\g}$ is typically referred to as the \textbf{Casimir tensor} of $\g$ (with respect to $(-, -)_{\g}$). 
    \end{convention}

    \subsection{Manin triples and quantisations of Lie bialgebras} \label{subsection: manin_triples_and_quantisations_of_lie_bialgebras}
        \begin{remark}
            Technically speaking, we do not need our field $k$ to be algebraically closed for the abstract machineries presented in this subsection to be true. This hypothesis is only needed for some of the examples. 
        \end{remark}
    
        \begin{definition}[Lie bialgebras] \label{def: lie_bialgebras}
            Let $\a$ be a Lie algebra over $k$ equipped with a $k$-linear map:
                $$\delta: \a \to \a \tensor_k \a$$
            $\a$ will then be a \textbf{Lie bialgebra} (over $k$) with \textbf{Lie cobracket} $\delta$ if the following two conditions are satisfied:
            \begin{itemize}
                \item Firstly, we require that the map:
                    $$\delta^*: (\a \tensor_k \a)^* \to \a^*$$
                induces a Lie bracket $\a^* \tensor_k \a^* \to \a^*$ on the $k$-vector space $\a^*$. Equivalently, this is saying $\delta: \a \to \a \tensor_k \a$ must firstly be \textbf{Lie cobracket}, i.e. it is to be skew-symmetric (i.e. its codomain is actually $\a \wedge \a$) and to satisfy the \textbf{co-Jacobi identity}:
                    $$( (1 \: 2 \: 3) + (2 \: 3 \: 1) + (3 \: 1 \: 2) ) \circ (\delta \circ \id_{\a}) \circ \delta = 0$$
                \item Secondly, we insist that the Lie cobracket $\delta$ is a Lie $1$-cocycle\footnote{In cohomological terms, one can write $\delta \in H^1_{\Lie}(\a, \a \tensor_k \a)$.} of $\a$ with coefficients in $\a \tensor_k \a$, which is to say that the following identity is to hold in $\rmU(\a) \tensor_k \rmU(\a)$:
                    $$\delta( [x, y] ) = [\bar{\Delta}(x), \delta(y)] + [\delta(x), \bar{\Delta}(y)]$$
                for all $x, y \in \a$.
            \end{itemize}
            If $(\a, [-, -], \delta)$ and $(\a', [-, -]', \delta')$ are Lie bialgebras then a homomorphism between them will be a homomorphism between the underlying Lie algebras $\phi: (\a, [-, -]) \to (\a', [-, -]')$ such that:
                $$\phi^{\tensor 2} \circ \delta = \delta' \circ \phi$$
        \end{definition}
        \begin{definition}[Graded Lie bialgebras] \label{def: graded_lie_bialgberas}
            Let $(\a, [-, -], \delta)$ be a Lie bialgebra over $k$, whose underlying vector space is graded by some abelian group $Z$, i.e.:
                $$\a := \bigoplus_{d \in Z} \a_d$$
            in such a way that the graded components $\a_d$ are all finite-dimensional as vector spaces over $k$. This Lie bialgebra is then said to be \textbf{$Z$-graded} if and only if the following conditions are satisfied:
            \begin{itemize}
                \item $[-, -]$ is a graded Lie bracket, i.e.:
                    $$[\a_m, \a_n] \subseteq \a_{m + n}$$
                for all $m, n \in Z$.
                \item $\delta$ is a graded Lie cobracket, in the sense that for any $d \in Z$, one has that:
                    $$\delta(\a_d) \subseteq \bigoplus_{m + n = d} \a_m \tensor_k \a_n$$
            \end{itemize}
        \end{definition}
        \begin{remark}[Duals of Lie bialgebras and Drinfeld's double construction] \label{remark: drinfeld_doubles}
            Let:
                $$(\a, [-, -]_{\a}, \delta_{\a})$$
            be a Lie bialgebra over $k$.
        
            Suppose firstly that $\a$ is a finite-dimensional. Then clearly, the full linear dual $\a^*$ will also be a Lie bialgebra, whose Lie cobracket:
                $$\delta_{\a^*}: \a^* \to \a^* \tensor_k \a^*$$
            is induced by the dual $[-, -]_{\a}^*: \a^* \to (\a \tensor_k \a)^*$ of the Lie bracket on $\a$. Furthermore, we note that there is no natural non-discrete topology on the linear dual of a finite-dimensional vector space.

            If $\a$ is an infinite-dimensional Lie bialgebra, on the other hand, then the full linear dual $\a^*$ will not generally be a Lie bialgebra but rather a \textbf{topological Lie bialgebra}, in the sense that the codomain of its Lie cobracket $\delta_{\a^*}$ may not $\a^* \tensor_k \a^*$, but rather in some appropriate topological completion $\a^* \hattensor_k \a^*$.

            In either event, the $k$-vector space:
                $$\frakDr(\a) := \a \oplus \a^*$$
            can be made into a Lie algebra over $k$, with Lie bracket given by\footnote{Note that $\frakDr(\a)^{\tensor 2} \cong \a^{\tensor 2} \oplus (\a^*)^{\tensor 2} \oplus (\a^* \tensor_k \a \oplus \a \tensor_k \a^*)$}:
                $$[-, -]_{\frakDr(\a)} := [-, -]_{\a} \oplus [-, -]_{\a^*} \oplus ( [-, -]_{\a} \circ (S_{\a^*} \tensor \id_{\a}) \oplus [-, -]_{\a^*}^{\op} \circ (\id_{\a^*} \tensor S_{\a}) )$$
            where:
                $$[-, -]_{\a^*}$$
            is the Lie bracket on $\a^*$ induced by $\delta_{\a}^*$ (well-defined since $\a$ is a Lie bialgebra by hypothesis) with opposite $[-, -]_{\a^*}^{\op} := -[-, -]_{\a^*}$, and:
                $$S_{\a}, S_{\a^*}$$
            denote the antipodes on the universal enveloping algebras of $\a, \a^*$ respectively. A natural non-degenerate and symmetric $k$-bilinear form $(-, -)_{\frakDr(\a)}$ on $\frakDr(\a)$ which is invariant with respect to $[-, -]_{\frakDr(\a)}$ can then be constructed by declaring that:
                $$(x + \varphi, y + \psi)_{\frakDr(\a)} := \psi(x) + \varphi(y)$$
            for all $x, y \in \a$ and all $\varphi, \psi \in \a^*$. 

            A topological Lie cobracket on $\frakDr(\a)$ can also be given as:
                $$\delta_{\frakDr(\a)} := \delta_{\a} \oplus \delta_{\a^*}^{\cop}$$
            with $\delta_{\a^*}^{\cop} := -\delta_{\a^*}$ being the opposite Lie cobracket on $\a^*$, and it can be verified that this is a $1$-cocycle of $\frakDr(\a)$ with values in $\frakDr(\a) \tensor_k \frakDr(\a)$, thus making $\frakDr(\a)$ a Lie bialgebra. This Lie bialgebra is typically called the \textbf{Drinfeld double}\footnote{Sometimes also called the \textbf{classical double}.} of $\a$ (or equivalently, of $\a^*$).
        \end{remark}
        \begin{remark}[Coboundary Lie algebras and classical R-matrices] \label{remark: classical_R_matrices}
            Let $\a$ be a Lie algebra.
        
            Let us note here that there is an alternative way to construct a Lie bialgebra structure on the vector space:
                $$\frakDr(\a) := \a \oplus \a^*$$
            as follows, which turns out to be more useful in practice (cf. e.g. example \ref{example: finite_type_yangian_manin_triple} and corollary \ref{coro: extended_toroidal_lie_bialgebras}). Namely, one makes use of the canonical elements:
                $$\sfr_{\a} \in \a \tensor_k \a^*, \sfr_{\a^*} \in \a^* \tensor_k \a$$
            which are to be the preimage of $\id_{\a} \in \End_k(\a), \id_{\a^*} \in \End_k(\a^*)$ under the canonical maps\footnote{Note that these maps are both injective, precisely because $(-, -)_{\frakDr(\a)}$ is non-degenerate by construction\footnote{Note that we are not making use of invariance here, and hence we do not need to assume that $\a$ is a Lie bialgebra from the start (this assumption is needed for the construction of the Lie bracket on $\frakDr(\a)$).}, and hence $\sfr_{\a}$ is well-defined.} $\a \tensor_k \a^* \to \End_k(\a), \a^* \tensor_k \a \to \End_k(\a^*)$ given by $x \tensor \varphi \mapsto (-, \varphi)_{\frakDr(\a)} x$ and by $x \tensor \varphi \mapsto (x, -)_{\frakDr(\a)} \varphi$, respectively, for all $x \in \a$ and all $\varphi \in \a^*$. A theorem of Drinfeld asserts that:
            \begin{enumerate}
                \item if:
                    $$\delta_{\a} := [\bar{\Delta}, \sfr_{\a}], \delta_{\a^*} := [\bar{\Delta}, \sfr_{\a^*}]$$
                then $\delta_{\a}$ and $\delta_{\a^*}$ will be Lie cobrackets on $\a$ and on $\a^*$ respectively, and
                \item if furthermore, the so-called \textbf{classical Yang-Baxter equation}:
                    $$[\sfr_{12}, \sfr_{13}] + [\sfr_{12}, \sfr_{23}] + [\sfr_{13}, \sfr_{13}]$$
                (where $\sfr := \sfr_{\a} - \sfr_{\a^*}$) is $\frakDr(\a)$-invariant, then:
                    $$\delta_{\frakDr(\a)} = \delta_{\a} \oplus \delta_{\a^*}^{\cop} = [\bar{\Delta}, \sfr]$$
                will be a Lie bialgebra structure on $\frakDr(\a)$.
            \end{enumerate}
            Consequently, $\delta_{\a}$ and $\delta_{\a^*}$ will be Lie bialgebra structures on $\a$ and $\a^*$, per the discussions in remark \ref{remark: drinfeld_doubles}.
        \end{remark}
        \begin{definition}[Classical R-matrices] \label{def: classical_R_matrices}
            Let $\a$ be a Lie bialgebra equipped with a \textit{non-degenerate} and \textit{invariant} symmetric bilinear form $(-, -)$. Then, the canonical element $\sfr_{\a} \in \a \tensor_k \a^*$ with respect to $(-, -)$ as in remark \ref{remark: classical_R_matrices} shall be referred to as the \textbf{classical R-matrix} of $\a$.
        \end{definition}
        \begin{definition}[(Quasi-)triangular Lie bialgebras] \label{def: (quasi)_triangular_lie_bialgebras}
            \todo{Will we need this ?}
        \end{definition}
        \begin{remark}[Graded Drinfeld doubles] \label{remark: graded_drinfeld_doubles}
            A graded analogue of Drinfeld doubles is also available for graded Lie bialgebras:
                $$(\a, [-, -]_{\a}, \delta_{\a})$$
            with finite-dimensional graded components. When $\a$ is finite-dimensional, there is no difference from the construction given in remark \ref{remark: drinfeld_doubles}. When $\a$ is infinite-dimensional, instead of considering full linear duals as in remark \ref{remark: drinfeld_doubles}, one considers graded duals. The rest can then be carried out in the same manner, yielding a graded Lie bialgebra structure on:
                $$\frakDr(\a) := \a \oplus \a^{\star}$$
        \end{remark}
        Even though we use the same notation for Drinfeld doubles of ungraded and graded Lie bialgebras, how the underlying vector space is given should be clear from context. We will specify if needed.
        
        The construction of (graded) Drinfeld doubles leads to the following notion of \say{Manin triples}, which in some ways is easier to work with than Lie bialgebras. We do not lose information by passing to these Manin triples, however, since each of them gives rise to a Lie bialgebra and \textit{vice versa}.
        \begin{definition}[Manin triples] \label{def: manin_triples}
            A \textbf{(graded) Manin triple} is the data of a triple of (graded) Lie algebras:
                $$(\a, \a^+, \a^-)$$
            as well as a non-degenerate and invariant symmetric bilinear form $(-, -)$ satisfying the following conditions:
            \begin{itemize}
                \item Either the Lie algebras $\a, \a^+, \a^-$ are finite-dimensional, or the graded components of the Lie algebras $\a, \a^+, \a^-$ are finite-dimensional.
                \item $\a \cong \a^+ \oplus \a^-$.
                \item The bilinear form $(-, -)$ pairs the Lie subalgebras $\a^{\pm}$ isotropically, i.e. $(\a^{\pm}, \a^{\pm}) = 0$. 
                \item Via $(-, -)$, one gets an identification $\a^- \cong (\a^+)^*$; respectively, to obtain a graded Manin triple, we require that $\a^- \cong (\a^+)^{\star}$ via $(-, -)$.
            \end{itemize}
        \end{definition}
        \begin{lemma}[Lie bialgebras from Manin triples] \label{lemma: lie_bialgebras_from_manin_triples}
            \begin{enumerate}
                \item \cite[Proposition 1.3.4 and Lemma 1.3.5]{chari_pressley_quantum_groups} There is a bijective function from the set of finite-dimensional Lie bialgebras to the set of finite-dimensional Manin triples. This function maps a given Lie bialgebra $(\a, [-, -], \delta)$ to the Manin triple $(\frakDr(\a), \a, \a^*)$.
                \item Let $Z$ be an abelian group. 

                There is a bijective function from the set of $Z$-graded Lie bialgebras with finite-dimensional graded components to the set of $Z$-graded Manin triples $(\p, \p^+, \p^-)$ where each of $\p, \p^+, \p^-$ has finite-dimensional graded components. This function maps such a $Z$-graded Lie bialgebra $(\a, [-, -], \delta)$ to the Manin triple $(\frakDr(\a), \a, \a^*)$.
            \end{enumerate}
        \end{lemma}
        \begin{definition}[Lie sub-bialgebras and Lie coideals] \label{def: lie_sub_bialgebras_and_lie_coideals}
            Let $(\a, [-, -], \delta)$ be a Lie bialgebra.
            \begin{itemize}
                \item A \textbf{Lie sub-bialgebra} of $\a$ is then determined by an injective Lie bialgebra homomorphism $\b \to \a$. 
                \item A \textbf{Lie coideal} therein is then a vector subspace $\b \subseteq \a$ such that:
                    $$\delta(\b) \subseteq \a \tensor_k \b \oplus \b \tensor_k \a$$
                or equivalently, if $\b$ is a Lie sub-bialgebra such that $\a/\b$ inherits a Lie bialgebra structure from the one on $\a$.
            \end{itemize}
        \end{definition}
        \begin{example}[The Kac-Moody Manin triple] \label{example: kac_moody_manin_triple}
            
        \end{example}
        \begin{example}[The Yangian Manin triple] \label{example: finite_type_yangian_manin_triple}
            There is a $\Z$-graded Manin triple:
                $$( \g[t^{\pm 1}], \g[t], t^{-1}\g[t^{-1}] )$$
            wherein $\g[t^{\pm 1}]$ is equipped with the following \textit{a priori} invariant inner product, given for all $x, y \in \g$ and all $p, q \in \Z$:
                $$(x t^p, y t^q)_{\g[t^{\pm 1}]} := (x, y)_{\g} \delta_{p + q, -1}$$
            Corresponding to this Manin triple is a topological Lie bialgebra structure on $\g[t]$, wherein the Lie cobracket:
                $$\delta: \g[t] \to \g[t_1] \hattensor_{k} \g[t_2]$$
            is given by:
                $$\delta(X)(t_1, t_2) = [ X(t_1) \tensor 1 + 1 \tensor X(t_2), \sfr_{\g} \frac{1}{t_2 - t_1} ]$$
            for all $X(t) \in \g[t]$, with $\frac{1}{t_2 - t_1}$ being understood as a shorthand for a geometric series.

            In this situation, we have that:
                $$\g[t^{\pm 1}] \cong \frakDr(\g[t])$$
            or equivalently, it is the Drinfeld double of $t^{-1}\g[t]$, with the Lie bialgebra structure graded-dual to that on $\g[t]$. 
        \end{example}

        \begin{definition}[Quantisations] \label{def: quantisations}
            Let $\a$ be a Lie bialgebra over $k$, say with Lie cobracket $\delta$. A \textbf{(graded) quantisation} of $\a$ is a Hopf $k[\![\hbar]\!]$-bialgebra $Y_{\hbar}$, say with coproduct $\Delta_{\hbar}$, such that:
            \begin{itemize}
                \item $Y_{\hbar}$ is a (graded) flat deformation\footnote{In the sense of definition \ref{def: graded_and_PBW_deformations}.} of $\rmU(\a)$, and
                \item For any $x \in \a$, one has that:
                    $$\delta(x) \equiv \frac{1}{\hbar}(\Delta_{\hbar} - \Delta_{\hbar}^{\cop})(x) \pmod{\hbar}$$
            \end{itemize}
            We say also, that $(\a, \delta)$ is the \textbf{classical limit} of $(Y_{\hbar}, \Delta_{\hbar})$. 
        \end{definition}
        \begin{remark}
            The question of existence and uniqueness of quantisations of Lie bialgebras is a difficult one. We refer the reader to \cite{etingof_kazhdan_quantisation_1} and its sequels for details.
        \end{remark}

        Let us now consider some examples of (non-cocommutative) Hopf algebras that quantise the Lie bialgebras constructed in examples \ref{example: kac_moody_manin_triple} and \ref{example: finite_type_yangian_manin_triple}.
        \begin{example}[Quantised enveloping algebras] \label{example: finite_type_QUEs}
            
        \end{example}
        \begin{example}[Finite-type Yangians] \label{example: finite_type_yangians}
            Set:
                $$n := \dim_k \g$$
        
            Interestingly, the finite-type formal Yangian:
                $$\rmY_{\hbar}(\g)$$
            admits a presentation so that it is isomorphic to the $k[\hbar]$-algebra $Y$ generated by the set:
                $$\{ x_{\lambda}, y_{\lambda} \}_{1 \leq \lambda \leq n}$$
            whose elements subjected to the following relations:
                $$[ x_{\lambda}, x_{\mu} ] = \sum_{1 \leq \lambda \leq n} c_{\lambda \mu \nu} x_{\nu}, [ x_{\lambda}, y_{\mu} ] = \sum_{1 \leq \lambda \leq n} c_{\lambda \mu \nu} y_{\nu}$$
                $$[ y_{\lambda}, [y_{\mu}, x_{\nu}] ] - [ x_{\lambda}, [y_{\mu}, y_{\nu}] ] = \hbar^2 \sum_{1 \leq \alpha, \beta, \gamma \leq n} a_{\lambda \mu \nu \alpha \beta \gamma} \{ x_{\alpha}, x_{\beta}, x_{\gamma} \}$$
                $$[ [y_{\lambda}, y_{\mu}], [x_r, x_s] ] + [ [y_r, y_s], [x_{\lambda}, x_{\mu}] ] = \hbar^2 \sum_{1 \leq \alpha, \beta, \gamma, \nu \leq n} ( a_{\lambda \mu \nu \alpha \beta \gamma} c_{r s \nu} + a_{r s \nu \alpha \beta \gamma} c_{\lambda \mu \nu} ) \{ x_{\alpha}, x_{\beta}, x_{\gamma} \}$$
            wherein $c_{\cdot \cdot \cdot}$ are the structural constants of $\g$, and:
                $$a_{\lambda \mu \nu \alpha \beta \gamma} = \frac{1}{24} \sum_{1 \leq i, j, k \leq n} c_{\lambda \alpha i} c_{\mu \beta j} c_{\nu \gamma k} c_{i j k}$$
            and we set:
                $$\deg x_{\lambda} := 0, \deg y_{\lambda} := 1$$
            for all $1 \leq \lambda \leq n$.
            
            Denote the isomorphism in question by $\varphi: Y \to \rmY_{\hbar}(\g)$. It is given by:
                $$\varphi(h_i) = d_{ii}^{-1} H_{i, 0}, \varphi(h_i t) = d_{ii}^{-1} H_{i, 0} + \varphi(v_i)$$
                $$\varphi(x_i^{\pm}) = d_{ii}^{-1} E_{i, 0}^{\pm}, \varphi(h_i t) = d_{ii}^{-1} H_{i, 0} + \varphi(w_i^{\pm})$$
            wherein:
                $$v_i := -\frac12 d_{ii} h_i^2 + \frac14 \sum_{\alpha \in \Phi^+} \height(\alpha)^2 d_{ii}^{-1} \alpha(h_i) \{x_{\alpha}^+, x_{\alpha}^-\}$$
                $$w_i^{\pm} := -\frac12 d_{ii} \{x_i^{\pm}, h_i\} + \frac14 \sum_{\alpha \in \Phi^+} \height(\alpha)^2 d_{ii}^{-1} \alpha(h_i) \{[x_i^{\pm}, x_{\alpha}^{\pm}], x_{\alpha}^{\mp}\}$$
            with choices of roots vectors $x_{\alpha}^{\pm} \in \g_{\pm \alpha}$ such that $(x_{\alpha}^-, x_{\alpha}^+)_{\g} = 1$. One notes also that $\varphi$ respects the $\Z_{\geq 0}$-grading on both algebras.
        
            As a consequence, there is a $\Z_{\geq 0}$-graded Hopf $k[\hbar]$-algebra structure:
                $$(\Delta_{\hbar}, S_{\hbar}, \e_{\hbar})$$
            on $\rmY_{\hbar}(\g)$ given by:
                $$\Delta_{\hbar}(x_{\lambda}) := \bar{\Delta}(x_{\lambda}), \Delta_{\hbar}(y_{\lambda}) := \bar{\Delta}(y_{\lambda}) + \frac12 \hbar [x_{\lambda} \tensor 1, \sfr_{\g}]$$
                $$S_{\hbar}(x_{\lambda}) = -x_{\lambda}, S_{\hbar}(y_{\lambda}) = -y_{\lambda} + \frac14 c_{} x_{\lambda}$$
                $$\e_{\hbar}(x_{\lambda}) = \e_{\hbar}(y_{\lambda}) = 0$$
            with $c$ being the eigenvalue of $\ad(\sfr_{\g})$. Its classical limit is the Lie bialgebra structure on $\g[t]$ as constructed in example \ref{example: finite_type_yangian_manin_triple}, and so the Hopf bialgebra $\rmY_{\hbar}(\g)$ is a quantisation of the current Lie bialgebra $\g[t]$; this can be proven by setting $\hbar = 0$, which degenerates $\rmY_{\hbar}(\g)$ to $\rmU(\g[t])$ with its usual Hopf structure. Also, by specialising to $\hbar = 1$, one obtains a $\Z_{\geq 0}$-filtered Hopf $k$-bialgebra structure $(\Delta_1, S_1, \e_1)$ on $\rmY(\g)$.
        \end{example}
        By cohomological arguments, one can also show that both of these quantisations are unique, but we will not go into those details.
        
        \todo[inline]{Not done}

        \todo[inline]{Write about how affine Yangians are not exactly quantisations.}

    \subsection{(Extended) toroidal Lie bialgebras}
        \begin{theorem} \label{theorem: extended_toroidal_manin_triples}
            There is a complete topological Manin triple:
                $$(\extendedtoroidal, \extendedtoroidal^{\positive}, \extendedtoroidal^{\negative})$$
            wherein $\extendedtoroidal$ is equipped with the non-degenerate invariant inner product $(-, -)_{\extendedtoroidal}$ (cf. convention \ref{conv: orthogonal_complement_of_toroidal_centres}).
        \end{theorem}
            \begin{proof}
                We know from lemma \ref{lemma: positive/negative_extended_toroidal_lie_algebras} that $\extendedtoroidal^{\positive/\negative}$ are Lie subalgebras of $\extendedtoroidal$, so it now remains to show that $(-, -)_{\extendedtoroidal}$ pairs the subalgebras $\extendedtoroidal^{\positive/\negative}$ isotropically, but this is true entirely due to how this invariant bilinear form was constructed in convention \ref{conv: orthogonal_complement_of_toroidal_centres}.
            \end{proof}
        \begin{corollary}[Lie cobracket on $\extendedtoroidal^{\positive}$] \label{coro: extended_toroidal_lie_bialgebras}
            On the extended toroidal Lie algebra $\extendedtoroidal^{\positive}$, there is a continuous Lie cobracket\footnote{Note the completion!}, making $\extendedtoroidal^{\positive}$ a complete topological Lie bialgebra:
                $$\delta_{\extendedtoroidal^{\positive}}: \extendedtoroidal^{\positive} \to \extendedtoroidal^{\positive} \hattensor_k \extendedtoroidal^{\positive}$$
            given for any $X \in \extendedtoroidal^{\positive}$ by the following formula (cf. \cite{etingof_kazhdan_quantisation_1}):
                $$\delta_{\extendedtoroidal^{\positive}}(X) = [ X \tensor 1 + 1 \tensor X, \sfr_{\extendedtoroidal^{\positive}} ]$$
            wherein:
                $$\sfr_{\extendedtoroidal^{\positive}} := \sfr_{\g} + \sfr_{\z_{[2]}^{\positive}} + \sfr_{\d_{[2]}^{\positive}} \in \extendedtoroidal^{\positive} \hattensor_k \extendedtoroidal^{\negative}$$
            with:
                $$\sfr_{\g_{[2]}^{\positive}} \in \g_{[2]}^{\positive} \hattensor_k \g_{[2]}^{\negative} := \sfr_{\g} v_2 \1(v_1, v_2) \1(t_1, t_2)$$
            and\footnote{Note how we are simply summing over tensor products of dual basis elements.} $\sfr_{\z_{[2]}^{\positive}} \in \z_{[2]}^{\positive} \hattensor_k \d_{[2]}^{\positive}$ and $\sfr_{\d_{[2]}^{\positive}} \in \d_{[2]}^{\positive} \hattensor_k \z_{[2]}^{\positive}$ being given by the following formulae:
                $$\sfr_{\z_{[2]}^{\positive}} := \sum_{(r, s) \in \Z \x \Z_{> 0}} K_{r, s} \tensor D_{r, s} + c_v \tensor D_v$$
                $$\sfr_{\d_{[2]}^{\positive}} := \sum_{(r, s) \in \Z \x \Z_{\leq 0}} D_{r, s} \tensor K_{r, s} + D_t \tensor c_t$$
        \end{corollary}
    
        \begin{convention}[Formal Dirac distributions] \label{conv: formal_dirac_distributions}
            We will be using the following shorthands:
                $$\1(z, w) = \sum_{m \in \Z} z^m w^{-m - 1}$$
                $$\1^+(z, w) = \sum_{m \in \Z_{\geq 0}} z^m w^{-m - 1}$$
            as opposed to the usual $\delta$ notation, in order to avoid confusion with Lie cobrackets.
        \end{convention}

        \begin{remark}[Total degrees of \say{Yangian} canonical elements] \label{remark: total_degrees_of_classical_yangian_R_matrices}
            One property of the R-matrix $\sfr_{\extendedtoroidal^{\positive}}$ from corollary \ref{coro: extended_toroidal_lie_bialgebras} that will help simplify some computations later on (see the proof of theorem \ref{theorem: toroidal_lie_bialgebras}) is that they are of total degree $-1$. 

            Recall that if $V := \bigoplus_{m \in \Z} V_m, W := \bigoplus_{n \in \Z} W_n$ are $\Z$-graded vector spaces then for any $d \in \Z$, we have that:
                $$(V \tensor_k W)_d \cong \bigoplus_{m + n = d} V_m \tensor_k W_n$$
                
            If we now take $V = W = \rmU(\toroidal)$ then the claim from above would read:
                $$\sfr_{\toroidal^{\positive}} \in ( \rmU(\toroidal^{\positive}) \tensor_k \rmU(\toroidal^{\negative}) )_{-1}$$
            with the $\Z$-grading on $\toroidal^{\positive/\negative}$ (and hence on $\rmU(\toroidal^{\positive/\negative})$) being the one on the second variable $t$ (cf. remark \ref{remark: Z_gradings_on_toroidal_lie_algebras}), and actually, this is entirely due to:
                $$\sfr_{\g_{[2]}^{\positive}} \in ( \rmU(\g_{[2]}^{\positive}) \tensor_k \rmU(\g_{[2]}^{\negative}) )_{-1}$$

            What this means for us is that, should we have $X \in \toroidal^{\positive}$ such that:
                $$\deg X \leq 0$$
            then it will automatically be the case that:
                $$\delta_{\toroidal^{\positive}}(X) = 0$$
        \end{remark}
        
        We are now finally able to put a Lie cobracket on the toroidal Lie algebra $\toroidal^{\positive}$, compatible with the Lie bracket thereon in a manner that produces a Lie bialgebra structure. This Lie bialgebra structure is the classical limit of the coproduct on the formal Yangian:
            $$\rmY_{\hbar}(\hat{\g}) := \rmY_{\hbar}(\hat{\g})$$
        associated to the affine Kac-Moody algebra $\hat{\g}$. 
        \begin{theorem}[Toroidal Lie bialgebras] \label{theorem: toroidal_lie_bialgebras}
            Assume convention \ref{conv: a_fixed_untwisted_affine_kac_moody_algebra} and let us abbreviate:
                $$\hat{\delta}^{\positive} := \delta_{\extendedtoroidal^{\positive}}$$
            with $\delta_{\extendedtoroidal^{\positive}}$ as in corollary \ref{coro: extended_toroidal_lie_bialgebras}. Let:
                $$\tilde{\delta}^{\positive} := \hat{\delta}^{\positive}|_{\toroidal}$$
            Then $(\toroidal^{\positive}, \tilde{\delta}^{\positive})$ will be a complete topological Lie sub-bialgebra of $(\extendedtoroidal^{\positive}, \hat{\delta}^{\positive})$ as given in corollary \ref{coro: extended_toroidal_lie_bialgebras}. Thanks to corollary \ref{coro: levendorskii_presentation_for_central_extensions_of_multiloop_algebras}, we know that it is enough to specify how $\tilde{\delta}^{\positive}$ is given on the set of generators:
                $$\{X_{i, 0}^{\pm}\}_{i \in \hat{\simpleroots}} \cup \{H_{i, r}\}_{ (i, r) \in \hat{\simpleroots} \x \{0, 1\} }$$
            and since we know that under the isomorphism in \textit{loc. cit.}, we have the following assignments:
                $$\forall i \in \hat{\simpleroots}: X_{i, 0}^{\pm} \mapsto x_i^{\pm}, H_{i, 0} \mapsto h_i$$
                $$\forall i \in \simpleroots: H_{i, 1} \mapsto h_i t$$
                $$H_{\theta, 1} \mapsto h_{\theta} t + t c_v$$
            it is enough to specify the following, wherein $h \in \h$ is arbitrary:
                $$\tilde{\delta}^{\positive}(h) = 0$$
                $$\tilde{\delta}^{\positive}(ht) = [h_1 \tensor 1, \sfr_{\g} v_2 \1(v_1, v_2)]$$
                $$\tilde{\delta}^{\positive}(t c_v) = 0$$
        \end{theorem}
            \begin{proof}
                \begin{enumerate}
                    \item Since $\deg x = 0$ for all $x \in \g$, we get via remark \ref{remark: total_degrees_of_classical_yangian_R_matrices} that:
                        $$\hat{\delta}^{\positive}(x) = 0$$
                    and in particular, we have that:
                        $$\hat{\delta}^{\positive}(h) = 0$$

                    \item Let us now compute $\hat{\delta}^{\positive}(ht)$ for an arbitrary $h \in \h$. 
                    \begin{enumerate}
                        \item \textbf{($\g_{[2]}^{\positive}$-component):} Firstly, to compute:
                            $$[\bar{\Delta}(ht), \sfr_{\g_{[2]}^{\positive}}]$$
                        let us firstly note that:
                            $$\sfr_{\g_{[2]}^{\positive}} = \sfr_{\g} v_2\1(v_1, v_2) \1^+(t_1, t_2)$$
                        Let us also choose a root basis for $\g$ for writing out $\sfr_{\g}$ explicitly: this is to say that for each positive root $\alpha \in \Phi^+$, we choose corresponding basis vectors $x_{\alpha}^{\pm} \in \g_{\pm \alpha}$ normalised so that:
                            $$(x_{\alpha}^-, x_{\alpha}^+)_{\g} = 1$$
                        to get the following basis for $\g$:
                            $$\{h_i\}_{i \in \simpleroots} \cup \{x_{\alpha}^-, x_{\alpha}^+\}_{\alpha \in \Phi^+}$$
                        From this, we see that:
                            $$
                                \begin{aligned}
                                    & [\bar{\Delta}(ht), \sfr_{\g_{[2]}^{\positive}}]
                                    \\
                                    = & 
                                    \begin{aligned}
                                        & -\sum_{i \in \simpleroots} [\bar{\Delta}(ht), h_i \tensor h_i v_2\1(v_1, v_2) \1^+(t_1, t_2)]
                                        \\
                                        - & \sum_{\alpha \in \Phi^+} [\bar{\Delta}(ht), (x_{\alpha}^- \tensor x_{\alpha}^+ + x_{\alpha}^+ \tensor x_{\alpha}^-) v_2\1(v_1, v_2) \1^+(t_1, t_2)]
                                    \end{aligned}
                                \end{aligned}
                            $$

                        Now, for each $i \in \simpleroots$, observe that:
                            $$
                                \begin{aligned}
                                    & [h t_1 \tensor 1, h_i \tensor h_i v_2\1(v_1, v_2) \1^+(t_1, t_2)]
                                    \\
                                    = & \sum_{(m, p) \in \Z \x \Z_{\geq 0}} [ht_1 \tensor 1, h_i v_1^m t_1^p \tensor h_i v_2^{-m} t_2^{-p - 1}]
                                    \\
                                    = & \sum_{(m, p) \in \Z \x \Z_{\geq 0}} [ht_1, h_i v_1^m t_1^p]_{\toroidal^{\positive}} \tensor h_i v_2^{-m} t_2^{-p - 1}
                                    \\
                                    = & \sum_{(m, p) \in \Z \x \Z_{\geq 0}} (h, h_i)_{\g} v_1^m t_1^p \bar{d}(t_1) \tensor h_i v_2^{-m} t_2^{-p - 1}
                                \end{aligned}
                            $$
                        and likewise, that:
                            $$[1 \tensor h t_2, h_i \tensor h_i v_2\1(v_1, v_2) \1^+(t_1, t_2)] = \sum_{(m, p) \in \Z \x \Z_{\geq 0}} h_i v_1^m t_1^p \tensor (h, h_i)_{\g} v_2^{-m} t_2^{-p - 1} \bar{d}(t_2)$$
                        Adding the two summands together then yields:
                            $$
                                \begin{aligned}
                                    & [\bar{\Delta}(ht), h_i \tensor h_i v_2\1(v_1, v_2) \1^+(t_1, t_2)]
                                    \\
                                    = & (h, h_i)_{\g} \sum_{(m, p) \in \Z \x \Z_{\geq 0}} \left( v_1^m t_1^p \bar{d}(t_1) \tensor h_i v_2^{-m} t_2^{-p - 1} + h_i v_1^m t_1^p \tensor v_2^{-m} t_2^{-p - 1} \bar{d}(t_2) \right)
                                    \\
                                    = & (h, h_i)_{\g} ( \bar{d}(t_1) \tensor h_i + h_i \tensor \bar{d}(t_2) ) v_2\1(v_1, v_2) \1^+(t_1, t_2)
                                \end{aligned}
                            $$
                        
                        Next, consider the following:
                            $$
                                \begin{aligned}
                                    & [ht_1 \tensor 1, x_{\alpha}^- \tensor x_{\alpha}^+ v_2\1(v_1, v_2) \1^+(t_1, t_2)]
                                    \\
                                    = & \sum_{(m, p) \in \Z \x \Z_{\geq 0}} [ht_1 \tensor 1, x_{\alpha}^- v_1^m t_1^p \tensor x_{\alpha}^+ v_2^{-m} t_2^{-p - 1}]
                                    \\
                                    = & \sum_{(m, p) \in \Z \x \Z_{\geq 0}} [ht_1, x_{\alpha}^- v_1^m t_1^p]_{\toroidal^{\positive}} \tensor x_{\alpha}^+ v_2^{-m} t_2^{-p - 1}
                                    \\
                                    = & \sum_{(m, p) \in \Z \x \Z_{\geq 0}} \left( -\alpha(h) x_{\alpha}^- v_1^m t_1^{p + 1} + (h, x_{\alpha}^-)_{\g} t_1 \bar{d}(v_1^m t_1^p) \right) \tensor x_{\alpha}^+ v_2^{-m} t_2^{-p - 1}
                                    \\
                                    = & \sum_{(m, p) \in \Z \x \Z_{\geq 0}} -\alpha(h) x_{\alpha}^- v_1^m t_1^{p + 1} \tensor x_{\alpha}^+ v_2^{-m} t_2^{-p - 1}
                                    \\
                                    & = -\alpha(h) ( x_{\alpha}^- \tensor x_{\alpha}^+ ) v_2 \1(v_1, v_2) t_1 \1^+(t_1, t_2)
                                \end{aligned}    
                            $$
                        wherein the second-to-last identity comes from the fact that\footnote{This can be proven easily by passing to the vector representation of $\g$, wherein $h$ is represented by a diagonal matrix while $x^{\pm}$ is represented by an upper/lower triangular matrix, and then using the fact that $(-, -)_{\g}$ differs from the trace form only by a non-zero constant.}:
                            $$(h, x^{\pm})_{\g} = 0$$
                        for every $h \in \h$ and every $x^{\pm} \in \n^{\pm}$. Similarly, we find that:
                            $$[ht_1 \tensor 1, x_{\alpha}^+ \tensor x_{\alpha}^- v_2\1(v_1, v_2) \1^+(t_1, t_2)] = \alpha(h) ( x_{\alpha}^+ \tensor x_{\alpha}^- ) v_2 \1(v_1, v_2) t_1 \1^+(t_1, t_2)$$
                        By putting the two together, one obtains:
                            $$[h t_1 \tensor 1, (x_{\alpha}^- \tensor x_{\alpha}^+ + x_{\alpha}^+ \tensor x_{\alpha}^-) v_2\1(v_1, v_2) \1^+(t_1, t_2)] = -\alpha(h) ( x_{\alpha}^- \tensor x_{\alpha}^+ - x_{\alpha}^+ \tensor x_{\alpha}^- ) v_2 \1(v_1, v_2) t_1 \1^+(t_1, t_2)$$
                        Likewise, we find that:
                            $$[1 \tensor h t_2, (x_{\alpha}^- \tensor x_{\alpha}^+ + x_{\alpha}^+ \tensor x_{\alpha}^-) v_2\1(v_1, v_2) \1^+(t_1, t_2)] = \alpha(h) ( x_{\alpha}^- \tensor x_{\alpha}^+ - x_{\alpha}^+ \tensor x_{\alpha}^- ) v_2 \1(v_1, v_2) t_2 \1^+(t_1, t_2)$$
                        and hence:
                            $$
                                \begin{aligned}
                                    & [\bar{\Delta}(ht), \sfr_{\g_{[2]}^{\positive}}]
                                    \\
                                    = &
                                    -\left(
                                    \begin{aligned}
                                        & \sum_{i \in \simpleroots} (h, h_i)_{\g} ( \bar{d}(t_1) \tensor h_i + h_i \tensor \bar{d}(t_2) )
                                        \\
                                        + & \sum_{\alpha \in \Phi^+} \alpha(h) ( x_{\alpha}^- \tensor x_{\alpha}^+ - x_{\alpha}^+ \tensor x_{\alpha}^- )(t_2 - t_1)
                                    \end{aligned}
                                    \right) v_2 \1(v_1, v_2) \1^+(t_1, t_2)
                                    \\
                                    & = -\left( \bar{d}(t_1) \tensor h + h \tensor \bar{d}(t_2) + [h_1 \tensor 1, \sfr_{\g}] (t_2 - t_1) \right) v_2 \1(v_1, v_2) \1^+(t_1, t_2)
                                    \\
                                    & = -\left( \bar{d}(t_1) \tensor h + h \tensor \bar{d}(t_2) \right) v_2 \1(v_1, v_2) \1^+(t_1, t_2) + [h_1 \tensor 1, \sfr_{\g}] v_2 \1(v_1, v_2)
                                \end{aligned}
                            $$
                        We note that the last equality holds thanks to the fact that:
                            $$(t_2 - t_1) \1^+(t_1, t_2) = (t_2 - t_1) \sum_{p \in \Z_{\geq 0}} t_1^p t_2^{-p - 1} = (t_2 - t_1) \frac{1}{t_2 - t_1} = 1$$
                            
                        \item \textbf{($\z_{[2]}^{\positive}$-component):} Recall from corollary \ref{coro: extended_toroidal_lie_bialgebras} that:
                            $$\sfr_{\z_{[2]}^{\positive}} := \sum_{(r, s) \in \Z \x \Z_{> 0}} K_{r, s} \tensor D_{r, s} + c_{v_1} \tensor D_{v_2}$$
                        and so:
                            $$
                                \begin{aligned}
                                    & [\bar{\Delta}(ht), \sfr_{\z_{[2]}^{\positive}}]
                                    \\
                                    = & \sum_{(r, s) \in \Z \x \Z_{> 0}} [\bar{\Delta}(ht), K_{r, s} \tensor D_{r, s}] + [\bar{\Delta}(ht), c_{v_1} \tensor D_{v_2}]
                                    \\
                                    = & -\sum_{(r, s) \in \Z \x \Z_{> 0}} K_{r, s} \tensor h D_{r, s}(t) - c_{v_1} \tensor h D_{v_2}(t_2)
                                    \\
                                    = & -\sum_{(r, s) \in \Z \x \Z_{> 0}} K_{r, s} \tensor r h v_2^{-r} t_2^{-s}
                                \end{aligned}
                            $$
                        where the minus sign in the third equation appeared because:
                            $$[ht, D_{r, s}] = -[D_{r, s}, ht] = -h D_{r, s}(t)$$
                            $$[ht, D_v] = -[D_v, ht] = -h D_v(t)$$
                        (cf. remark \ref{remark: derivation_action_on_multiloop_algebras}) and the last equality is due to the fact that:
                            $$D_{r, s} = -s v^{-r + 1} t^{-s - 1} \del_v + r v^{-r} t^{-s} \del_t$$
                            $$D_v = -v t^{-1} \del_v$$
                        (cf. remark \ref{remark: dual_of_toroidal_centres_contains_derivations}). 
                        
                        \item \textbf{($\d_{[2]}^{\positive}$-component):} Recall from corollary \ref{coro: extended_toroidal_lie_bialgebras} that:
                            $$\sfr_{\z_{[2]}^{\positive}} := \sum_{(r, s) \in \Z \x \Z_{\leq 0}} D_{r, s} \tensor K_{r, s} + D_{t_1} \tensor c_{t_2}$$
                        and so:
                            $$
                                \begin{aligned}
                                    & [\bar{\Delta}(ht), \sfr_{\z_{[2]}^{\positive}}]
                                    \\
                                    = & \sum_{(r, s) \in \Z \x \Z_{\leq 0}} [\bar{\Delta}(ht), D_{r, s} \tensor K_{r, s}] + [\bar{\Delta}(ht), D_{t_1} \tensor c_{t_2}]
                                    \\
                                    = & -\sum_{(r, s) \in \Z \x \Z_{\leq 0}} h D_{r, s}(t_1) \tensor K_{r, s} - h D_{t_1}(t_1) \tensor c_{t_2}
                                    \\
                                    = & -\sum_{(r, s) \in \Z \x \Z_{\leq 0}} r h v_1^{-r} t_1^{-s} \tensor K_{r, s} + h \tensor c_{t_2}
                                \end{aligned}
                            $$
                        where the minus sign in the third equation appeared because:
                            $$[ht, D_{r, s}] = -[D_{r, s}, ht] = -h D_{r, s}(t)$$
                            $$[ht, D_t] = -[D_t, ht] = -h D_t(t)$$
                        (cf. remark \ref{remark: derivation_action_on_multiloop_algebras}) and the the last equality is due to the fact that:
                            $$D_{r, s} = -s v^{-r + 1} t^{-s - 1} \del_v + r v^{-r} t^{-s} \del_t$$
                            $$D_t = -\del_t$$
                        (cf. remark \ref{remark: dual_of_toroidal_centres_contains_derivations}). 
                    \end{enumerate}

                    Since we know that:
                        $$[\bar{\Delta}(ht), \sfr_{\g_{[2]}^{\positive}}] = -\left( \bar{d}(t_1) \tensor h + h \tensor \bar{d}(t_2) \right) v_2 \1(v_1, v_2) \1^+(t_1, t_2) + [h_1 \tensor 1, \sfr_{\g}] v_2 \1(v_1, v_2)$$
                    we now claim that:
                        $$[\bar{\Delta}(ht), \sfr_{\z_{[2]}^{\positive}} + \sfr_{\d_{[2]}^{\positive}}] = \left( \bar{d}(t_1) \tensor h + h \tensor \bar{d}(t_2) \right) v_2 \1(v_1, v_2) \1^+(t_1, t_2)$$
                    (since ultimately, we would like to show that $\hat{\delta}^{\positive}(ht) = \sfr_{\g} v_2 \1(v_1, v_2)$), and to prove that this is the case, let us first note that we now have that:
                        $$
                            \begin{aligned}
                                & [\bar{\Delta}(ht), \sfr_{\z_{[2]}^{\positive}} + \sfr_{\d_{[2]}^{\positive}}]
                                \\
                                = & -\sum_{(r, s) \in \Z \x \Z_{> 0}} \left( K_{r, s} \tensor r h v_2^{-r} t_2^{-s} + r h v_1^{-r} t_1^s \tensor K_{r, -s} \right) - \sum_{r \in \Z} r h v_1^{-r} \tensor K_{r, 0} + h \tensor c_{t_2}
                            \end{aligned}
                        $$
                    wherein the first summand corresponds to the indices $(r, 0) \in \Z \x \Z_{\leq 0}$. From example \ref{example: toroidal_lie_algebras_centres}, we know that:
                        $$
                            K_{r, s} :=
                            \begin{cases}
                                \text{$\frac1s v^{r - 1} t^s \bar{d}(v)$ if $(r, s) \in \Z \x (\Z \setminus \{0\})$}
                                \\
                                \text{$-\frac1r v^r t^{-1} \bar{d}(t)$ if $(r, s) \in (\Z \setminus \{0\}) \x \{0\}$}
                                \\
                                \text{$0$ if $(r, s) = (0, 0)$}
                            \end{cases}
                        $$
                    from which one infers that:
                        $$
                            \begin{aligned}
                                & -\sum_{r \in \Z} r h v_1^{-r} \tensor K_{r, 0}
                                \\
                                = & -\sum_{r \in \Z} r h v_1^{-r} \tensor \left( -\frac1r v_2^r t_2^{-1} \bar{d}(t_2) \right)
                                \\
                                = & \sum_{r \in \Z} h v_1^{-r} \tensor v_2^r t_2^{-1} \bar{d}(t_2)
                                \\
                                = & \sum_{r \in \Z} h v_1^{-r} \tensor v_2^r t_2^{-1} \bar{d}(t_2)
                            \end{aligned}
                        $$
                        
                    Next, recall again from example \ref{example: toroidal_lie_algebras_centres} that:
                        $$(r, s) \in \Z^2 \implies K_{r, s} = \frac1s v^{r - 1} t^s \bar{d}(v) = -\frac1r v^r t^{s - 1} \bar{d}(t) \in \bar{\Omega}_{[2]}$$
                    and then consider the following:
                        $$
                            \begin{aligned}
                                & -\sum_{(r, s) \in \Z \x \Z_{> 0}} \left( K_{r, s} \tensor r h v_2^{-r} t_2^{-s} + r h v_1^{-r} t_1^s \tensor K_{r, -s} \right)
                                \\
                                = & \sum_{(r, s) \in \Z \x \Z_{> 0}} \left( v_1^r t_1^{s - 1} \bar{d}(t_1) \tensor h v_2^{-r} t_2^{-s} - h v_1^{-r} t_1^s \tensor v_2^r t_2^{-s - 1} \bar{d}(t_2) \right)
                            \end{aligned}
                        $$
                    wherein we note that for all $s \in \Z_{> 0}$, the summands corresponding to the indices $(0, s)$ vanish.

                    We now have that:
                        $$
                            \begin{aligned}
                                & [\bar{\Delta}(ht), \sfr_{\z_{[2]}^{\positive}} + \sfr_{\d_{[2]}^{\positive}}]
                                \\
                                = & \sum_{(r, s) \in \Z \x \Z_{> 0}} \left( K_{r, s} \tensor r h v_2^{-r} t_2^{-s} + r h v_1^{-r} t_1^s \tensor K_{r, -s} \right) - \sum_{r \in \Z} r h v_1^{-r} \tensor K_{r, 0} + h \tensor c_{t_2}
                                \\
                                = & \sum_{(r, s) \in \Z \x \Z_{> 0}} \left( v_1^r t_1^{s - 1} \bar{d}(t_1) \tensor h v_2^{-r} t_2^{-s} - h v_1^{-r} t_1^s \tensor v_2^r t_2^{-s - 1} \bar{d}(t_2) \right) + \sum_{r \in \Z} h v_1^{-r} \tensor v_2^r t_2^{-1} \bar{d}(t_2) + h \tensor t_2^{-1} \bar{d}(t_2)
                                \\
                                = & \sum_{(r, s) \in \Z \x \Z_{> 0}} \left( v_1^r t_1^{s - 1} \bar{d}(t_1) \tensor h v_2^{-r} t_2^{-s} + h v_1^r t_1^s \tensor v_2^{-r} t_2^{-s - 1} \bar{d}(t_2) \right) + \sum_{r \in \Z} h v_1^{-r} \tensor v_2^r t_2^{-1} \bar{d}(t_2)
                                \\
                                = & \sum_{(r, s) \in \Z \x \Z_{> 0}} \left( v_1^r t_1^{s - 1} \bar{d}(t_1) \tensor h v_2^{-r} t_2^{-s} + h v_1^{-r} t_1^s \tensor v_2^r t_2^{-s - 1} \bar{d}(t_2) \right) + \sum_{r \in \Z} h v_1^r \tensor v_2^{-r} t_2^{-1} \bar{d}(t_2)
                                \\
                                = & ( \bar{d}(t_1) \tensor h ) \sum_{(r, s) \in \Z \x \Z_{> 0}} v_1^r t_1^{s - 1} \tensor v_2^{-r} t_2^{-s} + ( h \tensor \bar{d}(t_2) ) \left( \sum_{(r, s) \in \Z \x \Z_{> 0}} v_1^r t_1^s \tensor v_2^{-r} t_2^{-s - 1} + \sum_{r \in \Z} v_1^r \tensor v_2^{-r} t_2^{-1} \right)
                                \\
                                = & ( \bar{d}(t_1) \tensor h ) \sum_{(r, s) \in \Z \x \Z_{\geq 0}} v_1^r t_1^s \tensor v_2^{-r} t_2^{-s - 1} + ( h \tensor \bar{d}(t_2) ) \left( \sum_{(r, s) \in \Z \x \Z_{> 0}} v_1^r t_1^s \tensor v_2^{-r} t_2^{-s - 1} + \sum_{r \in \Z} v_1^r \tensor v_2^{-r} t_2^{-1} \right)
                                \\
                                = & ( \bar{d}(t_1) \tensor h + h \tensor \bar{d}(t_2) ) \sum_{(r, s) \in \Z \x \Z_{\geq 0}} v_1^r t_1^s \tensor v_2^{-r} t_2^{-s - 1}
                                \\
                                = & ( \bar{d}(t_1) \tensor h + h \tensor \bar{d}(t_2) ) v_2 \1(v_1, v_2) \1^+(t_1, t_2)
                            \end{aligned}
                        $$

                    We can now add the three components together to yield:
                        $$[\bar{\Delta}(ht), \sfr_{\extendedtoroidal^{\positive}}] = [ \bar{\Delta}(ht), \sfr_{\g_{[2]}^{\positive}} + (\sfr_{\z_{[2]}^{\positive}} + \sfr_{\d_{[2]}^{\positive}}) ] =  [h_1 \tensor 1] v_2 \1(v_1, v_2)$$
                    precisely as claimed. 
                    
                    \item Finally, in order to compute $\hat{\delta}^{\positive}(t c_v)$, let us first notice that:
                        $$t c_v = K_{0, 1}$$
                    Since $[K_{0, 1}, \g_{[2]}] = 0$, we have that:
                        $$[\bar{\Delta}(K_{0, 1}), \sfr_{\extendedtoroidal^{\positive}}] = [\bar{\Delta}(K_{0, 1}), \sfr_{\z_{[2]}^{\positive}} + \sfr_{\d_{[2]}^{\positive}}] = [1 \tensor K_{0, 1}, \sfr_{\z_{[2]}^{\positive}}] + [K_{0, 1} \tensor 1, \sfr_{\d_{[2]}^{\positive}}]$$
                    From lemma \ref{lemma: explicit_commutators_between_central_basis_elements_and_derivations}, we know that:
                        $$
                            \forall (a, b) \in \Z^2: [D, K_{a, b}] =
                            \begin{cases}
                                \text{$((b - 1)r - sa) D_{a - r, b - s - 1}$ if $D = D_{r, s}$}
                                \\
                                \text{$-r K_{a, b - 1}$ if $D = D_v$}
                                \\
                                \text{$- D_{a, b - 1}$ if $D = D_t$}
                            \end{cases}
                        $$
                        $$[\d_{[2]}, c_v]_{\extendedtoroidal} = [\d_{[2]}, c_t]_{\extendedtoroidal} = 0 = 0$$
                    and so:
                        $$[1 \tensor K_{0, 1}, \sfr_{\z_{[2]}^{\positive}}] = \sum_{(r, s) \in \Z \x \Z_{> 0}} K_{r, s} \tensor [K_{0, 1}, D_{r, s}] = sum_{(r, s) \in \Z \x \Z_{> 0}} K_{r, s} \tensor 0 = 0$$
                        $$[K_{0, 1} \tensor 1, \sfr_{\d_{[2]}^{\positive}}] = \sum_{(r, s) \in \Z \x \Z_{\leq 0}} [K_{0, 1}, D_{r, s}] \tensor K_{r, s} = \sum_{(r, s) \in \Z \x \Z_{\leq 0}} 0 \tensor K_{r, s} = 0$$
                    We are thus able to conclude that:
                        $$\tilde{\delta}^{\positive}(t c_v) = 0$$
                    as claimed.
                \end{enumerate}
            \end{proof}

    \subsection{Topological issues}
        \todo[inline]{Not written}
    
    \subsection{Hopf coproducts and classical limits of completed affine Yangians}
        \begin{convention}
            In this subsection, we assume that $\g$ is simply laced, excluding the case where $\g$ is of type $\sfA_1$. 
        \end{convention}

        We begin this subsection by reviewing the construction of what we shall call the \say{Hopf coproduct} $\Delta$ on the Yangian $\rmY_{\hbar}(\hat{\g})$ associated to the affine Kac-Moody algebra $\hat{\g}$, as was done in \cite[Sections 4 and 5]{guay_nakajima_wendlandt_affine_yangian_coproduct}. We will then lift this map to the formal Yangian $\rmY_{\hbar}(\hat{\g})$ to get another \say{Hopf coproduct} $\Delta_{\hbar}$ thereon. The point of doing this is so that ultimately, we would obtain:
            $$\frac{1}{\hbar}(\Delta_{\hbar} - \Delta_{\hbar}^{\cop}) \pmod{\hbar} \equiv \tilde{\delta}^{\positive}$$
        and hence be able to realise the topological Lie bialgebra $(\toroidal^{\positive}, \tilde{\delta}^{\positive})$ from theorem \ref{theorem: toroidal_lie_bialgebras} as the classical limit of the formal Yangian $\rmY_{\hbar}(\hat{\g})$ in some sense (which, let us caution, is not exactly the same as in \cite{etingof_kazhdan_quantisation_1}).
        
        \begin{lemma}[The category $\calO$ for the affine Yangian $\rmY(\hat{\g})$] \label{lemma: category_O_affine_yangian}
            (Cf. \cite[Theorem 4.9]{guay_nakajima_wendlandt_affine_yangian_coproduct}).
        
            There is a full subcategory of the category of $\rmY(\hat{\g})$-modules, called the \textbf{category $\calO$}. This category satisfies the following properties:
            \begin{itemize}
                \item Every object $V \in \Ob(\calO)$ is $\hat{\h}$-diagonalisable and with finite-dimensional ($\hat{\h}$-)weight spaces, and
                \item For every object $V \in \Ob(\calO)$, there exist \textbf{maximal weights} $\lambda_1, ..., \lambda_k \in \hat{\h}^*$ such that, for any $\mu \in \Pi(V)$, one has that:
                    $$\forall 1 \leq i \leq k: \lambda_i - \mu \in \hat{Q}^+$$
            \end{itemize}

            The aforementioned category $\calO$ of $\rmY(\hat{\g})$ is closed under tensor products over $k$, i.e. if $V_1, V_2$ are any two objects of the category $\calO$, then there will be a $k$-algebra homomorphism:
                $$\Delta_{V_1, V_2}: \rmY(\hat{\g}) \to \End_k(V_1 \tensor_k V_2)$$
            Furthermore, these tensor products are coassociative in the sense that any $k$-vector space isomorphism:
                $$(V_1 \tensor_k V_2) \tensor_k V_3 \xrightarrow[]{\cong} V_1 \tensor_k (V_2 \tensor_k V_3)$$
            between objects $V_1, V_2, V_3 \in \Ob(\calO)$ upgrades to an isomorphism of left-$\rmY(\hat{\g})$-modules.

            Explicitly, for each $V_1, V_2 \in \Ob(\calO)$, the map $\Delta_{V_1, V_2}$ is given on the generating set\footnote{Using the Levendorskii presentation for $\rmY(\hat{\g})$, one sees that this generating set suffices.} $\hat{\h} \cup \{T_{i, 1}, X_{i, 0}^{\pm}\}_{i \in \hat{\simpleroots}}$ by:
                $$\forall h \in \hat{\h}: \Delta_{V_1, V_2}(h) := \bar{\Delta}(h)$$
                $$\forall i \in \hat{\simpleroots}: \Delta_{V_1, V_2}(X_{i, 0}^{\pm}) := \bar{\Delta}(X_{i, 0}^{\pm})$$
                $$\forall i \in \hat{\simpleroots}: \Delta_{V_1, V_2}(T_{i, 0}) = \bar{\Delta}(T_{i, 0}) + [H_{i, 0} \tensor 1, \sfr_{ \hat{\g} }^-]$$
            with $\sfr_{ \hat{\g} }^-$ being the Casimir tensor\footnote{This is denoted by $\Omega_+$ in \cite{guay_nakajima_wendlandt_affine_yangian_coproduct} and \cite{guay_nakajima_wendlandt_affine_yangian_vertex_representations_and_PBW}. We opted to designate this the \say{negative} half of the Casimir tensor of $\hat{\g}$ in accordance with the root-degree of the first tensor factor. Also, in \textit{loc. cit.}, the authors considered the Casimir tensor associated to the Kac-Moody pairing on $\hat{\h} \tensor_k \hat{\h} \oplus \hat{\n}^- \hattensor_k \hat{\n}^+$, but we need only the \say{triangular} component since the Cartan component will be killed by $[H_{i, 0} \tensor 1, -]$ anyway.} associated to the non-degenerate Kac-Moody pairing on $\hat{\n}^- \hattensor_k \hat{\n}^+$.
        \end{lemma}
        \begin{remark}
            The category $\calO$ as in lemma \ref{lemma: category_O_affine_yangian} is \textit{not} monoidal, since it lacks a monoidal unit. 
        \end{remark}
        \begin{remark}[Why involve the category $\calO$ ?]
            For a moment, let us pick the root bases $\{ x_{\alpha, k}^{\pm} \}_{(\alpha, k) \in \hat{\Phi}^+ \x \{1, ..., \dim_k (\hat{\g})_{\alpha} \}}$ for $\hat{\n}^{\pm}$ in such a way that they are dual to one another with respect to the Kac-Moody pairing on $\hat{\g}$. In terms of these bases, one can write:
                $$\sfr_{\hat{\g}}^- = \sum_{\alpha \in \hat{\Phi}^+} \sum_{k = 1}^{ \dim_k (\hat{\g})_{\alpha} } x_{\alpha, k}^- \tensor x_{\alpha, k}^+$$
        
            One notable detail is the fact that the sum\footnote{The completed tensor product $\rmY(\hat{\g}) \hattensor_k \rmY(\hat{\g})$ is only to be understood in the vague sense that it denotes some completion of the algebraic tensor product $\rmY(\hat{\g}) \tensor_k \rmY(\hat{\g})$ wherein the sum in question converges.}:
                $$\sum_{\alpha \in \hat{\Phi}^+} \sum_{k = 1}^{ \dim_k (\hat{\g})_{\alpha} } x_{\alpha, k}^- \tensor x_{\alpha, k}^+ \in \rmY(\hat{\g}) \hattensor_k \rmY(\hat{\g})$$
            is infinite \textit{a priori}, since the affine Kac-Moody algebra $\hat{\g}$ has infinitely many positive roots. However, this is precisely why we have restricted our attention down to the category $\calO$: notice that for any $V \in \Ob(\calO)$ and any $\mu \in \Pi(V)$, there exists a natural number $N \in \Z_{\geq 0}$ such that:
                $$\forall \alpha \in \hat{\Phi}^+: r \geq N \implies V_{\mu + r \alpha} \cong 0$$
            From this, one sees that even though it is given by an infinite sum, the operator:
                $$\sum_{\alpha \in \hat{\Phi}^+} \sum_{k = 1}^{ \dim_k (\hat{\g})_{\alpha} } x_{\alpha, k}^- \tensor x_{\alpha, k}^+ \in \End_k(V_1 \tensor_k V_2)$$
            is ultimately locally nilpotent on the vector spaces of the kind $V_1 \tensor_k V_2$, wherein $V_1, V_2 \in \Ob(\calO)$; as such, one sees that the infinite sum above actually becomes finite (and hence converges) after evaluation on elements of the $\rmY(\hat{\g})$-modules in the category $\calO$, and the maps $\Delta_{V_1, V_2}$ as in lemma \ref{lemma: category_O_affine_yangian} are therefore well-defined. 
        \end{remark}
        \begin{convention}
            If $\fraku$ is a Kac-Moody algebra of some simply laced untwisted affine type and then we will denote by $\hat{\rmY}(\fraku)$ the grading-completion of $\rmY(\fraku)$ with respect to its root grading.
        \end{convention}
        \begin{lemma}[$\rmY(\hat{\g})$-modules are $\hat{\rmY}(\hat{\g})$-modules] \label{lemma: lifting_representations_of_affine_yangians_to_root_grading_completions}
            (Cf. \cite[Proposition 5.14]{guay_nakajima_wendlandt_affine_yangian_coproduct}) Any left-$\rmY(\hat{\g})$-module $V$ in the category $\calO$, given by a $k$-algebra homomorphism:
                $$\rho: \rmY(\hat{\g}) \to \End_k(V)$$
            gives rise to a unique left-$\hat{\rmY}(\hat{\g})$-module structure on $V$, which is the same as a $k$-algebra homomorphism:
                $$\hat{\rho}: \hat{\rmY}(\hat{\g}) \to \End_k(V)$$
            fitting into the following commutative diagram of $k$-algebras and homomorphisms between them, where the vertical arrow is the canonical one as in \cite[Section 5, Lemma 5.3]{guay_nakajima_wendlandt_affine_yangian_coproduct}:
                $$
                    \begin{tikzcd}
                	{\hat{\rmY}(\hat{\g})} & {\End_k(V)} \\
                	{\rmY(\hat{\g})}
                	\arrow[from=2-1, to=1-1]
                	\arrow["{\hat{\rho}}", dashed, from=1-1, to=1-2]
                	\arrow["\rho"', from=2-1, to=1-2]
                    \end{tikzcd}
                $$
        \end{lemma}
        \begin{proposition}[Hopf coproduct on affine Yangians] \label{prop: hopf_coproduct_on_yangians}
            (Cf. \cite[Proposition 5.18]{guay_nakajima_wendlandt_affine_yangian_coproduct}) There exists a $k$-algebra homomorphism:
                $$\Delta: \rmY(\hat{\g}) \to \hat{\rmY}(\hat{\g} \oplus \hat{\g})$$
            satisfying:
                $$\Delta_{V_1, V_2} = (\hat{\rho}_1 \tensor \hat{\rho}_2) \circ \Delta$$
            for any objects $(V_1, \rho_1), (V_2, \rho_2)$ of the category $\calO$ of $\rmY(\hat{\g})$.
        \end{proposition}
        
        \begin{lemma}[The category $\calO_{\hbar}$ of the formal affine Yangian $\rmY_{\hbar}(\hat{\g})$] \label{lemma: category_O_formal_affine_yangian}
            For the formal affine Yangian $\rmY_{\hbar}(\hat{\g})$, one can define a category $\calO_{\hbar}$ in the exact same way\footnote{Ultimately, this is because we have that $[h, X_{i, r}^{\pm}] = \pm \alpha_i(h) X_{i, r}^{\pm} \in \rmY_{\hbar}(L) \setminus \hbar\rmY_{\hbar}(L) \cong \rmY^0(L)$ for all Cartan elements $h \in \hat{\h}$.} as how the category $\calO$ was defined for $\rmY(\hat{\g})$ in lemma \ref{lemma: category_O_affine_yangian}. 

            The category $\calO_{\hbar}$ is closed under $\tensor_k$ (cf. lemma \ref{lemma: category_O_affine_yangian}): for every $V_1, V_2 \in \Ob(\calO_{\hbar})$, there is a corresponding $k$-algebra homomorphism:
                $$\Delta_{V_1, V_2, \hbar}: \rmY_{\hbar}(\hat{\g}) \to \End_k(V_1 \tensor_k V_2)$$
            given by:
                $$\forall h \in \hat{\h}: \Delta_{V_1, V_2, \hbar}(h) := \bar{\Delta}(h)$$
                $$\forall i \in \hat{\simpleroots}: \Delta_{V_1, V_2, \hbar}(X_{i, 0}^{\pm}) := \bar{\Delta}(X_{i, 0}^{\pm})$$
                $$\forall i \in \hat{\simpleroots}: \Delta_{V_1, V_2, \hbar}(T_{i, 0}) = \bar{\Delta}(T_{i, 0}) + \hbar [H_{i, 0} \tensor 1, \sfr_{ \hat{\g} }^- ]$$
            Furthermore, the tensor products in $\calO_{\hbar}$ are coassociative in the same sense as in lemma \ref{lemma: category_O_affine_yangian}.
        \end{lemma}
            \begin{proof}
                This is a consequence of lemma \ref{lemma: category_O_affine_yangian} and the fact that we have a graded $k$-algebra isomorphism:
                    $$\rmY_{\hbar}(\hat{\g}) \xrightarrow[]{\cong} \Rees_{\hbar} \rmY(\hat{\g})$$
                (cf. lemma \ref{lemma: formal_yangians_as_rees_algebras}).
            \end{proof}
        \begin{convention}
            If $\fraku$ is a Kac-Moody algebra of some simply laced untwisted affine type and then we will denote by $\hat{\rmY}_{\hbar}(\fraku)$ the completion of $\rmY_{\hbar}(\fraku)$ with respect to its root grading, with \say{completion} being understood to be in the sense of \cite[Appendix A]{wendlandt_formal_shift_operators_on_yangian_doubles}. Note that this root grading is the same as the one on $\rmY(\fraku)$ due to the fact that:
                $$\forall h \in \hat{\h}: [h, X_{i, r}^{\pm}] = \pm \alpha_i(h) X_{i, r}^{\pm} \in \rmY_{\hbar}(\fraku) \setminus \hbar\rmY_{\hbar}(\fraku) \cong \rmY^0(\fraku)$$
            so the construction of $\hat{\rmY}_{\hbar}(\fraku)$ from $\rmY_{\hbar}(\fraku)$ is the same as that of $\hat{\rmY}(\fraku)$ from $\rmY(\fraku)$.
        \end{convention}
        \begin{lemma}[$\rmY_{\hbar}(\hat{\g})$-modules are $\hat{\rmY}_{\hbar}$-modules] \label{lemma: lifting_representations_of_formal_affine_yangians_to_root_grading_completions}
            Any left-$\rmY_{\hbar}(\hat{\g})$-module $V$ in the category $\calO$, given by a $k$-algebra homomorphism:
                $$\rho: \rmY_{\hbar}(\hat{\g}) \to \End_k(V)$$
            gives rise to a unique left-$\hat{\rmY}_{\hbar}$-module structure on $V$, which is the same as a $k$-algebra homomorphism:
                $$\hat{\rho}: \hat{\rmY}_{\hbar} \to \End_k(V)$$
            fitting into the following commutative diagram of $k$-algebras and homomorphisms between them, where the vertical arrow is the canonical inclusion (cf. \cite[Section 5, Lemma 5.3]{guay_nakajima_wendlandt_affine_yangian_coproduct}):
                $$
                    \begin{tikzcd}
                	{\hat{\rmY}_{\hbar}} & {\End_k(V)} \\
                	{\rmY_{\hbar}(\hat{\g})}
                	\arrow[from=2-1, to=1-1]
                	\arrow["{\hat{\rho}}", dashed, from=1-1, to=1-2]
                	\arrow["\rho"', from=2-1, to=1-2]
                    \end{tikzcd}
                $$
        \end{lemma}
            \begin{proof}
                
            \end{proof}
        \begin{theorem}[Hopf coproduct on formal affine Yangians] \label{theorem: hopf_coproduct_on_formal_yangians}
            The $k$-algebra homomorphism $\Delta: \rmY(\hat{\g}) \to \hat{\rmY}(\hat{\g} \oplus \hat{\g})$ from proposition \ref{prop: hopf_coproduct_on_yangians} lifts\footnote{... in the sense that $\Delta_{\hbar} \pmod{(\hbar - \hbar_0)} \equiv \Delta$ for any $\hbar_0 \in k^{\x}$.} to a $k$-algebra homomorphism:
                $$\Delta_{\hbar}: \rmY_{\hbar}(\hat{\g}) \to \hat{\rmY}_{\hbar}(\hat{\g} \oplus \hat{\g})$$
            satisfying:
                $$\Delta_{V_1, V_2, \hbar} = (\hat{\rho}_1 \tensor \hat{\rho}_2) \circ \Delta_{\hbar}$$
            for any $(V_1, \rho_1), (V_2, \rho_2) \in \Ob(\calO_{\hbar})$.
        \end{theorem}
            \begin{proof}
                
            \end{proof}
        
        \begin{theorem}[Toroidal Lie algebras as classical limits of formal affine Yangians] \label{theorem: toroidal_lie_algebras_as_classical_limits_of_formal_affine_yangians}
           The topological Lie bialgebra $(\toroidal^{\positive}, \tilde{\delta}^{\positive})$ from theorem \ref{theorem: toroidal_topological_lie_bialgebras} (see also theorem \ref{theorem: toroidal_lie_bialgebras}) is the classical limit of the formal affine Yangian $\rmY_{\hbar}(\hat{\g})$ with the \say{coproduct} $\Delta_{\hbar}$ (as in theorem \ref{theorem: hopf_coproduct_on_formal_yangians}), in the sense that:
                $$\frac{1}{\hbar}( \Delta_{\hbar} - \Delta_{\hbar}^{\cop} ) \equiv \tilde{\delta}^{\positive} \pmod{\hbar}$$
        \end{theorem}
            \begin{proof}
                Before we begin computing, let us make the preliminary observation that $T_{i, 1}(\hbar) \in \rmY_{\hbar}(\hat{\g})$ is a lift modulo $\hbar$ of $H_{i, 1} \in \toroidal^{\positive}$:
                    $$T_{i, 1}(\hbar) := H_{i, 1} - \frac12 \hbar H_{i, 0}^2 \equiv H_{i, 1} \pmod{\hbar}$$
                Also, let us note that, we know by lemma \ref{lemma: levendorskii_presentation} that it is enough to only check the value of $\Delta_{\hbar}^{\cop}$ on the generators $H_{i, 0}, X_{i, 0}^{\pm}$, and $T_{i, 1} \equiv H_{i, 1} \pmod{\hbar}$, for all $i \in \hat{\simpleroots}$.
            
                Firstly, from theorem \ref{theorem: hopf_coproduct_on_formal_yangians}, we know that:
                    $$\forall h \in \hat{\h}: \Delta_{\hbar}(h) := \bar{\Delta}(h)$$
                    $$\forall i \in \hat{\simpleroots}: \Delta_{\hbar}(X_{i, 0}^{\pm}) := \bar{\Delta}(X_{i, 0}^{\pm})$$
                    $$\forall i \in \hat{\simpleroots}: \Delta_{\hbar}(T_{i, 0}) = \bar{\Delta}(T_{i, 0}) + [H_{i, 0} \tensor 1, \sfr_{ \hat{\g} }^-]$$
                This tells us that:
                    $$\forall h \in \hat{\h}: \Delta_{\hbar}^{\cop}(h) := \bar{\Delta}(h)$$
                    $$\forall i \in \hat{\simpleroots}: \Delta_{\hbar}^{\cop}(X_{i, 0}^{\pm}) := \bar{\Delta}(X_{i, 0}^{\pm})$$
                    $$\forall i \in \hat{\simpleroots}: \Delta_{\hbar}^{\cop}(T_{i, 0}) = \bar{\Delta}(T_{i, 0}) + [1 \tensor H_{i, 0}, \sfr_{ \hat{\g} }^+]$$
                with $\sfr_{\hat{\g}}^-$ being the Casimir tensor associated to the non-degenerate Kac-Moody pairing on $\hat{\n}^- \hattensor_k \hat{\n}^+$.

                It is then trivial that:
                    $$\frac{1}{\hbar}( \Delta_{\hbar} - \Delta_{\hbar}^{\cop} )(X) = 0$$
                for:
                    $$X \in \hat{\h} \cup \{X_{i, 0}^{\pm}\}_{i \in \hat{\simpleroots}}$$
                which implies that:
                    $$\frac{1}{\hbar}( \Delta_{\hbar} - \Delta_{\hbar}^{\cop} )(X) \equiv \tilde{\delta}^{\positive}(X) \pmod{\hbar}$$
                which is because it is known from theorem \ref{theorem: toroidal_lie_bialgebras} that:
                    $$\tilde{\delta}^{\positive}(X) = 0$$
                whenever $\deg X = 0$, which is the case here.
                
                Now, let us verify that:
                    $$\frac{1}{\hbar}(\Delta_{\hbar} - \Delta_{\hbar}^{\cop})(T_{i, 1}) \equiv \tilde{\delta}^{\positive}(H_{i, 1})$$
                It is not hard to see that\footnote{One can prove this by e.g. picking the root bases foor $\hat{\n}^{\pm}$.}:
                    $$[1 \tensor H_{i, 0}, \sfr_{ \hat{\g} }^+] = -[H_{i, 0} \tensor 1, \sfr_{ \hat{\g} }^+]$$
                which tells us that:
                    $$\frac{1}{\hbar}( \Delta_{\hbar} - \Delta_{\hbar}^{\cop} )(T_{i, 1}) = [H_{i, 0} \tensor 1, \sfr_{ \hat{\g} }^- + \sfr_{ \hat{\g} }^+]$$
                Since $H_{i, 0}$ commutes with every element of $\hat{\h}$, we can equivalently rewrite the above into:
                    $$\frac{1}{\hbar}( \Delta_{\hbar} - \Delta_{\hbar}^{\cop} )(T_{i, 1}) = [H_{i, 0} \tensor 1, \sfr_{\hat{\h}} + \sfr_{ \hat{\g} }^- + \sfr_{ \hat{\g} }^+] = [H_{i, 0} \tensor 1, \sfr_{ \hat{\g} }]$$
                wherein $\sfr_{\hat{\h}}$ is the Casimir element associated to the Kac-Moody pairing on $\hat{\h} \tensor_k \hat{\h}$. 

                We know from theorem \ref{theorem: toroidal_lie_bialgebras} that:
                    $$\tilde{\delta}^{\positive}(H_{i, 1}) = [ H_{i, 0} \tensor 1, \sfr_{\g} v_2 \1(v_1, v_2) ]$$
                so we will be done if we can show that:
                    $$[H_{i, 0} \tensor 1, \sfr_{ \hat{\g} }] = [ H_{i, 0} \tensor 1, \sfr_{\g} v_2 \1(v_1, v_2) ]$$
                From the fact that:
                    $$\hat{\g} \cong \g \oplus \z \oplus \d \cong \g \oplus k c_v \oplus k D_{0, -1}$$
                we infer that:
                    $$\sfr_{ \hat{\g} } = \sfr_{\g} v_2 \1(v_1, v_2) + \sfr_{\z} + \sfr_{\d} = \sfr_{\g} v_2 \1(v_1, v_2) + K_{0, -1} \tensor D_{0, -1} + D_{0, -1} \tensor K_{0, -1}$$
                wherein $\sfr_{\z}, \sfr_{\d}$ respectively denote the Casimir elements corresponding to the Kac-Moody form on $\z \tensor_k \d$ and on $\d \tensor_k \z$ respectively. The element $K_{0, -1} \in \hat{\g} := \g \oplus \z$ is central and therefore commutes with $H_{i, 0}$, which implies that:
                    $$[H_{i, 0} \tensor 1, \sfr_{\z}] = [H_{i, 0} \tensor 1, K_{0, -1} \tensor D_{0, -1}] = 0$$
                At the same time, we also know that $D_{0, -1}$ acts as $\id_{\g} \tensor \left(-v \frac{d}{dv}\right)$ on $\g$ and hence as zero on the elements of $\g$ (i.e. degree-$0$ elements of $\g$), and so:
                    $$[H_{i, 0} \tensor 1, \sfr_{\d}] = [H_{i, 0} \tensor 1, D_{0, -1} \tensor K_{0, -1}] = 0$$
                as well. As such, we have demonstrated that:
                    $$[H_{i, 0} \tensor 1, \sfr_{ \hat{\g} }] = [ H_{i, 0} \tensor 1, \sfr_{\g} v_2 \1(v_1, v_2) ]$$
                as we sought to. As mentioned above, this allows us to conclude that:
                    $$\frac{1}{\hbar}( \Delta_{\hbar} - \Delta_{\hbar}^{\cop} )(T_{i, 1}) \equiv \tilde{\delta}^{\positive}(H_{i, 1}) \pmod{\hbar}$$
            \end{proof}