\section{A guide to \texorpdfstring{$\gamma$}{}-extended toroidal Lie algebras}
    \subsection{What is done in this thesis ?}
        A rather well-known story in the representation theory of Lie algebras is that of Kac-Moody algebras, specifically of Kac-Moody algebras of finite and (untwisted) affine types. Chronologically speaking, the former theory preceded the latter by around 50 to 80 years, dating back to the works of Wilhelm Killing and \'Elie Cartan in the late $19^{th}$ centry and early $20^{th}$ century on the classification of finite-dimensional semi-simple Lie algebras over an algebraically closed field of characteristic $0$ (e.g. $\bbC$), which are the same as the aforementioned finite-type Kac-Moody algebras. A salient feature of, and indeed, the essential technical ingredient in this theory is an essentially unique \textit{non-degenerate} and \textit{invariant} symmetric bilinear form that any finite-dimensional semi-simple Lie algebra, say $\g$, can be endowed with. This is the famous \textbf{Killing form} (named after Wilhelm Killing) and using it, one is able to more-or-less develop the entire structural and representation theory of these Lie algebras. In particular, using the Killing form, one is able to construct the so-called \textbf{Cartan matrix} (named after \'Elie Cartan), which contains within it a lot - if not even all - of the important information about finite-dimensional semi-simple Lie algebras. For instance, should one know only of a Cartan matrix, one can then write down a presentation in terms of generators and relations (which is due to Claude Chevalley and Jean-Pierre Serre) for a uniquely corresponding finite-dimensional semi-simple Lie algebras. 

        Now, one particular property that the Cartan matrix of a finite-dimensional semi-simple Lie algebra is that it is \textit{positive-definite}. However, should we not require that matrices of the \say{Cartan kind}\footnote{Many authors use the term \say{generalised Cartan matrix}.} are positive-definite but were to insist on writing down presentations \textit{\`a la} Chevalley-Serre nevertheless, then we would obtain certain infinite-dimensional Lie algebras that are nowadays known as \textbf{Kac-Moody algebras} (cf. \cite[Chapter 1]{kac_infinite_dimensional_lie_algebras}). Specifically, as Robert Moody and Victor Kac found out in the late 1960s and early 1970s, by requiring that the associated Cartan matrix be \textit{positive-semi-definite}, one shall obtain the so-called Kac-Moody algebras of \textbf{\textit{untwisted} affine types}. What is rather remarkable about these affine Kac-Moody algebras is that they admit something called \textbf{loop realisations}: starting with a finite-dimensional semi-simple Lie algebra $\g$ with Cartan matrix $C$, one can firstly form the \textbf{loop algebra} $\g[v^{\pm 1}] := \g \tensor_{\bbC} \bbC[v^{\pm 1}]$ - which will always be equipped with a non-degenerate and invariant symmetric bilinear form originating naturally from the Killing form - then consider its universal central extension (UCE), which happens to be by a $1$-dimensional centre, say $\bbC c_{\aff}$ for some $c_{\aff} \in \bbC^{\x}$, and then finally adding on a Lie derivation $D_{\aff} \in \der(\g[v^{\pm 1}] \oplus \bbC c_{\aff})$, with the purpose of it all being so that at the end of the process, one shall obtain a Lie algebra:
            $$\hat{\g} := (\g[v^{\pm 1}] \oplus \bbC c_{\aff}) \rtimes \bbC D_{\aff}$$
        that is isomorphic to the Kac-Moody algebra whose Cartan matrix is obtained from $C$ by adding one extra row and one extra column in a certain manner (for more details, see \cite[Chapter 7]{kac_infinite_dimensional_lie_algebras}). The idea here is that, because any invariant symmetric bilinear form on the UCE $\g[v^{\pm 1}] \oplus \bbC c_{\aff}$ is necessarily degenerate and with radical being equal to the subspace $\bbC c_{\aff}$, one introduces the extra element $D_{\aff}$ to pair non-trivially with $c_{\aff}$, thereby fixing the issue of degeneracy; in other words, $D_{\aff}$ is to be dual to $c_{\aff}$ to begin with. The fact that $D_{\aff}$ is a Lie derivation on $\g[v^{\pm 1}] \oplus \bbC c_{\aff}$ is actually a consequence of this construction. As a brief aside, let us remark that the loop realisations of untwisted affine Kac-Moody algebras are incredibly useful in practice (cf. e.g. \cite[Chapters 9 and 10]{kac_infinite_dimensional_lie_algebras}), not only because they are much more easier to work with than the Chevalley-Serre presentations, but also because they reveal interesting information about the structure of (not necessarily untwisted) affine Kac-Moody algebras, e.g. how such Kac-Moody algebras all arise as so-called \say{twists} of the untwisted ones (cf. e.g. \cite[Chapter 8]{kac_infinite_dimensional_lie_algebras}), and how this phenomenon gives rise to a classification of these Lie algebras in terms of Galois cohomology (such as in the works of Arturo Pianzola\footnote{... and our thanks to A. Pianzola for letting us know of such results!}; cf. e.g. \cite{pianzola_vanishing_of_H1_of_dedekind_rings} and its sequels). 

        In this thesis, out starting point is not the single-loop algebra $\g[v^{\pm 1}]$ as above, but rather the double-loop algebra:
            $$\g_{[2]} := \g[v^{\pm 1}, t^{\pm 1}]$$
        which, like above, will also be equipped with a non-degenerate and invriant symmetric bilinear form:
            $$(-, -)_{\g_{[2]}}$$
        originating naturally from the Killing form and depending on a distinguished linear map, a kind of modified formal residue map:
            $$\gamma: \bbC[v^{\pm 1}, t^{\pm 1}] \to \bbC$$
        on which our constructions will depend crucially (cf. subsection \ref{subsection: definition_of_yangian_extended_toroidal_lie_algebras}). We then again consider the UCE:
            $$\toroidal := \uce(\g_{[2]})$$
        but a large difference in contrast to the affine Kac-Moody situation is that now, the centre $\z(\toroidal)$ is \textit{infinite-dimensional}; luckily though, it is graded, and the graded components are all finite-dimensional. Regardless, the issue whereby invariant symmetric bilinear forms on the UCE must be degenerate persists, which leads us to consider the vector space:
            $$\extendedtoroidal := \toroidal \oplus \z(\toroidal)^{\star}$$
        on which we are able to define a \textit{non-degenerate} and invariant symmetric bilinear form:
            $$(-, -)_{\extendedtoroidal}$$
        that pairs elements of $\z(\toroidal)$ (which were causing the degeneracy issue) with those of its graded dual $\z(\toroidal)^{\star}$ non-trivially. Then, just like above, it can be shown that $\z(\toroidal)^{\star}$ is graded-isomorphic as a vector space to a certain Lie subalgebra $\der_{\gamma}(\bbC[v^{\pm 1}, t^{\pm 1}])$ of the Lie algebra $\der(\bbC[v^{\pm 1}, t^{\pm 1}])$ of derivations on $\bbC[v^{\pm 1}, t^{\pm 1}]$ (where the Lie structure is given by commutators). This, in turn, allows us to endow any vector space that is isomorphic to $\extendedtoroidal$ with a natural Lie algebra structure, coming from those on $\toroidal$ and on $\der_{\gamma}(\bbC[v^{\pm 1}, t^{\pm 1}])$. A \textbf{$\gamma$-extended toroidal Lie algebra} structure shall then be a Lie bracket:
            $$[-, -]_{\extendedtoroidal}$$
        on $\extendedtoroidal$, with respect to which the bilinear form $(-, -)_{\extendedtoroidal}$ is \textit{invariant} and the UCE $\toroidal$ becomes a Lie subalgebra of $\extendedtoroidal$ (cf. definition \ref{def: yangian_extended_toroidal_lie_algebras}).
        
        With all of these ingredients in place, we will be able to establish and subsequently prove the first main theorem of the thesis. A slightly imprecise version is as follows.
        \begin{theorem}[Cf. theorem \ref{theorem: yangian_extended_toroidal_lie_algebras_main_theorem}]
            A given Lie algebra will be a $\gamma$-extended toroidal Lie algebra if and only if it is isomorphic to a twisted semi-direct product:
                $$\toroidal \rtimes^{\sigma} \der_{\gamma}(\bbC[v^{\pm 1}, t^{\pm 1}])$$
            whose corresponding Lie $2$-cocycle $\sigma \in Z^2_{\Lie}(\der_{\gamma}(\bbC[v^{\pm 1}, t^{\pm 1}]), \toroidal)$ satisfies a certain invariance property that depends on $\gamma$.
        \end{theorem}

        The most important example of a $\gamma$-extended toroidal Lie algebra (at least for us) is the semi-direct product $\toroidal \rtimes \der_{\gamma}(\bbC[v^{\pm 1}, t^{\pm 1}])$, but we will also be able to provide an example of a $\gamma$-extended toroidal Lie algebra that is not isomorphic to this semi-direct product, which will be our second main result (see theorem \ref{theorem: billig_cocycle_main_theorem}) by analysing a particular Lie $2$-cocycle of $\der(\bbC[v^{\pm 1}, t^{\pm 1}])$ with values in $\z(\toroidal)$, which has been known since a work of Yuly Billig in the early 2000s, namely \cite{billig_energy_momentum_tensor}. 

    \subsection{History}
        Even though the above relationship between our construction of $\gamma$-extended toroidal Lie algebras and the older loop realisations of untwisted affine Kac-Moody algebras can indeed serve as a motivation for the main topic of this thesis all on its own, this was actually not how we were lead to considering these $\gamma$-extended toroidal Lie algebras.

        Back in July of 2023, the author was made aware\footnote{Thanks to Prof. Dr. Curtis Wendlandt and Prof. Dr. Alex Weekes.} of the idea that a good candidate for the classical limit of the so-called \say{affine Yangian}:
            $$\rmY_{\hbar}(\hat{\g})$$
        - a certain quantum group associated with the affine Kac-Moody algebra $\hat{\g}$ - would be a Lie bialgebra structure on:
            $$\toroidal^{\positive} := \uce(\g[v^{\pm 1}, t])$$
        This, in turn, came from an earlier realisation dating to \cite{wendlandt_formal_shift_operators_on_yangian_doubles} and \cite{guay_nakajima_wendlandt_affine_yangian_vertex_representations_and_PBW} that a kind of \say{Hopf-double} $\rmD\rmY_{\hbar}(\hat{\g})$ of the aforementioned affine Yangian $\rmY_{\hbar}(\hat{\g})$ should have $\toroidal$ as in the previous subsection as its classical limit. 
        
        Let us recall that under suitably mild finiteness assumptions, there is a bijective correspondence between so-called \textbf{Manin triples}, which are certain triples of Lie algebras:
            $$(\p, \p^+, \p^-)$$
        and isomorphism classes of Lie bialgebra structures on $\p^+$ (and indeed, on $\p$ and $\p^-$) as well (cf. \cite{etingof_kazhdan_quantisation_1}). One key detail here is that in defining such Manin triples, one is required to supply a \textit{non-degenerate} and \textit{invariant} symmetric bilinear form on $\toroidal$ (sastisfying certain conditions). As elaborated on earlier, any invariant symmetric bilinear form on a UCE such as $\toroidal$ is necessarily degenerate, so we ought not to attempt to directly construct a Manin triple of the form:
            $$(\toroidal, \toroidal^{\positive}, \toroidal^{\negative})$$
        We do, however, now have an extension of $\toroidal$ on which there is a non-degenerate and invariant symmetric bilinear form, namely $\extendedtoroidal$ equipped with $(-, -)_{\extendedtoroidal}$, and so an alternative strategy is to construct a Manin triple of the form:
            $$(\extendedtoroidal, \extendedtoroidal^{\positive}, \extendedtoroidal^{\negative})$$
        (wherein $\extendedtoroidal^{\positive} := \toroidal^{\positive} \rtimes \der_{\gamma}(\bbC[v^{\pm 1}, t])$), which helps us construct a Lie cobracket:
            $$\delta^{\positive}: \extendedtoroidal^{\positive} \to \extendedtoroidal^{\positive} \hattensor_{\bbC} \extendedtoroidal^{\positive}$$
        (where $\hattensor_{\bbC}$ denotes a suitable topological completion of the algebraic tensor product) that is compatible with the Lie bracket on $\extendedtoroidal^{\positive}$, thus making it a Lie bialgebra. Now, an easy consequence of our main theorem on $\gamma$-extended toroidal Lie algebras is that $\toroidal$ is a Lie ideal of any such Lie algebra. From this, one is able to infer that $\toroidal^{\positive}$ is a Lie ideal of $\extendedtoroidal^{\positive}$. However, this is not the end of it: $\toroidal^{\positive}$ can be shown to furthermore be a \say{Lie coideal} $\extendedtoroidal^{\positive}$, and hence a Lie sub-bialgebra of $\extendedtoroidal^{\positive}$.

        Finally, to make sense of the Lie bialgebra $(\toroidal^{\positive}, \delta^{\positive})$ as the classical limit of the affine Yangian $\rmY_{\hbar}(\hat{\g})$, the latter must be endowed with a bialgebra structure/comultiplication map (in the sense of say, \cite[Chapter 3]{kassel_quantum_groups}):
            $$\Delta: \rmY_{\hbar}(\hat{\g}) \to \rmY_{\hbar}(\hat{\g}) \hattensor_{\bbC} \rmY_{\hbar}(\hat{\g})$$
        such that the following \say{classical reduction} identity (cf. \cite{etingof_kazhdan_quantisation_1}) is satisfied for all $X \in \toroidal^{\positive}$:
            $$\delta^+(X) = \frac{1}{\hbar}( \Delta - \Delta^{\cop} )(\tilde{X}) \pmod{\hbar}$$
        where $\Delta^{\cop} = (1 \: 2) \circ \Delta$, with $(1 \: 2)$ being the map switching the two tensor factors, and $\tilde{X} \in \rmY_{\hbar}(\hat{\g})$ is a lift of $X$ (as an associative algebra, $\rmY_{\hbar}(\hat{\g})$ is supposed to be such that $\rmY_{\hbar}(\hat{\g})/\hbar \cong \rmU(\toroidal^{\positive})$). Somewhat surprisingly, it seems that no such comultiplication $\Delta$ exists for $\rmY_{\hbar}(\hat{\g})$. However, there does exist a version that is \say{almost} a bialgebra comultiplication, in the sense that only a slightly weaker version of the coassociativity condition (which normally defines comultiplications) is satisfied by $\Delta$ (cf. \cite{guay_nakajima_wendlandt_affine_yangian_coproduct}). Nevertheless, the classical reduction identity mentioned above is still satisfied, and thus a kind of classical limit of $\rmY_{\hbar}(\hat{\g})$ is found.

        \todo[inline]{Compare the situation to that of finite-type Yangians.}

    \subsection{What have we \textit{not} done ?}