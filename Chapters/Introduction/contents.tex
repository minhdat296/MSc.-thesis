\section{How and why have we constructed \texorpdfstring{$\gamma$}{}-extended toroidal Lie algebras ?}
    \subsection{Background}
        A rather well-known story in the representation theory of Lie algebras over algebraically closed fields of characteristic $0$ (e.g. $\bbC$, which from now on will be the default underlying field) is that of finite-dimensional semi-simple Lie algebras. This theory dates back to the works of Wilhelm Killing and \'Elie Cartan in the late $19^{th}$ century and early $20^{th}$ century on the classification of finite-dimensional semi-simple Lie algebras over $\bbC$. A salient feature of, and indeed, a very important technical ingredient in this theory, is an essentially unique\footnote{I.e. unique up to non-zero scalar multiples.} \textit{non-degenerate} and \textit{invariant} symmetric bilinear form that any finite-dimensional semi-simple Lie algebra can be endowed with. This is the famous \textbf{Killing form} (named after Wilhelm Killing) and using it, one is able to more-or-less develop the entire structural and representation theory of these Lie algebras. In particular, using the Killing form, one is able to construct the so-called \textbf{Cartan matrix} (named after \'Elie Cartan), which contains within it a lot - if not even all - of the important information about finite-dimensional semi-simple Lie algebras. For instance, should one know only of a Cartan matrix, one can then write down a presentation in terms of generators and relations (which is due to Claude Chevalley and Jean-Pierre Serre) for a uniquely corresponding finite-dimensional semi-simple Lie algebras. For more details, we refer the reader to \cite{humphreys_lie_algebras} and the first half of \cite{carter_affine_lie_algebras}, and we have also given a brief account of this story in subsection \ref{subsection: finite_dimensional_simple_lie_algebras}.

        Now, one particular property that the Cartan matrix of a finite-dimensional semi-simple Lie algebra is that it is \textit{positive-definite}. However, it is also possible to construct similar matrices that are not necessarily positive-definite, which are nowadays commonly called \textbf{generalised Cartan matrices} (cf. \cite[Chapter 1]{kac_infinite_dimensional_lie_algebras}). Should we then insist nevertheless on writing down presentations \textit{\`a la} Chevalley-Serre nevertheless, we would obtain certain infinite-dimensional Lie algebras that are nowadays known as \textbf{Kac-Moody algebras} (cf. \cite[Chapter 1]{kac_infinite_dimensional_lie_algebras}). It should be noted right away that generalised Cartan matrices - true to their names - admit the Cartan matrices of finite-dimensional semi-simple Lie algebras as special cases. Consequently, finite-dimensional semi-simple Lie algebras are certain instances of Kac-Moody algebras, and they are commonly referred to as \textbf{finite-type} Kac-Moody algebras when considered in this context.
        
        Amongst the Kac-Moody algebras, those of so-called \textbf{affine type} are of special interest. Specifically, as Robert Moody and Victor Kac independently discovered in the late 1960s and early 1970s, one obtains such Lie algebras by requiring that the associated Cartan matrix be \textit{positive-semi-definite}. What is rather remarkable about these affine Kac-Moody algebras is that they admit something called \textbf{loop realisations}: starting with a finite-dimensional semi-simple Lie algebra:
            $$\g$$
        with Cartan matrix $C$, one can firstly form the \textbf{loop algebra}:
            $$\g[v^{\pm 1}] := \g \tensor_{\bbC} \bbC[v^{\pm 1}]$$
        - which will always be equipped with a non-degenerate and invariant symmetric bilinear form originating naturally from the Killing form - then consider its universal central extension (UCE), which happens to be by a $1$-dimensional centre, say:
            $$\bbC c_{\aff}$$
        for some $c_{\aff} \in \bbC^{\x}$, and then finally adding on a Lie derivation:
            $$D_{\aff} \in \der(\g[v^{\pm 1}] \oplus \bbC c_{\aff})$$
        with the purpose of it all being so that at the end of the process, one shall obtain a Lie algebra:
            $$\hat{\g} := (\g[v^{\pm 1}] \oplus \bbC c_{\aff}) \rtimes \bbC D_{\aff}$$
        that is isomorphic to the Kac-Moody algebra whose Cartan matrix is obtained from $C$ by adding one extra row and one extra column in a certain manner (for more details, see \cite[Chapter 7]{kac_infinite_dimensional_lie_algebras}). The idea here is that, because any invariant symmetric bilinear form on the UCE $\g[v^{\pm 1}] \oplus \bbC c_{\aff}$ is necessarily degenerate and with radical being equal to the subspace $\bbC c_{\aff}$, one introduces the extra element $D_{\aff}$ to pair non-trivially with $c_{\aff}$, thereby fixing the issue of degeneracy; in other words, $D_{\aff}$ is to be dual to $c_{\aff}$ to begin with. The fact that $D_{\aff}$ is a Lie derivation on $\g[v^{\pm 1}] \oplus \bbC c_{\aff}$ is actually a consequence of this construction.
        
        As a brief aside, let us remark that the loop realisations of untwisted affine Kac-Moody algebras are incredibly useful in practice. For instance, they are useful for showing that all affine Kac-Moody algebras arise as \say{twists}\footnote{In the sense of \cite[Chapter 8]{kac_infinite_dimensional_lie_algebras}.} of the untwisted ones, and how these twists can be classified in terms of Galois cohomology (see \cite{pianzola_vanishing_of_H1_of_dedekind_rings} and its sequels\footnote{... and our thanks to A. Pianzola for letting us know of such results!}). For such reasons, we believe that the objects of main interest in this thesis, which arise from a double-loop realisation, are worthy of attention for reasons beyond their original motivation, which will be discussed in subsection \ref{subsection: history} below.

    \subsection{What is done in this thesis ?}
        Our starting point is not the single-loop algebra $\g[v^{\pm 1}]$ as above, but rather the \textbf{double-loop algebra}\footnote{Which are \textit{not} instances of Kac-Moody algebras!}:
            $$\g[v^{\pm 1}, t^{\pm 1}]$$
        which, like above, will also be equipped with a non-degenerate and invariant symmetric bilinear form originating naturally from the Killing form and depending on a distinguished linear map, a kind of modified formal residue map:
            $$\gamma: \bbC[v^{\pm 1}, t^{\pm 1}] \to \bbC$$
        on which our constructions will depend crucially (see subsection \ref{subsection: definition_of_yangian_extended_toroidal_lie_algebras}). We then again consider the UCE:
            $$\toroidal := \uce(\g[v^{\pm 1}, t^{\pm 1}])$$
        but a large difference in contrast to the affine Kac-Moody situation is that now, the centre $\z(\toroidal)$ is \textit{infinite-dimensional}; luckily though, it is graded, and the graded components are all finite-dimensional. Regardless, the issue whereby invariant symmetric bilinear forms on the UCE must be degenerate persists, which leads us to consider the vector space:
            $$\extendedtoroidal := \toroidal \oplus \z(\toroidal)^{\star}$$
        on which we are able to define a \textit{non-degenerate} and invariant symmetric bilinear form:
            $$(-, -)_{\extendedtoroidal}$$
        that pairs elements of $\z(\toroidal)$ (which were causing the degeneracy issue) with those of its graded dual $\z(\toroidal)^{\star}$ non-trivially. Then, just like above, it can be shown that $\z(\toroidal)^{\star}$ is graded-isomorphic as a vector space to a certain Lie subalgebra $\der_{\gamma}(\bbC[v^{\pm 1}, t^{\pm 1}])$ of the Lie algebra $\der(\bbC[v^{\pm 1}, t^{\pm 1}])$ of derivations on $\bbC[v^{\pm 1}, t^{\pm 1}]$ (where the Lie structure is given by commutators). This, in turn, allows us to endow any vector space that is isomorphic to $\extendedtoroidal$ with a natural Lie algebra structure, coming from those on $\toroidal$ and on $\der_{\gamma}(\bbC[v^{\pm 1}, t^{\pm 1}])$. A \textbf{$\gamma$-extended toroidal Lie algebra} structure shall then be a Lie bracket on $\extendedtoroidal$, with respect to which the bilinear form $(-, -)_{\extendedtoroidal}$ is \textit{invariant} and the UCE $\toroidal$ becomes a Lie subalgebra of $\extendedtoroidal$ (cf. definition \ref{def: yangian_extended_toroidal_lie_algebras}).
        
        With all of these ingredients in place, we will be able to establish and subsequently prove the first main theorem of the thesis. A slightly imprecise version is as follows.
        \begin{theorem}[Cf. theorem \ref{theorem: yangian_extended_toroidal_lie_algebras_main_theorem}]
            A given Lie algebra will be a $\gamma$-extended toroidal Lie algebra if and only if it is isomorphic to a twisted semi-direct product:
                $$\toroidal \rtimes^{\sigma} \der_{\gamma}(\bbC[v^{\pm 1}, t^{\pm 1}])$$
            whose corresponding Lie $2$-cocycle $\sigma \in Z^2_{\Lie}(\der_{\gamma}(\bbC[v^{\pm 1}, t^{\pm 1}]), \toroidal)$ satisfies a certain invariance property that depends on $\gamma$.
        \end{theorem}

        The most important example of a $\gamma$-extended toroidal Lie algebra (at least for us) is the semi-direct product $\toroidal \rtimes \der_{\gamma}(\bbC[v^{\pm 1}, t^{\pm 1}])$, but we will also be able to provide an example of a $\gamma$-extended toroidal Lie algebra that is not isomorphic to this semi-direct product, which will be our second main result (see theorem \ref{theorem: billig_cocycle_main_theorem}) by analysing a particular Lie $2$-cocycle of $\der(\bbC[v^{\pm 1}, t^{\pm 1}])$ with values in $\z(\toroidal)$, which has been known since a paper of Robert Moody and Senapathi Eswara Rao from 1990, namely \cite{moody_rao_n_toroidal_vertex_representations}.

    \subsection{History} \label{subsection: history}
        Even though the above relationship between our construction of $\gamma$-extended toroidal Lie algebras and the older loop realisations of untwisted affine Kac-Moody algebras can indeed serve as a motivation for the main topic of this thesis all on its own, this was actually not how we were lead to considering these $\gamma$-extended toroidal Lie algebras. Originally, we were motivated to consider $\gamma$-extended toroidal Lie algebras because they would help us realise a certain Lie bialgebra structure on:
            $$\toroidal^{\positive} := \uce(\g[v^{\pm 1}, t])$$
        as the classical limit of a quantum group:
            $$\rmY_{\hbar}(\hat{\g})$$
        known as the \textbf{Yangian} associated to the affine Kac-Moody algebra $\hat{\g}$, or simple \say{the} \textbf{affine Yangian} for short, when $\g$ is fixed.
        
        First of all, what is the affine Yangian $\rmY_{\hbar}(\hat{\g})$ ? This is a topological\footnote{A topological bialgebra is a bialgebra whose (co)multiplication maps are continuous.} bialgebra over $\bbC[\hbar]$ which deforms the universal enveloping algebra $\rmU(\toroidal^{\positive})$, in the sense that:
            $$\rmY_{\hbar}(\hat{\g})/\hbar \cong \rmU(\toroidal^{\positive})$$
        Per the general theory of quantisastions (as in \cite{etingof_kazhdan_quantisation_1}), the above tells us that $\rmY_{\hbar}(\hat{\g})$ is to quantise a topological \textbf{Lie bialgebra} structure:
            $$\delta^{\positive}: \extendedtoroidal^{\positive} \to \extendedtoroidal^{\positive} \hattensor_{\bbC} \extendedtoroidal^{\positive}$$
        on $\toroidal^{\positive}$; here, $\hattensor_{\bbC}$ denotes a suitable topological completion of the algebraic tensor product, which is necessary due to the appearance of certain infinite sums. Now, although it is certainly possible to endow $\toroidal^{\positive}$ with the Lie bialgebra structure coming directly from a topological bialgebra structure constructed in \cite{guay_nakajima_wendlandt_affine_yangian_coproduct}. However, the problem with this approach is that by the end of the process, it will not be clear whether or not the resulting Lie cobracket $\delta^{\positive}$ originates from the structure of $\toroidal^{\positive}$ itself. In turn, this will lead us towards difficulties in say, identifying the classical R-matrix of $\toroidal^{\positive}$ corresponding to the quantum R-matrix of the affine Yangian (which has been found in \cite{appel_gautam_wendlandt_R_matrices_of_affine_yangians}).
        
        Now, let us recall that under suitably mild finiteness assumptions, there is a bijective correspondence between so-called \textbf{Manin triples}, which are triples of Lie algebras $(\p, \p^+, \p^-)$ subjected to certain conditions, and isomorphism classes of Lie bialgebra structures on $\p^+$ (and indeed, on $\p$ and $\p^-$) as well (see \cite{etingof_kazhdan_quantisation_1} for more details). One key detail here is that in defining such Manin triples, one is required to supply a \textit{non-degenerate} and \textit{invariant} symmetric bilinear form on $\toroidal$ (sastisfying certain conditions). As elaborated on earlier, any invariant symmetric bilinear form on a UCE such as $\toroidal$ is necessarily degenerate, so we ought not to attempt to directly construct a Manin triple of the form $(\toroidal, \toroidal^{\positive}, \toroidal^{\negative})$. We do, however, now have an extension of $\toroidal$ on which there is a non-degenerate and invariant symmetric bilinear form, namely $\extendedtoroidal$ equipped with $(-, -)_{\extendedtoroidal}$ as constructed above, and so an alternative strategy is to construct a Manin triple of the form:
            $$(\extendedtoroidal, \extendedtoroidal^{\positive}, \extendedtoroidal^{\negative})$$
        (wherein $\extendedtoroidal^{\positive} := \toroidal^{\positive} \rtimes \der_{\gamma}(\bbC[v^{\pm 1}, t])$), which helps us construct the Lie cobracket $\delta^{\positive}$. Specifically, we can rely on the fact that $\toroidal \subset \extendedtoroidal$ is not just a Lie ideal but also a Lie coideal, and hence a Lie sub-bialgebra. 

        As an aside, let us note that the story that we have just outlined above runs in analogy with the more classical story of Drinfeld's construction of the Yangian $\rmY_{\hbar}(\g)$ of a finite-dimensional simple Lie algebra $\g$, as was done in \cite{drinfeld_original_yangian_paper}. This is the bialgebra quantising the Lie bialgebra structure on $\g[t]$ that is specified by the Manin triple $(\g[t^{\pm 1}], \g[t], t^{-1}\g[t^{-1}])$. Let us also remark that like in the affine setting where the classical limit is a UCE, $\g[t]$ is in fact also a UCE, namely the trivial UCE of itself (cf. example \ref{example: affine_lie_algebras_centres}).

    \subsection{What have we \textit{not} done ?}
        There are many natural questions that can be posed by the end of this thesis. We have chosen to highlight the following, which is a question that pertains directly to the latter parts of the thesis, particular to theorem \ref{theorem: billig_cocycle_main_theorem}.
        \begin{question}
            How many \say{$\gamma$-invariant} Lie $2$-cocycles (cf. definition \ref{def: yangian_toroidal_cocycles}) are there ? And for that matter, what is $H^2_{\Lie}(\der_{\gamma}(\bbC[v^{\pm 1}, t^{\pm 1}]), \z(\toroidal))$ ?
        \end{question}
        Some inspiration and guidance can perhaps be taken from \cite{billig_neeb_vector_field_cyclic_cohomology_parallelisable_manifolds}, where the authors have investigated the Lie algebra cohomology ($H^2_{\Lie}$, in particular) of the Lie algebra of all $C^{\infty}$-vector fields on a parallelisable smooth compact manifold $M$ with coefficients in the global section of $\Omega^p_M/d( \Omega^{p - 1}_M )$ (as we will see via theorem \ref{theorem: kassel_realisation}, $\z(\toroidal) \cong \Omega^1_{ \bbC[v^{\pm 1}, t^{\pm 1}]/\bbC }/d \bbC[v^{\pm 1}, t^{\pm 1}]$). That said, there is still much work to be done.

        Otherwise, we can continue pursuing the original goal of computing the classical limit of affine Yangians immediately after this thesis as well, and even though we have a somewhat detailed sketch of the proof at this point, there remain some difficulties at present. For instance, usually if one would like to prove that a topological bialgebra $(Y, \Delta)$ quantises a topological Lie bialgebra $(\fraku, \delta)$, then one would verify that the following equation holds true for all $x \in \fraku$ and any lift $\tilde{x} \in Y$ thereof:
            $$\delta(x) \equiv \frac{1}{\hbar}(\Delta - \Delta^{\cop})(\tilde{x}) \pmod{\hbar}$$
        However, it is still not known whether or not $\rmY_{\hbar}(\hat{\g})$ is torsion-free as a $\bbC[\hbar]$-module for all $\g$, even for those $\g$ for which $\rmY_{\hbar}(\hat{\g})$ is known to carry a topological bialgebra structure (see \cite[Section 5]{guay_nakajima_wendlandt_affine_yangian_coproduct}), meaning that the equation above might not make sense for all $\g$. When $\g$ is simply-laced, though, $\rmY_{\hbar}(\hat{\g})$ is known to be torsion-free (see \cite[Section 6]{guay_regelskis_wendlandt_affine_yangian_vertex_representations_and_PBW}).