\section{The structure of the thesis}
    We would also like to provide the reader with a short reading guide for the thesis.
    
    In chapter \ref{chapter: kassel_UCEs}, we begin by recalling some background information on perfect Lie algebras and on Lie algebra extensions, and then we will study UCEs in some detail, recalling in particular a realisation of Kassel, whereby centres of UCEs of current algebras (in the sense of definition \ref{def: current_algebras}) can be described in terms of algebraic differential $1$-forms modulo exact forms (see theorem \ref{theorem: kassel_realisation}). We will then end the chapter by analysing, using the aforementioned theorem, particular UCEs that will be of interest to us for the rest of the thesis (see, in particular, example \ref{example: toroidal_lie_algebras_centres}).

    Chapter \ref{chapter: yangian_EALAs} will begin with the construction of the technical ingredients necessary for defining $\gamma$-extended toroidal Lie algebras, and then of those Lie algebras themselves; the procedure will be as outlined in the previous section, and will be outlined in further details in the sections in chapter \ref{chapter: yangian_EALAs}; the main result of this portion of the chapter will be theorem \ref{theorem: yangian_extended_toroidal_lie_algebras_main_theorem}. Afterwards, we will be making some quick remarks about some structural features of $\gamma$-extended toroidal Lie algebras, such as an identification of its centre and intriguingly, a homomorphic image of the Witt algebra inside $\der_{\gamma}(\bbC[v^{\pm 1}, t^{\pm 1}])$ (see propositions \ref{prop: centres_of_yangian_extended_toroidal_lie_algebras} and \ref{prop: a_copy_of_the_witt_algebra_inside_the_lie_algebra_of_yangian_div_zero_vector_fields}, respectively). Finally, we will provide an example of a $\gamma$-extended toroidal Lie algebra that is \textit{not} isomorphic to $\toroidal \rtimes \der_{\gamma}(\bbC[v^{\pm 1}, t^{\pm 1}])$ by analysing a particular Lie $2$-cocycle known from \cite{billig_energy_momentum_tensor}; this will be done through theorem \ref{theorem: billig_cocycle_main_theorem}.