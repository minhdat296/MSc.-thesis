\section{The initial setup}
    Again, $\g$ is a finite-dimensional simple Lie algebra over an algebraically closed field of characteristic $0$. It is accompanied by all the data recalled in subsection \ref{subsection: finite_dimensional_simple_lie_algebras}. 

    \todo[inline]{All toroidal Lie algebra conventions have been moved here.}
    For our own convenience, we will also be adopting the following abbreviations:
        $$A := k[v^{\pm 1}, t^{\pm 1}], A^{\positive} := k[v^{\pm 1}, t]$$
        $$\bar{\Omega}_{[2]} := \bar{\Omega}^1_{A/k}, \bar{\Omega}_{[2]}^{\positive} := \bar{\Omega}^1_{A^{\positive}/k}$$
    and also that:
        $$\g_{[2]} := \g \tensor_k A, \g_{[2]}^{\positive} := \g \tensor_k A^{\positive}$$
    with both being understood as current algebras (in the sense of definition \ref{def: current_algebras}).

    We will then be interested in the Lie algebras:
        $$\toroidal := \uce(\g_{[2]})$$
        $$\toroidal^{\positive} := \uce(\g_{[2]}^{\positive})$$
    which, respectively, shall be referred to as the \textbf{toroidal Lie algebra} and \textbf{positive toroidal Lie algebra} associated to $\g$. In example \ref{example: toroidal_lie_algebras_centres} and remark \ref{remark: Z_gradings_on_toroidal_lie_algebras}, the structures of the underlying ($\Z$-grraded) vector spaces of these Lie algebras have already been described, and we refer the reader there for the details (in particular, the construction of a canonical basis for the underlying vector spaces of their centres). So that our notations would be suggestive, we shall be writing:
        $$\z_{[2]} := \z(\toroidal)$$
        $$\z_{[2]}^{\positive} := \z(\toroidal^{\positive})$$
    from now on; often, we might refer to these as the \textbf{(positive) toroidal centre}.
    
    Now, instead of being equipped with the usual residue bilinear form of degree $(0, 0)$, the Lie algebra $\g_{[2]}$ will be equipped with the residue bilinear form of degree $(0, -1)$:
        $$(x v^m t^p, y v^n t^q)_{\g_{[2]}} := (x, y)_{\g} \delta_{(m, p) + (n, q), (0, -1)}$$
    given for all $x, y \in \g$ and all $(m, p), (n, q) \in \Z^2$; sometimes, we will refer to this as the \textbf{Yangian form} or the \textbf{Yangian pairing}. It is easy to see that the bilinear form:
        $$(-, -)_{\g_{[2]}}$$
    is symmetric, non-degenerate, and invariant. Because $\toroidal$ has a non-trivial centre, any invariant (symmetric) bilinear form thereon (so in particular, any extension of $(-, -)_{\g_{[2]}}$ to $\toroidal$) is necessarily degenerate. The purpose of constructing \say{Yangian extended toroidal Lie algebras} is to remedy such degeneracy.