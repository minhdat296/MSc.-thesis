\section{The initial setup}
    Again, $\g$ is a finite-dimensional simple Lie algebra over an algebraically closed field of characteristic $0$. It is accompanied by all the data recalled in subsection \ref{subsection: finite_dimensional_simple_lie_algebras}. 

    \subsection{Conventions for toroidal Lie algebras}
        \todo[inline]{All toroidal Lie algebra conventions have been moved here.}
        For our own convenience, we will also be adopting the following abbreviations:
            $$A := k[v^{\pm 1}, t^{\pm 1}], A^{\positive} := k[v^{\pm 1}, t]$$
            $$\bar{\Omega}_{[2]} := \bar{\Omega}^1_{A/k}, \bar{\Omega}_{[2]}^{\positive} := \bar{\Omega}^1_{A^{\positive}/k}$$
        and also that:
            $$\g_{[2]} := \g \tensor_k A, \g_{[2]}^{\positive} := \g \tensor_k A^{\positive}$$
        with both being understood as current algebras (in the sense of definition \ref{def: current_algebras}).
    
        We will then be interested in the Lie algebras:
            $$\toroidal := \uce(\g_{[2]})$$
            $$\toroidal^{\positive} := \uce(\g_{[2]}^{\positive})$$
        which, respectively, shall be referred to as the \textbf{toroidal Lie algebra} and \textbf{positive toroidal Lie algebra} associated to $\g$. In example \ref{example: toroidal_lie_algebras_centres} and remark \ref{remark: Z_gradings_on_toroidal_lie_algebras}, the structures of the underlying ($\Z$-graded) vector spaces of these Lie algebras have already been described, and we refer the reader there for the details (in particular, the construction of a canonical basis for the underlying vector spaces of their centres). So that our notations would be suggestive, we shall be writing:
            $$\z_{[2]} := \z(\toroidal)$$
            $$\z_{[2]}^{\positive} := \z(\toroidal^{\positive})$$
        from now on; often, we might refer to these as the \textbf{(positive) toroidal centre}.
        
        Now, instead of being equipped with the usual residue bilinear form of degree $(0, 0)$, the Lie algebra $\g_{[2]}$ will be equipped with the residue bilinear form of degree $(0, -1)$:
            $$(x v^m t^p, y v^n t^q)_{\g_{[2]}} := (x, y)_{\g} \delta_{(m, p) + (n, q), (0, -1)}$$
        given for all $x, y \in \g$ and all $(m, p), (n, q) \in \Z^2$; sometimes, we will refer to this as the \textbf{Yangian form} or the \textbf{Yangian pairing}. It is easy to see that the bilinear form:
            $$(-, -)_{\g_{[2]}}$$
        is symmetric, non-degenerate, and invariant. Because $\toroidal$ has a non-trivial centre, any invariant (symmetric) bilinear form thereon (so in particular, any extension of $(-, -)_{\g_{[2]}}$ to $\toroidal$) is necessarily degenerate (see remark \ref{remark: extending_bilinear_forms_to_central_extensions}). The purpose of constructing \say{Yangian extended toroidal Lie algebras} is to remedy such degeneracy.

    \subsection{Untwisted affine Kac-Moody algebras} \label{subsection: a_fixed_untwisted_affine_kac_moody_algebra}
        \todo[inline]{All untwisted affine Kac-Moody algebra conventions have been moved here.}
        A construction that we will be making use of frequently is the untwisted affine Kac-Moody algebra associated to a finite-dimensional simple Lie algebra. Let us briefly recall how it is constructed. For details, we refer the reader to \cite[Chapters 6 and 7]{kac_infinite_dimensional_lie_algebras}.

        Recall from example \ref{example: affine_lie_algebras_centres} that the Lie algebra:
            $$\uce(\g[v^{\pm 1}]) \cong \g[v^{\pm 1}] \oplus k c_v$$
        (with $c_v := v^{-1} \bar{d}v$ as in example \ref{example: toroidal_lie_algebras_centres}) carries a uniquely determined invariant symmetric bilinear form $(-, -)_{\uce(\g[v^{\pm 1}])}$ which is necessarily degenerate (see remark \ref{remark: extending_bilinear_forms_to_central_extensions}). However, by adding an extra element $D_{\aff}$ to the Lie algebra $\uce(\g[v^{\pm 1}])$ and by requiring that:
            $$(c_v, D_{\aff})_{\uce(\g[v^{\pm 1}]) \oplus k D_{\aff}} = 1$$
            $$(c_v, c_v)_{\uce(\g[v^{\pm 1}]) \oplus k D_{\aff}} = (D_{\aff}, D_{\aff})_{\uce(\g[v^{\pm 1}]) \oplus k D_{\aff}} = 0$$
        one obtains the \textbf{untwisted affine Kac-Moody algebra} (in the sense of \cite[Chapter 7]{kac_infinite_dimensional_lie_algebras}):
            $$\hat{\g} := \uce(\g[v^{\pm 1}]) \rtimes k D_{\aff}$$
        The element $D_{\aff}$ can be shown to be the derivation on $\g[v^{\pm 1}]$ given by $\id_{\g} \tensor v \frac{d}{dv}$\footnote{Note that therefore $D_{\aff} \not = D_v$.}. It can also be shown that:
            $$[D_{\aff}, c_v]_{\hat{\g}} = 0$$

        We shall also fix once and for all the following non-degenerate, invariant, and symmetric $k$-bilinear form\footnote{Actually, any Kac-Moody algebra admits such a bilinear form, as noted in \cite[Chapter 2]{kac_infinite_dimensional_lie_algebras}. When the Kac-Moody algebra in question is of an untwisted affine type, the formula for bilinear form will be as we are presenting here.} on $\hat{\g}$:
            $$(-, -)_{\hat{\g}} := (-, -)_{\uce(\g[v^{\pm 1}]) \rtimes D_{\aff}}$$
        By construction, this restricts to $(-, -)_{\uce(\g[v^{\pm 1}])}$ on $\uce(\g[v^{\pm 1}])$, and for how this bilinear form is given, we refer the reader to the latter portion of example \ref{example: affine_lie_algebras_centres}. Also by construction\footnote{... and due also to the fact that $(-, -)_{\g}$ is non-degenerate when restricted down to $\h$.}, this bilinear form remains non-degenerate on the subspace:
            $$\hat{\h} := \h \oplus k c_v \oplus k D_{\aff}$$
        and hence induces a vector space isomorphism:
            $$\hat{\h} \xrightarrow[]{\cong} \hat{\h}^*$$
        given by:
            $$h \mapsto (h, -)_{\hat{\g}}$$
        It is not hard to see that $\hat{\h}$ is a maximal abelian Lie subalgebra of $\hat{\g}$ whose elements are $\ad$-diagonalisable, so it is reasonable to refer to $\hat{\h}$ as a \textbf{Cartan subalgebra} of $\hat{\g}$, in analogy with the Cartan subalgebra $\h$ of $\g$. The choice of Cartan subalgebra $\hat{\h}$ is fixed once and for all, and while it is true that all $\ad$-diagonalisable maximal abelian Lie subalgebras of $\hat{\g}$ are conjugate to one another in this setting as well (like for $\g$), this assertion is much more difficult (see \cite{kac_peterson_infinite_flag_varieties_and_conjugacy_of_cartan_subalgebras}), so we shall settle for defining \say{the} Cartan subalgebra of $\hat{\g}$ in the somewhat \textit{ad hoc} manner above.
        
        Now, any choice of basis for $\hat{\h}^*$ can be regarded as a choice of \textbf{simple roots} for $\hat{\g}$; let us fix once and for all such a basis:
            $$\{\alpha_i\}_{1 \leq i \leq \dim_k \hat{\h}}$$
        Like with finite-dimensional simple Lie algebras such as $\g$, this datum allows one to construct a Cartan matrix for $\hat{\g}$ in the same manner as was done for $\g$ in subsection \ref{subsection: finite_dimensional_simple_lie_algebras}:
            $$\hat{C} := (c_{ij})_{1 \leq i, j \leq \dim_k \hat{\h}} := \left( \frac{2(\alpha_i, \alpha_j)_{\hat{\g}}}{(\alpha_i, \alpha_i)_{\hat{\g}}} \right)_{1 \leq i, j \leq \dim_k \hat{\h}}$$
        Likewise, it can be shown that $\hat{C}$ is symmetrisable in the sense that there exists an invertible diagonal matrix:
            $$\hat{D} := (d_{ij})_{1 \leq i, j \leq \dim_k \hat{\h}} = \left(\frac{2\delta_{i, j}}{(\alpha_i, \alpha_j)_{\hat{\g}}}\right)_{1 \leq i, j \leq \dim_k \hat{\h}}$$
        and a symmetric matrix:
            $$\hat{A} := (a_{ij})_{1 \leq i, j \leq \dim_k \hat{\h}} = \left((\alpha_i, \alpha_j)_{\hat{\g}}\right)_{1 \leq i, j \leq \dim_k \hat{\h}}$$
        (which is nothing but the matrix representation of the bilinear form $(-, -)_{\hat{\g}}$ with respect to the basis $\{\alpha_i\}_{1 \leq i \leq \dim_k \hat{\h}}$), such that:
            $$\hat{C} = \hat{D} \hat{A}$$
        Furthermore, we have that $2\id - \hat{A}$ is the adjacency matrix of an undirected graph without loops, also typically referred to as the \textbf{Dynkin diagram} of $\hat{\g}$, and the \textbf{roots} of $\hat{\g}$ are the roots of this graph. The set of roots is denoted by:
            $$\hat{\Phi}$$
            
        Observe that because the vector space isomorphism $\hat{\h} \xrightarrow[]{\cong} \hat{\h}^*$ given by $h \mapsto (h, -)_{\hat{\g}}$ restricts down to an isomorphism $\h \mapsto \h^*$ given by the same formula, due to the construction of the bilinear form $(-, -)_{\hat{\g}}$ (cf. example \ref{example: affine_lie_algebras_centres}), there exists an injection:
            $$\Phi \subset \hat{\Phi}$$
        Let us denote its image by:
            $$\Re(\hat{\Phi})$$
        and refer to the elements in this image (i.e. regard the roots of $\g$) as \textbf{real roots} of $\hat{\g}$.
        
        Now, a point of deviation from the finite-dimensional theory is that the Cartan matrix $\hat{C}$ is only positive-semi-definite in general, and since positive-semi-definite matrices are positive-definite if and only if they are invertible, $\hat{C}$ is generally not of full rank. Let us write:
            $$\hat{l} := \rank \hat{C} \leq \dim_k \hat{\h}$$
        and it will also be convenient to have the notation:
            $$l := \rank C = \dim_k \h$$
        Using the construction of the the non-degenerate bilinear form $(-, -)_{\hat{\g}}$ from the degenerate form $(-, -)_{\uce(\g[v^{\pm 1}])}$ - particularly how one as a single extra basis element $D_{\aff}$ to $\uce(\g[v^{\pm 1}])$ - it can be shown that:
            $$\hat{l} = l + 1$$
        $\hat{C}$ is therefore \textit{not of full rank}, and hence only \textit{positive-semi-definite}. The set of simple roots of $\hat{\g}$ can now be defined to be the following linearly independent subset of $\hat{\h}^*$:
            $$\hat{\simpleroots} := \simpleroots \cup \{-\theta + \delta\}$$
        where $\delta$ is the linear functional on $\hat{\h}$ given by:
            $$\delta := (c_v, -)_{\hat{\g}}$$
        (cf. \cite[Section 7.4, p. 100]{kac_infinite_dimensional_lie_algebras}) is typically referred to as the \textbf{lowest positive imaginary root}. It can be shown that elements of the set:
            $$\Im(\hat{\Phi}) := \Z\delta \setminus \{0\}$$
        are in fact roots of $\hat{\g}$ (i.e. that $\Z\delta \setminus \{0\} = \Z\delta \cap \hat{\Phi}$); elements of this set are usually called \textbf{imaginary roots}. We will explain why the extra simple root needs to be $-\theta + \delta$ (as opposed to, say, $\delta$) shortly; for now, let us remark that this choice results in the slightly unfortunate fact that the lowest imaginary root $\delta$ is \textit{not} simple, though there is certainly a bijection $\hat{\simpleroots} \cong \simpleroots \cup \{\delta\}$.

        An intrinsic way to characterise real and imaginary roots is that $\beta \in \Im(\hat{\Phi})$ if and only if $(\beta, \beta)_{\hat{\g}} = 0$ and $\beta \in \Re(\hat{\Phi})$ otherwise. The set of roots of $\hat{\g}$ thus decomposes into a disjoint union of the subsets of real and of imaginary roots:
            $$\hat{\Phi} = \Re(\hat{\Phi}) \cup \Im(\hat{\Phi})$$
        
        The root lattice of $\hat{\g}$ can now be given as:
            $$\hat{Q} := \Z\hat{\simpleroots}$$
        Positive/negative roots are thus elements of:
            $$\hat{\Phi}^+ := \hat{\Phi} \cap \hat{Q}^{\pm}$$
        where:
            $$\hat{Q}^{\pm} := \pm\Z_{\geq 0}\hat{\simpleroots}$$
        Simple roots, in particular, are conventionally \textit{positive}.

        The $Q$-grading on $\g$ and the natural $\Z$-grading on $k[v^{\pm 1}]$ induce, together, a $Q \x \Z$-grading on $\g$. Explicitly, for each $\lambda \in \Phi$, each $x \in \g_{\alpha}$, and each $m \in \Z$, one has that:
            $$\deg x v^m = (\alpha, m)$$
        Following \cite[Chapter 6]{kac_infinite_dimensional_lie_algebras}, we know that there is an isomorphism of $\Z$-modules:
            $$\hat{Q} \xrightarrow[]{\cong} Q \x \Z$$
            $$\alpha + m\delta \mapsto (\alpha, m)$$
        (given for all $\alpha \in \Phi$ and all $m \in \Z$). As such, $\g$ can be equivalently viewed as being $\hat{Q}$-graded in the sense that for each $\alpha \in Q$, each $x \in \g_{\alpha}$, and each $m \in \Z$, one has that:
            $$\deg x v^m = \alpha + m\delta$$
        Additionally, let us recall from \cite[Chapter 7]{kac_infinite_dimensional_lie_algebras} that the root space decomposition of the untwisted affine Kac-Moody algebra $\hat{\g}$ takes the form:
            $$\hat{\g} \cong \hat{\h} \oplus \bigoplus_{\beta \in \Re(\hat{\Phi})} \hat{\g}_{\beta} \oplus \bigoplus_{\beta \in \Im(\hat{\Phi})} \hat{\g}_{\beta}$$
        with the root spaces being given by:
            $$\forall \alpha + m\delta \in \Re(\hat{\Phi}): \hat{\g}_{\alpha + m\delta} \cong \g_{\alpha} v^m$$
            $$\forall r\delta \in \Im(\hat{\Phi}): \hat{\g}_{r\delta} \cong \h v^r$$
        In particular, this tells us that real roots are all of multiplicity $1$ (since $\dim_k \g_{\alpha} = 1$ for all $\alpha \in \Phi$), while imaginary roots are of multiplicity $l$.
            
        Finally, let us note that $\hat{\g}$ is isomorphic to the Lie algebra generated by a set of \textbf{Chevalley-Serre generators} (cf. \cite[Theorems 1.2, 1.4, and 9.11]{kac_infinite_dimensional_lie_algebras}\footnote{Theorem 9.11 is where the Serre relations for symmetrisable Kac-Moody algebras are established.}). To describe these generators, write:
            $$\alpha_0 := -\theta + \delta$$
        $\hat{\g}$ is then isomorphic to the Lie algebra generated by the set:
            $$\{x_i^{\pm}, h_i\}_{i \in \hat{\simpleroots}}$$
        whose elements are subjected to the same relations as in theorem \ref{theorem: serre_theorem_for_finite_dimensional_simple_lie_algebras} (but now given for $i, j \in \hat{\simpleroots}$); the isomorphism in question is given by
            $$x_0^{\pm} \mapsto x_{\mp \theta} v^{\pm 1}, h_0 \mapsto -\check{\theta} + c_v$$
        where the root vectors $x_{\pm \theta} \in \g_{\pm \theta}$ have been chosen so that $(x_{\theta}, x_{-\theta})_{\g} = 1$, per the convention made in subsection \ref{subsection: finite_dimensional_simple_lie_algebras}. As for why the highest root $\theta \in \Phi$ might appear here in the first place, this is simply so that simultaneously, $h_0$ is not central in $\hat{\g}$ (which in particular, would imply that $h_0 \in k c_v$, which is absurd), and that:
            $$
                [h_0, x_i]_{\hat{\g}} =
                \begin{cases}
                    \text{$0$ if $i \in \simpleroots$}
                    \\
                    \text{$(-\theta + \delta, -\theta + \delta)_{\hat{\g}} = (\theta, \theta)_{\hat{\g}} \not = 0$ if $i = 0$}
                \end{cases}
            $$
        with the first case holding true thanks to the fact that the highest root $\theta$ being orthogonal to every other root $\alpha \in \Phi$ of the underlying finite-dimensional simple Lie algebra $\g$. Note also that, indeed, we have that:
            $$[x_0^+, x_0^-]_{\hat{\g}} = -\check{\theta} + c_v = h_0$$
        as well as:
            $$
                [h_i, x_0^{\pm}]_{\hat{\g}}
                =
                \begin{cases}
                    \text{$\mp \theta(h_i) x_{\mp \theta} v^{\pm 1}$ if $i \in \simpleroots$}
                    \\
                    \text{$\mp \theta(\check{\theta}) x_{\mp \theta} v^{\pm 1}$ if $i = 0$}
                \end{cases}
                =
                \pm (-\theta + \delta)(h_i) x_{\mp \theta} v^{\pm 1} = \pm \alpha_0(h_i) x_{\mp \theta} v^{\pm 1}
            $$
        for all $i \in \hat{\simpleroots}$, with the second-to-last equality holding thanks to the fact that $\delta := (c_v, -)_{\hat{\g}}$ and the fact that $(c_v, \g[v^{\pm 1}])_{\hat{\g}} = (c_v, \g[v^{\pm 1}])_{\uce(\g[v^{\pm 1}])} = 0$; this gives one reason for choosing the single extra simple root to be $-\theta + \delta$. The second reason is that we would like the \textbf{Chevalley involution}:
            $$\omega \in \Aut_{\Lie\Alg}(\hat{\g})$$
        (cf. \cite[p. 7]{kac_infinite_dimensional_lie_algebras}) to act on the Chevalley-Serre generators by:
            $$\omega(x_i^{\pm}) = -x_i^{\mp}, \omega(h_i) = -h_i$$
        for all $i \in \hat{\simpleroots}$, which forces $h_0 = -\check{\theta} + c_v$ and consequently, forces the extra simple root to be $-\theta + \delta$.
        
        \todo[inline]{Added explanation for why the extra simple root of $\hat{\g}$ is $-\theta + \delta$.}