\section{Some generalities on Lie algebras}
    \begin{convention}
        A very common notation that we will be liberally employing is that:
            $$\ad(x) := [x, -]$$
        for $x$ being an element of some Lie algebra. 

        Another common notation in the context of Lie algebras (over a fixed field) is $\gl(V)$, which is the Lie algebra of endomorphisms on a vector space $V$, with Lie bracket given by commutators.

        A particularly common representation of a Lie algebra $\a$ is the \textbf{adjoint representation}, which we will be denoting by:
            $$\ad: \a \to \gl(\a)$$
        and is given by $x \mapsto \ad(x)$.
    \end{convention}

    \subsection{Structure of finite-dimensional simple Lie algebras} \label{subsection: finite_dimensional_simple_lie_algebras}
        As a precursor to our main discussion, let us recall some features of the theory of finite-dimensional simple Lie algebras, particularly about their structure.

        We will be working over a field $k$ that is algebraically closed and of characteristic $0$, and no serious loss of generality will come from assuming, say, that $k \cong \bbC$. The assumption that $k$ is algebraically closed is so that we will have enough eigenvalues and hence, certain operators will be diagonalisable. The second assumption, that $\chara k = 0$, is made so that we can avoid having certain relations, e.g. as in theorem \ref{theorem: serre_theorem_for_finite_dimensional_simple_lie_algebras}, vanish.

        \begin{definition}[Simple Lie algebras]
            A Lie algebra is said to be \textbf{simple} if and only if it admits no non-zero Lie ideals. 
        \end{definition}

        Over a field $k$ that is algebraically closed and of characteristic $0$ (which can just be taken to be $\bbC$), much is known about the structure of a simple Lie algebra $\g$ that is finite-dimensional when regarded as a $k$-vector space. The bulk of the content presented above is discussed in further details in any standard textbook on Lie algebras (cf. e.g. \cite{humphreys_lie_algebras} or the first half of \cite{carter_affine_lie_algebras}). Let us give a very brief recap of this theory.

        One begins this process by choosing a \textbf{Cartan subalgebra} $\h$ for $\g$, which is a maximal abelian Lie subalgebra $\h$\footnote{We note also that, it is well-known that all Cartan subalgebras of $\g$ are conjugate to one another.}, whose elements are diagonalisable under the adjoint representation $\ad: \g \to \gl(\g)$. Now, let $V$ be a $\g$-module. Then, one can abstractly define the vector subspace of $V$ consisting of elements of \textbf{weight} $\mu \in \h^*$ to be:
            $$V_{\mu} := \{v \in V \mid \forall h \in \h: h \cdot v = \mu(h) v\}$$
        If we have a direct sum decomposition of $\h$-module:
            $$V \cong \bigoplus_{\mu \in \h^*} V_{\mu}$$
        then we will say that $V$ is a \textbf{weight module} for $\g$. Interestingly, elements of $\g_{\alpha}$ (with $\g$ acting on itself by the adjoint action) act by raising/lowering the weights of elements of $\g$-modules $V$ in the sense that:
            $$\g_{\alpha} \cdot V_{\mu} \subseteq V_{\mu + \alpha}$$
        for all weights $\alpha, \mu \in \h^*$. 
        
        As it turns out, when $\g$ is a weight module over itself via the adjoint representation. This particular weight space decomposition is usually referred to as the \textbf{root space decomposition} of $\g$. 
        \begin{theorem}[Root space decomposition for finite-dimensional simple Lie algebras] \label{theorem: root_space_decomposition_for_finite_dimensional_simple_lie_algebras}
            Let $\g$ be a module over itself via the adjoint representation.
            \begin{enumerate}
                \item $\g$ is a weight module over itself. As a particular consequence, $\g$ is graded by the abelian group $Q$: for every $x \in \g_{\alpha}$, one has that $\deg x = \alpha$.
                \item The weight space $\g_0$ is nothing but the Cartan subalgebra $\h$.
                \item For each non-zero weight $\alpha$ of this $\g$-module, $\dim_k \g_{\alpha} = 1$.
            \end{enumerate}
        \end{theorem}
        Typically, the non-zero weights $\alpha$ of the adjoint representations of $\g$ such that $\g_{\alpha} \not \cong 0$ are called \textbf{roots}. The set of roots is usually denoted by $\Phi$. 

        One of the most important features of finite-dimensional simple Lie algebras (henceforth implicitly understood to be defined over a characteristic-$0$ and algebraically closed field $k$) is that each such Lie algebra, say $\g$, posses an invariant and non-degenerate $k$-bilinear form:
            $$(-, -)_{\g}$$
        which is unique up to $k^{\x}$-multiples. The canonical choice is the so-called Killing form, given by $\kappa(x, y) := \trace(\ad(x) \circ \ad(y))$ for all $x, y \in \g$, but in various other context, other natural choices such as the trace form $\trace(\rho(x) \rho(y))$ (associated to some representation $\rho: \g \to \gl(V)$) are also very useful. What is important to us is that the Killing form is essentially unique: if $\kappa'$ is any invariant and non-degenerate symmetric $k$-bilinear form on $\g$ then there will exist a \textit{unique} $c \in k^{\x}$ such that $\kappa' = c \kappa$.

        Now, such a bilinear form $(-, -)_{\g}$ allows us to associate to $\g$ a \say{root system} (to be defined shortly), and the upshot is that these \say{roots systems} classify finite-dimensional simple Lie algebras (over algebraically closed fields of characteristic $0$) up to isomorphisms. Firstly, the restriction of $(-, -)_{\g}$ to $\h$ remains non-degenerate, so we can canonically identify $\h \xrightarrow[]{\cong} \h^*$ via said bilinear form. By picking a basis for our choice of a Catan subalgebra $\h$ and hence a dual basis:
            $$\{\alpha_i\}_{i \in \simpleroots}$$
        for $\h^*$, which we shall regard as a choice of a set of \textbf{simple roots}. The \textbf{Cartan matrix} of $\g$ can then be defined to be:
            $$C := (c_{ij})_{i, j \in \simpleroots} := \left( \frac{2(\alpha_i, \alpha_j)_{\g}}{(\alpha_i, \alpha_i)_{\g}} \right)_{i, j \in \simpleroots}$$
        It can be shown that $C$ can be symmetrised, in the sense that there exists an invertible diagonal matrix:
            $$D := (d_{ij})_{i, j \in \simpleroots} = \left(\frac{2\delta_{i, j}}{(\alpha_i, \alpha_j)_{\g}}\right)_{i, j \in \simpleroots}$$
        and a symmetric matrix:
            $$A := (a_{ij})_{i, j \in \simpleroots} = \left((\alpha_i, \alpha_j)_{\g}\right)_{i, j \in \simpleroots}$$
        (which is nothing but the matrix representation of the bilinear form $(-, -)_{\g}$ with respect to the basis $\{\alpha_i\}_{i \in \simpleroots}$), such that:
            $$C = DA$$
        Furthermore, we have that $2\id - A$ is the adjacency matrix of an undirected graph without loops, called the \textbf{Dynkin diagram} of $\g$, and the \textbf{roots} of $\g$ are the roots of this Dynkin diagram. The set of roots shall be denoted by:
            $$\Phi$$

        The \textbf{root lattice} of $\g$ is given by:
            $$Q := \Z \simpleroots$$
        and in light of the root space decomoposition of $\g$, one sees that $\g$ is graded by the abelian group $Q$. Given an element:
            $$\mu := \sum_{i \in \simpleroots} m_i \alpha_i \in Q$$
        we define its \textbf{height} to be the sum of the coefficients:
            $$\height \mu := \sum_{i \in \simpleroots} m_i$$
        $Q$ is therefore partially ordered by heights; the highest element of $\Phi$, i.e. the \textbf{highest root}, is usually denoted by:
            $$\theta$$
        We will be needing this distinguished element in subsection \ref{subsection: a_fixed_untwisted_affine_kac_moody_algebra}.

        Elements of $Q^+ := \Z_{\geq 0} \simpleroots$ are typically regarded as being \textbf{positive} (and in particular, the simple roots are positive by convention) and conversely, elements of $Q^- := \Z_{\leq 0} \simpleroots$ are typically said to be \textbf{negative}. One can subsequently construct the sets of positive/negative roots as:
            $$\Phi^{\pm} := \Phi \cap Q^{\pm}$$
        and note that $\Phi = \Phi^+ \cup \Phi^-$.

        From theorem \ref{theorem: root_space_decomposition_for_finite_dimensional_simple_lie_algebras}, we see that for any given positive root $\alpha \in \Phi^+$ and corresponding choices of root vectors\footnote{Choices of which are unique up to non-zero scalar multiples, since subspaces of non-zero weights are equally $1$-dimensional (see theorem \ref{theorem: root_space_decomposition_for_finite_dimensional_simple_lie_algebras}).} $x_{\pm\alpha} \in \g_{\pm\alpha}$, one has that:
            $$(h, [x_{\alpha}, x_{-\alpha}])_{\g} = ([h, x_{\alpha}], x_{-\alpha})_{\g} = \alpha(h) (x_{\alpha}, x_{-\alpha})_{\g}$$
        for all $h \in \h$. Via the non-degeneracy of the bilinear form $(-, -)_{\g}$, this tells us that:
            $$[x_{\alpha}, x_{-\alpha}] = (x_{\alpha}, x_{-\alpha})_{\g} \check{\alpha}$$
        where $\check{\alpha} \in \h$ being such that\todo{Fixed phrasing.}:
            $$(\check{\alpha}, -)_{\g} = \alpha$$
        as elements of $\h^*$; the element $\check{\alpha}$ is often referred to as the \textbf{coroot} dual to $\alpha$, and such elements form a set $\check{\Phi}$ of coroots, in which there is a linearly independent subset $\check{\simpleroots}$ of simple coroots, etc. For simplicity, the root vectors $x_{\pm \alpha}$ are therefore typically chosen so that:
            $$(x_{\alpha}, x_{-\alpha})_{\g} = 1$$
        which then yields:
            $$[x_{\alpha}, x_{-\alpha}] = \check{\alpha}$$
        A consequence of this is that for each positive root $\alpha \in \Phi^+$, there exists an injective Lie algebra homomorphism:
            $$\sl_2(k) \to \g$$
        given by $x^{\pm} \mapsto x_{\pm \alpha}$ and $h \mapsto \check{\alpha}$ (cf. example \ref{example: sl_2}).
        
        The next result is a fundamental theorem in the study of finite-dimensional simple Lie algebras over algebraically closed fields of characteristic $0$. It essentially asserts that to give such a Lie algebra via a presentation by generators and relations is the same as to specify its Cartan matrix. The result is not only practically useful, but also is the mean by which one approaches Kac-Moody algebras, where the Cartan matrix is no longer required to be positive-definite; we refer the reader to subsection \ref{subsection: a_fixed_untwisted_affine_kac_moody_algebra} for a partial recollection of this story, and to \cite[Chapters 1-8]{kac_infinite_dimensional_lie_algebras} for details. 
        \begin{theorem}[Serre's Theorem] \label{theorem: serre_theorem_for_finite_dimensional_simple_lie_algebras}
            $\g$ is isomorphic to the Lie algebra generated by the set:
                $$\{h_i, x_i^{\pm}\}_{i \in \simpleroots}$$
            whose elements are subjected to the following relations, given for all $i, j \in \simpleroots$:
                $$[h_i, h_j] = 0$$
                $$[h_i, x_j^{\pm}] = \pm c_{ij} x_j^{\pm}, [x_i^+, x_j^-] = \delta_{ij} h_i$$
            and for all $i \not = j \in \simpleroots$, there are also the so-called \textbf{Serre relations}:
                $$\ad(x_i^{\pm})^{1 - c_{ij}}(x_j^{\pm}) = 0$$
            This is usually referred to as the \textbf{Chevalley-Serre} presentation for $\g$, and the relations are usually referred to collectively as the \textbf{Chevalley-Serre relations}.
        \end{theorem}

        We end this subsection with a brief analysis of the easiest possible example of a finite-dimensional simple Lie algebra. 
        \begin{example}[$\sl_2$] \label{example: sl_2}
            Recall that $\sl_2(k)$ is the kernel of the trace map:
                $$\trace: \gl_2(k) \to k$$
            i.e. it is the Lie algebra of trace-zero $2 \x 2$-matrices whose Lie bracket is the usual commutator of matrices. It is of dimension $\dim_k \gl_2(k) - \dim_k k = 4 - 1 = 3$, and a common choice of basis is:
                $$\left\{ \begin{pmatrix} 1 & 0 \\ 0 & -1 \end{pmatrix}, \begin{pmatrix} 0 & 1 \\ 0 & 0 \end{pmatrix}, \begin{pmatrix} 0 & 0 \\ 1 & 0 \end{pmatrix} \right\}$$
                
            Because the first element is not nilpotent, whereas the other two elements are nilpotent, any Cartan subalgebra of $\sl_2(k)$ must therefore be isomorphic to $k \begin{pmatrix} 1 & 0 \\ 0 & -1 \end{pmatrix}$. One can then show that by setting $x^+ := \begin{pmatrix} 0 & 1 \\ 0 & 0 \end{pmatrix}$ and $x^- := \begin{pmatrix} 0 & 0 \\ 1 & 0 \end{pmatrix}$, the set:
                $$\{h, x^{\pm}\}$$
            then becomes a set of Chevalley-Serre generators for $\sl_2(k)$. The relations that they must satisfy, according to theorem \ref{theorem: serre_theorem_for_finite_dimensional_simple_lie_algebras}, are therefore:
                $$[h, x^{\pm}] = \pm 2 x^{\pm}, [x^+, x^-] = h$$
            
            In this case, the Cartan matrix is just:
                $$\begin{pmatrix} 2 \end{pmatrix}$$
            and the Dynkin diagram consists of only a single vertex and no edges:
                $$\bullet$$
        \end{example}

    \subsection{Perfect Lie algebras and their central extensions}
        \begin{definition}[Extensions of Lie algebras] \label{def: lie_algebra_extensions}
            Fix a Lie algebra $\a$.
        
            A \textbf{Lie algebra extension} is a short exact sequence of Lie algebras:
                $$0 \to \z \to \frake \xrightarrow[]{\pi} \a \to 0$$
            or equivalently, a Lie algebra epimorphism $\pi: \frake \to \a$ ($\z$ is then uniquely determined as $\ker \pi$). 
            
            A morphism between to such extensions of the given Lie algebra $\a$, say:
                $$\varphi: (\pi': \frake' \to \a) \to (\pi: \frake \to \a)$$
            is then a commutative diagram of Lie algebras and Lie algebra homomorphisms:
                $$
                    \begin{tikzcd}
                	0 & {\z'} & {\frake'} & \a & 0 \\
                	0 & \z & \frake & \a & 0
                	\arrow["{\varphi|_{\z'}}", from=1-2, to=2-2]
                	\arrow["\varphi", from=1-3, to=2-3]
                	\arrow["{\id_{\a}}", from=1-4, to=2-4]
                	\arrow[from=1-1, to=1-2]
                	\arrow[tail, from=1-2, to=1-3]
                	\arrow["{\pi'}", two heads, from=1-3, to=1-4]
                	\arrow[from=1-4, to=1-5]
                	\arrow[from=2-1, to=2-2]
                	\arrow[tail, from=2-2, to=2-3]
                	\arrow["\pi", two heads, from=2-3, to=2-4]
                	\arrow[from=2-4, to=2-5]
                    \end{tikzcd}
                $$
            wherein the rows are short exact sequences.

            An Lie algebra extension $p: \fraku \to \a$ of $\a$ is said to be \textbf{universal} if and only if, for all other extensions $\pi: \frake \to \a$ of $\a$, there exists a unique morphism of extensions $(p: \fraku \to \a) \to (\pi: \frake \to \a)$.
        \end{definition}
        \begin{remark}
            Universal extensions are unique up to unique isomorphisms. 
        \end{remark}
        \begin{remark}
            Since we are working over a field, all short exact sequences split when regarded as short exact sequences of vector spaces. In particular, this implies that given a Lie algebra extension:
                $$0 \to \z \to \frake \to \a \to 0$$
            the underlying vector space of $\frake$ will always be isomorphic to $\a \oplus \z$. From now on, this identification will be used without explicit mention.
        \end{remark}
        We will also need to know how the Lie brackets on extensions are given explicitly. 
        \begin{proposition}[Lie brackets on extensions] \label{prop: lie_brackets_on_extensions}
            Suppose that:
                $$0 \to \z \to \frake \to \a \to 0$$
            is an extension of Lie algebras. Then, up to a choice of so-called \say{$2$-cocycle} $\sigma$ (of $\a$ with coefficients in $\z$), which can be regarded as a $k$-linear map\footnote{Equivalently, one can pick a representative of an isomorphism class $[\sigma] \in H^2_{\Lie}(\a, \z)$.}:
                $$\sigma: \bigwedge^2 \a \to \z$$
            satisfying a certain condition, the Lie bracket on $\frake$ will be given by:
                $$[ (X, K), (Y, K') ]_{\frake} := [X, Y]_{\a} + ( [K, K']_{\z} + \ad(Y)(K) - \ad(X)(K') + \sigma(X, Y) )$$
            for all $X, Y \in \a$ and all $K, K' \in \z$.
            
            The aforementioned condition on $[-, -]_{\frake}$ is that it satisfies the Jacobi identity\footnote{Note that $[-, -]_{\frake}$ is already bilinear and skew-symmetric by construction}.
        \end{proposition}
        \begin{example}[Semi-direct products of Lie algebras] \label{example: lie_algebra_semi_direct_products}
            Let:
                $$\rho: \d \to \der_k(\z)$$
            be a Lie algebra homomorphism from a Lie algebra $\d$ to the Lie algebra of derivations on another Lie algebra $\z$, determining an action of $\d$ on $\z$. The canonical extension of $\d$ by $\z$ corresponding to the $2$-cocycle:
                $$\sigma = 0$$
            is known as the \textbf{semi-direct product} of $\d$ by $\z$, and commonly denoted by:
                $$\z \rtimes \d$$
            For the sake of completeness, let us note that the Lie bracket here is given by:
                $$[ (D, K), (D', K') ]_{\z \rtimes \d} := [D, D']_{\d} + ( [K, K]_{\z} + \rho(D')(K) - \rho(D)(K') )$$
            for all $D, D' \in \d$ and all $K, K' \in \z$.
        \end{example}
        \begin{definition}[Central extensions of Lie algebras]
            A Lie algebra extension:
                $$0 \to \z \to \frake \to \a \to 0$$
            is \textbf{central} if the elements of $\z$ are central in $\frake$; morphisms between central extensions are just morphisms of Lie algebra extensions (cf. definition \ref{def: lie_algebra_extensions}). A \textbf{universal central extension} (UCE) of a Lie algebra $\a$, typically denoted by $\uce(\a)$, is an extension that is initial amongst all central extensions of $\a$.
        \end{definition}

        A particular class of Lie algebras that happen to admit UCEs are the so-called \say{perfect} ones.
        \begin{definition}[Perfect Lie algebras]
            A Lie algebra is said to be \textbf{perfect} if it is equal to its derived subalgebra. 
        \end{definition}
        \begin{example}
            Since simple Lie algebras lack non-zero ideals by definition, they are perfect. 
        \end{example}
        \begin{example}
            Let $A$ be any commutative algebra over a field $k$, and let $\g$ be a simple Lie algebra over $k$. Endow the $k$-vector space $\g \tensor_k A$ with the Lie bracket:
                $$\forall x, y \in \g: \forall f, g \in A: [x f, y g]_{\g \tensor_k A} := [x, y]_{\g} \tensor fg$$
            Then, $\g \tensor_k A$ will be perfect when regraded as a Lie algebra over $k$, precisely because $\g$ is simple.
        \end{example}
        \begin{example}
            Counter-examples include nilpotent Lie algebras (e.g. abelian Lie algebras). For such Lie algebras, their derived subalgebras are always proper Lie subalgebras (and in particular, derived subalgebras of abelian Lie algebras are zero). 
        \end{example}
        \begin{proposition}[Perfect Lie algebras admit UCEs] \label{prop: perfect_lie_algebras_admit_UCEs}
            \cite[Lemma 1.10]{garland_arithmetics_of_loop_groups} Let $k$ be a field and $\a$ is a Lie algebra over $k$. Then $\a$ admits a UCE if and only if it is perfect.
        \end{proposition}

        Let us conclude this subsection by making note of the following fact about the process of extending invariant bilinear forms from Lie algebras to their central extensions.
        \todo[inline]{Added remark about extending bilinear forms to central extensions.}
        \begin{remark}[Extending invariant bilinear forms to central extensions] \label{remark: extending_bilinear_forms_to_central_extensions}
            Let $k$ be a field.
            
            Let $\a$ be a Lie algebra over $k$ and $(-, -)$ is an invariant symmetric $k$-bilinear form thereon. Additionally, let:
                $$0 \to \z \to \frake \to \a \to 0$$
            be a central Lie algebra extension of $\a$ such that $\z \not \cong 0$. By proposition \ref{prop: lie_brackets_on_extensions}, we know that for all $X, Y \in \a$, we have that:
                $$[X, Y]_{\frake} = [X, Y]_{\a} + K(X, Y)$$
            for some $K(X, Y) \in \z$ depending on the choices of $X, Y$.
            
            Suppose that $\<-, -\>$ is any extension of $(-, -)$ to $\frake$, i.e. $\<-, -\>$ is a symmetric $k$-bilinear form on $\frake$ such that:
                $$\<-, -\>|_{\Sym^2_k( \a )} = (-, -)$$
            Assume furthermore, that:
                $$\<\a, \z\> = 0, \<\z, \z\> = 0$$ 
            Then, note firstly that $\<-, -\>$ is necessarily invariant: indeed, for any $X, Y, Z \in \a$, we have that:
                $$
                    \begin{aligned}
                        \<[X, Y]_{\frake}, Z\> & = \< [X, Y]_{\a} + K(X, Y), Z \>
                        \\
                        & = ([X, Y]_{\a}, Z)
                        \\
                        & = (X, [Y, Z]_{\a})
                        \\
                        & = \< X, [Y, Z]_{\a} + K(Y, Z) \>
                        \\
                        & = \< X, [Y, Z]_{\frake} \>
                    \end{aligned}
                $$
            Furthermore, $\<-, -\>$ must be degenerate, regardless of whether or not $(-, -)$ is degenerate to begin with: to see why, it suffices to show that the radical\footnote{If $V$ is a vector space and $B$ is a symmetric bilinear form on $V$, then the \textbf{radical} of $B$ is then given by $\Rad B := \{v \in V \mid \forall w \in V: B(v, w) = 0\}$.} of $\<-, -\>$ is non-zero, for which we can simply consider the following for any $X, Y \in \a$ and any $K \in \z$, and then exploit the invariance of $\<-, -\>$:
                $$\<K, [X, Y]_{\frake}\> = \<[K, X]_{\frake}, Y\> = \<0, Y\> = 0$$

            Not every Lie algebra with a non-trivial centre arises as a central extension (see \cite[Section 1]{garland_arithmetics_of_loop_groups} for details), so it is not true that any invariant symmetric bilinear form on such a Lie algebra must be degenerate. Central elements can pair non-trivially with other elements, such as in the construction of untwisted affine Kac-Moody algebras (cf. subsection \ref{subsection: a_fixed_untwisted_affine_kac_moody_algebra}) or in the construction of so-called \say{extended toroidal Lie algebras} (cf. definition \ref{def: extended_toroidal_lie_algebras}).
        \end{remark}