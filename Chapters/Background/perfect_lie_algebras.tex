\section{Some generalities on Lie algebras}
    \subsection{Structure of finite-dimensional simple Lie algebras}
        As a precursor to our main discussion, let us recall some features of the theory of finite-dimensional simple Lie algebras, particularly about their structure.

        \begin{definition}[Simple Lie algebras]
            A Lie algebra over an arbitrary commutative ring $k$ is said to be \textbf{simple} if and only if it admits no non-zero Lie ideals. 
        \end{definition}

        Over a field $k$ that is algebraically closed and of characteristic $0$, much is known about the structure of a simple Lie algebra $\g$ that is finite-dimensional when regarded as a $k$-vector space. The bulk of the content presented above is discussed in further details in any standard textbook on Lie algebras (cf. e.g. \cite{humphreys_lie_algebras} or the first half of \cite{carter_affine_lie_algebras}). 

        Firsly, one of the most important features of finite-dimensional simple Lie algebras (henceforth implicitly understood to be defined over a characteristic-$0$ and algebraically closed field $k$) is that each such Lie algebra, say $\g$, posses an invariant and non-degenerate $k$-bilinear form which is unique up to $k^{\x}$-multiples. The canonical choice is the so-called Killing form, given as follows for all $x, y \in \g$:
            $$\kappa(x, y) := \trace(\ad(x) \circ \ad(y))$$
        but in various other context, other natural choices such as the trace form are also very useful. What is important to us is that the Killing form is essentially unique: if $\kappa'$ is any invariant and non-degenerate symmetric $k$-bilinear form on $\g$ then there will exist $c \in k^{\x}$ such that:
            $$\kappa' = c \kappa$$

        Now, the existence and uniqueness (up to non-zero scalings) of a non-degenerate and invariant symmetric bilinear form allows us to construct a natural grading of any simple Lie algebra $\g$ by its \say{root lattice} (to be defined shortly), which is a certain finite free $\Z$-module of combinatorial origin. One does this choosing a \textbf{Cartan subalgebra} $\h$ for $\g$, which is a maximal abelian Lie subalgebra of $\g$ (and it is well-known that all Cartan subalgebras of $\g$ are conjugate to one another).  
        \begin{lemma}[Non-degeneracy of invariant bilinear forms on Cartan subalgebras] \label{lemma: non_degeneracy_of_invariant_bilinear_forms_on_cartan_subalgebras}
            Choose a non-degenerate and invariant symmetric $k$-bilinear form $(-, -)_{\g}$ on $\g$, regarded as a $k$-linear map:
                $$(-, -)_{\g}: \Sym_k^2(\g) \to k$$
            Then, the domain restriction of this form to $\Sym_k^2(\h)$ remains non-degenerate. 
        \end{lemma}
        Thanks to this lemma, we see that with respect to a choice of a Cartan subalgebra $\h$ and a non-degenerate and invariant symmetric $k$-bilnear form on $\g$, one can canonically identify:
            $$\h \xrightarrow[]{\cong} \h^*$$
        via said bilinear form. By picking a basis for our choice of a Catan subalgebra $\h$ and hence a dual basis $\simpleroots$ for $\h^*$, which we shall regard as a choice of a set of \textbf{simple roots}, and writing our bilinear form $(-, -)_{\g}$ in terms of that matrix, we shall get the \textbf{Cartan matrix} of $\g$:
            $$C := (c_{ij})_{i, j \in \simpleroots}$$
        The aforementioned \textbf{root lattice} is then given by:
            $$Q := \Z \simpleroots$$
        It can then also be shown that $2\id - C$ is the adjacency matrix of an undirected graph without loops, called the \textbf{Dynkin diagram} of $\g$, and the \textbf{roots} of $\g$ are the roots of this Dynkin diagram.
            
        The following definition is mostly for the sake of setting up notations:
        \begin{definition}[Weight spaces and root spaces]
            Let $V$ be a $\g$-module. Then, one can abstractly define the vector subspace of $V$ consisting of elements of \textbf{weight} $\mu \in \h^*$ to be:
                $$V_{\mu} := \{v \in V \mid \forall h \in \h: h \cdot v = \mu(h) v\}$$
            The set of weights $\mu \in \h^*$ such that $V_{\mu} \not \cong 0$ is denoted by $\Pi(V)$, and if:
                $$V \cong \bigoplus_{\mu \in \Pi(V)} V_{\mu}$$
            then we will say that $V$ is a \textbf{weight module} for $\g$. 

            When $V \cong \g$ and carries the adjoint action of $\g$, we will instead refer to the non-zero weights as \textbf{roots}. These are the same as the roots constructed from the Cartan matrix of $\g$.
        \end{definition}
        The $Q$-grading of $\g$ that was mentioned earlier then takes the following form:
        \begin{theorem}[Root space decomposition for finite-dimensional simple Lie algebras] \label{theorem: root_space_decomposition_for_finite_dimensional_simple_lie_algebras}
            Under the adjoint action, $\g$ becomes a weight module over itself. Furthermore, for each $\alpha \in \Pi(\g) \setminus \{0\}$, one has that:
                $$\dim_k \g_{\alpha} = 1$$
        \end{theorem}
        \begin{convention}
            Let $\g$ act via the adjoint action on itself. Since Cartan subalgebras are abelian by definition, it is not hard to see that:
                $$\h \cong \g[0]$$
            and in light of this, one then often writes:
                $$\Phi := \Pi(\g) \setminus \{0\}$$
            to denote the set of roots of $\g$. The root space decomposition of $\g$ then takes the form:
                $$\g \cong \h \oplus \bigoplus_{\alpha \in \Phi} \g_{\alpha}$$
        \end{convention}
        The following is a very useful and conceptual way to think about elements:
            $$x \in \g_{\alpha}$$
        (i.e. \say{root vectors}). 
        \begin{lemma}[Action of root vectors]
            Let $V$ be an arbitrary $\g$-module. Then:
                $$\forall \alpha \in \Phi: \forall \mu \in \Pi(V): \g_{\alpha} \cdot V_{\mu} \subseteq V[\mu + \alpha]$$
        \end{lemma}
        \begin{corollary}
            For any given root:
                $$\alpha \in \Phi$$
            and corresponding root vectors $x_{\alpha} \in \g_{\alpha}, x_{\beta} \in \g_{\beta}$, one has that:
                $$[x_{\alpha}, x_{-\alpha}] = (x_{\alpha}, x_{\beta})_{\g} \check{\alpha}$$
            where:
                $$\check{\alpha} \in \h$$
            is such that:
                $$(\alpha, \check{\alpha})_{\g} = 2$$
        \end{corollary}
        The following result is a fundamental theorem in the study of finite-dimensional simple Lie algebras over algebraically closed fields of characteristic $0$. It essentially gives a bijective classification:
            $$\{ \text{connected Dynkin diagrams} \} \xrightarrow[]{\cong} \{ \text{finite-dimensional simple Lie algebras over $k$} \}/\cong$$
        which means that to give such a Lie algebra via a presentation by generators and relations is the same as giving its associated Cartan matrix. This is not only practically useful, but also is the mean by which one approaches Kac-Moody algebras, where the Cartan matrix is no longer required to be positive-definite (cf. \cite[Chapters 1-5]{kac_infinite_dimensional_lie_algebras}). 
        \begin{theorem}[Serre's Theorem]
            $\g$ is isomorphic to the Lie algebra generated by the set:
                $$\{h_i, x_i^{\pm}\}_{1 \leq i \leq l}$$
            whose elements are subjected to the following relations, given for all $1 \leq i, j \leq l$:
                $$[h_i, h_j] = 0$$
                $$[h_i, x_j^{\pm}] = \pm c_{ij} x_j^{\pm}, [x_i^+, x_j^-] = \delta_{ij} h_i$$
                $$\ad(x_i^{\pm})^{1 - c_{ij}}(x_j^{\pm}) = 0$$
            This is usually referred to as the \textbf{Chevalley-Serre} presentation for $\g$; final set of relations is usually known as the \textbf{Serre relations}.
        \end{theorem}
        \begin{corollary}
            For every root $\alpha \in \Phi$, any element $x \in \g_{\alpha}$ is nilpotent under the vector representation of $\g$. 
        \end{corollary}
        \begin{corollary}[Triangular decomposition for finite-dimensional simple Lie algebras]
            Let $\n^{\pm}$ denote the Lie subalgebras of $\g$ generated by the sets $\{x_i^{\pm}\}_{1 \leq i \leq l}$ respectively. Then:
                $$\g \cong \n^- \oplus \h \oplus \n^+$$
            This is usually called the \textbf{triangular decomposition} for $\g$. 
        \end{corollary}

        We end this subsection with a brief analysis of the easiest possible example of a finite-dimensional simple Lie algebra. 
        \begin{example}[$\sl_2$]
            Recall that $\sl_2(k)$ is the kernel of the trace map:
                $$\trace: \gl_2(k) \to k$$
            i.e. it is the Lie algebra of trace-zero $2 \x 2$-matrices whose Lie bracket is the usual commutator of matrices. It is of dimension $\dim_k \gl_2(k) - \dim_k k = 4 - 1 = 3$, and happens to be also generated by a set of cardinality $3$ (though this is a coincidence, due entirely to how \say{degenerate} of an example $\sl_2(k)$ is):
                $$\{h, e^+, e^-\}$$
            and the elements of this set are subjected to the relations:
                $$[h, e^{\pm}] = \pm 2 e^{\pm}, [e^+, e^-] = h$$
            One proves both of these statements by showing firstly that a basis for $\sl_2(k)$ is:
                $$\left\{ \begin{pmatrix} 1 & 0 \\ 0 & -1 \end{pmatrix}, \begin{pmatrix} 0 & 1 \\ 0 & 0 \end{pmatrix}, \begin{pmatrix} 0 & 0 \\ 1 & 0 \end{pmatrix} \right\}$$
            and then seeing that any Cartan subalgebra of $\sl_2(k)$ must therefore be isomorphic to $k \begin{pmatrix} 1 & 0 \\ 0 & -1 \end{pmatrix}$; the rest then follows. In particular, the Cartan matrix is just:
                $$\begin{pmatrix} 2 \end{pmatrix}$$
            and the Dynkin diagram consists of only a single vertex and no edges:
                $$\bullet$$
            From this, one gathers that $\sl_2(k)$ is of type $\sfA_1$. 
        \end{example}

    \subsection{Perfect Lie algebras and their central extensions}
        \begin{definition}[Extensions of Lie algebras]
            Let $k$ be a commutative ring and $\a$ be a Lie algebra over $k$. An \textbf{extension} of $\a$ by another Lie algebra $\z$ over $k$ is an extension of $k$-modules:
                $$0 \to \z \to \frake \xrightarrow[]{\pi} \a \to 0$$
            such that $\frake$ is also a Lie algebra over $k$. A morphism of extensions of a Lie algebra $\a$ by another Lie algebra $\z$ is a morphism of short exact sequences of Lie algebras:
                $$
                    \begin{tikzcd}
                	&& {\frake'} \\
                	0 & \z & \frake & \a & 0
                	\arrow["\varphi", from=1-3, to=2-3]
                	\arrow["{\pi'}", two heads, from=1-3, to=2-4]
                	\arrow["\pi", two heads, from=2-3, to=2-4]
                	\arrow[from=2-2, to=2-3]
                	\arrow[from=2-2, to=1-3]
                	\arrow[from=2-1, to=2-2]
                	\arrow[from=2-4, to=2-5]
                    \end{tikzcd}
                $$
            An isomorphism of extension occurs if $\varphi: \frake' \to \frake$ is a Lie algebra isomorphism. An extension:
                $$0 \to \z \to \tilde{\a} \to \a \to 0$$
            of $\a$ by $\z$ is said to be \textbf{universal} if it is initial amongst all such Lie algebra extensions, in the sense that for every other extension:
                $$0 \to \z \to \frake \to \a \to 0$$
            there must exist a unique Lie algebra homomorphism $\tilde{\a} \to \frake$ fitting into the following morphism of extensions:
                $$
                    \begin{tikzcd}
                	&& {\tilde{\a}} \\
                	0 & \z & \frake & \a & 0
                	\arrow[dashed, from=1-3, to=2-3]
                	\arrow[two heads, from=1-3, to=2-4]
                	\arrow[two heads, from=2-3, to=2-4]
                	\arrow[from=2-2, to=2-3]
                	\arrow[from=2-2, to=1-3]
                	\arrow[from=2-1, to=2-2]
                	\arrow[from=2-4, to=2-5]
                    \end{tikzcd}
                $$
        \end{definition}
        \begin{remark}
            Universal extensions are unique up to unique isomorphisms. 
        \end{remark}
        \begin{definition}[Central extensions of Lie algebras]
            A Lie algebra extension:
                $$0 \to \z \to \frake \to \a \to 0$$
            is \textbf{central} if:
                $$\z \subseteq \z(\frake)$$
            A \textbf{universal central extension} of a Lie algebra $\a$ by another Lie algebra $\z$ (UCE) is an extension that is initial amongst all central extensions.
        \end{definition}
        \begin{example}[A non-central extension of Lie algebras]
            Let $n \geq 2$ and let $k$ be a field of characteristic $\not = 2$. Then, one can construct the Lie algebra $\gl_n(k)$ of $n \x n$ matrices with coefficients in $k$ as the non-central extension:
                $$0 \to \sl_n(k) \to \gl_n(k) \xrightarrow[]{\trace} k \to 0$$
        \end{example}
        \begin{proposition}[$H^2_{\Lie}$ = extensions]
            \cite[Theorem VIII.3.3]{hilton_stammbach_homological_algebra} Suppose that $\a$ is a Lie algebra over a commutative ring $k$. Then, the isomorphism classes of Lie algebra extensions of $\a$ (regarded as a left-$\rmU(\a)$-module) by some left-$\rmU(\a)$-module $M$ is in bijection with elements of\footnote{... understood to be computed using the Chevalley-Eilenberg complex.}:
                $$H^2_{\Lie}(\a, M) := \Ext^2_{{}^l\rmU(\a)\mod}(k, M)$$
            More explicitly, given an extension:
                $$0 \to M \to \frake \to \a \to 0$$
            then up to a choice of $2$-cocyle:
                $$\sigma \in H^2_{\Lie}(\a, M)$$
            the Lie bracket on $\frake$ will be given by:
                $$\forall X, Y \in \a: \forall m, m' \in M: [ (X, m), (Y, m') ]_{\frake} := [X, Y]_{\a} + ( m \cdot Y - m' \cdot Y + \sigma(X, Y) )$$
        \end{proposition}
        \begin{corollary}[Trivial cohomological coefficients = central extensions]
            The isomorphism classes of central extensions of $\a$ are parametrised bijectively by elements of $H^2_{\Lie}(\a, k)$.
        \end{corollary}
        \begin{example}[Semi-direct products of Lie algebras]
            Let $k$ be an arbitrary commutative ring. 
            
            Let $\d$ be a Lie $k$-algebra acting on another Lie $k$-algebra $\z$, i.e. let $\t$ be a $\d$-module that happens also to be a Lie algebra over $k$. The canonical extension of $\d$ by $\z$ corresponding to the element:
                $$0 \in H^2_{\Lie}(\d, \z)$$
            is known as the \textbf{semi-direct product} of $\d$ by $\z$, and commonly denoted by:
                $$\z \rtimes \d$$
            For the sake of completeness, let us note that the Lie bracket here is given by:
                $$\forall D, D' \in \d: \forall K, K' \in \z: [ (D, K), (D', K') ]_{\z \rtimes \d} := [D, D']_{\d} + ( D \cdot K' - D' \cdot K )$$
            
            Interestingly, semi-direct products are split extensions, in the sense that there is a section Lie algebra homomorphism $\d \to \t \rtimes_{\rho} \d$, given by $D \mapsto (0, D)$.
        \end{example}

        At this point, a natural question to pose is as follows.
        \begin{question}
            Which class of Lie algebras naturally admits UCEs ? 
        \end{question}
        As it so happens, so-called \say{perfect Lie algebras} form one such class of Lie algebras.
        \begin{definition}[Perfect Lie algebras]
            A Lie algebra over a commutative ring is said to be \textbf{perfect} if it is equal to its derived subalgebra. 
        \end{definition}
        \begin{example}
            Since simple Lie algebras lack non-zero ideals by definition, they are perfect. 
        \end{example}
        \begin{example}
            Let $A$ be any commutative algebra over some base commutative ring $k$, and let $\g$ be a simple Lie algebra over $k$. Endow the $k$-module $\g_A := \g \tensor_k A$ with the Lie bracket:
                $$\forall x, y \in \g: \forall f, g \in A: [x f, y g]_{\g_A} := [x, y]_{\g} fg$$
            Then, $\g_A$ will be perfect when regraded as a Lie algebra over $k$, precisely because $\g$ is simple.
        \end{example}
        \begin{definition}[Covering Lie algebras]
            Suppose that:
                $$0 \to \z \to \frake \xrightarrow[]{\pi} \a \to 0$$
            is a central extension of Lie algebras over some base commutative ring $k$. If $\frake$ is perfect then we will call the Lie algebra epimorphism:
                $$\pi: \frake \to \a$$
            a \textbf{covering} or a \textbf{cover} of $\a$. 

            There is an evident full subcategory of $\Cent\Lie\Ext^1_k(\a, -)$ where the objects are coverings of $\a$. Let us denote it by $\Cov_k(\a)$
        \end{definition}
        \begin{proposition}[Perfect Lie algebras admit UCEs] \label{prop: perfect_lie_algebras_admit_UCEs}
            \cite[Lemma 1.10]{garland_arithmetics_of_loop_groups} If $k$ is a field and $\a$ is a perfect Lie algebra over $k$, then $\Cov_k(\a)$ will have initial objects. Any such object will be referred to as a \textbf{universal cover} of $\a$, but still typically denoted by $\uce(\a)$.
        \end{proposition}