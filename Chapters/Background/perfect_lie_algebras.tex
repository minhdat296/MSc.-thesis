\section{Some generalities on Lie algebras}
    \begin{convention}
        A very common notation that we will be liberally employing is that:
            $$\ad(x) := [x, -]$$
        for $x$ being an element of some Lie algebra $\a$ with bracket $[-, -]$. 

        Another common notation in the context of Lie algebras (over a fixed field) is $\gl(V)$, which is the Lie algebra of endomorphisms on a vector space $V$, with Lie bracket given by commutators.

        A particularly common representation of a Lie algebra $\a$ is the \textbf{adjoint representation}, which we will be denoting by:
            $$\ad: \a \to \gl(\a)$$
        and is given by $x \mapsto \ad(x)$.
    \end{convention}

    \subsection{Structure of finite-dimensional simple Lie algebras} \label{subsection: finite_dimensional_simple_lie_algebras}
        As a precursor to our main discussion, let us recall some features of the theory of finite-dimensional simple Lie algebras, particularly about their structure.

        We will be working over a field $k$ that is algebraically closed and of characteristic $0$, and no serious loss of generality will come from assuming, say, that $k \cong \bbC$. The assumption that $k$ is algebraically closed is so that we will have enough eigenvalues and hence, certain operators will be diagonalisable. The second assumption, that $\chara k = 0$, is made so that we can avoid having certain relations, e.g. as in theorem \ref{theorem: serre_theorem_for_finite_dimensional_simple_lie_algebras}, vanish.

        \begin{definition}[Simple Lie algebras]
            A Lie algebra is said to be \textbf{simple} if and only if it admits no non-zero Lie ideals. 
        \end{definition}

        Over a field $k$ that is algebraically closed and of characteristic $0$ (which can just be taken to be $\bbC$), much is known about the structure of a simple Lie algebra $\g$ that is finite-dimensional when regarded as a $k$-vector space. The bulk of the content presented above is discussed in further details in any standard textbook on Lie algebras (cf. e.g. \cite{humphreys_lie_algebras} or the first half of \cite{carter_affine_lie_algebras}). Let us give a very brief recap of this theory. Also, let us fix once and for all such a finite-dimensional simple Lie algebra:
            $$\g$$

        One begins this process by choosing a \textbf{Cartan subalgebra} $\h$ for $\g$, which is a maximal abelian Lie subalgebra $\h$\footnote{We note also that, it is well-known that all Cartan subalgebras of $\g$ are conjugate to one another.}, whose elements are diagonalisable under the adjoint representation $\ad: \g \to \gl(\g)$. Now, let $V$ be a $\g$-module. Then, one can abstractly define the vector subspace of $V$ consisting of elements of \textbf{weight} $\mu \in \h^*$ to be:
            $$V_{\mu} := \{v \in V \mid \forall h \in \h: h \cdot v = \mu(h) v\}$$
        If we have a direct sum decomposition of $\h$-module:
            $$V \cong \bigoplus_{\mu \in \h^*} V_{\mu}$$
        then we will say that $V$ is a \textbf{weight module} for $\g$. Interestingly, elements of $\g_{\alpha}$ (with $\g$ acting on itself by the adjoint action) act by raising/lowering the weights of elements of $\g$-modules $V$ in the sense that:
            $$\g_{\alpha} \cdot V_{\mu} \subseteq V_{\mu + \alpha}$$
        for all weights $\alpha, \mu \in \h^*$. 
        
        As it turns out, when $\g$ is a weight module over itself via the adjoint representation. This particular weight space decomposition is usually referred to as the \textbf{root space decomposition} of $\g$. 
        \begin{theorem}[Root space decomposition for finite-dimensional simple Lie algebras] \label{theorem: root_space_decomposition_for_finite_dimensional_simple_lie_algebras}
            Let $\g$ be a module over itself via the adjoint representation.
            \begin{enumerate}
                \item $\g$ is a weight module over itself.
                \item The weight space $\g_0$ is nothing but the Cartan subalgebra $\h$.
                \item For each non-zero weight $\alpha$ of this $\g$-module, $\dim_k \g_{\alpha} = 1$.
            \end{enumerate}
        \end{theorem}
        Typically, the non-zero weights $\alpha$ of the adjoint representations of $\g$ such that $\g_{\alpha} \not \cong 0$ are called \textbf{roots}, and the subset of $\h^*$ consisting of such roots is denoted by:
            $$\Phi$$
        It is then possible to \textit{(non-canonically) choose} a partition of $\Phi$ into two disjoint subsets $\Phi^{\pm}$, consisting of \textbf{positive} and \textbf{negative} roots respectively. There are various ways of doing so, but for our purposes, we find it \textit{convenient} to make use of invariant and non-deegenerate bilinear forms on $\g$ as ultimately, we will be very often concerned with how such bilinear forms pair elements of $\Phi$ together\footnote{Strictly speaking, this sentence does not make sense yet, since $\Phi \subset \h^* \subset \g^*$, while the bilinear forms mentioned are on $\g$. We shall elaborate shortly.}. Before moving on, however, let us note that for any root $\alpha \in \Phi$, we have that:
            $$[\g_{-\alpha}, \g_{\alpha}] \subset \h$$
        Let us also observe that:
            $$\g = [\g, \g]$$
        which is true per the assumption that $\g$ is simple, and implies that any element $h \in \h$ arises as $h = [x, y]$ for some $x \in \g_{-\alpha}, y \in \g_{\alpha}$ (for some $\alpha \in \Phi$). Together with the fact that $\h$ is finite-dimensional, meaning that $\h \cong \h^*$, these facts imply that the vector space $\h^*$ is generated by the set $\Phi$.

        The \textbf{root lattice} of $\g$ is the $\Z$-module generated by the ($\Z$-linearly \textit{depedent}) set $\Phi$:
            $$Q := \Z \Phi$$
        and in light of the root space decomposition of $\g$, one sees that $\g$ is graded by the $\Z$-module $Q$. Given an element:
            $$\mu := \sum_{\alpha \in \Phi} m_{\alpha} \alpha \in Q$$
        we define its \textbf{height} to be the sum of the coefficients:
            $$\height \mu := \sum_{\alpha \in \Phi} m_{\alpha}$$
        $Q$ is therefore partially ordered by heights. The highest element of $\Phi$, i.e. the \textbf{highest root}, is usually denoted by:
            $$\theta$$
        This element is indeed uniquely defined, thanks to the fact that for any two elements $\alpha, \beta \in \Phi \cup \{0\}$, one has that:
            $$[\g_{\alpha}, \g_{\beta}] = \g_{\alpha + \beta}$$
        which implies the existence of $\theta \in \Phi$ such that:
            $$\forall \beta \in \Phi \cup \{0\}: \g_{\theta + \beta} \cong 0$$
        We will be needing this distinguished element in subsection \ref{subsection: a_fixed_untwisted_affine_kac_moody_algebra}. Let us also note that because $\Phi$ generated $\h^*$, the $\Z$-module $Q$ is actually a lattice inside $\h^*$, in the sense that:
            $$Q \tensor_{\Z} k \cong \h^*$$
        thus justifying the terminology \say{root lattice}.

        Now, as eluded to above, one of the most important features of $\g$ is that it posses an invariant and non-degenerate $k$-bilinear form:
            $$(-, -)_{\g}$$
        which is unique up to $k^{\x}$-multiples. The canonical choice is the so-called Killing form, given by $\kappa(x, y) := \trace(\ad(x) \circ \ad(y))$ for all $x, y \in \g$, but in various other context, other natural choices such as the more general trace form $\trace(\rho(x) \rho(y))$ (associated to some representation $\rho: \g \to \gl(V)$; one recovers $\kappa$ by taking $\rho := \ad$) are also very useful. What is important to us is that the Killing form is essentially unique: if $\kappa'$ is any invariant and non-degenerate symmetric $k$-bilinear form on $\g$ then there will exist a \textit{unique} $c \in k^{\x}$ such that $\kappa' = c \kappa$.

        Now, such a bilinear form $(-, -)_{\g}$ helps us associate to $\g$ a \say{root system} (to be defined shortly), and the upshot is that these \say{roots systems} classify finite-dimensional simple Lie algebras (over algebraically closed fields of characteristic $0$) up to isomorphisms; again, we remark that root systems can be constructed in the absence of such a bilinear form, but we find it more convenient to make use of $(-, -)_{\g}$. A choice of \textbf{simple roots}:
        \todo[inline]{Fixed simple root construction.}
            $$\{\alpha_i\}_{i \in \simpleroots}$$
        can then be made: this is to be a basis for the $\Z$-module $Q$ (cf. \cite[Subsection 10.1, p. 47]{humphreys_lie_algebras}) - and hence a basis for the vector space $\h^*$ - such that:
            $$\Phi \not \subset \Z_{\geq 0}\{\alpha_i\}_{i \in \simpleroots} \cap \Z_{\leq 0}\{\alpha_i\}_{i \in \simpleroots}$$
        In other words, any root $\alpha \in \Phi$ of the form:
            $$\alpha = \sum_{i \in \simpleroots} m_i \alpha_i \in \Z\{\alpha_i\}_{i \in \simpleroots}$$
        where the coefficients $m_i \in \Z$ are (exclusively) either non-negative or non-positive. The sets:
            $$\Phi^{\pm} := \Phi \cap \pm \Z_{\geq 0} \{\alpha_i\}_{i \in \simpleroots}$$
        shall be referred to, respectively, as the sets of \textbf{positive} and \textbf{negative} roots. Elements of $Q^+ := \Z_{\geq 0} \simpleroots$ are typically regarded as being \textbf{positive} (and in particular, the simple roots are positive by convention) and conversely, elements of $Q^- := \Z_{\leq 0} \simpleroots$ are typically said to be \textbf{negative}. One can also easily show that $\Phi^{\pm} = \Phi \cap Q^{\pm}$.
        
        The \textbf{Cartan matrix} of $\g$ can then be defined to be:
            $$C := (c_{ij})_{i, j \in \simpleroots} := \left( \frac{2(\alpha_i, \alpha_j)_{\g}}{(\alpha_i, \alpha_i)_{\g}} \right)_{i, j \in \simpleroots}$$
        It can be shown that $C$ can be symmetrised, in the sense that there exists an invertible diagonal matrix:
            $$D := (d_{ij})_{i, j \in \simpleroots} = \left(\frac{2\delta_{i, j}}{(\alpha_i, \alpha_j)_{\g}}\right)_{i, j \in \simpleroots}$$
        and a symmetric matrix:
            $$A := (a_{ij})_{i, j \in \simpleroots} = \left((\alpha_i, \alpha_j)_{\g}\right)_{i, j \in \simpleroots}$$
        (which is nothing but the matrix representation of the bilinear form $(-, -)_{\g}$ with respect to the basis $\{\alpha_i\}_{i \in \simpleroots}$), such that:
            $$C = DA$$
        \todo[inline]{Removed mentions of Dynkin diagrams}

        From theorem \ref{theorem: root_space_decomposition_for_finite_dimensional_simple_lie_algebras}, we see that for any given positive root $\alpha \in \Phi^+$ and corresponding choices of root vectors\footnote{Choices of which are unique up to non-zero scalar multiples, since subspaces of non-zero weights are equally $1$-dimensional (see theorem \ref{theorem: root_space_decomposition_for_finite_dimensional_simple_lie_algebras}).} $x_{\pm\alpha} \in \g_{\pm\alpha}$, one has that:
            $$(h, [x_{\alpha}, x_{-\alpha}])_{\g} = ([h, x_{\alpha}], x_{-\alpha})_{\g} = \alpha(h) (x_{\alpha}, x_{-\alpha})_{\g}$$
        for all $h \in \h$. By choosing the root vectors $x_{\pm \alpha} \in \g_{\pm \alpha}$ such that:
            $$(x_{\alpha}, x_{-\alpha})_{\g} = 1$$
        the above reduces to:
            $$(h, [x_{\alpha}, x_{-\alpha}])_{\g} = \alpha(h)$$
        for all $h \in \h$. Per the non-degneracy of the bilinear form $(-, -)_{\g}$, there must then exist an element:
            $$h_{\alpha} := [x_{\alpha}, x_{-\alpha}] \in \h$$
        such that:
            $$(-, h_{\alpha})_{\g} = \alpha$$

        A consequence of this is that for each positive root $\alpha \in \Phi^+$, there exists an injective Lie algebra homomorphism:
            $$\sl_2(k) \to \g$$
        given by:
            $$x^{\pm} \mapsto x_{\pm \alpha}, h \mapsto \alpha^{\vee}$$
        where $\alpha^{\vee} := \frac{2}{(h_{\alpha}, h_{\alpha})_{\g}} h_{\alpha}$. The reasoning for introducing the renormalisation $\frac{2}{(h_{\alpha}, h_{\alpha})_{\g}}$ is that we would like to have that:
            $$[\alpha^{\vee}, x_{\pm \alpha}] = \pm 2 x_{\pm \alpha}$$
        which forces the element $\alpha^{\vee} \in k h_{\alpha}$ to be such that:
            $$2 = \alpha( \alpha^{\vee} ) = (\alpha^{\vee}, h_{\alpha})_{\g}$$
        If we were to suppose that $\alpha^{\vee} := N_{\alpha} h_{\alpha}$ for some scalar $N \in k$ then we would see that:
            $$N_{\alpha} (h_{\alpha}, h_{\alpha})_{\g} = 2$$
        which means that:
            $$N_{\alpha} = \frac{2}{(h_{\alpha}, h_{\alpha})_{\g}}$$
        \todo[inline]{Fixed coroot scaling.}
        
        The next result is a fundamental theorem in the study of finite-dimensional simple Lie algebras over algebraically closed fields of characteristic $0$. It essentially asserts that to give such a Lie algebra via a presentation by generators and relations is the same as to specify its Cartan matrix. The result is not only practically useful, but also is the mean by which one approaches Kac-Moody algebras, where the Cartan matrix is no longer required to be positive-definite; we refer the reader to subsection \ref{subsection: a_fixed_untwisted_affine_kac_moody_algebra} for a partial recollection of this story, and to \cite[Chapters 1-8]{kac_infinite_dimensional_lie_algebras} for details. 
        \begin{theorem}[Serre's Theorem] \label{theorem: serre_theorem_for_finite_dimensional_simple_lie_algebras}
            $\g$ is isomorphic to the Lie algebra generated by the set:
                $$\{h_i, x_i^{\pm}\}_{i \in \simpleroots}$$
            whose elements are subjected to the following relations, given for all $i, j \in \simpleroots$:
                $$[h_i, h_j] = 0$$
                $$[h_i, x_j^{\pm}] = \pm c_{ij} x_j^{\pm}, [x_i^+, x_j^-] = \delta_{ij} h_i$$
            and for all $i \not = j \in \simpleroots$, there are also the so-called \textbf{Serre relations}:
                $$\ad(x_i^{\pm})^{1 - c_{ij}}(x_j^{\pm}) = 0$$
            This is usually referred to as the \textbf{Chevalley-Serre} presentation for $\g$, and the relations are usually referred to collectively as the \textbf{Chevalley-Serre relations}.
        \end{theorem}

        We end this subsection with a brief analysis of the easiest possible example of a finite-dimensional simple Lie algebra. 
        \begin{example}[$\sl_2$] \label{example: sl_2}
            Recall that $\sl_2(k)$ is the kernel of the trace map:
                $$\trace: \gl_2(k) \to k$$
            i.e. it is the Lie algebra of trace-zero $2 \x 2$-matrices whose Lie bracket is the usual commutator of matrices. It is of dimension $\dim_k \gl_2(k) - \dim_k k = 4 - 1 = 3$, and a common choice of basis is:
                $$\left\{ \begin{pmatrix} 1 & 0 \\ 0 & -1 \end{pmatrix}, \begin{pmatrix} 0 & 1 \\ 0 & 0 \end{pmatrix}, \begin{pmatrix} 0 & 0 \\ 1 & 0 \end{pmatrix} \right\}$$
                
            Because the first element is not nilpotent, whereas the other two elements are nilpotent, any Cartan subalgebra of $\sl_2(k)$ must therefore be spanned by the matrix $\begin{pmatrix} 1 & 0 \\ 0 & -1 \end{pmatrix}$. One can then show that by setting $x^+ := \begin{pmatrix} 0 & 1 \\ 0 & 0 \end{pmatrix}$ and $x^- := \begin{pmatrix} 0 & 0 \\ 1 & 0 \end{pmatrix}$, the set:
                $$\{h, x^{\pm}\}$$
            then becomes a set of Chevalley-Serre generators for $\sl_2(k)$. The relations that they must satisfy, according to theorem \ref{theorem: serre_theorem_for_finite_dimensional_simple_lie_algebras}, are therefore:
                $$[h, x^{\pm}] = \pm 2 x^{\pm}, [x^+, x^-] = h$$
            
            In this case, the Cartan matrix is just:
                $$\begin{pmatrix} 2 \end{pmatrix}$$
            and the Dynkin diagram consists of only a single vertex and no edges:
                $$\bullet$$
        \end{example}

    \subsection{Perfect Lie algebras and their central extensions}
        \begin{definition}[Extensions of Lie algebras] \label{def: lie_algebra_extensions}
            Fix a Lie algebra $\a$.
        
            A \textbf{Lie algebra extension} is a short exact sequence of Lie algebras:
                $$0 \to \Omega \to \frake \xrightarrow[]{\pi} \a \to 0$$
            or equivalently, a Lie algebra epimorphism $\pi: \frake \to \a$ ($\Omega$ is then uniquely determined as $\ker \pi$). 
            
            A morphism between two such extensions of the given Lie algebra $\a$, say:
                $$\varphi: (\pi': \frake' \to \a) \to (\pi: \frake \to \a)$$
            is then a commutative diagram of Lie algebras and Lie algebra homomorphisms:
                $$
                    \begin{tikzcd}
                	0 & {\Omega'} & {\frake'} & \a & 0 \\
                	0 & \Omega & \frake & \a & 0
                	\arrow["{\varphi|_{\z'}}", from=1-2, to=2-2]
                	\arrow["\varphi", from=1-3, to=2-3]
                	\arrow["{\id_{\a}}", from=1-4, to=2-4]
                	\arrow[from=1-1, to=1-2]
                	\arrow[tail, from=1-2, to=1-3]
                	\arrow["{\pi'}", two heads, from=1-3, to=1-4]
                	\arrow[from=1-4, to=1-5]
                	\arrow[from=2-1, to=2-2]
                	\arrow[tail, from=2-2, to=2-3]
                	\arrow["\pi", two heads, from=2-3, to=2-4]
                	\arrow[from=2-4, to=2-5]
                    \end{tikzcd}
                $$
            wherein the rows are short exact sequences.
        \end{definition}
        \begin{definition}[Central extensions of Lie algebras]
            A Lie algebra extension:
                $$0 \to \Omega \to \frake \to \a \to 0$$
            is \textbf{central} if the elements of $\Omega$ are central in $\frake$; morphisms between central extensions are just morphisms of Lie algebra extensions (cf. definition \ref{def: lie_algebra_extensions}).
            
            A central extension $(p: \fraku \to \a)$ is said to be \textbf{universal} if and only if for every other central extension $(\pi: \frake \to \a)$, there exists a \textit{unique} morphism:
                $$(p: \fraku \to \a) \to (\pi: \frake \to \a)$$
            If a Lie algebra $\a$ admits a universal central extension (UCE) then we will denote it by $\uce(\a)$.
        \end{definition}
        \begin{remark}
            Universal central extensions are unique up to unique isomorphisms. 
        \end{remark}
        \begin{remark}
            Since we are working over a field, all short exact sequences split when regarded as short exact sequences of vector spaces. In particular, this implies that given a Lie algebra extension:
                $$0 \to \Omega \to \frake \to \a \to 0$$
            the underlying vector space of $\frake$ will always be isomorphic to $\a \oplus \z$. From now on, this identification will be used without explicit mention.
        \end{remark}
        We will also need to know how the Lie brackets on extensions are given explicitly, though only for a certain class of extensions. The fundamental example is the construction of semi-direct products of Lie algebras.
        \begin{example}[Semi-direct products of Lie algebras] \label{example: lie_algebra_semi_direct_products}
            Let:
                $$\rho: \d \to \der_k(\t)$$
            be a Lie algebra homomorphism from a Lie algebra $\d$ to the Lie algebra of derivations on another Lie algebra $\t$, making $\t$ a $\d$-module. The canonical extension of $\d$ by $\t$, known as the \textbf{semi-direct product} of $\d$ by $\t$, and commonly denoted by:
                $$\t \rtimes \d$$
            is the extension with Lie bracket given by:
                $$[ K + D, K' + D' ]_{\t \rtimes \d} := ( [K, K']_{\t} + \rho(D)(K') - \rho(D')(K) ) + [D, D']_{\d}$$
            for all $D, D' \in \d$ and all $K, K' \in \t$.

            Observe, also, that should $\t$ be abelian, the summand $[K, K']_{\t}$ shall vanish, and the bracket on $\t \rtimes \d$ shall then take the form:
                $$[ K + D, K' + D' ]_{\t \rtimes \d} = ( \rho(D)(K') - \rho(D')(K) ) + [D, D']_{\d}$$
            If we also have that $\t \subseteq \z( \t \rtimes \d )$, i.e. that the extension of $\d$ by $\t$ is central, then the bracket on $\t \rtimes \d$ will reduce furthermore down to:
                $$[ K + D, K' + D' ]_{\t \rtimes \d} = [D, D']_{\d}$$
            meaning that in this case, one has an isomorphism of Lie algebras:
                $$\t \rtimes \d \cong \t \oplus \d$$
        \end{example}
        Of course, there are central Lie algebra extensions other than trivial ones (i.e. direct sums of Lie algebras), and more generally, there are definitely extensions of Lie algebras that are not semi-direct products. One way to construct such extensions is to add an extra term - a \say{$2$-cocycle} - whose function is to \say{twist} the Lie brackets on semi-direct products into a slight variant thereof. More preecisely, if we are granted a Lie algebra extension:
            $$0 \to \t \to \frake \to \d \to 0$$
        then the aforementioned $2$-cocycles shall serve to measure the difference between $[D, D']_{\frake}$ and $[D, D']_{\d}$, for $D, D' \in \d$. One small technical assumption that we must make in order to be able to even make this comparison in the first place, is that $\t$ must be a $\d$-module.
        \todo[inline]{Twisted semi-direct product and $2$-cocycle definition}
        \begin{definition}[$2$-cocycles and twisted semi-direct products] \label{def: twisted_semi_direct_products} 
            Suppose that:
                $$0 \to \t \to \frake \to \d \to 0$$
            is an extension of Lie algebras. This is said to be a \textbf{twisted semi-direct product} corresponding to a \textbf{$2$-cocycle}\footnote{The terminology stems from the fact that isomorphism classes of \textit{central} extensions, say of $\d$ by $\t$, are in bijection with the elements of $H^2_{\Lie}(\d, \t)$. This is a standard homological algebraic fact; see e.g. \cite{hilton_stammbach_homological_algebra}. We will not, however, be making use of any homological algebra. This remark is purely for an etymological purpose.} $\sigma$ of $\d$ with values in $\t$ if the Lie bracket on $\frake$ is of the form\footnote{Note that in writing $[-, -]_{\frake} = [-, -]_{\t \rtimes \d} + \sigma$, we have actually slightly abused notations. Technically, this should have been $[-, -]_{\frake} = [-, -]_{\t \rtimes \d} + \sigma \circ ( \pi \wedge \pi )$, with $\pi: \frake \to \d$ being the canonical projection.}:
                $$[-, -]_{\frake} = [-, -]_{\t \rtimes \d} + \sigma$$
            for some skew-symmetric bilinear map:
                $$\sigma: \bigwedge^2 \d \to \t$$
            making $[-, -]_{\frake}$ satisfy the Jacobi identity. Equivalently, one can simply insist that $\sigma$ itself satisfies the Jacobi identity. 
            
            In that case, we shall write:
                $$\frake \cong \t \rtimes^{\sigma} \d$$
            Should $\frake$ be a central extension of $\d$ by $\t$, i.e. $\t \subseteq \z(\frake)$, and hence the $\d$-action on $\t$ is trivial, then to put emphasis on this fact (as it is entirely possible for $\t$ to be abelian while $\t \not \subset \z(\frake)$), we might instead write:
                $$\frake \cong \t \oplus^{\sigma} \d$$
        \end{definition}
        \begin{remark}
            Clearly, semi-direct products are twisted semi-direct products corresponding to the $2$-cocycle $0$.
        \end{remark}
        \todo[inline]{Added twisted semi-direct product criterion.}
        \begin{proposition}[When is a Lie algebra extension a twisted semi-direct product ?] \label{prop: twisted_semi_direct_product_criterion}
            Suppose that:
                $$0 \to \t \to \frake \xrightarrow[]{\pi} \d \to 0$$
            is an extension of Lie algebras. Then, $\t$ will be a $\d$-module if and only if there exists a Lie algebra isomorphism:
                $$\frake \cong \t \rtimes^{\sigma} \d$$
            for some $2$-cocycle $\sigma$ of $\d$ with values in $\t$.
        \end{proposition}
            \begin{proof}
                Suppose firstly that $\t$ is a $\d$-module, say defined by a Lie algebra homomorphism:
                    $$\rho: \d \to \gl(\t)$$
                Also, pick a section $\d \to \frake$. Given $K, K' \in \t$ and $D, D' \in \d$, one then has that:
                    $$
                        \begin{aligned}
                            [ K + D, K' + D' ]_{\frake} & = ( [K, K']_{\t} + [D, K']_{\frake} - [D', K]_{\frake} ) + [D, D']_{\frake}
                            \\
                            & = ( [K, K']_{\t} + \rho(D)(K') - \rho(D')(K) ) + [D, D']_{\frake}
                        \end{aligned}
                    $$
                If we can now define a skew-symmetric bilinear map:
                    $$\sigma: \bigwedge^2 \d \to \t$$
                to measure the difference between $[D, D']_{\frake}$ and $[D, D']_{\d}$ (implicitly understood to be the image under the previously fixed section $\d \to \frake$), i.e. by the following formula:
                    $$\sigma(D, D') := [D, D']_{\frake} - [D, D']_{\d}$$
                then the Lie bracket $[-, -]_{\frake}$ can then be specified by:
                    $$[ K + D, K' + D' ]_{\frake} := ( [K, K']_{\t} + \rho(D)(K') - \rho(D')(K) ) + [D, D']_{\d} + \sigma(D, D')$$
                Note also that because both $[-, -]_{\frake}$ and $[-, -]_{\d}$ are Lie brackets and hence satisfy the Jacobi identity, so must $\sigma$ too. $\sigma$ must then be a $2$-cocycle of $\d$ with values in $\t$ by definition, so it remains to show that $\sigma$ is well-defined, namely that the codomain of $\sigma$ lies inside $\t$. For this, simply note that the following holds for all $D, D' \in \d$:
                    $$\pi( \sigma(D, D') ) = \pi( [D, D']_{\frake} ) - \pi( [D, D']_{\d} ) = [D, D']_{\d} - [D, D']_{\d} = 0$$
                where the second equality holds thanks to the fact that $\pi: \frake \to \d$ is a Lie algebra homomorphism. 

                Conversely, suppose that the extension is a twisted semi-direct product, say:
                    $$\frake \cong \t \rtimes^{\sigma} \d$$
                for some $2$-cocycle $\sigma$ of $\d$ with values in $\t$. By the definition of twisted semi-direct products, the operators of the form:
                    $$[D, -]_{\frake}$$
                must then be derivations on $\t$, clearly making $\t$ a $\d$-module. 
            \end{proof}
        The following corollary follows suite, though we would like to state it nevertheless, as it will be useful later on when we get to discussing Kassel's description of UCEs of current algebras (see \ref{theorem: kassel_realisation} and the discussion preceding it).
        \todo[inline]{Central extensions are twisted semi-direct products.}
        \begin{corollary}[Central extensions are twisted semi-direct products] \label{coro: lie_brackets_on_central_extensions}
            Suppose that:
                $$0 \to \Omega \to \frake \to \a \to 0$$
            is an extension of a Lie algebra $\a$ by an $\a$-module $\Omega$, equipped with an abelian Lie algebra structure. Such an extension is central if and only if there exists a $2$-cocycle:
                $$\e: \bigwedge^2 \a \to \Omega$$
            such that:
                $$\frake \cong \a \oplus^{\e} \Omega$$
        \end{corollary}
            \begin{proof}
                Clear from proposition \ref{prop: twisted_semi_direct_product_criterion} and the discussion at the end of example \ref{example: lie_algebra_semi_direct_products}.
            \end{proof}
        \begin{remark}
            In cohomological terms, corollary \ref{coro: lie_brackets_on_central_extensions} can interpreted as the fact that there is a bijection a between elements $\e \in H^2_{\Lie}(\a, \Omega)$ and isomorphism classes of \textit{central} extensions of $\a$ by an $\a$-module $\Omega$ equipped with the trivial Lie bracket. This is a very common formulation of the result, especially in textbooks on homological algebra, e.g. \cite{hilton_stammbach_homological_algebra}.
        \end{remark}

        Let us now turn our attention towards the problem of existence of UCEs, now that we know how Lie brackets on central extensions are given in general. A particular class of Lie algebras that happen to admit UCEs are the so-called \say{perfect} ones.
        \begin{definition}[Perfect Lie algebras]
            A Lie algebra is said to be \textbf{perfect} if it is equal to its derived subalgebra. 
        \end{definition}
        \begin{example}
            Since simple Lie algebras lack non-zero ideals by definition, they are perfect. 
        \end{example}
        \begin{example}
            Let $A$ be any commutative algebra over a field $k$, and let $\g$ be a simple Lie algebra over $k$. Endow the $k$-vector space $\g \tensor_k A$ with the Lie bracket:
                $$\forall x, y \in \g: \forall f, g \in A: [x f, y g]_{\g \tensor_k A} := [x, y]_{\g} \tensor fg$$
            Then, $\g \tensor_k A$ will be perfect when regraded as a Lie algebra over $k$, precisely because $\g$ is simple.
        \end{example}
        \begin{example}
            Counter-examples include nilpotent Lie algebras (e.g. abelian Lie algebras). For such Lie algebras, their derived subalgebras are always proper Lie subalgebras (and in particular, derived subalgebras of abelian Lie algebras are zero). 
        \end{example}
        \begin{proposition}[Perfect Lie algebras admit UCEs] \label{prop: perfect_lie_algebras_admit_UCEs}
            \cite[Lemma 1.10]{garland_arithmetics_of_loop_groups} Let $k$ be a field and $\a$ is a Lie algebra over $k$. Then $\a$ admits a UCE if and only if it is perfect.
        \end{proposition}

        Let us conclude this subsection by making note of the following fact about the process of extending invariant bilinear forms from Lie algebras to their central extensions.
        \begin{remark}[Extending invariant bilinear forms to central extensions] \label{remark: extending_bilinear_forms_to_central_extensions}
            Let $k$ be a field.
            
            Let $\a$ be a Lie algebra over $k$ and $(-, -)$ is an invariant symmetric $k$-bilinear form thereon. Additionally, let:
                $$0 \to \Omega \to \frake \to \a \to 0$$
            be a central Lie algebra extension of $\a$ such that $\Omega \not \cong 0$. By proposition \ref{prop: twisted_semi_direct_product_criterion}, we know that for all $X, Y \in \a$, we have that:
                $$[X, Y]_{\frake} = [X, Y]_{\a} + K(X, Y)$$
            for some $K(X, Y) \in \Omega$ depending on the choices of $X, Y$.
            
            Suppose that $\<-, -\>$ is any extension of $(-, -)$ to $\frake$, i.e. $\<-, -\>$ is a symmetric $k$-bilinear form on $\frake$ such that:
                $$\<-, -\>|_{\Sym^2_k( \a )} = (-, -)$$
            Assume furthermore, that:
                $$\<\a, \Omega\> = 0, \<\Omega, \Omega\> = 0$$ 
            Now, note that $\<-, -\>$ is necessarily invariant: indeed, for any $X, Y, Z \in \a$, we have that:
                $$
                    \begin{aligned}
                        \<[X, Y]_{\frake}, Z\> & = \< [X, Y]_{\a} + K(X, Y), Z \>
                        \\
                        & = ([X, Y]_{\a}, Z)
                        \\
                        & = (X, [Y, Z]_{\a})
                        \\
                        & = \< X, [Y, Z]_{\a} + K(Y, Z) \>
                        \\
                        & = \< X, [Y, Z]_{\frake} \>
                    \end{aligned}
                $$
            Furthermore, $\<-, -\>$ must be degenerate, regardless of whether or not $(-, -)$ is degenerate to begin with: to see why, it suffices to show that the radical\footnote{If $V$ is a vector space and $B$ is a symmetric bilinear form on $V$, then the \textbf{radical} of $B$ is then given by $\Rad B := \{v \in V \mid \forall w \in V: B(v, w) = 0\}$.} of $\<-, -\>$ is non-zero, for which we can simply consider the following for any $X, Y \in \a$ and any $K \in \Omega$, and then exploit the invariance of $\<-, -\>$:
                $$\<K, [X, Y]_{\frake}\> = \<[K, X]_{\frake}, Y\> = \<0, Y\> = 0$$

            Not every Lie algebra with a non-trivial centre arises as a central extension (see \cite[Section 1]{garland_arithmetics_of_loop_groups} for details), so it is not true that any invariant symmetric bilinear form on such a Lie algebra must be degenerate. Central elements can pair non-trivially with other elements, such as in the construction of untwisted affine Kac-Moody algebras (cf. subsection \ref{subsection: a_fixed_untwisted_affine_kac_moody_algebra}) or in the construction of so-called \say{extended toroidal Lie algebras} (cf. definitions \ref{def: yangian_extended_toroidal_lie_algebras} and \ref{def: general_extended_toroidal_lie_algebras}).
        \end{remark}