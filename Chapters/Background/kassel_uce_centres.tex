\section{The Kassel realisation of UCEs of current Lie algebras}
    \subsection{A recollection of K\"ahler differentials}
        There are many perspectives on algebraic differential forms, but for our purposes, the following will be the easiest to use.
        \begin{definition}[Modules of K\"ahler differentials] \label{def: kahler_differentials}
            Let $k$ be a base commutative ring and let $A$ be a commutative $k$-algebra, defined by a multiplication map:
                $$\mu_{A/k}: A \tensor_k A \to A$$
            The $A$-module of K\"ahler differentials $\Omega^1_{A/k}$ relative to the ring map $k \to A$ is then given by:
                $$\Omega^1_{A/k} := I/I^2$$
            where $I := \ker \mu_{A/k}$. Elements of $\Omega^1_{A/k}$ are typically referred to as (differential) $1$-forms.
        \end{definition}
        \begin{remark}[Diffentials satify the Leibniz rule]
            Observe that the $A \tensor_k A$-ideal:
                $$I := \ker \mu_{A/k}$$
            is generated by elements of the form $1 \tensor f - f \tensor 1$, for all $f \in A$. We then see that:
                $$\Omega^1_{A/k} := I/I^2 \cong I \tensor_{A \tensor_k A} ( (A \tensor_k A)/I ) \cong I \tensor_{A \tensor_k A} A$$
            should we regard $A$ as a commutative $A \tensor_k A$-algebra; note that the last isomorphism holds thanks to the fact that the multiplication map $\mu_{A/k}: A \tensor_k A \to A$ is \textit{a priori} surjective. From this, one infers that there exists a canonical $k$-linear derivation:
                $$d: A \to \Omega^1_{A/k}$$
                $$f \mapsto 1 \tensor f - f \tensor 1$$
            Indeed, for every $f, g \in A$, the Leibniz rule is satisfied:
                $$g df + f dg = g(1 \tensor f - f \tensor 1) + f(1 \tensor g - g \tensor 1) = gf \tensor 1 - fg \tensor 1 = d(fg)$$
        \end{remark}
        The following well-known lemmas are very useful. A proof can be be found in any standard reference on general commutative algebra (cf. e.g. \cite[\href{https://stacks.math.columbia.edu/tag/00AO}{Tag 00AO}]{stacks}).
        \begin{lemma}[Universal property of modules of K\"ahler differentials]
            Let $k$ be a base commutative ring and let $A$ be a commutative $k$-algebra. Then, the $A$-module of K\"ahler differentials relative to the ring map $k \to A$ corepresents the functor of $k$-linear derivations from $A$, i.e. there is a natural isomorphism of functors $A\mod \to A\mod$ as follows:
                $$\Der_k(A, -) \cong \Hom_A(\Omega^1_{A/k}, -)$$
            Because of this, the category of $k$-linear derivations $d_M: A \to M$ admit an initial object, namely $d: A \to \Omega^1_{A/k}$. 
        \end{lemma}
        \begin{corollary}[$1$-forms are dual to derivations]
            $k$-linear derivations from $A$ to itself are dual to differential $1$-forms relative to $k \to A$ in the following manner:
                $$\Der_k(A) := \Der_k(A, A) \cong \Hom_A(\Omega^1_{A/k}, A)$$
        \end{corollary}
        \begin{lemma}[$1$-forms over polynomial algebras]
            \cite[\href{https://stacks.math.columbia.edu/tag/00RX}{Tag 00RX}]{stacks} Let $k$ be a commutative ring and fix some $n \in \Z_{\geq 0}$, and consider the canonical ring homomorphism $k \to k[v_1, ..., v_n]$. In this case, $\Omega^1_{[n]} := \Omega^1_{k[v_1, ..., v_n]/k}$ will be free and of finite rank $n$ as an $k[v_1, ..., v_n]$-module; in particular, it admits the set $\{dv_1, ..., dv_n\}$ as a $k[v_1, ..., v_n]$-linear basis.
        \end{lemma}
        \begin{corollary}[Derivations as differential operators]
            The $k$-module of $k$-linear derivations from $k[v_1, ..., v_n]$ to itself admit a particularly simple and useful description:
                $$\Der_k(k[v_1, ..., v_n]) \cong \bigoplus_{1 \leq i \leq n} k[v_1, ..., v_n] \del_{v_i}$$
        \end{corollary}

    \subsection{Centres of UCEs of current Lie algebras}
        \begin{convention} 
            From now on, we fix a finite-dimensional simple Lie algebra:
                $$\g$$
            over an algebraically closed field $k$ of characteristic $0$, equipped with a symmetric and non-degenerate invariant $k$-bilinear form $(-, -)_{\g}$. It is known that such a bilinear form is unique up to $k^{\x}$-multiples, so for all intents and purposes, it can be assumed to be the Killing form, though this assumption is not necessary. 
        \end{convention}

        \begin{convention}
            Let us fix a commutative $k$-algebra $A$.

            We shall be writing:
                $$\g_A := \g \tensor_k A$$
            and endow this $k$-vector space with the following Lie bracket:
                $$\forall x, y \in \g: \forall f, g \in A: [x f, y g]_{\g_A} := [x, y]_{\g} fg$$
        \end{convention}

        \begin{convention}
            Let $R \to S$ be a homomorphism of commutative rings. Then, let us write:
                $$\bar{\Omega}^1_{S/R} := \Omega^1_{S/R}/dS$$
            Note that this is only an $R$-module, not an $S$-module.
        \end{convention}
        
        \begin{theorem}[The Kassel realisation] \label{theorem: kassel_realisation}
            \cite[Corollary 3.5]{kassel_universal_central_extensions_of_lie_algebras} Let $k$ now an algebraically closed field of characteristic $0$ again, as in convention \ref{conv: a_fixed_finite_dimensional_simple_lie_algebra}. 

            For the perfect Lie $k$-algebra $\g_A$, we have that:
                $$\z(\uce(\g_A)) \cong \bar{\Omega}^1_{A/k}$$
        \end{theorem}
            \begin{proof}[Proof sketch]
                Kassel constructed in the proof of \cite[Theorem 3.3]{kassel_universal_central_extensions_of_lie_algebras} a $k$-linear map:
                    $$\e: \bigwedge^2 \g_A \to \bar{\Omega}^1_{A/k}$$
                by the formula:
                    $$\forall x, y \in \g, \forall f, g \in A: \e(x f, y g) := (x, y)_{\g} f \bar{d}g$$
                which can be shown - relying on the $\g$-invariance of the bilinear form $(-, -)_{\g}$ - to be an element of $H^2_{\Lie}(\g_A, k)$ and hence gives a central extension $\fraku$ of $\g_A$ by $\bar{\Omega}_{A/k}^1$, whose underlying $k$-vector space is:
                    $$\g_A \oplus \bar{\Omega}_{A/k}^1$$
                and whose Lie bracket is:
                    $$[-, -]_{\fraku} = [-, -]_{\g_A} + \e$$
                It is also not hard to see that there is a section map $\g_A \to \fraku$, implying that $\fraku$ is simply connected in the sense of definition \ref{def: simply_connected_lie_algebras}. Via proposition \ref{prop: UCEs_are_simply_connected}, we then see that:
                    $$\fraku \cong \uce(\g_A)$$
                concluding the proof.
            \end{proof}
        \begin{example}[UCEs of multiloop Lie algebras]
            Fix some $n \in \Z_{\geq 0}$.
        
            Consider the case:
                $$A := A_{[n]} := k[v_1^{\pm 1}, ..., v_n^{\pm 1}]$$
            (let $A_{[0]} := k$). Set:
                $$\g_{[n]} := \g \tensor_k A_{[n]}$$
                $$\tilde{\g}_{[n]} := \uce(\g_{[n]})$$
                $$\Omega^1_{[n]} := \Omega^1_{A_{[n]}/k}, \bar{\Omega}^1_{[n]} := \bar{\Omega}^1_{A_{[n]}/k}$$
            Much can be said about the centre $\bar{\Omega}^1_{[n]}$ of the UCE $\tilde{\g}_{[n]}$ of $\g_{[n]}$, especially in the cases where $n \leq 2$, where it is rather easy to provide an explicit basis for the $k$-vector space $\bar{\Omega}^1_{[n]}$. 

            We know that $\Omega^1_{[n]}$ is free and of rank $n$ on the set:
                $$\{dv_1, ..., dv_n\}$$
            This implies that $\bar{\Omega}^1_{[n]}$ is generated by elements:
                $$\bar{d}v_j$$
            that are subjected to the following relation:
                $$0 = \bar{d}( v_1^{m_1} ... v_n^{m_n} ) = \sum_{1 \leq j \leq n} m_j v_1^{m_1} ... v_j^{m_j - 1} ... v_n^{m_n} \bar{d}v_j$$
            From this, one infers that the elements:
                $$m_j^{-1} v_1^{m_1} ... v_j^{m_j - 1} ... v_n^{m_n} \bar{d}v_j$$
            form a basis for $\bar{\Omega}_{[n]}$ as a $k$-vector space.

            When $n = 0$ or $n = 1$, it is trivial to see that:
                $$\dim_k \bar{\Omega}^1_{[0]} \cong 0, \dim_k \bar{\Omega}^1_{[1]} = 1$$

            When $n = 2$, consider the following: $\bar{\Omega}^1_{[2]}$ now decomposes as a $k$-vector space in the following manner:
                $$\bar{\Omega}^1_{[2]} \cong ( \bigoplus_{(r, s) \in \Z^2} k K_{r, s}) \oplus k c_v \oplus k c_t$$
            wherein:
                $$
                    K_{r, s} :=
                    \begin{cases}
                        \text{$\frac1s v^{r - 1} t^s \bar{d}v$ if $(r, s) \in \Z \x (\Z \setminus \{0\})$}
                        \\
                        \text{$-\frac1r v^r t^{-1} \bar{d}t$ if $(r, s) \in (\Z \setminus \{0\}) \x \{0\}$}
                        \\
                        \text{$0$ if $(r, s) = (0, 0)$}
                    \end{cases}
                $$
                $$c_v := v^{-1} \bar{d}v, c_t := t^{-1} \bar{d}t$$
            In fact, any element of the form:
                $$v^m t^p \bar{d}(v^n t^q) \in \bar{\Omega}^1_{[2]}$$
            can be written in terms of the basis vectors $K_{r, s}, c_v, c_t$ in the following manner:
                $$v^m t^p \bar{d}(v^n t^q) = \delta_{(m, p) + (n, q), (0, 0)} ( n c_v + q c_t ) + (np - mq) K_{m + n, p + q}$$
        \end{example}
        \begin{remark}[The $\Z$-grading on $\tilde{\g}_{[2]}$] \label{remark: Z_gradings_on_toroidal_lie_algebras}
            If $k$ is an arbitrary commutative ring and $A$ is a $\Z$-graded commutative $k$-algebra, say:
                $$A := \bigoplus_{n \in \Z} A_n$$
            and if $\a$ is a perfect Lie algebra over $k$, then $\a_A$ will also be $\Z$-graded, specifically in the following manner:
                $$\a_A := \a \tensor_k A \cong \bigoplus_{n \in \Z} \a \tensor_k A_n$$
            and for convenience, let us write $\a_{A_n} := \a \tensor_k A_n$ for each $n \in \Z$. This grading on $\a_A$ actually extends to the whole of $\uce(\a_A)$. Because the $A$-module $\Omega^1_{A/k}$ is generated by the set:
                $$\{da\}_{a \in A}$$
            whose elements are subjected to the relations:
                $$\forall a, b \in A: d(ab) - a d(b) - d(a) b = 0$$
           there is an induced $\Z$-grading on $\Omega^1_{A/k}$ given by:
                $$\deg d(ab) = \deg a d(b) = \deg d(a) b = \deg a + \deg b - 1$$
            for all $a, b \in A$. Inside $\Omega^1_{A/k}$, now viewed as a $k$-module, one has the $k$-submodule $\im d$, which is also $\Z$-graded: the grading is given like above, namely:
                $$\deg d(a) = \deg a - 1$$
            This $\Z$-grading induces another one on $\bar{\Omega}^1_{A/k}$, given by:
                $$\deg \bar{d}(ab) = \deg a \bar{d}(b) = \deg \bar{d}(a) b = \deg a + \deg b - 1$$
            for all $a, b \in A$.

            Now, let us focus once more on the case:
                $$A := A_{[2]}$$
            wherein the relevant $\Z$-grading is given by:
                $$\deg v := 0, \deg t := 1$$
            Since we know that the basis elements of $\bar{\Omega}^1_{[2]}$ are given by:
                $$
                    K_{r, s} :=
                    \begin{cases}
                        \text{$\frac1s v^{r - 1} t^s \bar{d}v$ if $(r, s) \in \Z \x (\Z \setminus \{0\})$}
                        \\
                        \text{$-\frac1r v^r t^{-1} \bar{d}t$ if $(r, s) \in (\Z \setminus \{0\}) \x \{0\}$}
                        \\
                        \text{$0$ if $(r, s) = (0, 0)$}
                    \end{cases}
                $$
                $$c_v := v^{-1} \bar{d}v, c_t := t^{-1} \bar{d}t$$
            (cf. \textit{loc. cit.}) their respective degrees with respect to the $\Z$-grading on $\bar{\Omega}_{[2]} \cong \bar{\Omega}^1_{[2]}$ are:
                $$
                    \deg K_{r, s} =
                    \begin{cases}
                        \text{$s - 1$ if $(r, s) \in \Z \x (\Z \setminus \{0\})$}
                        \\
                        \text{$-1$ if $(r, s) \in (\Z \setminus \{0\}) \x \{0\}$}
                        \\
                        \text{$0$ if $(r, s) = (0, 0)$}
                    \end{cases}
                $$
                $$\deg c_v = \deg c_t = -1$$
        \end{remark}