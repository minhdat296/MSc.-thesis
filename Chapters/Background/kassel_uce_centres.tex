\section{The Kassel realisation of UCEs of current Lie algebras}
    \subsection{A recollection of K\"ahler differentials}
        In \cite{kassel_universal_central_extensions_of_lie_algebras}, Kassel showed that centres of UCEs of current algebras (i.e. Lie algebras of the form $\g \tensor_k A$ for some commutative $k$-algebra $A$) are identified with spaces of algebraic differential $1$-forms modulo exact forms, so before proceeding, let us take the time to recall some relevant details about algebraic differential forms. 

        There are many approaches to algebraic differential forms. Ultimately, however, we will be relying on the description of modules of differential forms by generators and relations\footnote{One reason is that this makes it clear how, should a commutative $R$-algebra $S$ be graded by some abelian group $Z$, then that grading will induce a $Z$-grading on $\Omega^1_{S/R}$ as well (see remark \ref{remark: Z_gradings_on_toroidal_lie_algebras}).}, so let us define them that way.
        \begin{definition}[Modules of K\"ahler differentials] \label{def: kahler_differentials}
            Let $R$ be a base commutative ring and let $S$ be a commutative $R$-algebra. The $S$-module of K\"ahler differentials $\Omega^1_{S/R}$ relative to the ring map $R \to S$ is then the quotient of the $S$-module $S \tensor_R S$ by the $S$-submodule generated by the relations:
                $$ss' \tensor 1 - s' \tensor s - s \tensor s'$$
            given for all $s, s' \in S$
        \end{definition}
        \begin{remark}[Diffentials and derivations] \label{remark: differentials_and_derivations}
            The definition suggests to us that K\"ahler differentials might have something to do with derivations, and indeed they do. In fact, this relationship between algebraic $1$-forms and derivations comes from a universal property that the $S$-module $\Omega^1_{S/R}$ enjoys. Namely, for any $R$-module $M$, there exists a natural isomorphism of $S$-modules\footnote{The LHS is the $S$-module of $R$-linear derivations from $S$ into $M$.}:
                $$\Der_R(S, M) \cong \Hom_S(\Omega^1_{S/R}, M)$$
            (cf. \cite[\href{https://stacks.math.columbia.edu/tag/00RO}{Tag 00RO}]{stacks}). From this universal property, one infers that $\Omega^1_{S/R}$ is isomorphic to the $S$-module generated by the set:
                $$\{ds\}_{s \in S}$$
            whose elements are constrained by the relations:
                $$d(ss') = s' ds + s ds'$$
            given for all $s, s' \in S$. The isomorphism in question is given by:
                $$ds \mapsto s \tensor 1$$
            for all $s \in S$.
            
            A particular instance of this phenomenon is that $R$-linear derivations from $S$ to itself are dual to differential $1$-forms relative to $R \to S$, in the sense that there is an $S$-module isomorphism:
                $$\Der_R(S) := \Der_R(S, S) \cong \Hom_A(\Omega^1_{S/R}, A)$$
        \end{remark}
        \begin{remark}
            If $\Omega^1_{S/R}$ is finite free of rank $n$ over $A$, e.g.:
                $$\Omega^1_{S/R} \cong \bigoplus_{1 \leq i \leq n} A dv_i$$
            then we can identify:
                $$\Der_R(S) \cong \bigoplus_{1 \leq i \leq n} A \del_{v_i}$$
            where $\del_{v_i} \in \Der_R(S)$ are the preimages under the isomorphism $\Der_R(S) \xrightarrow[]{\cong} \Hom_S(\Omega^1_{S/R}, S)$ of the $S$-linear duals of the generators $dv_i \in \Omega^1_{S/R}$. 
        \end{remark}
        
        The following well-known lemmas are very useful. Proofs can be be found in any standard reference on general commutative algebra (e.g. \cite[\href{https://stacks.math.columbia.edu/tag/00AO}{Tag 00AO}]{stacks}).
        \begin{lemma}[$1$-forms over polynomial algebras] \label{lemma: 1_forms_over_polynomial_algebras}
            \cite[\href{https://stacks.math.columbia.edu/tag/00RX}{Tag 00RX}]{stacks} Let $R$ be a commutative ring and fix some $n \in \Z_{\geq 0}$. In this case, $\Omega^1_{R[v_1, ..., v_n]/R}$ will be free and of finite rank $n$ as an $R[v_1, ..., v_n]$-module; in particular, it admits the set $\{dv_1, ..., dv_n\}$ as a $R[v_1, ..., v_n]$-linear basis.
        \end{lemma}
        \begin{lemma}[Localisation of $1$-forms] \label{lemma: localisation_1_forms}
            \cite[\href{https://stacks.math.columbia.edu/tag/031G}{Tag 031G}]{stacks} Let $k$ be a field\footnote{... so that the only prime ideal of $k$ would be $(0)$.} and fix some $n \in \Z_{\geq 0}$, and consider the canonical ring homomorphism $k \to k[v_1, ..., v_n]$. Then, for any $1 \leq i \leq n$, there will be a $k[v_1, ..., v_n][v_i^{-1}]$-module isomorphism:
                $$\Omega^1_{k[v_1, ..., v_n][v_i^{-1}]/k} \cong \Omega^1_{k[v_1, ..., v_n]/k}[v_i^{-1}]$$
        \end{lemma}

        Let us end this subsection with the following examples, which will be useful for what comes later on.
        \begin{example}
            Let $k$ be a field.
        
            Per lemma \ref{lemma: 1_forms_over_polynomial_algebras}, we know that:
                $$\Omega^1_{k[v, t]/k} \cong k[v, t] dv \oplus k[v, t] dv$$
            Using lemma \ref{lemma: localisation_1_forms}, we then see that:
                $$\Omega^1_{k[v^{\pm 1}, t^{\pm 1}]/k} \cong k[v^{\pm 1}, t^{\pm 1}] dv \oplus k[v^{\pm 1}, t^{\pm 1}] dt$$
                $$\Omega^1_{k[v^{\pm 1}, t]/k} \cong k[v^{\pm 1}, t] dv \oplus k[v^{\pm 1}, t] dt$$
        \end{example}

    \subsection{UCEs of current Lie algebras}
        \begin{convention} \label{conv: a_fixed_finite_dimensional_simple_lie_algebra}
            From now on, we fix a finite-dimensional simple Lie algebra $\g$ over an algebraically closed field $k$ of characteristic $0$, equipped with a symmetric and non-degenerate invariant $k$-bilinear form $(-, -)_{\g}$. We fix also a Cartan subalgebra $\h$ of $\g$, along with all the accompanying data (e.g. root system, Cartan matrix, etc.) as in subsection \ref{subsection: finite_dimensional_simple_lie_algebras}. 
        \end{convention}

        For the sake of establishing the terminology, let us make the following definition:
        \begin{definition}[Current algebras] \label{def: current_algebras}
            Let $A$ be a commutative algebra over $k$. The vector space:
                $$\g \tensor_k A$$
            with the following Lie bracket:
                $$[x f, y g]_{\g \tensor_k A} := [x, y]_{\g} \tensor fg$$
            (given for all $x, y \in \g$ and all $f, g \in A$) shall then be referred to as a \textbf{current algebra}. 
                
            Also, we will be abbreviating:
                $$xf := x \tensor f$$
            for $x \in \g$ and $f \in A$.
        \end{definition}
        \begin{remark}[(Multi)loop algebras]
            When $A \cong k[v_1^{\pm 1}, ..., v_n^{\pm 1}]$, it is common to refer to $\g \tensor_k A$ as a \textbf{multiloop algebra}. When $n = 1$, we will only be saying \textbf{loop algebra}.
        \end{remark}

        \begin{convention}
            Let $R \to S$ be a homomorphism of commutative rings. Then, let us write:
                $$\bar{\Omega}^1_{S/R} := \Omega^1_{S/R}/dS$$
            Note that this is only an $R$-module, not an $S$-module. Let us also write $\bar{d}: S \to \bar{\Omega}^1_{S/R}$ for the canonical composition:
                $$
                    \begin{tikzcd}
                	S & {\Omega^1_{S/R}} \\
                	& {\bar{\Omega}^1_{S/R}}
                	\arrow["d", from=1-1, to=1-2]
                	\arrow[two heads, from=1-2, to=2-2]
                	\arrow["{\bar{d}}"', from=1-1, to=2-2]
                    \end{tikzcd}
                $$
        \end{convention}

        In order to characterise centres of UCE Kassel constructed in the proof of \cite[Theorem 3.3(iii)]{kassel_universal_central_extensions_of_lie_algebras} a $k$-linear map:
            $$\e: \bigwedge^2 (\g \tensor_k A) \to \bar{\Omega}^1_{A/k}$$
        by the formula\footnote{One can also take $\e(x f, y g) := -(x, y)_{\g} f \bar{d}g$, since $-f \bar{d}g \equiv g \bar{d}f \pmod{d(A)}$. This results in an isomorphic Lie algebra. We will switch back and forth between these choices depending on necessity.}:
            $$\e(x f, y g) := (x, y)_{\g} g \bar{d}f$$
        for all $x, y \in \g$ and for all $f, g \in A$. This can be shown - relying on the $\g$-invariance of the bilinear form $(-, -)_{\g}$ - to be a $2$-cocycle of $\g \tensor_k A$ with coefficients in $\bar{\Omega}^1_{A/k}$\footnote{... i.e. a representative of an isomorphism class $[\e] \in H^2_{\Lie}(\g \tensor_k A, \bar{\Omega}^1_{A/k})$.} and hence gives a central extension $\fraku$ of $\g \tensor_k A$ by $\bar{\Omega}_{A/k}^1$ (cf. proposition \ref{prop: lie_brackets_on_extensions}), whose underlying $k$-vector space is:
            $$(\g \tensor_k A) \oplus \bar{\Omega}_{A/k}^1$$
        and whose Lie bracket is:
            $$[-, -]_{\fraku} := [-, -]_{\g \tensor_k A} + \e$$
        \begin{lemma} \label{lemma: lie_brackets_on_UCEs_of_current_algebras}
            $[-, -]_{\fraku}$ as constructed above is a well-defined Lie bracket.
        \end{lemma}
            \begin{proof}
                By construction, it is already bilinear and skew-symmetric, so the only thing to show is that it satisfies the Jacobi identity. To this end, pick $x, y, z \in \g$ and $f, g, h \in A$ and then consider the following\footnote{We chose $\e(xf, yg) := (x, y)_{\g} g \bar{d}f$ because it makes the verification that $\e$ satisfies the Jacobi identity easier.} computations in $\bar{\Omega}^1_{A/k}$:
                    $$
                        \begin{aligned}
                            & [xf, [yg, zh]_{\fraku}]_{\fraku} + [yg, [zh, xf]_{\fraku}]_{\fraku} + [zh, [xf, yg]_{\fraku}]_{\fraku}
                            \\
                            & = [xf, [y, z]_{\g} gh + \e(yg, zh)]_{\fraku} + [yg, [z, x]_{\g} hf + \e(zh, xf)]_{\fraku} + [zh, [x, y]_{\g} fg + \e(xf, yg)]_{\fraku}
                            \\
                            & = [xf, [y, z]_{\g} gh]_{\fraku} + [yg, [z, x]_{\g} hf]_{\fraku} + [zh, [x, y]_{\g} fg]_{\fraku}
                            \\
                            & = 
                            \begin{aligned}
                                & \left( [x, [y, z]_{\g}]_{\g} fgh + \e(xf, [y, z]_{\g} gh) \right)
                                \\
                                + & \left( [y, [z, x]_{\g}]_{\g} ghf + \e(yg, [z, x]_{\g} hf) \right)
                                \\
                                + & \left( [z, [x, y]_{\g}]_{\g} hfg + \e(zh, [x, y]_{\g} fg) \right)
                            \end{aligned}
                            \\
                            & = \e(xf, [y, z]_{\g} gh) + \e(yg, [z, x]_{\g} hf) + \e(zh, [x, y]_{\g} fg)
                            \\
                            & = (x, [y, z]_{\g})_{\g} gh \bar{d}f + (y, [z, x]_{\g})_{\g} hf \bar{d}g + (z, [x, y]_{\g})_{\g} fg \bar{d}h
                            \\
                            & = (x, [y, z]_{\g})_{\g} ( gh \bar{d}f + hf \bar{d}g + fg\bar{d}h )
                            \\
                            & = 0
                        \end{aligned}
                    $$
                The fourth equality comes from the fact that $\g$ is a Lie algebra, and hence any triple of elements $x, y, z \in \g$ therein satisfies the Jacobi identity:
                    $$[x, [y, z]_{\g}]_{\g} + [y, [z, x]_{\g}]_{\g} + [z, [x, y]_{\g}]_{\g} = 0$$
                The last equality came from the fact that:
                    $$gh df + hf dg + fg dh = d(fgh)$$
                in $\Omega^1_{A/k}$, per definition \ref{def: kahler_differentials} (see also remark \ref{remark: differentials_and_derivations}), which hence implies that:
                    $$gh \bar{d}f + hf \bar{d}g + fg \bar{d}h \equiv 0 \pmod{d(A)}$$
                in $\bar{\Omega}^1_{A/k}$.
            \end{proof}
        Kassel then showed that the Lie algebra $\fraku$ as above is a UCE of $\g \tensor_k A$. 
        \begin{theorem}[The Kassel realisation] \label{theorem: kassel_realisation}
            \cite[Corollary 3.5]{kassel_universal_central_extensions_of_lie_algebras} For the (perfect) Lie $k$-algebra $\g \tensor_k A$, we have that:
                $$\uce(\g \tensor_k A) \cong (\g \tensor_k A) \oplus \bar{\Omega}^1_{A/k}$$
            with Lie bracket as in lemma \ref{lemma: lie_brackets_on_UCEs_of_current_algebras}.
        \end{theorem}