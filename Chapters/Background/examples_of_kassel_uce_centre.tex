\section{Some useful examples of UCEs}
    This section is for analysing some of the instances of UCEs of current algebras that are particularly useful to our purposes. Namely, we are interested in the UCEs of the Lie algebras:
        $$\g, \g[v^{\pm 1}], \g[v^{\pm 1}, t^{\pm 1}]$$
    (corresponding to $A$ being isomorphic to $k, \bbC[v^{\pm 1}]$, and $\bbC[v^{\pm 1}, t^{\pm 1}]$ respectively). 

    \subsection{Finite-dimensional simple Lie algebras}
        Consider, firstly, the case:
            $$A := \bbC$$
        It is trivial to see that:
            $$\dim_{\bbC} \bar{\Omega}^1_{\bbC/\bbC} \cong 0$$
        from which one sees that:
            $$\uce(\g) \cong \g$$
        i.e. $\g$ is its own universal central extension, and hence every central extension of $\g$ is trivial. Of course, there are other more conventional ways to see that $\g$ admits no non-trivial central extensions, but we thought we would include this example as a particularly degenerate case of Kassel's realisation of UCEs of current algebras.

    \subsection{Affine Lie algebras}
        Now, consider:
            $$A := \bbC[v^{\pm 1}]$$
    
        \begin{example}[Affine Lie algebras] \label{example: affine_lie_algebras_centres}
            Let us compute the UCE of $\g[v^{\pm 1}]$. From this, we can construct the so-called \say{untwisted affine Kac-Moody algebra} attached to $\g$ (cf. \cite[Chapter 7]{kac_infinite_dimensional_lie_algebras}). 

            To this end, let us firstly compute the underlying vector space of the centre of $\uce(\g[v^{\pm 1}])$. Abstractly, we know that it is isomorphic to $\bar{\Omega}^1_{\bbC[v^{\pm 1}]/\bbC}$, and it is also known that:
                $$\Omega^1_{\bbC[v^{\pm 1}]/\bbC} \cong \bbC[v^{\pm 1}] dv \cong \bigoplus_{m \in \Z} \bbC  v^m dv$$
            so the only non-trivial computation to make is that of $d( \bbC[v^{\pm 1}] )$. For this, let us consider how $d$ acts on the basis elements $v^m \in \bbC[v]$:
                $$d(v^m) = m v^{m - 1} dv$$
            We see that $d(v^m) = 0$ if and only if $m = 0$, and since the set $\{v^m\}_{m \in \Z}$ is a $\bbC$-linear basis for $\bbC[v^{\pm 1}]$, the set:
                $$\{m v^{m - 1} dv\}_{m \in \Z \setminus \{0\}}$$
            therefore spans $d(\bbC[v^{\pm 1}])$. It is also easy to see this subset of $d(\bbC[v^{\pm 1}])$ is $\bbC$-linearly independent and hence is a basis for $d(\bbC[v^{\pm 1}])$. This then tells us that:
                $$\bar{\Omega}^1_{\bbC[v^{\pm 1}]/\bbC} \cong \bbC  v^{-1} \bar{d}v$$
            The underlying vector space of $\uce(\g[v^{\pm 1}])$ is thus isomorphic to:
                $$\g[v^{\pm 1}] \oplus \bbC  v^{-1} \bar{d}v$$

            We know that the Lie bracket on $\uce(\g[v^{\pm 1}])$ is given by:
                $$[x f, y g]_{\g[v^{\pm 1}]} = [x, y]_{\g} fg + (x, y)_{\g} g \bar{d}f$$
            for all $x, y \in \g$ and all $f, g \in \bbC[v^{\pm 1}]$. Since:
                $$g \bar{d}f \in \bbC  v^{-1} \bar{d}v$$
            necessarily, the bracket can be given simplier as:
                $$[x f, y g]_{\g[v^{\pm 1}]} = [x, y]_{\g} fg + (x, y)_{\g} c(f, g) v^{-1} \bar{d}v$$
            for a uniquely determined scalar $c(f, g) \in k$, which can be computed explicitly. To do this, if suffices to perform the computation for basis elements of $\bbC[v^{\pm 1}]$, i.e. we can pick $f := v^m, g := v^n$ for some $m, n \in \Z$ and then consider the following:
                $$g \bar{d}f = v^n \bar{d}(v^m) = m v^{n + m} v^{-1} \bar{d}v$$
            This expression vanishes if and only if $m + n = 0$, so we have that:
                $$c(v^m, v^n) = m \delta_{n + m, 0}$$
            For general $f, g \in \bbC[v^{\pm 1}]$, we can write this more succinctly as:
                $$c(f, g) = \Res_{v = 0}( gdf )$$
            The Lie bracket on $\uce(\g[v^{\pm 1}])$ then takes the form:
                $$[x f, y g]_{\g[v^{\pm 1}]} = [x, y]_{\g} fg + (x, y)_{\g} \Res_{v = 0}(g df) v^{-1} \bar{d}v$$
                
            Note also, that there is a non-degenerate and invariant\footnote{Because $(-, -)_{\g}$ is invariant.} symmetric bilinear form on $\g[v^{\pm 1}]$ given by:
                $$(xf, yg)_{\g[v^{\pm 1}]} := (x, y)_{\g} \Res_{v = 0}(g df)$$
            for all $x, y \in \g$ and all $f, g \in \bbC[v^{\pm 1}]$. By invariance, the extension of this bilinear form to $\uce(\g[v^{\pm 1}])$ is necessarily invariant and degenerate (see lemma \ref{lemma: extending_bilinear_forms_to_central_extensions}).

            Let us also note that unlike $\uce(\g[v^{\pm 1}])$, the UCE of the perfect Lie algebra $\g[v]$ is trivial, since:
                $$\forall f, g \in \bbC[v]: \Res_{v = 0}(g df) = 0$$
        \end{example}

    \subsection{Toroidal Lie algebras}
        \begin{example}[Toroidal Lie algebras] \label{example: toroidal_lie_algebras_centres}
            Next, let us compute the UCE of $\g[v^{\pm 1}, t^{\pm 1}]$. For this, $\bbC[v^{\pm 1}, t^{\pm 1}]$ shall be endowed with the $\Z^2$-grading given by:
                $$\deg v^r t^s := (r, s)$$
            for all $(r, s) \in \Z^2$ so that remarks \ref{remark: gradings_on_1_forms} and \ref{remark: induced_gradings_on_UCEs} can be applied. 
            
            Firstly, let us compute its underlying vector space. Thanks to remark \ref{remark: induced_gradings_on_UCEs}, we know that the only non-trivial computation to make is that of (a basis of) the $\bbC$-vector space:
                $$\bar{\Omega}^1_{\bbC[v^{\pm 1}, t^{\pm 1}]/\bbC} := \Omega^1_{\bbC[v^{\pm 1}, t^{\pm 1}]/\bbC}/d\bbC[v^{\pm 1}, t^{\pm 1}]$$
            From remark \ref{remark: gradings_on_1_forms}, it can be inferred that:
                $$( \Omega^1_{\bbC[v^{\pm 1}, t^{\pm 1}]/\bbC} )_{(r, s)} \cong \bbC v^{r - 1} t^s dv \oplus \bbC v^r t^{s - 1} dt$$
                $$( d\bbC[v^{\pm 1}, t^{\pm 1}] )_{(r, s)} = d( \bbC[v^{\pm 1}, t^{\pm 1}]_{(r, s)} ) = d( \bbC v^r t^s ) = \bbC d(v^r t^s)$$
            which hold for all $(r, s) \in \Z^2$. We thus get:
                $$\bar{\Omega}^1_{\bbC[v^{\pm 1}, t^{\pm 1}]/\bbC} \cong \bigoplus_{(r, s) \in \Z^2} \frac{ \bbC v^{r - 1} t^s dv \oplus \bbC v^r t^{s - 1} dt }{ \bbC d(v^r t^s) }$$
            Next, let us compute the direct summands.
            \begin{itemize}
                \item To begin, observe that when $(r, s) = (0, 0)$, we have that:
                    $$\bbC v^{0 - 1} t^0 dv \oplus \bbC v^0 t^{0 - 1} dt = \bbC v^{-1} dv \oplus \bbC t^{-1} dt$$
                    $$\bbC d(v^0 t^0) = \bbC d(1) = \bbC 0 = 0$$
                This implies that:
                    $$\frac{ \bbC v^{r - 1} t^s dv \oplus \bbC v^r t^{s - 1} dt }{ \bbC d(v^r t^s) } \cong \bbC v^{-1} \bar{d}v \oplus \bbC t^{-1} \bar{d}t$$
                when $(r, s) = (0, 0)$.
                \item Next, consider $(r, s) \not = (0, 0)$. Any element $\omega \in \bbC v^{r - 1} t^s dv \oplus \bbC v^r t^{s - 1} dt$ can be written in the form:
                    $$\omega := a v^{r - 1} t^s dv + b v^r t^{s - 1} dt$$
                for some $a, b \in \bbC$, and by using the fact that $0 \equiv d(v^r t^s) \equiv r v^{r - 1} t^s dv + s v^r t^{s - 1} dt \pmod{d\bbC[v^{\pm 1}, t^{\pm 1}]}$, and hence $r v^{r - 1} t^s dv \equiv -s v^r t^{s - 1} dt \pmod{d\bbC[v^{\pm 1}, t^{\pm 1}]}$, we can rewrite:
                    $$
                        \begin{aligned}
                            \omega & \equiv a v^{r - 1} t^s dv + \frac{b}{s} s v^r t^{s - 1} dt \pmod{d\bbC[v^{\pm 1}, t^{\pm 1}]}
                            \\
                            & \equiv a v^{r - 1} t^s dv - \frac{b}{s} r v^{r - 1} t^s dv \pmod{d\bbC[v^{\pm 1}, t^{\pm 1}]}
                            \\
                            & \equiv \frac{as - br}{s} v^{r - 1} t^s dv \pmod{d\bbC[v^{\pm 1}, t^{\pm 1}]}
                        \end{aligned}
                    $$
                whenever $s \not = 0$ and similarly, we can rewrite:
                    $$\omega \equiv -\frac{as - br}{r} v^r t^{s - 1} dt \pmod{d\bbC[v^{\pm 1}, t^{\pm 1}]}$$
                whenever $r \not = 0$. Since $a, b \in \bbC$ are arbitrary, any element of $\bbC$ can be written in the form $as - br$ (as $\bbC$ is a vector space over itself), and hence we gather from the rewritings of $\omega$ from above that:
                    $$
                        \frac{ \bbC v^{r - 1} t^s dv \oplus \bbC v^r t^{s - 1} dt }{ \bbC d(v^r t^s) } \cong
                        \begin{cases}
                            \text{$\bbC \frac1s v^{r - 1} t^s \bar{d}v$ if $(r, s) \in \Z \x (\Z \setminus \{0\})$}
                            \\
                            \text{$\bbC \frac{-1}{r} v^r t^{-1} \bar{d}t$ if $(r, s) \in (\Z \setminus \{0\}) \x \{0\}$}
                        \end{cases}
                    $$
                for all $(r, s) \not = (0, 0)$.
            \end{itemize}
            In conclusion, we have computed the direct summands to be:
                $$
                    \frac{ \bbC v^{r - 1} t^s dv \oplus \bbC v^r t^{s - 1} dt }{ \bbC d(v^r t^s) } \cong 
                    \begin{cases}
                        \text{$\bbC \frac1s v^{r - 1} t^s \bar{d}v$ if $(r, s) \in \Z \x (\Z \setminus \{0\})$}
                        \\
                        \text{$\bbC \frac{-1}{r} v^r t^{-1} \bar{d}t$ if $(r, s) \in (\Z \setminus \{0\}) \x \{0\}$}
                        \\
                        \text{$\bbC v^{-1}\bar{d}v \oplus \bbC t^{-1} \bar{d}t$ if $(r, s) = (0, 0)$}
                    \end{cases}
                $$
            As such, we are now able to write $\bar{\Omega}^1_{\bbC[v^{\pm 1}, t^{\pm 1}]/\bbC}$ as a direct sum as follows:
                $$\bar{\Omega}^1_{\bbC[v^{\pm 1}, t^{\pm 1}]/\bbC} \cong ( \bigoplus_{(r, s) \in \Z^2} \bbC K_{r, s}) \oplus \bbC c_v \oplus \bbC c_t$$
            wherein the basis elements are given by:
                $$
                    K_{r, s} :=
                    \begin{cases}
                        \text{$\frac1s v^{r - 1} t^s \bar{d}v$ if $(r, s) \in \Z \x (\Z \setminus \{0\})$}
                        \\
                        \text{$-\frac1r v^r t^{-1} \bar{d}t$ if $(r, s) \in (\Z \setminus \{0\}) \x \{0\}$}
                        \\
                        \text{$0$ if $(r, s) = (0, 0)$}
                    \end{cases}
                $$
                $$c_v := v^{-1} \bar{d}v, c_t := t^{-1} \bar{d}t$$
            Moreover, any element of the form:
                $$v^n t^q \bar{d}(v^m t^p) \in \bar{\Omega}^1_{\bbC[v^{\pm 1}, t^{\pm 1}]/\bbC}$$
            can be written in terms of the basis vectors $K_{r, s}, c_v, c_t$ in the following manner:
                $$v^n t^q \bar{d}(v^m t^p) = (mq - np) K_{m + n, p + q} + \delta_{(m, p) + (n, q), (0, 0)} ( m c_v + p c_t )$$
            (cf. \cite[p. 35]{wendlandt_formal_shift_operators_on_yangian_doubles}).

            Finally, let us note that the Lie bracket on $\uce(\g[v^{\pm 1}, t^{\pm 1}])$ is given by:
                $$
                    \begin{aligned}
                        & [x v^m t^p, y v^n t^q]_{\uce(\g[v^{\pm 1}, t^{\pm 1}])}
                        \\
                        = & [x, y]_{\g} v^{m + n} t^{p + q} + (x, y)_{\g} v^n t^q \bar{d}(v^m t^p)
                        \\
                        = & [x, y]_{\g} v^{m + n} t^{p + q} + (x, y)_{\g} \left( (mq - np) K_{m + n, p + q} + \delta_{(m, p) + (n, q), (0, 0)} ( m c_v + p c_t ) \right)
                    \end{aligned}
                $$
            for all $x, y \in \g$ and all $(m, p), (n, q) \in \Z^2$. As a side note, let us note that interestingly, unlike how $\g[v]$ admits only the trivial UCE, $\g[v^{\pm 1}, t]$ admits a non-trivial UCE, on which the Lie bracket is given by:
                $$[x v^m t^p, y v^n t^q]_{\uce(\g[v^{\pm 1}, t])} = [x, y]_{\g} v^{m + n} t^{p + q} + (x, y)_{\g} \left( (mq - np) K_{m + n, p + q} + \delta_{m + n, 0} m c_v \right)$$
            for all $x, y \in \g$ and all $(m, p), (n, q) \in \Z \x \Z_{\geq 0}$. This also demonstrates that the vector subspace $\uce(\g[v^{\pm 1}, t])$ is closed under the Lie bracket on the larger vector space $\uce(\g[v^{\pm 1}, t^{\pm 1}])$, and therefore is a Lie subalgebra thereof.
        \end{example}